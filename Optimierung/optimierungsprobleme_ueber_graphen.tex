\section{Optimierungsprobleme über Graphen}
\subsection{Grundbegriffe}
Ausgangssituation (Graph):
$X = \set{x_1, \dots, x_m}$ ist Bogenmenge ($|X| = m$), 
$V = \set{v_1, \dots, v_m}$ Knotenmenge ($|V| = n$), 
Abbildungen $a, e:X \rightarrow V$.
\begin{definition}
  Die Abbildungen $a, e$ heißen \keyword{Inzidenzabbildungen}.

$G = \set{X, V, a, e}$ ist Graph mit Anfangsknoten $a(x)$ und Endknoten $e(x)$. 
$x_{ik} \in X$ mit $a(a_{ik}) = v_i$, $e(x_{ik}) = v_k$. 
\end{definition}

\begin{definition}
  Die \keyword{Adjazenzmatrix} $A$ ist definiert als
\begin{align*}
  A =(a_{ij})_{i, j = 1}^n :=
  \begin{cases}
    0, &x_{ij} \notin X \\
    1, &x_{ij} \in X 
  \end{cases}
\end{align*}
und die \keyword{Inzidenzmatrix} als
\begin{align*}
  B=(b_{ij})_{i, j = 1}^{n, m} :=
  \begin{cases}
    1, &v_i = a(x_j)\\
    0, & \text{sonst}\\
    1, &v_i = e(x_j)
  \end{cases}
\end{align*}

\end{definition}