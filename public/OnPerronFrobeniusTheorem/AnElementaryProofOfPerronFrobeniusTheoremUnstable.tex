\documentclass{article}
%My personal maths package
%\bibliography{Bibliography.bib}

%%%%%%%%%%%%%%%%%%%%%%%%%%%%%%%%%%%%%%%%%%%%%%%%%%%%%%%%%%%%%%%%%%%%%%%%%%%%%%
%%%%% MATH PACKAGES %%%%%%%%%%%%%%%%%%%%%%%%%%%%%%%%%%%%%%%%%%%%%%%%%%%%%%%%%%
%%%%%%%%%%%%%%%%%%%%%%%%%%%%%%%%%%%%%%%%%%%%%%%%%%%%%%%%%%%%%%%%%%%%%%%%%%%%%%

% very good package
%\usepackage{mathtools}

%%% font stuff
\usepackage[T1]{fontenc}        % for capitals in section /paragraph etc.
\usepackage[utf8]{inputenc}     % use utf8 symbols in code

\usepackage{amssymb,amsmath,amsfonts} % amsthm not needed -- use my own envs
%\usepackage{mathtools}
%\mathtoolsset{showonlyrefs}
% further alternative math packages: unicode-math, abx-math
\usepackage{bm}
\usepackage{mathrsfs} % used for: fraktal math letters
\usepackage[bigsqcap]{stmaryrd} % used for: big square cap symbol
\usepackage{xargs}

% standard packages
%\usepackage{color} % for color

%%
%%index

\newcommand*{\keyword}[2][\empty]{\emph{#2}\ifx#1\empty\index{#2}\else\index{#1}\fi}
\newcommand*{\person}[1]{\textsc{#1}}

%%%%%%%%%%%%%%%%%%%%%%%%%%%%%%%%%%%%%%%%%%%%%%%%%%%%%%%%%%%%%%%%%%%%%%%%%%%%%%%%%%%%%%%%%%%%%%%%%%%%%%%%%%%%
%%%%% MATH ALPHABETS & SYMBOLS %%%%%%%%%%%%%%%%%%%%%%%%%%%%%%%%%%%%%%%%%%%%%%%%%%%%%%%%%%%%%%%%%%%%%%%%%%%%%
%%%%%%%%%%%%%%%%%%%%%%%%%%%%%%%%%%%%%%%%%%%%%%%%%%%%%%%%%%%%%%%%%%%%%%%%%%%%%%%%%%%%%%%%%%%%%%%%%%%%%%%%%%%%

% w: http://milde.users.sourceforge.net/LUCR/Math/math-font-selection.xhtml

% ===== Set quick commands for math letters ================================================================
% calagraphic letters (only upper case available; standard)
\newcommand{\cA}{\mathcal{A}}
\newcommand{\cB}{\mathcal{B}}
\newcommand{\cC}{\mathcal{C}}
\newcommand{\cD}{\mathcal{D}}
\newcommand{\cE}{\mathcal{E}}
\newcommand{\cF}{\mathcal{F}}
\newcommand{\cG}{\mathcal{G}}
\newcommand{\cH}{\mathcal{H}}
\newcommand{\cI}{\mathcal{I}}
\newcommand{\cJ}{\mathcal{J}}
\newcommand{\cK}{\mathcal{K}}
\newcommand{\cL}{\mathcal{L}}
\newcommand{\cM}{\mathcal{M}}
\newcommand{\cN}{\mathcal{N}}
\newcommand{\cO}{\mathcal{O}}
\newcommand{\cP}{\mathcal{P}}
\newcommand{\cQ}{\mathcal{Q}}
\newcommand{\cR}{\mathcal{R}}
\newcommand{\cS}{\mathcal{S}}
\newcommand{\cT}{\mathcal{T}}
\newcommand{\cU}{\mathcal{U}}
\newcommand{\cV}{\mathcal{V}}
\newcommand{\cW}{\mathcal{W}}
\newcommand{\cX}{\mathcal{X}}
\newcommand{\cY}{\mathcal{Y}}
\newcommand{\cZ}{\mathcal{Z}}

% bold math letters (standard)
\newcommand{\bfA}{\mathbf{A}}
\newcommand{\bfB}{\mathbf{B}}
\newcommand{\bfC}{\mathbf{C}}
\newcommand{\bfD}{\mathbf{D}}
\newcommand{\bfE}{\mathbf{E}}
\newcommand{\bfF}{\mathbf{F}}
\newcommand{\bfG}{\mathbf{G}}
\newcommand{\bfH}{\mathbf{H}}
\newcommand{\bfI}{\mathbf{I}}
\newcommand{\bfJ}{\mathbf{J}}
\newcommand{\bfK}{\mathbf{K}}
\newcommand{\bfL}{\mathbf{L}}
\newcommand{\bfM}{\mathbf{M}}
\newcommand{\bfN}{\mathbf{N}}
\newcommand{\bfO}{\mathbf{O}}
\newcommand{\bfP}{\mathbf{P}}
\newcommand{\bfQ}{\mathbf{Q}}
\newcommand{\bfR}{\mathbf{R}}
\newcommand{\bfS}{\mathbf{S}}
\newcommand{\bfT}{\mathbf{T}}
\newcommand{\bfU}{\mathbf{U}}
\newcommand{\bfV}{\mathbf{V}}
\newcommand{\bfW}{\mathbf{W}}
\newcommand{\bfX}{\mathbf{X}}
\newcommand{\bfY}{\mathbf{Y}}
\newcommand{\bfZ}{\mathbf{Z}}
\newcommand{\bfa}{\mathbf{a}}
\newcommand{\bfb}{\mathbf{b}}
\newcommand{\bfc}{\mathbf{c}}
\newcommand{\bfd}{\mathbf{d}}
\newcommand{\bfe}{\mathbf{e}}
\newcommand{\bff}{\mathbf{f}}
\newcommand{\bfg}{\mathbf{g}}
\newcommand{\bfh}{\mathbf{h}}
\newcommand{\bfi}{\mathbf{i}}
\newcommand{\bfj}{\mathbf{j}}
\newcommand{\bfk}{\mathbf{k}}
\newcommand{\bfl}{\mathbf{l}}
\newcommand{\bfm}{\mathbf{m}}
\newcommand{\bfn}{\mathbf{n}}
\newcommand{\bfo}{\mathbf{o}}
\newcommand{\bfp}{\mathbf{p}}
\newcommand{\bfq}{\mathbf{q}}
\newcommand{\bfr}{\mathbf{r}}
\newcommand{\bfs}{\mathbf{s}}
\newcommand{\bft}{\mathbf{t}}
\newcommand{\bfu}{\mathbf{u}}
\newcommand{\bfv}{\mathbf{v}}
\newcommand{\bfw}{\mathbf{w}}
\newcommand{\bfx}{\mathbf{x}}
\newcommand{\bfy}{\mathbf{y}}
\newcommand{\bfz}{\mathbf{z}}

% fractal math letters (standard)
\newcommand{\fkA}{\mathfrak{A}}
\newcommand{\fkB}{\mathfrak{B}}
\newcommand{\fkC}{\mathfrak{C}}
\newcommand{\fkD}{\mathfrak{D}}
\newcommand{\fkE}{\mathfrak{E}}
\newcommand{\fkF}{\mathfrak{F}}
\newcommand{\fkG}{\mathfrak{G}}
\newcommand{\fkH}{\mathfrak{H}}
\newcommand{\fkI}{\mathfrak{I}}
\newcommand{\fkJ}{\mathfrak{J}}
\newcommand{\fkK}{\mathfrak{K}}
\newcommand{\fkL}{\mathfrak{L}}
\newcommand{\fkM}{\mathfrak{M}}
\newcommand{\fkN}{\mathfrak{N}}
\newcommand{\fkO}{\mathfrak{O}}
\newcommand{\fkP}{\mathfrak{P}}
\newcommand{\fkQ}{\mathfrak{Q}}
\newcommand{\fkR}{\mathfrak{R}}
\newcommand{\fkS}{\mathfrak{S}}
\newcommand{\fkT}{\mathfrak{T}}
\newcommand{\fkU}{\mathfrak{U}}
\newcommand{\fkV}{\mathfrak{V}}
\newcommand{\fkW}{\mathfrak{W}}
\newcommand{\fkX}{\mathfrak{X}}
\newcommand{\fkY}{\mathfrak{Y}}
\newcommand{\fkZ}{\mathfrak{Z}}
\newcommand{\fka}{\mathfrak{a}}
\newcommand{\fkb}{\mathfrak{b}}
\newcommand{\fkc}{\mathfrak{c}}
\newcommand{\fkd}{\mathfrak{d}}
\newcommand{\fke}{\mathfrak{e}}
\newcommand{\fkf}{\mathfrak{f}}
\newcommand{\fkg}{\mathfrak{g}}
\newcommand{\fkh}{\mathfrak{h}}
\newcommand{\fki}{\mathfrak{i}}
\newcommand{\fkj}{\mathfrak{j}}
\newcommand{\fkk}{\mathfrak{k}}
\newcommand{\fkl}{\mathfrak{l}}
\newcommand{\fkm}{\mathfrak{m}}
\newcommand{\fkn}{\mathfrak{n}}
\newcommand{\fko}{\mathfrak{o}}
\newcommand{\fkp}{\mathfrak{p}}
\newcommand{\fkq}{\mathfrak{q}}
\newcommand{\fkr}{\mathfrak{r}}
\newcommand{\fks}{\mathfrak{s}}
\newcommand{\fkt}{\mathfrak{t}}
\newcommand{\fku}{\mathfrak{u}}
\newcommand{\fkv}{\mathfrak{v}}
\newcommand{\fkw}{\mathfrak{w}}
\newcommand{\fkx}{\mathfrak{x}}
\newcommand{\fky}{\mathfrak{y}}
\newcommand{\fkz}{\mathfrak{z}}

% script math symbols (only uppercase; package: mathrsfs)
\newcommand{\sA}{\mathscr{A}}
\newcommand{\sB}{\mathscr{B}}
\newcommand{\sC}{\mathscr{C}}
\newcommand{\sD}{\mathscr{D}}
\newcommand{\sE}{\mathscr{E}}
\newcommand{\sF}{\mathscr{F}}
\newcommand{\sG}{\mathscr{G}}
\newcommand{\sH}{\mathscr{H}}
\newcommand{\sI}{\mathscr{I}}
\newcommand{\sJ}{\mathscr{J}}
\newcommand{\sK}{\mathscr{K}}
\newcommand{\sL}{\mathscr{L}}
\newcommand{\sM}{\mathscr{M}}
\newcommand{\sN}{\mathscr{N}}
\newcommand{\sO}{\mathscr{O}}
\newcommand{\sP}{\mathscr{P}}
\newcommand{\sQ}{\mathscr{Q}}
\newcommand{\sR}{\mathscr{R}}
\newcommand{\sS}{\mathscr{S}}
\newcommand{\sT}{\mathscr{T}}
\newcommand{\sU}{\mathscr{U}}
\newcommand{\sV}{\mathscr{V}}
\newcommand{\sW}{\mathscr{W}}
\newcommand{\sX}{\mathscr{X}}
\newcommand{\sY}{\mathscr{Y}}
\newcommand{\sZ}{\mathscr{Z}}

%%%%%%%%%%%%%%%%%%%%%%%%%%%%%%%%%%%%%%%%%%%%%%%%%%%%%%%%%%%%%%%%%%%%%%%%%%%%%%%%%%%%%%
%%%% BOLD MATH IN BOLD TEXT ENVIRONMENT %%%%%%%%%%%%%%%%%%%%%%%%%%%%%%%%%%%%%%%%%%%%%%
%%%%%%%%%%%%%%%%%%%%%%%%%%%%%%%%%%%%%%%%%%%%%%%%%%%%%%%%%%%%%%%%%%%%%%%%%%%%%%%%%%%%%%

% for bold math in bold text (e.g. sections)
\makeatletter
\g@addto@macro\bfseries{\boldmath}
\makeatother

\def\brackets#1{\ifx#1\empty\else\left(#1\right)\fi}

%%%%%%%%%%%%%%%%%%%%%%%%%%%%%%%%%%%%%%%%%%%%%%%%%%%%%%%%%%%%%%%%%%%%%%%%%%%%%%%%%%%%%%
%%%%% CATEGORY THEORY %%%%%%%%%%%%%%%%%%%%%%%%%%%%%%%%%%%%%%%%%%%%%%%%%%%%%%%%%%%%%%%%
%%%%%%%%%%%%%%%%%%%%%%%%%%%%%%%%%%%%%%%%%%%%%%%%%%%%%%%%%%%%%%%%%%%%%%%%%%%%%%%%%%%%%%

% ===== Category theory concepts =====================================================

\newcommand{\Ob}{\mathop\mathrm{Ob}}
\newcommand{\Mor}{\mathop\mathrm{Mor}}

\newcommand{\ccoprod}{\bigsqcup}
\newcommand{\cprod}{\bigsqcap}
\newcommand{\cincl}{\mathop\mathrm{incl}}
\newcommand{\cproj}{\mathop\mathrm{pr}}


% ===== Define standard categories ===================================================

% Define sets
\newcommand{\Set}{\mathbf{Set}}
% Define set-builder operator (equivalent to gen for algebras)
\newcommand{\set}[1]{\left\{#1\right\}}
% define interval operator: o - open, c - closed
\newcommand{\intervalcc}[2]{\left[#1,#2\right]}
\newcommand{\intervalco}[2]{\left[#1,#2\right)}
\newcommand{\intervaloc}[2]{\left(#1,#2\right]}
\newcommand{\intervaloo}[2]{\left(#1,#2\right)}

\newcommand{\inter}{\mathop\mathrm{int}}
\newcommand{\face}{\mathop\mathrm{F}}
\newcommand{\Pol}{\mathop\mathrm{Pol}}
\newcommand{\Inv}{\mathop\mathrm{Inv}}
\def\struct#1{\gen{#1}}

\let\originaltimes\times%
\renewcommand{\times}{\mathbin{\sqcap}}
\newcommand\settimes{\originaltimes}
\newcommand{\setleq}{\subseteq}
\newcommand{\setgeq}{\supseteq}

\newcommand{\pderive}[2]{\frac{\partial{#1}}{\partial{#2}}}
\renewcommand{\div}{\mathop\mathrm{div}}

%% Diffgeo
\def\Ric{\mathop\mathrm{Ric}}
\def\ric{\mathop\mathrm{ric}}
\def\tr{\mathop\mathrm{tr}}

\def\cotimes{\mathbin{\sqcup}}

\def\@rightopen#1{\ifx#1]{\right]}\else{\interval@errmessage}\fi}
\def\@leftclosed[#1){\left[#1\right)}
\makeatother

% finite
\newcommand{\fin}{\mathrm{fin}}

% Define groups (optarg: properties such as -> abelian, noetherian (acc), artinian (dcc) etc.)
\newcommand{\Grp}[1][\empty]{\if\empty{#1}{\mathbf{Grp}}\else{\mathbf{Grp}_{#1}}}
\def\PGL{\mathrm{PGL}}
\def\PGammaL{\mathrm{P\Gamma L}}
\def\GL{\mathrm{GL}}
% Define rings
\newcommand{\rg}{\mathrm{rg}} %rank of a matrix
\newcommand{\Rg}[1][\empty]{\if\empty{#1}{\mathbf{Rg}}\else{\mathbf{Rg}_{#1}}}
\edef\units#1{#1^{\settimes}}
\def\dual#1{#1^{\ast}}

%% redefine the command \P to produce the projective functor in math mode
\let\parsymb\P%
\def\P{\ifmmode\mathrm{P}\else\parsymb\fi}
\renewcommand{\iff}{\ifmmode\equival\else{if and only if}\fi}
\newcommand{\quotring}{\mathop{\mathrm{Q}}}
\newcommand{\rad}{\mathrm{rad}}
% Standard rings
% integral domains
\newcommand{\ID}{\mathbf{ID}}
% unique factorization domains
\newcommand{\UFD}{\mathbf{UFD}}
% principal ideal domains
\newcommand{\PID}{\mathbf{PID}}

% Define modules over a group or ring
\newcommand{\Mod}[1]{\mathbf{Mod}_{#1}}
% Define vector space over a field
\renewcommand{\Vec}[1]{\mathbf{Vec}_{#1}}%

% when cases
\def\otherwise{\textrm{otherwise}}

% new concepts
\newcommand{\new}[1]{\emph{#1}}

\usepackage{xifthen,xstring}

% replace the bar command by overline when argument just one character (shorter and better)
%$\let\oldbar\bar
%\renewcommand{\bar}[1]{\StrLen{#1}[\length]\ifthenelse{\length > 1}{\overline{#1}}{\oldbar{#1}}}
\def\bar{\overline}

%% argument in equatoin
\def\arg{\bullet}

% groups and algebras
\newcommand{\Con}{\mathop\mathrm{Con}}
\newcommand{\Sub}{\mathop\mathrm{Sub}}
\newcommand{\Hom}{\mathop\mathrm{Hom}}
\newcommand{\Aut}{\mathop\mathrm{Aut}}
\newcommand{\Out}{\mathop\mathrm{Out}}
\newcommand{\End}{\mathop\mathrm{End}}
\newcommand{\id}{\mathop\mathrm{id}}
\newcommand{\rk}{\mathop\mathrm{rk}} % rank of a group module/ lattice
\newcommandx{\con}[1][1=\empty]{\ifx#1\empty{\mathop{\mathrm{con}}}\else{\mathop{\mathrm{con}}\left(#1\right)}\fi}
\newcommand{\leftsemidirprod}[1][]{\mathbin{\ifx&#1&\ltimes\else{\ltimes_{#1}}\fi}}
\newcommand{\rightsemidirprod}[1][]{\ifx#1\empty\rtimes\else{\rtimes_{#1}}\fi}
\newcommand{\normalisor}[2][]{\ifx#1\empty{\mathrm{N}\left(#2\right)}\else{\mathrm{N}_{#1} \left(#2\right)}\fi}
% support
\newcommand{\spt}{\mathop{\mathrm{spt}}}
% commutator
\newcommand{\gcom}[2]{\left[#1,#2\right]}

% physics stuff
\newcommand{\float}[3][\empty]{\ifx#1\empty{{#2}\cdot{10^{#3}}}\else{{#2}\cdot{{#1}^{#3}}}\fi}
\makeatletter
\def\newunit#1{\@namedef{#1}{\mathrm{#1}}}
\def\mum{\mathrm{\mu m}}
\def\ohm{\Omega}
\newunit{V}
\newunit{mV}
\newunit{kV}
\newunit{s}
\newunit{ms}
\def\mus{\mathrm{\mu s}}
\newunit{m}
\newunit{nm}
\newunit{cm}
\newunit{mm}
\newunit{fF}
\newunit{A}

\newunit{fA}
\newunit{C}

% elements
\def\newelement#1{\@namedef{#1}{\mathrm{#1}}}
\newelement{Si}
\makeatother


% groups
\newcommand{\ord}{\mathop\mathrm{ord}}
\newcommand{\divides}{|}

\newcommand{\conleq}{\trianglelefteq}
\newcommand{\congeq}{\trianglerighteq}

% common algebraic objects
\newcommand{\reals}{\mathbb{R}} 			% real numbers
\newcommand{\nats}{\mathbb{N}} 				% natural numbers
\newcommand{\ints}{\mathbb{Z}} 				% integers
\newcommand{\rats}{\mathbb{Q}}				% rationals
\newcommand{\complex}{\mathbb{C}}			% complex numbers
\newcommand{\field}[1]{\mathbb{F}_{#1}}  		% finite field
\newcommand{\cards}{\boldsymbol{Cn}}                     % The cardinal numbers
\newcommand{\ords}{\boldsymbol{On}}                      % The ordinal numbers

% graphs
\def\KG{\mathop\mathrm{KG}}                     % Knesergraph

\newcommand{\uvect}{\boldsymbol{e}}

% new operators and relations

%%%%%%%%%%%%%%%%%%%
% complex numbers %
%%%%%%%%%%%%%%%%%%%

\renewcommand{\Re}{\mathop\mathrm{Re}}		% real part
\renewcommand{\Im}{\mathop\mathrm{Im}}		% imaginary part
\newcommand{\sgn}{\mathop\mathrm{sgn}}				% the sign operator (0 for 0)

%%%%%%%%%%%%%%%%
% reel numbers %
%%%%%%%%%%%%%%%%

\newcommand{\floor}[1]{\left\lfloor#1\right\rfloor}
\newcommand{\ceil}[1]{\left\lceil#1\right\rceil}

%%%%%%%%%%%%%%%%%%
% set operations %
%%%%%%%%%%%%%%%%%%

\newcommand{\intersect}{\cap}			% intersect to sets
\newcommand{\setjoin}{\cup}				% join two sets
\newcommand{\setmeet}{\cap}                     % intersect to sets
\newcommand{\bigsetjoin}{\bigcup}			% the union of sets ... subscripts to be added
\newcommand{\distunion}{\dot{\bigcup}}	% disjoint union of sets ... subscripts to be added
\newcommand{\bigsetmeet}{\bigcap}		% intersection of sets
\newcommand{\powerset}[1][]{\ifx&#1&\mathcal{P}\else\mathcal{P}_{#1}\fi}		% powerset ... to be customized
\newcommand{\card}[1]{\left|#1\right|}

%%%%%%%%%%%%%%%%%%%%%%%%%%%%%%%%%%%%%%%%
% composition operations of structures %
%%%%%%%%%%%%%%%%%%%%%%%%%%%%%%%%%%%%%%%%

%\newcommand{\setprod}{\bigtimes}			% setproduct - needed
\newcommand{\dirprod}{\bigotimes} 			% direct product for groups and spaces
\newcommand{\dirtimes}{\otimes}				% direct multiply for groups and spaces
\newcommand{\dirsum}{\bigoplus} 			% direct sum for groups and spaces
\newcommand{\dirplus}{\oplus}				% direct add for groups and spaces
\newcommand{\inprod}[2]{\left\langle #1,#2 \right\rangle}

\newcommand{\tuple}{\meet}
\newcommand{\cotuple}{\join}

%%%%%%%%%%%%%%
% categories %
%%%%%%%%%%%%%%

% combinatorics
%%

\renewcommand{\binom}[3][\empty]{\if\empty{#1}{{#2 \choose #3}}\else{{#2 \choose #3}_{#1}}}

%%%%%%%%%%%%%%%%%%%%%%%%%%%%%%%%%%%%%%%%%%%%%%%%%%%%%
% metric spaces and normed spaces and vector spaces %
%%%%%%%%%%%%%%%%%%%%%%%%%%%%%%%%%%%%%%%%%%%%%%%%%%%%%

\newcommand{\dist}{\mathop\mathrm{dist}}				% distance operator ... dist(A,b), where A is a set and b a point
\newcommand{\diam}{\mathop\mathrm{diam}}	% diameter operator for sets
\newcommand{\norm}[1]{\left\Vert #1 \right\Vert}	% norm in a normed space ... subscript to be added
\newcommand{\conv}{\mathop\mathrm{conv}} 			% convex hull - vectorspaces
\newcommand{\lin}{\mathop\mathrm{lin}} 				% linear hull - vectorspaces
\newcommand{\aff}{\mathop\mathrm{aff}}				% affine hull - vectorspaces

%%%%%%%%%%%%%%%%%%%%%%
% operators in rings %
%%%%%%%%%%%%%%%%%%%%%%

\newcommand{\lcm}{\mathop\mathrm{lcm}}				% least common multiple - in euclidean rings
\renewcommand{\gcd}{\mathop\mathrm{gcd}}				% greatest command devisor - in euclidean rings
\newcommand{\res}{\mathop\mathrm{res}}				% residue of p mod q is res(p,q)
\renewcommand{\mod}{\textrm{ mod }}

%%%%%%%%%%%%%%%%%%%
% logical symbols %
%%%%%%%%%%%%%%%%%%%

%\newcommand{\impliedby}{\Leftarrow}				% reverse implicatoin arrow
%\newcommand{\implies}{\Rightarrow}				% implication
\newcommand{\equival}{\Leftrightarrow}				% equivalence

%%%%%%%%%%%%%%%%%%%%%%%%%%%%%%%%%%
% functions - elementary symbols %
%%%%%%%%%%%%%%%%%%%%%%%%%%%%%%%%%%

\newcommand{\rest}[1]{\left. #1\right\vert}		% restriction of a function to a set / also used as restriction in other terms like differential expressions / evaluation of a function
\newcommand{\rto}[3][]{#2\ifx&#1&\rightarrow\else\stackrel{#1}{\rightarrow}\fi#3}
\renewcommand{\to}{\rightarrow}							% arrow between domain and image
\newcommand{\dom}{\mathop\mathrm{dom}}					% domain of a function
\newcommand{\im}{\mathop\mathrm{im}}						% image of a function
\newcommand{\compose}{\circ}							% compose two functions
\newcommand{\cont}{\mathop\mathrm{C}}					% continuous functions from a domain into the reels or complex numbers 

%%%%%%%%%%%%%%%%%%%%
% groups - symbols %
%%%%%%%%%%%%%%%%%%%%

\newcommand{\stab}[1][]{\if&#1{\mathop\mathrm{stab}}&\else{\mathop\mathrm{stab}_{#1}}\fi}					% the stabilizer ... subscripts to be added
\newcommand{\orb}[1][]{\ifx&#1&\mathrm{orb}\else\mathrm{orb}_{#1}\fi} 					% orbit ... subscrit to be added (group)
\newcommand{\gen}[2][\empty]{\ifx#1\empty{\left\langle#2\right\rangle}\else{\left\langle#2\right\rangle_{#1}}\fi}					% generate ... kind of hull operator ---- to be thought of !!!!!!!

\def\Clo{\mathrm{Clo}}
\def\Loc{\mathrm{Loc}}
%% nets
\newcommand{\net}[2][\empty]{\ifx#1\empty{\left(#2\right)}\else{{\left(#2\right)}_{#1}}\fi}

%% open half ray
\newcommand{\ray}[2]{R_{#1}(#2)}

%% new
\let\oldcong\cong%
\newcommand{\iso}{\oldcong}

\def\cong{\equiv}
\newcommand{\base}[2]{\left[#2\right]_{#1}}                                   % base n expansion of some number
%%%%%%%%%%%%%%%%%%%%%%%%%%%%%%%%%
% matrices and linear operators %
%%%%%%%%%%%%%%%%%%%%%%%%%%%%%%%%%

\newcommand{\diag}{\mathop\mathrm{diag}}				% diagonal matrix or operator
\newcommand{\Eig}[1]{\mathop\mathrm{Eig}_{#1}}		% eigenspace for a certain eigenvalie		
\newcommand{\trace}{\mathop\mathrm{tr}}				% trace of a matrix
\newcommand{\trans}{\top} 							% transponse matrix

%%%%%%%%%%%%%
% constants %
%%%%%%%%%%%%%

\renewcommand{\i}{\boldsymbol{i}}			% imaginary unit
\newcommand{\e}{\boldsymbol{e}}				% the Eulerian constant

%%%%%%%%%%%%%%%%%%%%%%%%%%%%%%
% limit operators and arrows %
%%%%%%%%%%%%%%%%%%%%%%%%%%%%%%

\newcommand{\upto}{\uparrow}				% convergence from above
\newcommand{\downto}{\downarrow}			% convergence from below

%%
% other
%%
\newcommand{\cind}{\mathop\mathrm{Ind}}		% Cauchy index
\newcommand{\sgnc}{\sigma}					% sign changes
\newcommand{\wnumb}{\omega}					% winding number
\newcommand{\cfunc}{\mathop\mathrm{Cf}}		% Cauchy function of a compact curve in complex\setminus\{0\}


% evaluation of a function as a difference or single value

\newcommand{\abs}[1]{\left|#1\right|}
\newcommand{\conj}[1]{\overline{#1}}
\newcommand{\diff}{\mathop\mathrm{d}}

%%%%% test
\newcommand{\distjoin}{\mathaccent\cdot\cup}	% to be modified (name)
\newcommand{\cl}{\mathop\mathrm{cl}}				% topological closure
\newcommand{\sphere}{\mathbb{S}} % n-sphere
\newcommand{\ball}{\mathbb{B}} % n-ball
\newcommand{\bound}{\partial}
\newcommand{\bigmeet}{\mathop\mathrm{\bigwedge}}
\newcommand{\bigjoin}{\mathop\mathrm{\bigvee}}
\newcommandx{\rchar}[1][1=\empty]{\mathop\mathrm{char}\brackets{#1}}              % characteristic of a ring
\newcommand{\lgor}{\vee}                               % logical
\newcommand{\lgand}{\wedge}
\newcommand{\codim}{\mathop\mathrm{codim}}
\newcommand{\row}{\mathop\mathrm{row}}
\newcommand{\cone}{\mathop\mathrm{cone}}
\newcommand{\comp}{\mathop\mathrm{comp}}
\newcommand{\proj}{\mathrm{P}}
\def\PG{\mathrm{PG}}           % projective space
\newcommand{\meet}{\wedge}
\newcommand{\join}{\vee}
\newcommand{\col}{\mathop\mathrm{col}}
\newcommand{\vol}{\mathop\mathrm{vol}\nolimits}
%arrangements
\newcommand{\tpert}{\mathop\mathrm{tpert}}
\renewcommand{\epsilon}{\varepsilon}
% groups
\newcommand{\symgr}{\mathop\mathrm{Sym}}
\newcommand{\symalg}{\mathop\mathrm{S}}
\newcommand{\extalg}{\mathop\mathrm{\Lambda}}
\newcommand{\extpow}[1]{\mathop\mathrm{\Lambda}^{#1}}

%% set hulloperators -> define



\newcommandx{\homl}[3][1=1,2=2,3=3]{\ifx#1\empty{\mathrm{Hom}}\else{\mathrm{Hom}_{#1}}\fi(#2,#3)}
\makeatletter
\newenvironment{myproofof}[1]{\par
  \pushQED{\qed}%
  \normalfont \topsep6\p@\@plus6\p@\relax
  \trivlist
  \item[\hskip\labelsep
        \bfseries
    Proof of #1\@addpunct{.}]\ignorespaces
}{%
  \popQED\endtrivlist\@endpefalse
}
\makeatother



%%% environment test with enumerates

    

% Local variables:
% mode: tex
% End:



\title{A Simple Generalization of Perron-Frobenius Theorem}
\author{Jakob Schneider, Darth RAIDer (aka RAID - bring up the data)}

\begin{document}

\maketitle
%\tableofcontents

\begin{abstract}
The theorem of Perron-Frobenius in a finite dimensional space is usually proven using means like Brouwers fixed point theorem or contracting metrics in which the corresponding mapping is somehow contractive (Banachs fixed point theorem).
We will give an elementary proof on the subject to emphesize that it is not necessary to apply means which work for general compact (continuous) mappings. Moreover we will give a characterization of the existence of a unique fixed point.
\end{abstract}

\section{Set up}

At first we want to give a definition which reprensents a natural extension of the notion of an $n$-simplex to infinite dimensional spaces.

\begin{definition}[simplex]
Let $V$ be a linear space and $S\subset V$ such that for any two homotheties $\alpha, \beta:V\to V$ the intersection $\alpha(S)\intersect\beta(S)$ is either empty or the image $\gamma(S)$ of $S$ under another homothety $\gamma:V\to V$. Then we call $S$ a \emph{simplex}.
\end{definition}

Of course this definition might differ slightly from the readers intuition of an $n$-simplex in the euclidean space $\reels^n$, ignoring compactness. So we will need a further

\begin{definition}[line-compactness]
Let $V$ be a linear space and $C\subset V$. Then $V$ is called \emph{line-compact} if for some line $l\subset V$ (one-dimensional subspace) the set $C\intersect l$ is compact (in the topology of the line - i.e. the default topology of $\reels$).
\end{definition}

It should be clear that we may replace $l$ in this definition by a finite dimensional subspace (using the euclidean topology of this space). Now it is a routine matter to check that a line-compact simplex in the sense of these two definition coincides with the notion of an $n$-simplex (in finite-dimensional spaces $V$).

\section{Elementary proof of Perron-Frobenius theorem}

The main goal of this section is to establish Perron-Frobenius theorem as a special case of a more general theorem without the use of other fixed point theorems or contractivity arguments, giving a probably new impression to the reader.
The following lemma would directly follow form Brouwers theorem.

\begin{lemma}\label{lem1}
Let $V$ be an $n$-dimensional archimedian vector space ($n\in\nats$)\footnote{by $\nats$ we denote the set $\{1,2,3,\ldots\}$} with a polytopic base $B$ (thus $B$ is a (line-)compact polytope). Let $T$ be a strictly positive operator (i.e. $\forall v\in V^+\setminus\{0\}:T(v)>0$). Define the equivalence relation $\sim:=\{(u,v)\in V^+\setminus\{0\}:\exists \lambda>0:u=\lambda v\}$. Then the map $T\restrict_{V\setminus\{0\}}/\sim=:\chi$ has a fixed point in $V^+\setminus\{0\}/\sim$ (i.e. $T$ has an eigenvector in $V^+$).  
\end{lemma}

\begin{remark}
The space $V\setminus\{0\}/\sim$ is isomorphic to $S^{n-1}$. For $n$ even any continuous map from $S^{n-1}$ to $S^{n-1}$ admits a fixed point. 
\end{remark}

\begin{remark}
In our further considerations we set $B$ to be generating (otherwise consider the lowerdimensional problem in $\lin B$). In this case we already know that $B=V^+\intersect f^{-1}\{1\}$ for some strictly positive functional $f$.
\end{remark}

During the proof of this lemma we also want to establish some kind of characterization.
The (short) proof of the lemma will be a minor part of that and will be given later. At first we impose some more general considerations.

\subsection{The map $\rho$}

\begin{definition}
Introduce the map $\rho:B\to B$ by

\begin{equation}
\rho=\phi^{-1}\frac{T\restrict_{V^+\setminus\{0\}}}{\sim}\phi
\end{equation}

where $\phi$ is the \emph{natural projection map} $\phi:B\to\frac{V^+\setminus\{0\}}{\sim}$ with $\phi(v):=[v]_\sim$ for $v\in V$.
\end{definition}

\begin{remark}
By definition of $\rho$, $T$ admits a positive eigenvector if and only if $\rho$ has a fixed point.
\end{remark}

Now there are several important properties arising from this definition of $\rho$. But at first we need some definitions.

\begin{definition}[degree, boundary, face]
Let $P$ be a line-compact convex polytope in a vectorspace $V$ and let $p\in P$. We define the \emph{degree} of $p$ (with respect to $P$) as the supremum of the dimension of the faces in whose interior $p$ lies. Formally,

\begin{equation}
\deg_P p := \sup\{|Q|:Q\subset P, Q \text{ affinely independent } \wedge p\in\int \conv Q\}-1
\end{equation} 

Then we define the \emph{$\alpha$-dimensional boundary} $\partial_\alpha P$ by

\begin{equation}
\partial_{\alpha} P:= \deg_P^{-1}\{\beta\in\cardinals:\beta\leq\alpha\}\text{.}\footnote{Here $\cardinals$ denotes the cardinal numbers.}
\end{equation}

Furthermore, any maximal convex subset of $\partial_\alpha S$ is called a \emph{$\alpha$-dimensional face}.
\end{definition}  

\begin{remark}
The set $\partial_0 P$ will also be denoted as \emph{vertices}. It is an easy exercise to show that for any $n$-dimensional (line-)compact convex polytope $P$ ($n\in\nats$) the sets $\partial_0 P,\ldots, \partial_n P$ are not empty, but in general a convex polytope does not need to have finite-dimensional faces. The reader is invited to think of such polytopes (hint: consider the base of the archimedian vector lattice $C[0,1]$ with the pointwise order).
\end{remark}

\begin{remark}\label{rem1}
For the finite-dimensional one may substitude $Q$ by $\partial_0 P$ in the above definition.
\end{remark}

Now we are ready for the following important

\begin{lemma}\label{lem2}
The map $\rho$ has the following three properties
\begin{enumerate}
\item $\rho$ is continuous.
\item For any set $A\subset B$ we have $\conv\rho[A]=\rho\conv A$.
\item If $P\subset B$ is a compact convex polytope such that $\dim\aff P = \dim\aff \rho[P]$ then $\rho$ preserves $\deg$, i.e. $\deg_P p = \deg_{\rho[P]} \rho(p) = k\leq n-1$.
\end{enumerate}
\end{lemma}

\begin{proof}
\begin{enumerate}
\item Clearly, as $V$ is a finite-dimensional euclidean space, we have $T$ being continuous. Now taking the quotient topology and conjugating with the natural projection map $\phi$ does not destroy anything. 
\item Let us pick a set $A\subset B$. By definition any point $a\in\conv A$ admits a representation $a=\sum_{i=1}^l{\alpha_i a_i}$ with $a_i\in A, \alpha_i>0$ forall $i=1,\ldots,l$ and $\sum_{i=1}^l{\alpha_i}=1$. We then have by 
\begin{equation}
\rho(a)=\frac{\sum_{i=1}^l{\alpha_i T(a_i)}}{f\left(\sum_{i=1}^l{\alpha_i T(a_i)}\right)}=\frac{\sum_{i=1}^l{\alpha_i\lambda_i\rho(a_i)}}{f\left(\sum_{i=1}^l{\alpha_i T(a_i)}\right)}\label{eq1}
\end{equation}
that $\rho(a)\in\conv\rho[A]$ as we may obtain a convex combination of $\rho(a)$ in $\rho(a_i)$ ($i=1,\ldots,l$) by considering the fact that there exist (unique) $\lambda_i>0$ such that $\lambda_i\rho(a_i)=T(a_i)$ ($i=1,\ldots,l$). Thus $\conv\rho[A]\supset\rho\conv A$.

For the other inclusion consider $\tilde{a}=\sum_{i=1}^l{\alpha_i \rho(a_i)}\in\conv\rho[A]$ for $a_i\in A, \alpha_i>0$ forall $i=1,\ldots,l$ and $\sum_{i=1}^l{\alpha_i}=1$. By taking the $\lambda_i$ ($i=1,\ldots,n$) as in the previous case, we obtain that
\begin{equation}
\tilde{a}=\sum_{i=1}^l{\frac{\alpha_i}{\lambda_i} T(a_i)}=T\left(\sum_{i=1}^l{\frac{\alpha_i}{\lambda_i} a_i}\right)=\rho\left(\frac{\sum_{i=1}^l{\frac{\alpha_i}{\lambda_i} a_i}}{f\left(\sum_{i=1}^l{\frac{\alpha_i}{\lambda_i} a_i}\right)}\right)\text{.}\label{eq2}
\end{equation}
Thus we obtain the desired equality $\conv\rho[A]=\rho\conv A$.
\item We will show now: a subset $C\subset P$ is linearly (affinely) independent if and only if $\rho[C]$ is linearly independent (linearly and affinely independence are equivalent for such $C$ as $0\not\in P$).

\textbf{$\impliedby$:} So let us suppose $C$ is linearly dependent, i.e. for some numbering of the elements of $C$ by $c_1,\ldots,c_p,\overline{c}_1,\ldots,\overline{c}_q$ we have $\delta_1,\ldots,\delta_p,\gamma_1,\ldots,\gamma_q\geq 0$ with $\sum_{i=1}^p{\delta_i}=\sum_{i=1}^q{\gamma_i}=1$ (this condition comes from $f(0)=0$ for the corresponding strictly positive functional) such that 
\begin{equation}
\sum_{i=1}^p{\delta_i c_i}=\sum_{i=1}^q{\gamma_i \overline{c}_i}\text{.}\label{eq3}
\end{equation}

Plugging the $c_i$ and $\overline{c}_i$ in the role of the $a_i$ in equation (\ref{eq1}) one obtains that $\rho[C]$ is affinely dependent.

\textbf{$\implies$:} For the other direction we will need the dimension condition. 

Suppose there is a set $C$ of linearly independent points such that $\rho[C]$ is linearly dependent. Clearly, $|C|\leq k+1$ as $C\subset P$ and $\dim\aff P=k$, but we may extend $C$ to a $(k+1)$-element set $\tilde{C}$ such that $\rho[\tilde{C}]$ is stil linearly dependent. If we now pick some point $p\in P$ we get a unique representation of it as $p = \sum_{i=1}^{k+1}{\epsilon_i \tilde{c}_i}$ (with $\sum_{i=1}^{k+1}{\epsilon_i}=1$). Now, shifting the negative $\epsilon_i$ to the side of $p$ and applying equation (\ref{eq1}) we get that $\rho(p)\in\aff\rho[\tilde{C}]$ a contradiction to the condition $\dim\aff\rho[P]=k$.

As we now have shown that affinely independence and lying in the interior of a convex set are preserved (to see this consider equations (\ref{eq1}) and (\ref{eq2})), we may deduce the desired result.
\end{enumerate}
\end{proof}

\begin{remark}
In the case where $T$ does not act regular on the set $P$, $\deg$ may increase and decrease. This can be seen from a tetrahedran which degrades to a triangle (i.e. the degree of one vertex increases) and a tetrahedran which degrades to a line (where two antipodal edges degrade to point - here the degree of the internal points of theses edges decreases).
\end{remark}

\section{The $\rho$-invariant polytope $\tilde{B}$}

The first thing we do is to simplify the problem a bit by the following consideration. Define 

\begin{equation} 
\tilde{B}:=\intersection_{m\in\nats}{\rho^m[B]}\text{.}
\end{equation} 

Then $\tilde{B}$ as an intersection of a chain of compact sets cannot be empty by Dedekinds principle.
Note that if $\rho$ has a fixed point in $B$ then it has to be in $\tilde{B}$. Furthermore, for $n\in\nats$ the set $\rho^m[B]$ must also be a convex compact polytope. This can be seen by observing that $\rho[B]=\conv\rho[\partial_0 B]$ where $\partial_0 B$ are the vertices of $B$ (cf. lemma (\ref{lem2})). 
Using a similar argument we obtain that $\tilde{P}$ also has to be a compact convex polytope.

THIS LAST ONE NEEDS A BIT MORE PRECISION. 

\paragraph{The degraded case.} If $\tilde{B}$ has lower dimension than $B$ we may consider the 'projected' problem in the ordered subspace $\lin\tilde{B}\subset\img{T}$ equipped with the order induced by the cone $V^+\intersect\lin\tilde{B}$ on which $T$ acts as a bijection. 
Maybe, we shall mention at this point that the 'initial' case where $V$ is one-dimensional (if $\dim V=0$ we had no strictly positive operator and no base $B$) is trivial. 

\subsection{The case $T$ being regular.}
Thus let us assume that $\dim\lin\tilde{B}=n\geq 1$ (actually this implies $T$ being regular). 
Then it is immediately clear that $\rho[\partial_0{\tilde{B}}]=\partial_0{\tilde{B}}$ by considering the fact that $\tilde{B}$ is fixed under $\rho$ and $\rho$ preserves $\deg$ as $T$ is regular (cf. lemma (\ref{lem2}) part 3). 
Thus $\rho$ acts on the set of vertices of $\tilde{B}$ as a finite permutation which has some representation of the form $\rho\restrict_{\partial_0{\tilde{B}}}=c_1\compose\cdots\compose c_l$ (where $c_i$ for $i=1,\ldots,l$ are disjoint cycles). 
Let us denote the vertices of $\tilde{B}$ by $v_1,\ldots,v_m$ and define $\tau$ implicitly by $\rho\restrict_{\partial_0{\tilde{B}}}(v_i)=v_{\tau(i)}$ for $i=1,\ldots,m$. 

From the following we will see that the number of fixed points of $\rho$ is at least $l\geq 1$ and thus establish lemma (\ref{lem1}).

\paragraph{The case $\tau$ being itself a cycle.}

Let us now consider the case that $\rho\restrict_{\partial_0{\tilde{B}}}$ is itself a cycle (which is not trivial). 
We then have some positive $\lambda_1,\ldots,\lambda_n$ such that $T(v_i)=\lambda_iv_{\tau(i)}$ for ($i=1,\ldots,n$). 
It is now an easy matter to solve the arising equation (for an eigenvector of $T$ in $\tilde{B}$)

\begin{equation}
T\left(\sum_{i=1}^n\mu_iv_i\right)=\sum_{i=1}^m\mu_i\lambda_iv_{\tau(i)}=\lambda \sum_{i=1}^m\mu_iv_i
\end{equation}

from which one obtains $\frac{\mu_i}{\mu_{\tau(i)}}=\frac{\lambda}{\lambda_i}$ ($i=1,\ldots,m$), which obviously has the solution $\lambda = \sqrt[m]{\prod_{i=1}^m{\lambda_i}}$ 
and $\mu_{\tau^k(1)}=\frac{1}{\lambda^k}\prod_{i=0}^k{\lambda_{\tau^i(1)}}\mu_1$ (all solutions $(\mu_1,\ldots,\mu_m)$ are a multiple of each other). 
Thus $\rho$ in this special case has a unique fixed point in the interior of $\tilde{B}$ (and thus $B$).

\begin{remark}
In this case the operator $T^m$ coincides with the homothety $v\mapsto \lambda^m v$.
\end{remark}

\paragraph{Some power of $\tau$.}

Let us now observe that $\tau$ as a member of the finite group $S_m$ has finite order. Thus we see that there must be some power of $T$ (e.g. $T^{\ord\tau}$) which is diagonalisable such that all vertices in a cycle of $\rho\restrict_{\partial_0 \tilde{B}}$ are in the same eigenspace of $T^{\ord{\tau}}$. 

Moreover, from the previous section we may deduce that $T$ has a unique positive eigenvector if and only if $\rho\restrict_{\tilde{B}}$ is a single cycle.

\begin{lemma}\label{lem4} Let $v$ be a vertex of $\tilde{B}$ such that $v$ is an eigenvector of $T^{\ord\tau}$ of the smallest (positive) eigenvalue $\lambda_{\min}$. Then $v$ is a vertex of $B$. 

\end{lemma}

\begin{proof}
...
\end{proof}

\section{Conclusion}

Now we want to reassamble all facts and extract the detailed characterization.

//
But before a last lemma seems necessary.

\begin{lemma}\label{lem5}
Let $T$ have and eigenvector $s\in\int F$ where $F$ is a $k$-dimensional face of $B$. Then $\rho[F]\subset F$.
\end{lemma}

\begin{proof}
By definition of a $k$-dimensional face we have $b\in\int\conv\{b_0,\ldots,b_k\}$ for some $u_0\ldots,u_k\subset \partial_0 B$. Thus assuming that $\rho(u_i)\not\in F$ (for some $i$) we obtain that $\rho(b)$ does not lie in $\rho[F]$ which can for example be observed using the well known fact, that any point of $B$ can be described by a unique convex combination of its vertices.
\end{proof}
//

Now we are able to give a characterization for the unique existence of a positive eigenvector of a strictly positive operator $T$.

\begin{theorem}[Perron-Frobenius]
Let $V$ be an $n$-dimensional archimedian vector space with base $B$ and $T$ be some strictly positive operator ($n\geq 1$). 
Then $T$ admits a positive eigenvector. 
Moreover, this eigenvector is unique if and only if the action of the $\rho$ (i.e. the action of the group $\langle \rho\rangle$) on linear subspaces of $\lin B$ has is at most one nontrivial (i.e. with more than one element) finite orbit $L\subset B$ (that is the permutation $\tau$ is a cycle). 

If $\conv L$ contains an internal point of $V^+$ then the algebraic multiplicity of the eigenvector is one.
Especially, if $T^m[V^+]\subset\int V^+$ for some $m\in\nats$ then the above condition is fulfilled with $|L|=1$ and the algebraic multiplicity of the unique eigenvector is one. 
\end{theorem}

\begin{proof}
By the previous section we know that the eigenvector is unique if and only if $\rho$ acts cyclic on the vertices of $\tilde{B}$. If $|\tilde{B}|=1$ we are done with the first fact, because all elements with finite orbit under $\rho$ must lie in $\tilde{B}$. In the second case ..
\end{proof}

\end{document}
