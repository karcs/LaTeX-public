\message{ !name(On The Routh-Hurwitz Theorem.tex)}\documentclass[10pt,a4paper]{article}
\usepackage[utf8]{inputenc}
\usepackage{amsmath}
\usepackage{amsfonts}
\usepackage{amssymb}
\usepackage{enumerate}
%My personal maths package
%\bibliography{Bibliography.bib}

%%%%%%%%%%%%%%%%%%%%%%%%%%%%%%%%%%%%%%%%%%%%%%%%%%%%%%%%%%%%%%%%%%%%%%%%%%%%%%
%%%%% MATH PACKAGES %%%%%%%%%%%%%%%%%%%%%%%%%%%%%%%%%%%%%%%%%%%%%%%%%%%%%%%%%%
%%%%%%%%%%%%%%%%%%%%%%%%%%%%%%%%%%%%%%%%%%%%%%%%%%%%%%%%%%%%%%%%%%%%%%%%%%%%%%

% very good package
%\usepackage{mathtools}

%%% font stuff
\usepackage[T1]{fontenc}        % for capitals in section /paragraph etc.
\usepackage[utf8]{inputenc}     % use utf8 symbols in code

\usepackage{amssymb,amsmath,amsfonts} % amsthm not needed -- use my own envs
%\usepackage{mathtools}
%\mathtoolsset{showonlyrefs}
% further alternative math packages: unicode-math, abx-math
\usepackage{bm}
\usepackage{mathrsfs} % used for: fraktal math letters
\usepackage[bigsqcap]{stmaryrd} % used for: big square cap symbol
\usepackage{xargs}

% standard packages
%\usepackage{color} % for color

%%
%%index

\newcommand*{\keyword}[2][\empty]{\emph{#2}\ifx#1\empty\index{#2}\else\index{#1}\fi}
\newcommand*{\person}[1]{\textsc{#1}}

%%%%%%%%%%%%%%%%%%%%%%%%%%%%%%%%%%%%%%%%%%%%%%%%%%%%%%%%%%%%%%%%%%%%%%%%%%%%%%%%%%%%%%%%%%%%%%%%%%%%%%%%%%%%
%%%%% MATH ALPHABETS & SYMBOLS %%%%%%%%%%%%%%%%%%%%%%%%%%%%%%%%%%%%%%%%%%%%%%%%%%%%%%%%%%%%%%%%%%%%%%%%%%%%%
%%%%%%%%%%%%%%%%%%%%%%%%%%%%%%%%%%%%%%%%%%%%%%%%%%%%%%%%%%%%%%%%%%%%%%%%%%%%%%%%%%%%%%%%%%%%%%%%%%%%%%%%%%%%

% w: http://milde.users.sourceforge.net/LUCR/Math/math-font-selection.xhtml

% ===== Set quick commands for math letters ================================================================
% calagraphic letters (only upper case available; standard)
\newcommand{\cA}{\mathcal{A}}
\newcommand{\cB}{\mathcal{B}}
\newcommand{\cC}{\mathcal{C}}
\newcommand{\cD}{\mathcal{D}}
\newcommand{\cE}{\mathcal{E}}
\newcommand{\cF}{\mathcal{F}}
\newcommand{\cG}{\mathcal{G}}
\newcommand{\cH}{\mathcal{H}}
\newcommand{\cI}{\mathcal{I}}
\newcommand{\cJ}{\mathcal{J}}
\newcommand{\cK}{\mathcal{K}}
\newcommand{\cL}{\mathcal{L}}
\newcommand{\cM}{\mathcal{M}}
\newcommand{\cN}{\mathcal{N}}
\newcommand{\cO}{\mathcal{O}}
\newcommand{\cP}{\mathcal{P}}
\newcommand{\cQ}{\mathcal{Q}}
\newcommand{\cR}{\mathcal{R}}
\newcommand{\cS}{\mathcal{S}}
\newcommand{\cT}{\mathcal{T}}
\newcommand{\cU}{\mathcal{U}}
\newcommand{\cV}{\mathcal{V}}
\newcommand{\cW}{\mathcal{W}}
\newcommand{\cX}{\mathcal{X}}
\newcommand{\cY}{\mathcal{Y}}
\newcommand{\cZ}{\mathcal{Z}}

% bold math letters (standard)
\newcommand{\bfA}{\mathbf{A}}
\newcommand{\bfB}{\mathbf{B}}
\newcommand{\bfC}{\mathbf{C}}
\newcommand{\bfD}{\mathbf{D}}
\newcommand{\bfE}{\mathbf{E}}
\newcommand{\bfF}{\mathbf{F}}
\newcommand{\bfG}{\mathbf{G}}
\newcommand{\bfH}{\mathbf{H}}
\newcommand{\bfI}{\mathbf{I}}
\newcommand{\bfJ}{\mathbf{J}}
\newcommand{\bfK}{\mathbf{K}}
\newcommand{\bfL}{\mathbf{L}}
\newcommand{\bfM}{\mathbf{M}}
\newcommand{\bfN}{\mathbf{N}}
\newcommand{\bfO}{\mathbf{O}}
\newcommand{\bfP}{\mathbf{P}}
\newcommand{\bfQ}{\mathbf{Q}}
\newcommand{\bfR}{\mathbf{R}}
\newcommand{\bfS}{\mathbf{S}}
\newcommand{\bfT}{\mathbf{T}}
\newcommand{\bfU}{\mathbf{U}}
\newcommand{\bfV}{\mathbf{V}}
\newcommand{\bfW}{\mathbf{W}}
\newcommand{\bfX}{\mathbf{X}}
\newcommand{\bfY}{\mathbf{Y}}
\newcommand{\bfZ}{\mathbf{Z}}
\newcommand{\bfa}{\mathbf{a}}
\newcommand{\bfb}{\mathbf{b}}
\newcommand{\bfc}{\mathbf{c}}
\newcommand{\bfd}{\mathbf{d}}
\newcommand{\bfe}{\mathbf{e}}
\newcommand{\bff}{\mathbf{f}}
\newcommand{\bfg}{\mathbf{g}}
\newcommand{\bfh}{\mathbf{h}}
\newcommand{\bfi}{\mathbf{i}}
\newcommand{\bfj}{\mathbf{j}}
\newcommand{\bfk}{\mathbf{k}}
\newcommand{\bfl}{\mathbf{l}}
\newcommand{\bfm}{\mathbf{m}}
\newcommand{\bfn}{\mathbf{n}}
\newcommand{\bfo}{\mathbf{o}}
\newcommand{\bfp}{\mathbf{p}}
\newcommand{\bfq}{\mathbf{q}}
\newcommand{\bfr}{\mathbf{r}}
\newcommand{\bfs}{\mathbf{s}}
\newcommand{\bft}{\mathbf{t}}
\newcommand{\bfu}{\mathbf{u}}
\newcommand{\bfv}{\mathbf{v}}
\newcommand{\bfw}{\mathbf{w}}
\newcommand{\bfx}{\mathbf{x}}
\newcommand{\bfy}{\mathbf{y}}
\newcommand{\bfz}{\mathbf{z}}

% fractal math letters (standard)
\newcommand{\fkA}{\mathfrak{A}}
\newcommand{\fkB}{\mathfrak{B}}
\newcommand{\fkC}{\mathfrak{C}}
\newcommand{\fkD}{\mathfrak{D}}
\newcommand{\fkE}{\mathfrak{E}}
\newcommand{\fkF}{\mathfrak{F}}
\newcommand{\fkG}{\mathfrak{G}}
\newcommand{\fkH}{\mathfrak{H}}
\newcommand{\fkI}{\mathfrak{I}}
\newcommand{\fkJ}{\mathfrak{J}}
\newcommand{\fkK}{\mathfrak{K}}
\newcommand{\fkL}{\mathfrak{L}}
\newcommand{\fkM}{\mathfrak{M}}
\newcommand{\fkN}{\mathfrak{N}}
\newcommand{\fkO}{\mathfrak{O}}
\newcommand{\fkP}{\mathfrak{P}}
\newcommand{\fkQ}{\mathfrak{Q}}
\newcommand{\fkR}{\mathfrak{R}}
\newcommand{\fkS}{\mathfrak{S}}
\newcommand{\fkT}{\mathfrak{T}}
\newcommand{\fkU}{\mathfrak{U}}
\newcommand{\fkV}{\mathfrak{V}}
\newcommand{\fkW}{\mathfrak{W}}
\newcommand{\fkX}{\mathfrak{X}}
\newcommand{\fkY}{\mathfrak{Y}}
\newcommand{\fkZ}{\mathfrak{Z}}
\newcommand{\fka}{\mathfrak{a}}
\newcommand{\fkb}{\mathfrak{b}}
\newcommand{\fkc}{\mathfrak{c}}
\newcommand{\fkd}{\mathfrak{d}}
\newcommand{\fke}{\mathfrak{e}}
\newcommand{\fkf}{\mathfrak{f}}
\newcommand{\fkg}{\mathfrak{g}}
\newcommand{\fkh}{\mathfrak{h}}
\newcommand{\fki}{\mathfrak{i}}
\newcommand{\fkj}{\mathfrak{j}}
\newcommand{\fkk}{\mathfrak{k}}
\newcommand{\fkl}{\mathfrak{l}}
\newcommand{\fkm}{\mathfrak{m}}
\newcommand{\fkn}{\mathfrak{n}}
\newcommand{\fko}{\mathfrak{o}}
\newcommand{\fkp}{\mathfrak{p}}
\newcommand{\fkq}{\mathfrak{q}}
\newcommand{\fkr}{\mathfrak{r}}
\newcommand{\fks}{\mathfrak{s}}
\newcommand{\fkt}{\mathfrak{t}}
\newcommand{\fku}{\mathfrak{u}}
\newcommand{\fkv}{\mathfrak{v}}
\newcommand{\fkw}{\mathfrak{w}}
\newcommand{\fkx}{\mathfrak{x}}
\newcommand{\fky}{\mathfrak{y}}
\newcommand{\fkz}{\mathfrak{z}}

% script math symbols (only uppercase; package: mathrsfs)
\newcommand{\sA}{\mathscr{A}}
\newcommand{\sB}{\mathscr{B}}
\newcommand{\sC}{\mathscr{C}}
\newcommand{\sD}{\mathscr{D}}
\newcommand{\sE}{\mathscr{E}}
\newcommand{\sF}{\mathscr{F}}
\newcommand{\sG}{\mathscr{G}}
\newcommand{\sH}{\mathscr{H}}
\newcommand{\sI}{\mathscr{I}}
\newcommand{\sJ}{\mathscr{J}}
\newcommand{\sK}{\mathscr{K}}
\newcommand{\sL}{\mathscr{L}}
\newcommand{\sM}{\mathscr{M}}
\newcommand{\sN}{\mathscr{N}}
\newcommand{\sO}{\mathscr{O}}
\newcommand{\sP}{\mathscr{P}}
\newcommand{\sQ}{\mathscr{Q}}
\newcommand{\sR}{\mathscr{R}}
\newcommand{\sS}{\mathscr{S}}
\newcommand{\sT}{\mathscr{T}}
\newcommand{\sU}{\mathscr{U}}
\newcommand{\sV}{\mathscr{V}}
\newcommand{\sW}{\mathscr{W}}
\newcommand{\sX}{\mathscr{X}}
\newcommand{\sY}{\mathscr{Y}}
\newcommand{\sZ}{\mathscr{Z}}

%%%%%%%%%%%%%%%%%%%%%%%%%%%%%%%%%%%%%%%%%%%%%%%%%%%%%%%%%%%%%%%%%%%%%%%%%%%%%%%%%%%%%%
%%%% BOLD MATH IN BOLD TEXT ENVIRONMENT %%%%%%%%%%%%%%%%%%%%%%%%%%%%%%%%%%%%%%%%%%%%%%
%%%%%%%%%%%%%%%%%%%%%%%%%%%%%%%%%%%%%%%%%%%%%%%%%%%%%%%%%%%%%%%%%%%%%%%%%%%%%%%%%%%%%%

% for bold math in bold text (e.g. sections)
\makeatletter
\g@addto@macro\bfseries{\boldmath}
\makeatother

\def\brackets#1{\ifx#1\empty\else\left(#1\right)\fi}

%%%%%%%%%%%%%%%%%%%%%%%%%%%%%%%%%%%%%%%%%%%%%%%%%%%%%%%%%%%%%%%%%%%%%%%%%%%%%%%%%%%%%%
%%%%% CATEGORY THEORY %%%%%%%%%%%%%%%%%%%%%%%%%%%%%%%%%%%%%%%%%%%%%%%%%%%%%%%%%%%%%%%%
%%%%%%%%%%%%%%%%%%%%%%%%%%%%%%%%%%%%%%%%%%%%%%%%%%%%%%%%%%%%%%%%%%%%%%%%%%%%%%%%%%%%%%

% ===== Category theory concepts =====================================================

\newcommand{\Ob}{\mathop\mathrm{Ob}}
\newcommand{\Mor}{\mathop\mathrm{Mor}}

\newcommand{\ccoprod}{\bigsqcup}
\newcommand{\cprod}{\bigsqcap}
\newcommand{\cincl}{\mathop\mathrm{incl}}
\newcommand{\cproj}{\mathop\mathrm{pr}}


% ===== Define standard categories ===================================================

% Define sets
\newcommand{\Set}{\mathbf{Set}}
% Define set-builder operator (equivalent to gen for algebras)
\newcommand{\set}[1]{\left\{#1\right\}}
% define interval operator: o - open, c - closed
\newcommand{\intervalcc}[2]{\left[#1,#2\right]}
\newcommand{\intervalco}[2]{\left[#1,#2\right)}
\newcommand{\intervaloc}[2]{\left(#1,#2\right]}
\newcommand{\intervaloo}[2]{\left(#1,#2\right)}

\newcommand{\inter}{\mathop\mathrm{int}}
\newcommand{\face}{\mathop\mathrm{F}}
\newcommand{\Pol}{\mathop\mathrm{Pol}}
\newcommand{\Inv}{\mathop\mathrm{Inv}}
\def\struct#1{\gen{#1}}

\let\originaltimes\times%
\renewcommand{\times}{\mathbin{\sqcap}}
\newcommand\settimes{\originaltimes}
\newcommand{\setleq}{\subseteq}
\newcommand{\setgeq}{\supseteq}

\newcommand{\pderive}[2]{\frac{\partial{#1}}{\partial{#2}}}
\renewcommand{\div}{\mathop\mathrm{div}}

%% Diffgeo
\def\Ric{\mathop\mathrm{Ric}}
\def\ric{\mathop\mathrm{ric}}
\def\tr{\mathop\mathrm{tr}}

\def\cotimes{\mathbin{\sqcup}}

\def\@rightopen#1{\ifx#1]{\right]}\else{\interval@errmessage}\fi}
\def\@leftclosed[#1){\left[#1\right)}
\makeatother

% finite
\newcommand{\fin}{\mathrm{fin}}

% Define groups (optarg: properties such as -> abelian, noetherian (acc), artinian (dcc) etc.)
\newcommand{\Grp}[1][\empty]{\if\empty{#1}{\mathbf{Grp}}\else{\mathbf{Grp}_{#1}}}
\def\PGL{\mathrm{PGL}}
\def\PGammaL{\mathrm{P\Gamma L}}
\def\GL{\mathrm{GL}}
% Define rings
\newcommand{\rg}{\mathrm{rg}} %rank of a matrix
\newcommand{\Rg}[1][\empty]{\if\empty{#1}{\mathbf{Rg}}\else{\mathbf{Rg}_{#1}}}
\edef\units#1{#1^{\settimes}}
\def\dual#1{#1^{\ast}}

%% redefine the command \P to produce the projective functor in math mode
\let\parsymb\P%
\def\P{\ifmmode\mathrm{P}\else\parsymb\fi}
\renewcommand{\iff}{\ifmmode\equival\else{if and only if}\fi}
\newcommand{\quotring}{\mathop{\mathrm{Q}}}
\newcommand{\rad}{\mathrm{rad}}
% Standard rings
% integral domains
\newcommand{\ID}{\mathbf{ID}}
% unique factorization domains
\newcommand{\UFD}{\mathbf{UFD}}
% principal ideal domains
\newcommand{\PID}{\mathbf{PID}}

% Define modules over a group or ring
\newcommand{\Mod}[1]{\mathbf{Mod}_{#1}}
% Define vector space over a field
\renewcommand{\Vec}[1]{\mathbf{Vec}_{#1}}%

% when cases
\def\otherwise{\textrm{otherwise}}

% new concepts
\newcommand{\new}[1]{\emph{#1}}

\usepackage{xifthen,xstring}

% replace the bar command by overline when argument just one character (shorter and better)
%$\let\oldbar\bar
%\renewcommand{\bar}[1]{\StrLen{#1}[\length]\ifthenelse{\length > 1}{\overline{#1}}{\oldbar{#1}}}
\def\bar{\overline}

%% argument in equatoin
\def\arg{\bullet}

% groups and algebras
\newcommand{\Con}{\mathop\mathrm{Con}}
\newcommand{\Sub}{\mathop\mathrm{Sub}}
\newcommand{\Hom}{\mathop\mathrm{Hom}}
\newcommand{\Aut}{\mathop\mathrm{Aut}}
\newcommand{\Out}{\mathop\mathrm{Out}}
\newcommand{\End}{\mathop\mathrm{End}}
\newcommand{\id}{\mathop\mathrm{id}}
\newcommand{\rk}{\mathop\mathrm{rk}} % rank of a group module/ lattice
\newcommandx{\con}[1][1=\empty]{\ifx#1\empty{\mathop{\mathrm{con}}}\else{\mathop{\mathrm{con}}\left(#1\right)}\fi}
\newcommand{\leftsemidirprod}[1][]{\mathbin{\ifx&#1&\ltimes\else{\ltimes_{#1}}\fi}}
\newcommand{\rightsemidirprod}[1][]{\ifx#1\empty\rtimes\else{\rtimes_{#1}}\fi}
\newcommand{\normalisor}[2][]{\ifx#1\empty{\mathrm{N}\left(#2\right)}\else{\mathrm{N}_{#1} \left(#2\right)}\fi}
% support
\newcommand{\spt}{\mathop{\mathrm{spt}}}
% commutator
\newcommand{\gcom}[2]{\left[#1,#2\right]}

% physics stuff
\newcommand{\float}[3][\empty]{\ifx#1\empty{{#2}\cdot{10^{#3}}}\else{{#2}\cdot{{#1}^{#3}}}\fi}
\makeatletter
\def\newunit#1{\@namedef{#1}{\mathrm{#1}}}
\def\mum{\mathrm{\mu m}}
\def\ohm{\Omega}
\newunit{V}
\newunit{mV}
\newunit{kV}
\newunit{s}
\newunit{ms}
\def\mus{\mathrm{\mu s}}
\newunit{m}
\newunit{nm}
\newunit{cm}
\newunit{mm}
\newunit{fF}
\newunit{A}

\newunit{fA}
\newunit{C}

% elements
\def\newelement#1{\@namedef{#1}{\mathrm{#1}}}
\newelement{Si}
\makeatother


% groups
\newcommand{\ord}{\mathop\mathrm{ord}}
\newcommand{\divides}{|}

\newcommand{\conleq}{\trianglelefteq}
\newcommand{\congeq}{\trianglerighteq}

% common algebraic objects
\newcommand{\reals}{\mathbb{R}} 			% real numbers
\newcommand{\nats}{\mathbb{N}} 				% natural numbers
\newcommand{\ints}{\mathbb{Z}} 				% integers
\newcommand{\rats}{\mathbb{Q}}				% rationals
\newcommand{\complex}{\mathbb{C}}			% complex numbers
\newcommand{\field}[1]{\mathbb{F}_{#1}}  		% finite field
\newcommand{\cards}{\boldsymbol{Cn}}                     % The cardinal numbers
\newcommand{\ords}{\boldsymbol{On}}                      % The ordinal numbers

% graphs
\def\KG{\mathop\mathrm{KG}}                     % Knesergraph

\newcommand{\uvect}{\boldsymbol{e}}

% new operators and relations

%%%%%%%%%%%%%%%%%%%
% complex numbers %
%%%%%%%%%%%%%%%%%%%

\renewcommand{\Re}{\mathop\mathrm{Re}}		% real part
\renewcommand{\Im}{\mathop\mathrm{Im}}		% imaginary part
\newcommand{\sgn}{\mathop\mathrm{sgn}}				% the sign operator (0 for 0)

%%%%%%%%%%%%%%%%
% reel numbers %
%%%%%%%%%%%%%%%%

\newcommand{\floor}[1]{\left\lfloor#1\right\rfloor}
\newcommand{\ceil}[1]{\left\lceil#1\right\rceil}

%%%%%%%%%%%%%%%%%%
% set operations %
%%%%%%%%%%%%%%%%%%

\newcommand{\intersect}{\cap}			% intersect to sets
\newcommand{\setjoin}{\cup}				% join two sets
\newcommand{\setmeet}{\cap}                     % intersect to sets
\newcommand{\bigsetjoin}{\bigcup}			% the union of sets ... subscripts to be added
\newcommand{\distunion}{\dot{\bigcup}}	% disjoint union of sets ... subscripts to be added
\newcommand{\bigsetmeet}{\bigcap}		% intersection of sets
\newcommand{\powerset}[1][]{\ifx&#1&\mathcal{P}\else\mathcal{P}_{#1}\fi}		% powerset ... to be customized
\newcommand{\card}[1]{\left|#1\right|}

%%%%%%%%%%%%%%%%%%%%%%%%%%%%%%%%%%%%%%%%
% composition operations of structures %
%%%%%%%%%%%%%%%%%%%%%%%%%%%%%%%%%%%%%%%%

%\newcommand{\setprod}{\bigtimes}			% setproduct - needed
\newcommand{\dirprod}{\bigotimes} 			% direct product for groups and spaces
\newcommand{\dirtimes}{\otimes}				% direct multiply for groups and spaces
\newcommand{\dirsum}{\bigoplus} 			% direct sum for groups and spaces
\newcommand{\dirplus}{\oplus}				% direct add for groups and spaces
\newcommand{\inprod}[2]{\left\langle #1,#2 \right\rangle}

\newcommand{\tuple}{\meet}
\newcommand{\cotuple}{\join}

%%%%%%%%%%%%%%
% categories %
%%%%%%%%%%%%%%

% combinatorics
%%

\renewcommand{\binom}[3][\empty]{\if\empty{#1}{{#2 \choose #3}}\else{{#2 \choose #3}_{#1}}}

%%%%%%%%%%%%%%%%%%%%%%%%%%%%%%%%%%%%%%%%%%%%%%%%%%%%%
% metric spaces and normed spaces and vector spaces %
%%%%%%%%%%%%%%%%%%%%%%%%%%%%%%%%%%%%%%%%%%%%%%%%%%%%%

\newcommand{\dist}{\mathop\mathrm{dist}}				% distance operator ... dist(A,b), where A is a set and b a point
\newcommand{\diam}{\mathop\mathrm{diam}}	% diameter operator for sets
\newcommand{\norm}[1]{\left\Vert #1 \right\Vert}	% norm in a normed space ... subscript to be added
\newcommand{\conv}{\mathop\mathrm{conv}} 			% convex hull - vectorspaces
\newcommand{\lin}{\mathop\mathrm{lin}} 				% linear hull - vectorspaces
\newcommand{\aff}{\mathop\mathrm{aff}}				% affine hull - vectorspaces

%%%%%%%%%%%%%%%%%%%%%%
% operators in rings %
%%%%%%%%%%%%%%%%%%%%%%

\newcommand{\lcm}{\mathop\mathrm{lcm}}				% least common multiple - in euclidean rings
\renewcommand{\gcd}{\mathop\mathrm{gcd}}				% greatest command devisor - in euclidean rings
\newcommand{\res}{\mathop\mathrm{res}}				% residue of p mod q is res(p,q)
\renewcommand{\mod}{\textrm{ mod }}

%%%%%%%%%%%%%%%%%%%
% logical symbols %
%%%%%%%%%%%%%%%%%%%

%\newcommand{\impliedby}{\Leftarrow}				% reverse implicatoin arrow
%\newcommand{\implies}{\Rightarrow}				% implication
\newcommand{\equival}{\Leftrightarrow}				% equivalence

%%%%%%%%%%%%%%%%%%%%%%%%%%%%%%%%%%
% functions - elementary symbols %
%%%%%%%%%%%%%%%%%%%%%%%%%%%%%%%%%%

\newcommand{\rest}[1]{\left. #1\right\vert}		% restriction of a function to a set / also used as restriction in other terms like differential expressions / evaluation of a function
\newcommand{\rto}[3][]{#2\ifx&#1&\rightarrow\else\stackrel{#1}{\rightarrow}\fi#3}
\renewcommand{\to}{\rightarrow}							% arrow between domain and image
\newcommand{\dom}{\mathop\mathrm{dom}}					% domain of a function
\newcommand{\im}{\mathop\mathrm{im}}						% image of a function
\newcommand{\compose}{\circ}							% compose two functions
\newcommand{\cont}{\mathop\mathrm{C}}					% continuous functions from a domain into the reels or complex numbers 

%%%%%%%%%%%%%%%%%%%%
% groups - symbols %
%%%%%%%%%%%%%%%%%%%%

\newcommand{\stab}[1][]{\if&#1{\mathop\mathrm{stab}}&\else{\mathop\mathrm{stab}_{#1}}\fi}					% the stabilizer ... subscripts to be added
\newcommand{\orb}[1][]{\ifx&#1&\mathrm{orb}\else\mathrm{orb}_{#1}\fi} 					% orbit ... subscrit to be added (group)
\newcommand{\gen}[2][\empty]{\ifx#1\empty{\left\langle#2\right\rangle}\else{\left\langle#2\right\rangle_{#1}}\fi}					% generate ... kind of hull operator ---- to be thought of !!!!!!!

\def\Clo{\mathrm{Clo}}
\def\Loc{\mathrm{Loc}}
%% nets
\newcommand{\net}[2][\empty]{\ifx#1\empty{\left(#2\right)}\else{{\left(#2\right)}_{#1}}\fi}

%% open half ray
\newcommand{\ray}[2]{R_{#1}(#2)}

%% new
\let\oldcong\cong%
\newcommand{\iso}{\oldcong}

\def\cong{\equiv}
\newcommand{\base}[2]{\left[#2\right]_{#1}}                                   % base n expansion of some number
%%%%%%%%%%%%%%%%%%%%%%%%%%%%%%%%%
% matrices and linear operators %
%%%%%%%%%%%%%%%%%%%%%%%%%%%%%%%%%

\newcommand{\diag}{\mathop\mathrm{diag}}				% diagonal matrix or operator
\newcommand{\Eig}[1]{\mathop\mathrm{Eig}_{#1}}		% eigenspace for a certain eigenvalie		
\newcommand{\trace}{\mathop\mathrm{tr}}				% trace of a matrix
\newcommand{\trans}{\top} 							% transponse matrix

%%%%%%%%%%%%%
% constants %
%%%%%%%%%%%%%

\renewcommand{\i}{\boldsymbol{i}}			% imaginary unit
\newcommand{\e}{\boldsymbol{e}}				% the Eulerian constant

%%%%%%%%%%%%%%%%%%%%%%%%%%%%%%
% limit operators and arrows %
%%%%%%%%%%%%%%%%%%%%%%%%%%%%%%

\newcommand{\upto}{\uparrow}				% convergence from above
\newcommand{\downto}{\downarrow}			% convergence from below

%%
% other
%%
\newcommand{\cind}{\mathop\mathrm{Ind}}		% Cauchy index
\newcommand{\sgnc}{\sigma}					% sign changes
\newcommand{\wnumb}{\omega}					% winding number
\newcommand{\cfunc}{\mathop\mathrm{Cf}}		% Cauchy function of a compact curve in complex\setminus\{0\}


% evaluation of a function as a difference or single value

\newcommand{\abs}[1]{\left|#1\right|}
\newcommand{\conj}[1]{\overline{#1}}
\newcommand{\diff}{\mathop\mathrm{d}}

%%%%% test
\newcommand{\distjoin}{\mathaccent\cdot\cup}	% to be modified (name)
\newcommand{\cl}{\mathop\mathrm{cl}}				% topological closure
\newcommand{\sphere}{\mathbb{S}} % n-sphere
\newcommand{\ball}{\mathbb{B}} % n-ball
\newcommand{\bound}{\partial}
\newcommand{\bigmeet}{\mathop\mathrm{\bigwedge}}
\newcommand{\bigjoin}{\mathop\mathrm{\bigvee}}
\newcommandx{\rchar}[1][1=\empty]{\mathop\mathrm{char}\brackets{#1}}              % characteristic of a ring
\newcommand{\lgor}{\vee}                               % logical
\newcommand{\lgand}{\wedge}
\newcommand{\codim}{\mathop\mathrm{codim}}
\newcommand{\row}{\mathop\mathrm{row}}
\newcommand{\cone}{\mathop\mathrm{cone}}
\newcommand{\comp}{\mathop\mathrm{comp}}
\newcommand{\proj}{\mathrm{P}}
\def\PG{\mathrm{PG}}           % projective space
\newcommand{\meet}{\wedge}
\newcommand{\join}{\vee}
\newcommand{\col}{\mathop\mathrm{col}}
\newcommand{\vol}{\mathop\mathrm{vol}\nolimits}
%arrangements
\newcommand{\tpert}{\mathop\mathrm{tpert}}
\renewcommand{\epsilon}{\varepsilon}
% groups
\newcommand{\symgr}{\mathop\mathrm{Sym}}
\newcommand{\symalg}{\mathop\mathrm{S}}
\newcommand{\extalg}{\mathop\mathrm{\Lambda}}
\newcommand{\extpow}[1]{\mathop\mathrm{\Lambda}^{#1}}

%% set hulloperators -> define



\newcommandx{\homl}[3][1=1,2=2,3=3]{\ifx#1\empty{\mathrm{Hom}}\else{\mathrm{Hom}_{#1}}\fi(#2,#3)}
\makeatletter
\newenvironment{myproofof}[1]{\par
  \pushQED{\qed}%
  \normalfont \topsep6\p@\@plus6\p@\relax
  \trivlist
  \item[\hskip\labelsep
        \bfseries
    Proof of #1\@addpunct{.}]\ignorespaces
}{%
  \popQED\endtrivlist\@endpefalse
}
\makeatother



%%% environment test with enumerates

    

% Local variables:
% mode: tex
% End:



%\usepackage{xparse}
%\let\rsum\sum
%\RenewDocumentCommand\sum{mo}{%sum command with optional upper index
%  \IfNoValueTF{#2}
%    {\rsum_{#1}}
%    {\rsum_{#1}^{#2}}%
%}

\begin{document}

\message{ !name(On The Routh-Hurwitz Theorem.tex) !offset(-3) }


\begin{definition}[Algebraic winding number]
Let $\gamma:[a,b]\to\complex\setminus\{0\}$ be a continuous curve ($a,b\in\reels,a\leq b$). Then there exists a unique (modulo $2\pi$) continuous function $\arg{\gamma}$ such that 
\begin{equation}
\gamma = \abs{\gamma}\exp(\i\arg{\gamma}) = \abs{\gamma}\sgn{\gamma}
\end{equation}
The \emph{algebraic winding number} of $\gamma$ with respect to the origin 0 is then defined as
\begin{equation}
\wnumb_0(\gamma) := \rest{\arg{\gamma}}^b_a = \arg{\gamma(b)}-\arg{\gamma(a)}\text{.}
\end{equation}

\end{definition}

%% for rectifiable $\sgn(\gamma)$
\begin{remark}
If $\sgn{\gamma}:I\to\complex$ is a rectifiable function, one has the identity
\begin{equation}
\wnumb_0(\gamma)=\frac{1}{2\pi\i}\int_{\sgn{\gamma}}{\frac{1}{z}}\d{z}\text{.}
\end{equation}
\end{remark}

A natural idea is now to store the rotational information (i.e. the information of the function $\arg{\gamma}$) in another function which then can be analyzed using generalized \textsc{Sturm} chains and \textsc{Cauchy} indices. However, the 'function' which is introduced may have some gaps in its domain.

\begin{definition}[\textsc{Cauchy} function] Let $\gamma:I\to\complex\setminus\{0\}$ be a continuous curve for a compact interval $I$. The \textsc{Cauchy} \emph{function} $\cfunc(\gamma)$ of $\gamma$ is then defined as
\begin{equation}
\cfunc(\gamma):I\to\reels\setminus\{x\in I:\Re{\gamma(x)}=0\},x\mapsto\frac{\Im{\gamma(x)}}{\Re{\gamma(x)}}\text{.}
\end{equation}
Any partially defined function which can be obtained in such manner is called a \textsc{Cauchy} \emph{function} over $I$.
\end{definition}


%%% unstable
Now, one needs to redefine the operations standard arithmetic operations for such partial functions.

\begin{definition}[\textsc{Cauchy} function inverse]
Let $f$ be a \textsc{Cauchy} be a function on the compact interval $I$ corresponding to the path $\gamma:I\to\complex\setminus\{0\}$. Then the \textsc{Cauchy} function $1/f$ is defined as
\begin{equation}
1/f:\dom(f)\setminus f^{-1}\{0\}\unify \intersection_{n\in\nats}f^{-1}[\reels\setminus (-n,n)]\to\reels,x\mapsto\lim_{\xi\to x}{\frac{1}{f(x)}}
\end{equation}
which gives the \textsc{Cauchy}-function corresponding to the path $\i\conj{\gamma}$.
\end{definition}

\begin{definition}[\textsc{Cauchy} function of the product path]
Let $f,g$ be \textsc{Cauchy} be a function on the compact interval $I$ corresponding to paths $\gamma,\eta:I\to\complex\setminus\{0\}$. Then the \textsc{Cauchy} function $f\star g=\frac{f+g}{1-fg}$ is defined as
...
which gives the \textsc{Cauchy}-function corresponding to the path product $\gamma\eta$.
\end{definition}

%% end unstable

%%% give a characterization of Cauchy functions

%% first definition needed

\begin{definition}[maximal rotational interval, rotational domain of a partial function]
Let $f:J\to\reels$ be continuous where $J\subset I$ is open in the subspace $I$ for a compact interval $I\subset\reels$. Let $\{I_n\}_n$ be the partition of $J$ into maximal (with respect to inclusion) intervals (which exists as $J$ lies relatively open in $I$). An interval $I_i\in\{I_n\}_n$ is called \emph{rotational} if it is not an interval of the form $(a,b)$ ($a,b\in I, a<b$) which satisfies
\begin{equation}
\lim_{\xi\downto a}{f(\xi)}=\lim_{\xi\upto b}{f(\xi)}=\pm\infty\text{.}
\end{equation}
The \emph{rotational domain} of $f$ is defined as $\conv\left(\union\{\bar{I}_i\}_i\right)$ where $\{\bar{I}_i\}_i$ are all rotational intervals among $\{I_n\}_n$.
\end{definition}

%% the characterization of Cauchy functions

\begin{lemma}[Characterization of \textsc{Cauchy} functions]
Let $I\subset\reels$ be a compact interval and $J\subset I$ be a an open set in the subspace topology $I$. Then for a continuous function $f:J\to\reels$ the following two are equivalent
\begin{enumerate}
\item The function $f$ is a \textsc{Cauchy} function over $I$.
\item Let $\{I_n\}_{n\in\nats}$ be the partition of $J$ into maximal (with respect to inclusion) intervals (which exists as $J$ lies relatively open in $I$). Then there are only finitely many rotational intervals $\{\bar{I}_i\}_{i=0}^l$ among them.
\item There exist continuous functions $g,h:I\to\reels$ with no common zero $\xi\in I$ with $g(\xi)=h(\xi)$ such that $\rest{\frac{g}{h}}_{\dom(f)}=f$ (especially the expression on the left side is well-defined on $\dom(f)$).
\end{enumerate}
\end{lemma}

\begin{proof}
\textbf{$1\implies 2$.}
Let $f$ be a \textsc{Cauchy} function corresponding to the path $\gamma:I\to\complex\setminus\{0\}$.
Now assume there would be infinitely many maximal rotational intervals. Then at most two of them are half-open intervals (at most one at each bound $a$ and $b$). Thus let $\{\bar{I}_i\}_{i\in\nats}\subset\powerset(\dom(f))=I\setminus\{x\in I:\Re{\gamma(x)}=0\}$ be all maximal open rotational intervals.
Let $\{L_i\}=\{(\hat{a}_i,\hat{b}_i)\}\subset\{\bar{I}_i\}_{i\in\nats}$ be the set of maximal open rotational intervals for which
\begin{equation}
\lim_{\xi\downto \hat{a}_i}{f(\xi)}=-\infty \vee \lim_{\xi\upto \hat{b}_i}{f(\xi)}=\infty
\end{equation}
and $\{R_i\}=\{(\check{a}_i,\check{b}_i)\}\subset\{\tilde{I}_i\}_{i\in\nats}$ such that
\begin{equation}
\lim_{\xi\downto \hat{a}_i}{f(\xi)}=\infty \vee \lim_{\xi\upto \hat{b}_i}{f(\xi)}=-\infty\text{.}
\end{equation}
(The '$\vee$' is just needed to include the half-open intervals at the boundary of $I$ which might exist.)
It is then clear that in each interval $L_i$ which is not half-open the curves $\gamma$ and $\sgn{\gamma}$ rotates the angle $\pi$ around $0$ staying in one of the half planes $\{z\in\complex:\Re(z)\geq 0\}$ or $\{z\in\complex:\Re(z)\leq 0\}$, counterclockwise. In the intervals $R_i$ which is not half-open happens the same but the angle of rotation is $-\pi$ (clockwise rotation). Thus we obtain for the unique (modulo $2\pi$) continuous function $\arg{\gamma}$ (defined by $\exp(\i\arg{\gamma})=\sgn{\gamma}$) an infinite monotone sequence $(\xi_n)_{n\in\nats}\subset I$ of distinct boundary points of these intervals such that 
\begin{equation}
|\arg{\gamma(\xi_i)}-\arg{\gamma(\xi_{i+1})}|=\pi\text{.}
\end{equation}
But this cannot happen as such sequence would converge and thus as $\arg{\gamma}$ is continuous $\lim_{i\to\infty}{|\arg{\gamma(\xi_i)}-\arg{\gamma(\xi_{i+1})}|}$.

\textbf{$2\implies1$.} Let $f$ have the desired property. Then one directly constructs a path $\gamma:I=[a,b]\to\complex\setminus\{0\}$ such that $\cfunc(\gamma)=f$. Let $\{L_i\}$ and $\{R_i\}$ be defined as in the previous part.
We then define $\gamma$ as
\begin{equation}
\gamma(x):=\exp\left(\i(\arctan{f(x)}+\pi|\{L_i\}\intersect\powerset{[a,x]}|-|\{R_i\}\intersect\powerset{[a,x]}|)\right)
\end{equation}
for $x\in I\intersect\dom(f)$. For the other points $x\in I\setminus\dom(f)$ set $\rest{\gamma}_{I\setminus\dom(f)}$ constant on path components (values $\pm\i$) such that $\gamma$ is continuous.

It is routine to check that $\cfunc(\gamma)=f$ is fulfilled. 
\end{proof}

%% remark on the connection between the pathes of a Cauchy function
\begin{remark}
Any function $\tilde{\gamma}$ with $\cfunc(\tilde{\gamma})=f$ satisfies $\tilde{\gamma}=\alpha\gamma$ where $\alpha:I\to\reels$ is continuous and of constant sign as one easily verifies. 
\end{remark}

\begin{definition}[\textsc{Cauchy} index via paths] Let $f$ be a \textsc{Cauchy} be a function on the compact interval $I=[a,b]$ corresponding to the path $\gamma:I\to\complex\setminus\{0\}$ and let $\arg{\gamma}$ be the unique (modulo $2\pi$) continuous function such that 
\begin{equation}
\gamma = \abs{\gamma}\exp(\i\arg{\gamma})\text{.}
\end{equation}
Then its \textsc{Cauchy} index is defined as
\begin{equation}
\cind(f):=\sgn(\arg{\gamma(a)}-\arg{\gamma(b)})\abs{\conv\{\arg{\gamma(a)},\arg{\gamma(b)}\}\intersect\left(\pi\ints+\frac{\pi}{2}\right)}\text{.}
\end{equation}
\end{definition}

% explanation on the definition -> rotations clockwise (traverses of imag. axis)
\begin{remark} From this we see that the Cauchy index basically counts how many times the curve $\gamma$ traverses the imaginary line clockwise, where counterclockwise traverses are counted negative (roughly speaking). 
\end{remark} % welldefined !!!

\begin{lemma}[\textsc{Cauchy} index inversion formula, most general] Let $f$ be a Cauchy function over $I=[a,b]$ and let $\{\bar{I}_i\}_{i=0}^l$ be the maximal rotational intervals of $f$. Moreover, let $\bar{I}=\conv\left(\union\{\bar{I}_i\}_{i=0}^l\right)$ be the \emph{rotational domain} of $f$ and $\bar{a}<\bar{b}$ the bounds of the interval $\bar{I}$ (which can be open or closed at each side).  
Then it holds
\begin{equation}
\cind\left(f\right)+\cind\left(\frac{1}{f}\right)=\frac{1}{2}\left(\lim_{\xi\upto\bar{b},\xi\in\bar{I}}{\sgn{f(\xi)}}-\lim_{\xi\downto\bar{b},\xi\in\bar{I}}{\sgn{f(\xi)}}\right)\text{.}
\end{equation}
\end{lemma}

\begin{proof}
Let $\tilde{\gamma}:\bar{I}\to\complex\setminus\{0\}$ be a corresponding continuous curve to $\rest{f}_{\bar{I}}$ (i.e. $\cfunc(\tilde{\gamma})=\rest{f}_{\bar{I}}$). Then $\cfunc(\i\conj{\tilde{\gamma}})=\rest{(1/f)}_{\bar{I}}$. It is clear that $\cind(\rest{f}_{\bar{I}})=\cind(f)$ from the definition of the \textsc{Cauchy} index since if $\gamma:I\to\complex\setminus\{0\}$ such that $\cfunc(\gamma)=f$ then $\gamma(a)=\tilde{\gamma}(\bar{a})$ and $\gamma(b)=\tilde{\gamma}(\bar{b})$. 
Thus in the definition of the \textsc{Cauchy} indices of $f$ and $1/f$ we get
\begin{align}
\cind(f)+\cind\left(\frac{1}{f}\right) & =\sgn(\arg{\tilde{\gamma}(\bar{a})}-\arg{\tilde{\gamma}(\bar{b})})\abs{\conv\{\arg{\tilde{\gamma}(\bar{a})},\arg{\tilde{\gamma}(\bar{b})}\}\intersect\left(\pi\ints+\frac{\pi}{2}\right)}\\
& \quad+\sgn(\arg{\i\conj{\tilde{\gamma}}(\bar{a})}-\arg{\i\conj{\tilde{\gamma}}(\bar{b})})\abs{\conv\{\arg{\delta(\bar{a})},\arg{\delta(\bar{b})}\}\intersect\left(\pi\ints+\frac{\pi}{2}\right)}\\
& = \sgn(\arg{\tilde{\gamma}(\bar{a})}-\arg{\tilde{\gamma}(\bar{b})})\abs{\conv\{\arg{\tilde{\gamma}(\bar{a})},\arg{\tilde{\gamma}(\bar{b})}\}\intersect\left(\pi\ints+\frac{\pi}{2}\right)}\\
& \quad -\sgn(\arg{\tilde{\gamma}(\bar{a})}-\arg{\tilde{\gamma}(\bar{b})})\abs{\conv\{\arg{\tilde{\gamma}(\bar{a})},\arg{\tilde{\gamma}(\bar{b})}\}\intersect\pi\ints}
\end{align}
Now set $\eta:=\max\{\arg{\tilde{\gamma}(\bar{a})},\arg{\tilde{\gamma}(\bar{b})}\}$ and $\nu:=\min\{\arg{\tilde{\gamma}(\bar{a})},\arg{\tilde{\gamma}(\bar{b})}\}$. Then we obtain from our previous calculation
\begin{align}
\cind(f)+\cind\left(\frac{1}{f}\right) & = \sgn(\arg{\tilde{\gamma}(\bar{a})}-\arg{\tilde{\gamma}(\bar{b})})\begin{cases}
1 	&: \eta\in\pi\ints+\left[-\frac{\pi}{2},0\right)\wedge\nu\in\pi\ints+\left(0,\frac{\pi}{2}\right]\\
-1 	&: \eta\in\pi\ints+\left[0,\frac{\pi}{2}\right)\wedge\nu\in\pi\ints+\left(-\frac{\pi}{2},0\right]\\
0	&: \text{ otherwise }
\end{cases}
\end{align}
Now let $L$ be the rotational interval whose lower bound is $\bar{a}$ and $U$ be the rotational interval whose upper bound is $\bar{b}$ (these exist as there are only finitely many rotational intervals as $f$ is \textsc{Cauchy} function). Then there are two cases to be distinguished at each bound namely whether $L$ or $U$ are boundary intervals (and thus half-open) or if they are open intervals. In the second case it is clear that $\arg{\tilde{\gamma}(a)}\in\pi\ints+\pi/2$ or respectively $\arg{\tilde{\gamma}(b)}\in\pi\ints+\pi/2$.

\end{proof}

\section{Some remarks on Routh-Hurwitz-Theorem}

\begin{definition}[\textsc{Haar} space]
Let $K\subset \reels$ be a non-empty set. Then a $n$-dimensional linear subspace $H$ of $\cont(K)$ is called \textsc{Haar} \emph{space} if any function $g\in H\setminus\{0\}$ has at most $n-1$ zeros.
\end{definition}
\begin{definition}[\textsc{Chebyshev} system]
A system $\{\varphi_i\}^n_{i=1}$ is called \textsc{Chebyshev} \emph{system} or \textsc{Haar} \emph{system} if its spanned vector space is a \textsc{Haar} space.
\end{definition}
\begin{definition}[FM space]
Let $K$ be a compact space. Then any linear subspace $V$ of $\cont(K)$ is called an \emph{FM space} if there exists a number $N\in\nats$ such that any function $g\in V$ has at most $N$ maxima.
\end{definition}
\begin{definition}[DZ space]
Let $K\subset \reels$ be a non-empty set. Then a linear subspace $H$ of $\cont(K)$ is called \emph{DZ space} if any function $g\in H\setminus\{0\}$ has only isolated zeros.
\end{definition}

\section{Cauchy's Argument principle}

In this section we will deal with the question, how many zeros and poles of a meromorphic function function lie in a open set $D \subset \complex$ which is bounded by a rectifiable curve $\gamma$.

\begin{lemma}[Cauchy's Argument Principle] 
Let $f:D\to\complex$ be meromorphic inside a set $C$ and have no zeros on its contour $\gamma$. Moreover, let $P$ be the number of poles and $Z$ be the number of zeros of $f$ inside $C$. Then
%\begin{equation}
%\oint_{\gamma}{\frac{f'(z)}{f(z)}dz=2\pi\i
%\end{equation}
\end{lemma}

The theorem of Routh-Hurwitz is of remarkable importance in system and control theory taught in virtually any undergraduate course. In this section we want to take a closer look of the proof of this theorem.

\paragraph{The statement.}

To check the stability of a rational transfer function it is necessary to check weather all roots of its denominator lie in the left half-plane of $\complex$.

Such polynomials having all its roots in the left half-plane are often called \textsc{Hurwitz}-polynomials.

The check is often formulated as follows

\paragraph{Check.} Given a monic polynomial $P\in\reels[X]$ where $P=X^n+\cdots+a_0$. Such polynomial is \textsc{Hurwitz} if and only if 
\begin{enumerate}
\item All its coefficients are of the same sign (that is $\forall i\in[n] : a_i>0$).
\item All principal minors of the \textsc{Hurwitz}-Matrix $H_n(P)\in\reels^{n\timesvgfd n}$ have positive determinant where
\begin{equation}
H_n(P) := \begin{pmatrix}
a_{n-1} & a_{n-3} & \cdots & 0 & \cdots & 0\\
1 		 & a_{n-2} & \cdots & 0 & \cdots & 0\\
\vdots & \ddots & \ddots & \ddots & \ddots & \vdots\\
0 & \cdots & 0 & a_{n-1} & a_{n-3} & \cdots\\
0 & \cdots & 0 & 1 		 & a_{n-2} & \cdots\\
\end{pmatrix}\text{.}
\end{equation}
\end{enumerate}

\paragraph{The actual statement.} The theorem of \textsc{Routh-Hurwitz} as stated originally gives the above as an easy consequence. The idea is to count the difference between the number of zeros of in the left half-plane $L$ and the number of zeros in the right half-plane $R$ of $\complex$ of a polynomial $P\in\complex[X]$. 

\paragraph{The idea.}
A really simple but useful idea is here to consider the behavior of $\arg{P}$ on the imaginary axis.

To do this one needs the condition that $P$ has no root on the imaginary line. Furthermore, we assume that $P$ is monic ($a_n=1$).

Assume $P$ factors over $\complex$ such that 

\begin{equation}
P(z)=(z-l_1)\cdots(l-l_L)(z-r_1)\cdots(z-r_R)
\end{equation}

clearly this holds by fundamental theorem of algebra and by condition $R+L=n$. In the following equations let us abbreviate the operator $\lim_{\omega\to\infty}-\lim_{\omega\to-\infty}$ by $\Delta^\omega$ where $\omega\in\reels$.

\begin{equation}
\Delta^\omega{\arg{P(\omega\i)}} = \sum_{i=1}^L{\Delta^\omega{\arg(\omega\i-l_i)}}+\sum_{i=1}^R{\Delta^\omega{\arg(\omega\i-r_i)}}\label{eq1}
\end{equation}

Here $\arg$ is of course considered as continuous and takes values in $\reels$.

Let us evaluate each of the two sums separately. Thus, consider a left zero $l_i=a^l_i+b^l_i\i$ with $a^l_i<0$. One obtains that

\begin{equation}
\arg(\omega\i-l_i)=\arctan\left(\frac{\omega-b^l_i}{-a^l_i}\right)-\pi \text{ mod }2\pi
\end{equation}

As $a^l_i$ is negative, the above term increases as $\omega$ increases. Thus

\begin{equation}
\Delta^\omega{\arg(\omega\i-l_i)}=\pi
\end{equation}

and analogously for any $r_i$ we have

\begin{equation}
\Delta^\omega{\arg(\omega\i-r_i)}=-\pi\text{.}
\end{equation}

From this we get in equation (\ref{eq1}) that

\begin{equation}
\Delta^\omega{\arg{P(\omega\i)}} = \pi(L-R)\text{.}\label{eq2}
\end{equation}

\paragraph{Proceeding the argument.} A natural question is now, how we can get back from this last condition to the original polynomial. However, this is also not very difficult. Note that $P$ can be uniquely decomposed into polynomials $R$ and $S$ such that

\begin{equation}
P(\omega\i) = R(\omega)+S(\omega)\i\text{.}
\end{equation}

Now, it is a natural question to translate the term $\Delta^\omega{\arg{P(\omega\i)}}$ in a term depending on the rational function $\frac{S}{R}$ as the argument $\arg$ is connected to this function via $\arctan$.

Therefore one introduces the following definition. 

\section{The \textsc{Cauchy} index}

At this point it becomes reasonable to introduce a special parameter for rational functions from $\reels$ to $\reels$

\begin{definition}[\textsc{Cauchy} index] Let $I\subset\reels$ be an open interval and $F:I\setminus P\to \reels$ be a continuous function where $P$ is a finite set of poles of $F$, i.e. for each $p\in P$ we have 
\begin{equation}
\lim_{x\upto p}{F(x)}=\pm\infty
\end{equation}
and 
\begin{equation}
\lim_{x\downto p}{F(x)}=\pm\infty\text{,}
\end{equation}
respectively. For $r\in\reels$ one defines the pointwise \textsc{Cauchy} \emph{index} of $F$ as
\begin{equation}
\cind_r(F):=\begin{cases}
1 &: \lim_{x\upto r}{F(x)}=-\infty\wedge \lim_{x\downto r}{F(x)}=\infty\\
-1 &: \lim_{x\upto r}{F(x)}=\infty\wedge \lim_{x\downto r}{F(x)}=-\infty\\
0 &: \text{otherwise}
\end{cases}\text{.}
\end{equation}
Then the Cauchy index of $F$ on the interval $I$ is defined as
\begin{equation}
\cind(F) := \sum_{r\in I}{I_r(F)}\text{.}
\end{equation}
\end{definition}

Starting from this definition it is obvious that the operator $\mathcal{I}$ has several nice properties.

\begin{lemma}
Let $F:\reels\to\reels$ be rational and non-zero. Then one has 
\begin{equation}
\mathcal{I}(F)=-\mathcal{I}(F^{-1})\text{.}
\end{equation}
Moreover, if $P,Q\in\reels[X]$ have no common zero then one has 
\begin{equation}
\mathcal{I}\left(\frac{P}{Q}\right) = \mathcal{I}\left(\frac{\res(P,Q)}{Q}\right)\text{.}
\end{equation}
\end{lemma}

\begin{proof}
\textbf{The first statement:} Let $p_1,\ldots,p_l\in\reels\unify\{-\infty,\infty\}$ be the poles of $F$ on the real line ordered by ascending order. Consider the interval $(p_i,p_{i+1})$ for $i\in\{1,\ldots,l-1\}$. If 

\begin{equation}
\sgn(\lim_{r\downto p_i}{F(r)})=\sgn(\lim_{r\upto p_{i+1}}{F(r)})\label{eq5}
\end{equation}

then one gets that there must be equally many zeros $r_0$ of $F$ in $(p_i,p_{i+1})$ with $F'(r_0)<0$ as with $F'(r_0)>0$. Thus it directly follows that than

\begin{equation}
\mathcal{I}_{(p_i,p_{i+1})}(F^{-1})=0\text{.}
\end{equation}  

In the opposite case where

\begin{equation}
\sgn(\lim_{r\downto p_i}{F(r)})=-\sgn(\lim_{r\upto p_{i+1}}{F(r)})\label{eq6}
\end{equation}

one gets analogously that 

\begin{equation}
\mathcal{I}_{(p_i,p_{i+1})}(F^{-1})=\sgn(\lim_{r\downto p_i}{F(r)})\text{.}
\end{equation} 

Moreover, we then may compute $\mathcal{I}(F^{-1})$ by

\begin{equation}
\mathcal{I}(F^{-1}) = \sum_{i=1}^{l-1}{\mathcal{I}_{(p_i,p_{i+1})}(F^{-1})}+\mathcal{I}_\infty(F^{-1})
\end{equation}

From this consideration it is now clear, that we may assume there are no intervals $(p_i,p_{i+1})$ of the first type. To see this one checks that deleting them does not change $\mathcal{I}(F^{-1})$ and $\mathcal{I}(F)$.

Analogously, we may delete two consecutive intervals of the second type $(p_i,p_{i+1})$ and $(p_{i+1},p_{i+2})$ ($i\in\{1,\ldots,l-2\}$) where one has 

\begin{equation}
\lim_{r\downto p_{i+1}}{F(r)} = \pm\infty\text{.}
\end{equation}

These arguments can also be done if one bound of an interval is $\pm\infty$.
Proceeding this argument, one is left with the case that all intervals are of the second type and that for all $i,j\in\{1,\ldots,l\}$ we have 

\begin{equation}
\sgn(\lim_{r\downto p_i}{F(r)})=\sgn(\lim_{r\downto p_j}{F(r)}) \wedge \sgn(\lim_{r\upto p_i}{F(r)})=\sgn(\lim_{r\upto p_j}{F(r)})\text{.}
\end{equation}

It is then a routine matter to check the statement holds true.

\textbf{The second statement} is obvious.
\end{proof}

\begin{remark}
One can prove the first statement much faster considering the functions $\omega\mapsto P(\omega)\pm Q(\omega)\i$ where $F=P/Q$. The number of rotations of the curves of these functions in the complex plane around 0 is the same and directly connected with the two \textsc{Cauchy}-indices.
\end{remark}

\begin{definition}[Generalized \textsc{Sturm} chain]
Let $I\subset\reels$ be an open interval. A \emph{generalized} \textsc{Sturm} \emph{chain} over $I$ is a finite sequence of continuous functions $(f_i)_{i=0}^n$ such that
\begin{enumerate}
	\item The functions $f_i:I\to\reels$ have only finitely many zeros ($i=0,\ldots,n$).
	\item The function $\sgn{f_n}$ is constant on $I$.
	\item If $f_i(\xi)=0$ for $\xi\in I$ and $i\in\{1,\ldots,n-1\}$, then $\sgn(f_{i-1})=-\sgn(f_{i+1})$.
\end{enumerate}
\end{definition}

%% this one is needed for the next theorem

\begin{definition}[Number of sign changes for a finite sequence]
Let $(x_i)_{i=0}^n$ be a finite sequence. Let $(x_{i_j})_{j=0}^m$ be a subsequence of $(x_i)_{i=0}^n$ %which contains all non-zero entries of $x$ in the original order (i.e. $i_j$ is increasing)%
. Then we define the \emph{number of sign changes} of $x=(x_i)_{i=0}^n$ by
\begin{equation}
\sgnc(x):=\frac{1}{2}\sum_{i=1}^m{\abs{\sgn(x_{i_j})-\sgn(x_{i_{j-1}})}}\text{.}
\end{equation}
\end{definition}

%% the theorem -> connectoin Cauchy Index and gen. Sturm chains

\begin{theorem}[Calculation of the \textsc{Cauchy} index by a generalized \textsc{Sturm} chain]
Let $I=(a,b)\subset\reels$ be an open interval and $F:I\setminus P\to \reels$ be a continuous function where $P$ is a finite set of poles of $F$, i.e. for each $p\in P$ we have 
\begin{equation}
\lim_{x\upto p}{F(x)}=\pm\infty
\end{equation}
and 
\begin{equation}
\lim_{x\downto p}{F(x)}=\pm\infty\text{,}
\end{equation}
respectively. Let $(f_i)_{i=0}^n$ be a generalized \textsc{Sturm} chain such that 
\begin{equation}
F(\xi)=\frac{f_1(\xi)}{f_0(\xi)}
\end{equation}
for $\xi\in I\setminus P$. Then we have
\begin{equation}
\cind(F)=\sgnc(f_i(b))_{i=0}^n-\sgnc(f_i(a))_{i=0}^n\text{.}
\end{equation}
\end{theorem}

\begin{proof}

\end{proof}

\paragraph{Explanation.} The obvious intention of this definition is to count how many times a function $P(\omega\i)=R(\omega)+S(\omega)\i$ rotates around the origin $0$ of $\complex$. Therefore one may plugin $F=\frac{S}{R}$ are its reciprocal.

But we want to give some more detailed explanation.

\paragraph{Details.} %% Pictures needed! ATTENTION _ WORK TO DO
Let $\omega^{-1}_1,\ldots,\omega^{-1}_d\in\reels$ be the numbers where the function $P(\cdot\i):\reels\to\complex$ traverses the imaginary line counterclockwise and $\omega^1_1,\ldots,\omega^1_e\in\reels$ the numbers where $P(\cdot\i)$ traverses the imaginary line clockwise with respect to the origin. By condition, $P(\cdot\i)$ has no zero, thus for all these $\omega$'s we have $P(\omega\i)\neq 0$.

Formally, this means when $f:\reels\to\reels$ denotes the differentiable function $\omega\mapsto\arg{P(\omega\i)}$

\begin{equation}
R(\omega^i_j)=0 \wedge \sgn{f'(\omega^i_j)}=-i
\end{equation}

for $i=\pm 1$ and $j$ in the appropriate index set.
More, from this we see that

\begin{equation}
I_{\omega^i_j}\left(\frac{S}{R}\right)=-i\text{.}
\end{equation}

One thus may deduce the following equation

\begin{equation}
\Delta^\omega{\arg{P(\omega\i)}}=-\pi\mathcal{I}\left(\frac{S}{R}\right)+\Delta^\omega{\arctan\left(\frac{S}{R}\right)}\text{.}\label{eq4}
\end{equation}

Now, at this point, the most interesting argument comes in. Obviously, we have  $-\Delta^\omega\arg{\i\overline{P}(\omega\i)}=\Delta^\omega{\arg{P(\omega\i)}}$ and $\i\overline{P}(\omega\i)=S(\omega)+P(\omega)\i$. We thus may deduce from (\ref{eq4}) that 

\begin{align}
-\pi\mathcal{I}\left(\frac{S}{R}\right)+\Delta^\omega{\arctan\left(\frac{S}{R}\right)} = \pi\mathcal{I}\left(\frac{R}{S}\right)-\Delta^\omega{\arctan\left(\frac{R}{S}\right)}\text{.}
\end{align}

Now, let us consider the $\arctan$-terms. Indeed, here only the leading coefficients of $R$ and $S$ matter and it is easy to observe that 

BSBS
\begin{equation}
\Delta^\omega{\arctan\left(\frac{S}{R}\right)}=\pi(-1)^{\deg{R}-\deg{S}}
\end{equation}


From this definition one sees  that if $F=\frac{S}{R}$ for polynomials $R$ and $S$ then
\begin{equation}
-\pi\mathcal{I}(F)=-\pi\mathcal{I}\left(\frac{S}{R}\right)=\Delta^\omega{\arg(R(\omega)+S(\omega)\i)}\label{eq3}\text{.}
\end{equation}
 
From equations (\ref{eq2}) and (\ref{eq3}) we obtain that 

\begin{equation}
L-R = -\mathcal{I}_J\left(\frac{S}{R}\right)
\end{equation}
 


\end{document}
\message{ !name(On The Routh-Hurwitz Theorem.tex) !offset(-324) }
