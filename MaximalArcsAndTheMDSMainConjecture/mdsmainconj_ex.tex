
\begin{exercise}
    Prove that the \person{Hamming} distance is indeed a metric on $K^n$.
\end{exercise}

\begin{solution}
    As $d_H(a,b)=w_H(a-b)=w_H(b-a)$ it is clearly symmetric and identifying, i.e. $d_H(a,b)=0$ implies that $a_i=b_i$ for all $i\in n$ so $a=b$.
    The triangle inequality follows from the fact that $w_H(a+b)\leq w_H(a)+w_H(b)$ since for any $i\in n$ it follows from $a_i+b_i\neq 0$ that either $a_i\neq 0$ or $b_i\neq 0$.    
\end{solution}

\begin{exercise}
    Compute the size of a (closed) \person{Hamming} ball of radius $r\in\nats$.
\end{exercise}

\begin{solution}
    We count the elements of the closed ball of radius $r$ by counting seperately the elements of weight $w$.
    This yields
    $$
    \card{\ball_r}=\sum_{w=0}^r{\binom{n}{w}{(q-1)}^w}
    $$
\end{solution}

\begin{exercise}
    Let $\cC$ be an $(n,k,d)$-linear code which is MDS. Define the \emph{weight} of a vector $w(a)$ as $d_H(a,0)$. Analogously, for the elements of the corresponding projective space $\proj{\field{q}^n}$ define the weight of a point $a$ as the weight of some  representant of the linear subspace.
    Compute the size of the sets
    $$
    A_w:=\set{a\in\proj{\cC}:w(a)=w}.
    $$
\end{exercise}

\begin{solution}
    Recall the mapping $\pi:\field{q}^n\to\field{q}^{n-d+1}$ being injective on $\cC$ forgetting $d-1$ coordinates. The the preimage of any $w-d+1$-dimensional subspace of $\field{q}^{n-d+1}$ which is generated by a vectors attaining non-zero values only on a set $X\setleq n$ of size $w-d+1$ gives a $w-d+1$-dimensional subspace of $\cC$.
    
    Proceed in the following manner. Define for $X\setleq n$ that
    $$
    W(X):=\card{\proj\set{v\in \cC:v_i\neq 0\equival i\in X}}
    $$
    Then we have
    $$
    \sum_{Y\setleq X}{W(Y)}=
    \begin{cases}
        \frac{q^{\card{X}-d+1}-1}{q-1} &: \card X\geq d\\
        0 &: \otherwise    
    \end{cases}
    $$
    \person{Möbius} inversion gives us
    $$
    \sum_{\substack{Y\setleq X\\ \card{Y}\geq d}}{\mu(X,Y)\frac{q^{\card{Y}-d+1}-1}{q-1}}=W(X)
    $$
    where it is simple to show that $\mu(X,Y)={(-1)}^{\card X - \card Y}$ for $Y\setleq X$ and $\mu(X,Y)=0$ otherwise. This reveals
    $$
    \sum_{i=d}^{\card X}{{(-1)}^{\card{X}-i}\binom{\card X}{i}\frac{q^{i-d+1}-1}{q-1}}=\sum_{j=0}^{\card X-d}{q^j\sum_{i=d+j}^{\card X}{{(-1)}^{\card X -i}\binom{\card X}{i}}}=W(X)
    $$
    Using the fact that
    $$
    \sum_{i=0}^k{{(-1)}^i\binom{n}{i}}=(-1)^k\binom{n-1}{k}
    $$
    and letting $X$ run over all $w$-element subsets of $n$ we get the final result
    $$
    A_w=\binom{n}{w}\sum_{i=0}^{w-d}{{(-1)}^i\binom{w-1}{i}q^{w-d-i}}.
    $$
\end{solution}
