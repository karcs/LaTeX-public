\documentclass{article}
%My personal maths package
%\bibliography{Bibliography.bib}

%%%%%%%%%%%%%%%%%%%%%%%%%%%%%%%%%%%%%%%%%%%%%%%%%%%%%%%%%%%%%%%%%%%%%%%%%%%%%%
%%%%% MATH PACKAGES %%%%%%%%%%%%%%%%%%%%%%%%%%%%%%%%%%%%%%%%%%%%%%%%%%%%%%%%%%
%%%%%%%%%%%%%%%%%%%%%%%%%%%%%%%%%%%%%%%%%%%%%%%%%%%%%%%%%%%%%%%%%%%%%%%%%%%%%%

% very good package
%\usepackage{mathtools}

%%% font stuff
\usepackage[T1]{fontenc}        % for capitals in section /paragraph etc.
\usepackage[utf8]{inputenc}     % use utf8 symbols in code

\usepackage{amssymb,amsmath,amsfonts} % amsthm not needed -- use my own envs
%\usepackage{mathtools}
%\mathtoolsset{showonlyrefs}
% further alternative math packages: unicode-math, abx-math
\usepackage{bm}
\usepackage{mathrsfs} % used for: fraktal math letters
\usepackage[bigsqcap]{stmaryrd} % used for: big square cap symbol
\usepackage{xargs}

% standard packages
%\usepackage{color} % for color

%%
%%index

\newcommand*{\keyword}[2][\empty]{\emph{#2}\ifx#1\empty\index{#2}\else\index{#1}\fi}
\newcommand*{\person}[1]{\textsc{#1}}

%%%%%%%%%%%%%%%%%%%%%%%%%%%%%%%%%%%%%%%%%%%%%%%%%%%%%%%%%%%%%%%%%%%%%%%%%%%%%%%%%%%%%%%%%%%%%%%%%%%%%%%%%%%%
%%%%% MATH ALPHABETS & SYMBOLS %%%%%%%%%%%%%%%%%%%%%%%%%%%%%%%%%%%%%%%%%%%%%%%%%%%%%%%%%%%%%%%%%%%%%%%%%%%%%
%%%%%%%%%%%%%%%%%%%%%%%%%%%%%%%%%%%%%%%%%%%%%%%%%%%%%%%%%%%%%%%%%%%%%%%%%%%%%%%%%%%%%%%%%%%%%%%%%%%%%%%%%%%%

% w: http://milde.users.sourceforge.net/LUCR/Math/math-font-selection.xhtml

% ===== Set quick commands for math letters ================================================================
% calagraphic letters (only upper case available; standard)
\newcommand{\cA}{\mathcal{A}}
\newcommand{\cB}{\mathcal{B}}
\newcommand{\cC}{\mathcal{C}}
\newcommand{\cD}{\mathcal{D}}
\newcommand{\cE}{\mathcal{E}}
\newcommand{\cF}{\mathcal{F}}
\newcommand{\cG}{\mathcal{G}}
\newcommand{\cH}{\mathcal{H}}
\newcommand{\cI}{\mathcal{I}}
\newcommand{\cJ}{\mathcal{J}}
\newcommand{\cK}{\mathcal{K}}
\newcommand{\cL}{\mathcal{L}}
\newcommand{\cM}{\mathcal{M}}
\newcommand{\cN}{\mathcal{N}}
\newcommand{\cO}{\mathcal{O}}
\newcommand{\cP}{\mathcal{P}}
\newcommand{\cQ}{\mathcal{Q}}
\newcommand{\cR}{\mathcal{R}}
\newcommand{\cS}{\mathcal{S}}
\newcommand{\cT}{\mathcal{T}}
\newcommand{\cU}{\mathcal{U}}
\newcommand{\cV}{\mathcal{V}}
\newcommand{\cW}{\mathcal{W}}
\newcommand{\cX}{\mathcal{X}}
\newcommand{\cY}{\mathcal{Y}}
\newcommand{\cZ}{\mathcal{Z}}

% bold math letters (standard)
\newcommand{\bfA}{\mathbf{A}}
\newcommand{\bfB}{\mathbf{B}}
\newcommand{\bfC}{\mathbf{C}}
\newcommand{\bfD}{\mathbf{D}}
\newcommand{\bfE}{\mathbf{E}}
\newcommand{\bfF}{\mathbf{F}}
\newcommand{\bfG}{\mathbf{G}}
\newcommand{\bfH}{\mathbf{H}}
\newcommand{\bfI}{\mathbf{I}}
\newcommand{\bfJ}{\mathbf{J}}
\newcommand{\bfK}{\mathbf{K}}
\newcommand{\bfL}{\mathbf{L}}
\newcommand{\bfM}{\mathbf{M}}
\newcommand{\bfN}{\mathbf{N}}
\newcommand{\bfO}{\mathbf{O}}
\newcommand{\bfP}{\mathbf{P}}
\newcommand{\bfQ}{\mathbf{Q}}
\newcommand{\bfR}{\mathbf{R}}
\newcommand{\bfS}{\mathbf{S}}
\newcommand{\bfT}{\mathbf{T}}
\newcommand{\bfU}{\mathbf{U}}
\newcommand{\bfV}{\mathbf{V}}
\newcommand{\bfW}{\mathbf{W}}
\newcommand{\bfX}{\mathbf{X}}
\newcommand{\bfY}{\mathbf{Y}}
\newcommand{\bfZ}{\mathbf{Z}}
\newcommand{\bfa}{\mathbf{a}}
\newcommand{\bfb}{\mathbf{b}}
\newcommand{\bfc}{\mathbf{c}}
\newcommand{\bfd}{\mathbf{d}}
\newcommand{\bfe}{\mathbf{e}}
\newcommand{\bff}{\mathbf{f}}
\newcommand{\bfg}{\mathbf{g}}
\newcommand{\bfh}{\mathbf{h}}
\newcommand{\bfi}{\mathbf{i}}
\newcommand{\bfj}{\mathbf{j}}
\newcommand{\bfk}{\mathbf{k}}
\newcommand{\bfl}{\mathbf{l}}
\newcommand{\bfm}{\mathbf{m}}
\newcommand{\bfn}{\mathbf{n}}
\newcommand{\bfo}{\mathbf{o}}
\newcommand{\bfp}{\mathbf{p}}
\newcommand{\bfq}{\mathbf{q}}
\newcommand{\bfr}{\mathbf{r}}
\newcommand{\bfs}{\mathbf{s}}
\newcommand{\bft}{\mathbf{t}}
\newcommand{\bfu}{\mathbf{u}}
\newcommand{\bfv}{\mathbf{v}}
\newcommand{\bfw}{\mathbf{w}}
\newcommand{\bfx}{\mathbf{x}}
\newcommand{\bfy}{\mathbf{y}}
\newcommand{\bfz}{\mathbf{z}}

% fractal math letters (standard)
\newcommand{\fkA}{\mathfrak{A}}
\newcommand{\fkB}{\mathfrak{B}}
\newcommand{\fkC}{\mathfrak{C}}
\newcommand{\fkD}{\mathfrak{D}}
\newcommand{\fkE}{\mathfrak{E}}
\newcommand{\fkF}{\mathfrak{F}}
\newcommand{\fkG}{\mathfrak{G}}
\newcommand{\fkH}{\mathfrak{H}}
\newcommand{\fkI}{\mathfrak{I}}
\newcommand{\fkJ}{\mathfrak{J}}
\newcommand{\fkK}{\mathfrak{K}}
\newcommand{\fkL}{\mathfrak{L}}
\newcommand{\fkM}{\mathfrak{M}}
\newcommand{\fkN}{\mathfrak{N}}
\newcommand{\fkO}{\mathfrak{O}}
\newcommand{\fkP}{\mathfrak{P}}
\newcommand{\fkQ}{\mathfrak{Q}}
\newcommand{\fkR}{\mathfrak{R}}
\newcommand{\fkS}{\mathfrak{S}}
\newcommand{\fkT}{\mathfrak{T}}
\newcommand{\fkU}{\mathfrak{U}}
\newcommand{\fkV}{\mathfrak{V}}
\newcommand{\fkW}{\mathfrak{W}}
\newcommand{\fkX}{\mathfrak{X}}
\newcommand{\fkY}{\mathfrak{Y}}
\newcommand{\fkZ}{\mathfrak{Z}}
\newcommand{\fka}{\mathfrak{a}}
\newcommand{\fkb}{\mathfrak{b}}
\newcommand{\fkc}{\mathfrak{c}}
\newcommand{\fkd}{\mathfrak{d}}
\newcommand{\fke}{\mathfrak{e}}
\newcommand{\fkf}{\mathfrak{f}}
\newcommand{\fkg}{\mathfrak{g}}
\newcommand{\fkh}{\mathfrak{h}}
\newcommand{\fki}{\mathfrak{i}}
\newcommand{\fkj}{\mathfrak{j}}
\newcommand{\fkk}{\mathfrak{k}}
\newcommand{\fkl}{\mathfrak{l}}
\newcommand{\fkm}{\mathfrak{m}}
\newcommand{\fkn}{\mathfrak{n}}
\newcommand{\fko}{\mathfrak{o}}
\newcommand{\fkp}{\mathfrak{p}}
\newcommand{\fkq}{\mathfrak{q}}
\newcommand{\fkr}{\mathfrak{r}}
\newcommand{\fks}{\mathfrak{s}}
\newcommand{\fkt}{\mathfrak{t}}
\newcommand{\fku}{\mathfrak{u}}
\newcommand{\fkv}{\mathfrak{v}}
\newcommand{\fkw}{\mathfrak{w}}
\newcommand{\fkx}{\mathfrak{x}}
\newcommand{\fky}{\mathfrak{y}}
\newcommand{\fkz}{\mathfrak{z}}

% script math symbols (only uppercase; package: mathrsfs)
\newcommand{\sA}{\mathscr{A}}
\newcommand{\sB}{\mathscr{B}}
\newcommand{\sC}{\mathscr{C}}
\newcommand{\sD}{\mathscr{D}}
\newcommand{\sE}{\mathscr{E}}
\newcommand{\sF}{\mathscr{F}}
\newcommand{\sG}{\mathscr{G}}
\newcommand{\sH}{\mathscr{H}}
\newcommand{\sI}{\mathscr{I}}
\newcommand{\sJ}{\mathscr{J}}
\newcommand{\sK}{\mathscr{K}}
\newcommand{\sL}{\mathscr{L}}
\newcommand{\sM}{\mathscr{M}}
\newcommand{\sN}{\mathscr{N}}
\newcommand{\sO}{\mathscr{O}}
\newcommand{\sP}{\mathscr{P}}
\newcommand{\sQ}{\mathscr{Q}}
\newcommand{\sR}{\mathscr{R}}
\newcommand{\sS}{\mathscr{S}}
\newcommand{\sT}{\mathscr{T}}
\newcommand{\sU}{\mathscr{U}}
\newcommand{\sV}{\mathscr{V}}
\newcommand{\sW}{\mathscr{W}}
\newcommand{\sX}{\mathscr{X}}
\newcommand{\sY}{\mathscr{Y}}
\newcommand{\sZ}{\mathscr{Z}}

%%%%%%%%%%%%%%%%%%%%%%%%%%%%%%%%%%%%%%%%%%%%%%%%%%%%%%%%%%%%%%%%%%%%%%%%%%%%%%%%%%%%%%
%%%% BOLD MATH IN BOLD TEXT ENVIRONMENT %%%%%%%%%%%%%%%%%%%%%%%%%%%%%%%%%%%%%%%%%%%%%%
%%%%%%%%%%%%%%%%%%%%%%%%%%%%%%%%%%%%%%%%%%%%%%%%%%%%%%%%%%%%%%%%%%%%%%%%%%%%%%%%%%%%%%

% for bold math in bold text (e.g. sections)
\makeatletter
\g@addto@macro\bfseries{\boldmath}
\makeatother

\def\brackets#1{\ifx#1\empty\else\left(#1\right)\fi}

%%%%%%%%%%%%%%%%%%%%%%%%%%%%%%%%%%%%%%%%%%%%%%%%%%%%%%%%%%%%%%%%%%%%%%%%%%%%%%%%%%%%%%
%%%%% CATEGORY THEORY %%%%%%%%%%%%%%%%%%%%%%%%%%%%%%%%%%%%%%%%%%%%%%%%%%%%%%%%%%%%%%%%
%%%%%%%%%%%%%%%%%%%%%%%%%%%%%%%%%%%%%%%%%%%%%%%%%%%%%%%%%%%%%%%%%%%%%%%%%%%%%%%%%%%%%%

% ===== Category theory concepts =====================================================

\newcommand{\Ob}{\mathop\mathrm{Ob}}
\newcommand{\Mor}{\mathop\mathrm{Mor}}

\newcommand{\ccoprod}{\bigsqcup}
\newcommand{\cprod}{\bigsqcap}
\newcommand{\cincl}{\mathop\mathrm{incl}}
\newcommand{\cproj}{\mathop\mathrm{pr}}


% ===== Define standard categories ===================================================

% Define sets
\newcommand{\Set}{\mathbf{Set}}
% Define set-builder operator (equivalent to gen for algebras)
\newcommand{\set}[1]{\left\{#1\right\}}
% define interval operator: o - open, c - closed
\newcommand{\intervalcc}[2]{\left[#1,#2\right]}
\newcommand{\intervalco}[2]{\left[#1,#2\right)}
\newcommand{\intervaloc}[2]{\left(#1,#2\right]}
\newcommand{\intervaloo}[2]{\left(#1,#2\right)}

\newcommand{\inter}{\mathop\mathrm{int}}
\newcommand{\face}{\mathop\mathrm{F}}
\newcommand{\Pol}{\mathop\mathrm{Pol}}
\newcommand{\Inv}{\mathop\mathrm{Inv}}
\def\struct#1{\gen{#1}}

\let\originaltimes\times%
\renewcommand{\times}{\mathbin{\sqcap}}
\newcommand\settimes{\originaltimes}
\newcommand{\setleq}{\subseteq}
\newcommand{\setgeq}{\supseteq}

\newcommand{\pderive}[2]{\frac{\partial{#1}}{\partial{#2}}}
\renewcommand{\div}{\mathop\mathrm{div}}

%% Diffgeo
\def\Ric{\mathop\mathrm{Ric}}
\def\ric{\mathop\mathrm{ric}}
\def\tr{\mathop\mathrm{tr}}

\def\cotimes{\mathbin{\sqcup}}

\def\@rightopen#1{\ifx#1]{\right]}\else{\interval@errmessage}\fi}
\def\@leftclosed[#1){\left[#1\right)}
\makeatother

% finite
\newcommand{\fin}{\mathrm{fin}}

% Define groups (optarg: properties such as -> abelian, noetherian (acc), artinian (dcc) etc.)
\newcommand{\Grp}[1][\empty]{\if\empty{#1}{\mathbf{Grp}}\else{\mathbf{Grp}_{#1}}}
\def\PGL{\mathrm{PGL}}
\def\PGammaL{\mathrm{P\Gamma L}}
\def\GL{\mathrm{GL}}
% Define rings
\newcommand{\rg}{\mathrm{rg}} %rank of a matrix
\newcommand{\Rg}[1][\empty]{\if\empty{#1}{\mathbf{Rg}}\else{\mathbf{Rg}_{#1}}}
\edef\units#1{#1^{\settimes}}
\def\dual#1{#1^{\ast}}

%% redefine the command \P to produce the projective functor in math mode
\let\parsymb\P%
\def\P{\ifmmode\mathrm{P}\else\parsymb\fi}
\renewcommand{\iff}{\ifmmode\equival\else{if and only if}\fi}
\newcommand{\quotring}{\mathop{\mathrm{Q}}}
\newcommand{\rad}{\mathrm{rad}}
% Standard rings
% integral domains
\newcommand{\ID}{\mathbf{ID}}
% unique factorization domains
\newcommand{\UFD}{\mathbf{UFD}}
% principal ideal domains
\newcommand{\PID}{\mathbf{PID}}

% Define modules over a group or ring
\newcommand{\Mod}[1]{\mathbf{Mod}_{#1}}
% Define vector space over a field
\renewcommand{\Vec}[1]{\mathbf{Vec}_{#1}}%

% when cases
\def\otherwise{\textrm{otherwise}}

% new concepts
\newcommand{\new}[1]{\emph{#1}}

\usepackage{xifthen,xstring}

% replace the bar command by overline when argument just one character (shorter and better)
%$\let\oldbar\bar
%\renewcommand{\bar}[1]{\StrLen{#1}[\length]\ifthenelse{\length > 1}{\overline{#1}}{\oldbar{#1}}}
\def\bar{\overline}

%% argument in equatoin
\def\arg{\bullet}

% groups and algebras
\newcommand{\Con}{\mathop\mathrm{Con}}
\newcommand{\Sub}{\mathop\mathrm{Sub}}
\newcommand{\Hom}{\mathop\mathrm{Hom}}
\newcommand{\Aut}{\mathop\mathrm{Aut}}
\newcommand{\Out}{\mathop\mathrm{Out}}
\newcommand{\End}{\mathop\mathrm{End}}
\newcommand{\id}{\mathop\mathrm{id}}
\newcommand{\rk}{\mathop\mathrm{rk}} % rank of a group module/ lattice
\newcommandx{\con}[1][1=\empty]{\ifx#1\empty{\mathop{\mathrm{con}}}\else{\mathop{\mathrm{con}}\left(#1\right)}\fi}
\newcommand{\leftsemidirprod}[1][]{\mathbin{\ifx&#1&\ltimes\else{\ltimes_{#1}}\fi}}
\newcommand{\rightsemidirprod}[1][]{\ifx#1\empty\rtimes\else{\rtimes_{#1}}\fi}
\newcommand{\normalisor}[2][]{\ifx#1\empty{\mathrm{N}\left(#2\right)}\else{\mathrm{N}_{#1} \left(#2\right)}\fi}
% support
\newcommand{\spt}{\mathop{\mathrm{spt}}}
% commutator
\newcommand{\gcom}[2]{\left[#1,#2\right]}

% physics stuff
\newcommand{\float}[3][\empty]{\ifx#1\empty{{#2}\cdot{10^{#3}}}\else{{#2}\cdot{{#1}^{#3}}}\fi}
\makeatletter
\def\newunit#1{\@namedef{#1}{\mathrm{#1}}}
\def\mum{\mathrm{\mu m}}
\def\ohm{\Omega}
\newunit{V}
\newunit{mV}
\newunit{kV}
\newunit{s}
\newunit{ms}
\def\mus{\mathrm{\mu s}}
\newunit{m}
\newunit{nm}
\newunit{cm}
\newunit{mm}
\newunit{fF}
\newunit{A}

\newunit{fA}
\newunit{C}

% elements
\def\newelement#1{\@namedef{#1}{\mathrm{#1}}}
\newelement{Si}
\makeatother


% groups
\newcommand{\ord}{\mathop\mathrm{ord}}
\newcommand{\divides}{|}

\newcommand{\conleq}{\trianglelefteq}
\newcommand{\congeq}{\trianglerighteq}

% common algebraic objects
\newcommand{\reals}{\mathbb{R}} 			% real numbers
\newcommand{\nats}{\mathbb{N}} 				% natural numbers
\newcommand{\ints}{\mathbb{Z}} 				% integers
\newcommand{\rats}{\mathbb{Q}}				% rationals
\newcommand{\complex}{\mathbb{C}}			% complex numbers
\newcommand{\field}[1]{\mathbb{F}_{#1}}  		% finite field
\newcommand{\cards}{\boldsymbol{Cn}}                     % The cardinal numbers
\newcommand{\ords}{\boldsymbol{On}}                      % The ordinal numbers

% graphs
\def\KG{\mathop\mathrm{KG}}                     % Knesergraph

\newcommand{\uvect}{\boldsymbol{e}}

% new operators and relations

%%%%%%%%%%%%%%%%%%%
% complex numbers %
%%%%%%%%%%%%%%%%%%%

\renewcommand{\Re}{\mathop\mathrm{Re}}		% real part
\renewcommand{\Im}{\mathop\mathrm{Im}}		% imaginary part
\newcommand{\sgn}{\mathop\mathrm{sgn}}				% the sign operator (0 for 0)

%%%%%%%%%%%%%%%%
% reel numbers %
%%%%%%%%%%%%%%%%

\newcommand{\floor}[1]{\left\lfloor#1\right\rfloor}
\newcommand{\ceil}[1]{\left\lceil#1\right\rceil}

%%%%%%%%%%%%%%%%%%
% set operations %
%%%%%%%%%%%%%%%%%%

\newcommand{\intersect}{\cap}			% intersect to sets
\newcommand{\setjoin}{\cup}				% join two sets
\newcommand{\setmeet}{\cap}                     % intersect to sets
\newcommand{\bigsetjoin}{\bigcup}			% the union of sets ... subscripts to be added
\newcommand{\distunion}{\dot{\bigcup}}	% disjoint union of sets ... subscripts to be added
\newcommand{\bigsetmeet}{\bigcap}		% intersection of sets
\newcommand{\powerset}[1][]{\ifx&#1&\mathcal{P}\else\mathcal{P}_{#1}\fi}		% powerset ... to be customized
\newcommand{\card}[1]{\left|#1\right|}

%%%%%%%%%%%%%%%%%%%%%%%%%%%%%%%%%%%%%%%%
% composition operations of structures %
%%%%%%%%%%%%%%%%%%%%%%%%%%%%%%%%%%%%%%%%

%\newcommand{\setprod}{\bigtimes}			% setproduct - needed
\newcommand{\dirprod}{\bigotimes} 			% direct product for groups and spaces
\newcommand{\dirtimes}{\otimes}				% direct multiply for groups and spaces
\newcommand{\dirsum}{\bigoplus} 			% direct sum for groups and spaces
\newcommand{\dirplus}{\oplus}				% direct add for groups and spaces
\newcommand{\inprod}[2]{\left\langle #1,#2 \right\rangle}

\newcommand{\tuple}{\meet}
\newcommand{\cotuple}{\join}

%%%%%%%%%%%%%%
% categories %
%%%%%%%%%%%%%%

% combinatorics
%%

\renewcommand{\binom}[3][\empty]{\if\empty{#1}{{#2 \choose #3}}\else{{#2 \choose #3}_{#1}}}

%%%%%%%%%%%%%%%%%%%%%%%%%%%%%%%%%%%%%%%%%%%%%%%%%%%%%
% metric spaces and normed spaces and vector spaces %
%%%%%%%%%%%%%%%%%%%%%%%%%%%%%%%%%%%%%%%%%%%%%%%%%%%%%

\newcommand{\dist}{\mathop\mathrm{dist}}				% distance operator ... dist(A,b), where A is a set and b a point
\newcommand{\diam}{\mathop\mathrm{diam}}	% diameter operator for sets
\newcommand{\norm}[1]{\left\Vert #1 \right\Vert}	% norm in a normed space ... subscript to be added
\newcommand{\conv}{\mathop\mathrm{conv}} 			% convex hull - vectorspaces
\newcommand{\lin}{\mathop\mathrm{lin}} 				% linear hull - vectorspaces
\newcommand{\aff}{\mathop\mathrm{aff}}				% affine hull - vectorspaces

%%%%%%%%%%%%%%%%%%%%%%
% operators in rings %
%%%%%%%%%%%%%%%%%%%%%%

\newcommand{\lcm}{\mathop\mathrm{lcm}}				% least common multiple - in euclidean rings
\renewcommand{\gcd}{\mathop\mathrm{gcd}}				% greatest command devisor - in euclidean rings
\newcommand{\res}{\mathop\mathrm{res}}				% residue of p mod q is res(p,q)
\renewcommand{\mod}{\textrm{ mod }}

%%%%%%%%%%%%%%%%%%%
% logical symbols %
%%%%%%%%%%%%%%%%%%%

%\newcommand{\impliedby}{\Leftarrow}				% reverse implicatoin arrow
%\newcommand{\implies}{\Rightarrow}				% implication
\newcommand{\equival}{\Leftrightarrow}				% equivalence

%%%%%%%%%%%%%%%%%%%%%%%%%%%%%%%%%%
% functions - elementary symbols %
%%%%%%%%%%%%%%%%%%%%%%%%%%%%%%%%%%

\newcommand{\rest}[1]{\left. #1\right\vert}		% restriction of a function to a set / also used as restriction in other terms like differential expressions / evaluation of a function
\newcommand{\rto}[3][]{#2\ifx&#1&\rightarrow\else\stackrel{#1}{\rightarrow}\fi#3}
\renewcommand{\to}{\rightarrow}							% arrow between domain and image
\newcommand{\dom}{\mathop\mathrm{dom}}					% domain of a function
\newcommand{\im}{\mathop\mathrm{im}}						% image of a function
\newcommand{\compose}{\circ}							% compose two functions
\newcommand{\cont}{\mathop\mathrm{C}}					% continuous functions from a domain into the reels or complex numbers 

%%%%%%%%%%%%%%%%%%%%
% groups - symbols %
%%%%%%%%%%%%%%%%%%%%

\newcommand{\stab}[1][]{\if&#1{\mathop\mathrm{stab}}&\else{\mathop\mathrm{stab}_{#1}}\fi}					% the stabilizer ... subscripts to be added
\newcommand{\orb}[1][]{\ifx&#1&\mathrm{orb}\else\mathrm{orb}_{#1}\fi} 					% orbit ... subscrit to be added (group)
\newcommand{\gen}[2][\empty]{\ifx#1\empty{\left\langle#2\right\rangle}\else{\left\langle#2\right\rangle_{#1}}\fi}					% generate ... kind of hull operator ---- to be thought of !!!!!!!

\def\Clo{\mathrm{Clo}}
\def\Loc{\mathrm{Loc}}
%% nets
\newcommand{\net}[2][\empty]{\ifx#1\empty{\left(#2\right)}\else{{\left(#2\right)}_{#1}}\fi}

%% open half ray
\newcommand{\ray}[2]{R_{#1}(#2)}

%% new
\let\oldcong\cong%
\newcommand{\iso}{\oldcong}

\def\cong{\equiv}
\newcommand{\base}[2]{\left[#2\right]_{#1}}                                   % base n expansion of some number
%%%%%%%%%%%%%%%%%%%%%%%%%%%%%%%%%
% matrices and linear operators %
%%%%%%%%%%%%%%%%%%%%%%%%%%%%%%%%%

\newcommand{\diag}{\mathop\mathrm{diag}}				% diagonal matrix or operator
\newcommand{\Eig}[1]{\mathop\mathrm{Eig}_{#1}}		% eigenspace for a certain eigenvalie		
\newcommand{\trace}{\mathop\mathrm{tr}}				% trace of a matrix
\newcommand{\trans}{\top} 							% transponse matrix

%%%%%%%%%%%%%
% constants %
%%%%%%%%%%%%%

\renewcommand{\i}{\boldsymbol{i}}			% imaginary unit
\newcommand{\e}{\boldsymbol{e}}				% the Eulerian constant

%%%%%%%%%%%%%%%%%%%%%%%%%%%%%%
% limit operators and arrows %
%%%%%%%%%%%%%%%%%%%%%%%%%%%%%%

\newcommand{\upto}{\uparrow}				% convergence from above
\newcommand{\downto}{\downarrow}			% convergence from below

%%
% other
%%
\newcommand{\cind}{\mathop\mathrm{Ind}}		% Cauchy index
\newcommand{\sgnc}{\sigma}					% sign changes
\newcommand{\wnumb}{\omega}					% winding number
\newcommand{\cfunc}{\mathop\mathrm{Cf}}		% Cauchy function of a compact curve in complex\setminus\{0\}


% evaluation of a function as a difference or single value

\newcommand{\abs}[1]{\left|#1\right|}
\newcommand{\conj}[1]{\overline{#1}}
\newcommand{\diff}{\mathop\mathrm{d}}

%%%%% test
\newcommand{\distjoin}{\mathaccent\cdot\cup}	% to be modified (name)
\newcommand{\cl}{\mathop\mathrm{cl}}				% topological closure
\newcommand{\sphere}{\mathbb{S}} % n-sphere
\newcommand{\ball}{\mathbb{B}} % n-ball
\newcommand{\bound}{\partial}
\newcommand{\bigmeet}{\mathop\mathrm{\bigwedge}}
\newcommand{\bigjoin}{\mathop\mathrm{\bigvee}}
\newcommandx{\rchar}[1][1=\empty]{\mathop\mathrm{char}\brackets{#1}}              % characteristic of a ring
\newcommand{\lgor}{\vee}                               % logical
\newcommand{\lgand}{\wedge}
\newcommand{\codim}{\mathop\mathrm{codim}}
\newcommand{\row}{\mathop\mathrm{row}}
\newcommand{\cone}{\mathop\mathrm{cone}}
\newcommand{\comp}{\mathop\mathrm{comp}}
\newcommand{\proj}{\mathrm{P}}
\def\PG{\mathrm{PG}}           % projective space
\newcommand{\meet}{\wedge}
\newcommand{\join}{\vee}
\newcommand{\col}{\mathop\mathrm{col}}
\newcommand{\vol}{\mathop\mathrm{vol}\nolimits}
%arrangements
\newcommand{\tpert}{\mathop\mathrm{tpert}}
\renewcommand{\epsilon}{\varepsilon}
% groups
\newcommand{\symgr}{\mathop\mathrm{Sym}}
\newcommand{\symalg}{\mathop\mathrm{S}}
\newcommand{\extalg}{\mathop\mathrm{\Lambda}}
\newcommand{\extpow}[1]{\mathop\mathrm{\Lambda}^{#1}}

%% set hulloperators -> define



\newcommandx{\homl}[3][1=1,2=2,3=3]{\ifx#1\empty{\mathrm{Hom}}\else{\mathrm{Hom}_{#1}}\fi(#2,#3)}
\makeatletter
\newenvironment{myproofof}[1]{\par
  \pushQED{\qed}%
  \normalfont \topsep6\p@\@plus6\p@\relax
  \trivlist
  \item[\hskip\labelsep
        \bfseries
    Proof of #1\@addpunct{.}]\ignorespaces
}{%
  \popQED\endtrivlist\@endpefalse
}
\makeatother



%%% environment test with enumerates

    

% Local variables:
% mode: tex
% End:



% links
\usepackage{hyperref}

%setup
\hypersetup{
    bookmarks=true,         % show bookmarks bar?
    unicode=false,          % non-Latin characters in Acrobat’s bookmarks
    pdftoolbar=true,        % show Acrobat’s toolbar?
    pdfmenubar=true,        % show Acrobat’s menu?
    pdffitwindow=false,     % window fit to page when opened
    pdfstartview={FitH},    % fits the width of the page to the window
    pdftitle={My title},    % title
    pdfauthor={Author},     % author
    pdfsubject={Subject},   % subject of the document
    pdfcreator={Creator},   % creator of the document
    pdfproducer={Producer}, % producer of the document
    pdfkeywords={keyword1} {key2} {key3}, % list of keywords
    pdfnewwindow=true,      % links in new window
    colorlinks=false,       % false: boxed links; true: colored links
    linkcolor=blue,         % color of internal links (change box color with linkbordercolor)
    citecolor=green,        % color of links to bibliography
    filecolor=magenta,      % color of file links
    urlcolor=cyan           % color of external links
}

%\usepackage{biblatex} & bullshit 

\title{A Geometric Approach on Perron-Frobenius Theorem (and some applications in graph theory)}
\author{Jakob Schneider}

\begin{document}

\maketitle
\tableofcontents

\begin{abstract}
The theorem of Perron-Frobenius in a finite-dimensional space is usually proven using means like Brouwers fixed point theorem or metrics in which the corresponding mapping is somehow contractive (Banachs fixed point theorem).
We will give a geometric proof on the subject for polyhedral cones using a different technique which is based on the finiteness of vertices of the corresponding polytopes. 
This shall be done for ideal irreducible and primitive operators.
In the last section, some graph-theoretical application of the theorem is presented.
\end{abstract}

\section{Symbols}

The following symbols will be used through this paper:

\begin{itemize}
\item $V$ denotes some ordered vector space over $\reals$ with generating cone $V^+$ admitting base $B$\footnote{Recall that a base $B\subset V^+$ of a cone $V^+$ is a convex set such that for all $v>0$ there is a unique $\lambda>0$ and $b\in B$ such that $\lambda b=v$. For any base $B$ there exists a strictly positive linear functional $f:V\to \reals$ such that $B=f^{-1}\{1\}\intersect V^+$.} (the symbol $B$ will only be used for such base).
If stated, it is finite-dimensional (dimension is usually denoted by $n$) or its cone has polyhedral base.
\item $P$ denotes some compact polytope in $V$.
\item $T$ denotes some positive linear operator from $V$ to $V$.
\item $\norm{\cdot}$ denotes the Euclidean norm in finite-dimensional space.
\item $\varrho(A)$ the spectral radius of operator $A$.
\item $\sigma(A)$ the spectrum of operator $A$.
\end{itemize}

\section{Introduction}

\subsection{Some definitions around polytopes}

As the theorem of Perron-Frobenius is especially interesting when stated for polyhedral cones, we will need the following

\begin{definition}[compact finite-dimensional polytope]\label{def1}
Let $V$ be a linear space\footnote{This will always mean a vectorspace over $\reals$.} and $P\subset V$. Then $P$ is called a \emph{compact\footnote{All topological expressions are meant in Euclidean topology.} finite-dimensional polytope} if $P=\conv S$ for a finite set $S\subset V$. The \emph{dimension} of $P$ as $\dim\lin P$.\footnote{Clearly $\dim\lin P< |S|$.}
\end{definition}

Next, we introduce $k$-dimensional faces and the $k$-dimensional boundary of a polytope as they are heavily used in our proof.

\begin{definition}[degree, boundary, face]\label{def2}
Let $P$ be a finite-dimensional compact polytope and let $p\in P$. We define the \emph{degree}\footnote{Roughly speaking, this is the maximum of the dimension of the faces in whose interior $p$ lies.} of $p$ (with respect to $P$) as
\begin{equation}
\deg_P p := \max\{|Q|:Q\subset P, Q \text{ affinely independent } \wedge p\in\int \conv Q\}-1\text{.}
\end{equation} 

Then we define the \emph{$k$-dimensional boundary} $\partial_k{P}$ by\footnote{By $\nats$ we denote the set $\{0,1,2\ldots\}$} 
\begin{equation}
\partial_k{P}:= \deg_P^{-1}\{l\in\nats_0:l\leq k\}\text{.}
\end{equation}

Furthermore, any maximal\footnote{with respect to inclusion} convex subset of $\partial_k{P}$ is called a \emph{$k$-dimensional face}.
\end{definition}  

\begin{remark}\label{rem1}
The set $\partial_0 P$ will also be denoted as \emph{vertices}. It is an easy exercise to show that for any $n$-dimensional compact convex polytope $P$ ($n\in\nats$) the sets $\partial_0 P,\ldots, \partial_n P$ are not empty. 
\end{remark}

\begin{remark}\label{rem2}
One may take $Q\subset\partial_0 P$ in the above definition.
\end{remark}

\begin{remark}
If $P=V^+$ is a cone then $\deg_{V^+}{v}=\dim I_v$ where $I_v$ denotes the ideal generated by $v$.   
\end{remark}

We will see that the spectral properties of the linear maps we are going to consider become especially nice when we have a lattice cone. Thus we give the following definition.

\begin{definition}[simplex]\label{def3}\footnote{this definition is also given in \cite{PerOrdTopVec}.}
Let $V$ be a linear space and $S\subset V$ a convex set such that for any two homotheties\footnote{By a homothety we denote any map of the form $x\mapsto \lambda x+y$ acting on a vectorspace ($\lambda\geq 0$).} $\alpha, \beta:V\to V$ the intersection $\alpha(S)\intersect\beta(S)$ is either empty or the image $\gamma(S)$ of $S$ under another homothety $\gamma:V\to V$. Then we call $S$ a \emph{simplex}\footnote{A simplex $S$ does not need to have 'finite-dimensional faces' (in the sense that there is some $s\in S$ in definition \ref{def2} such that $\deg_S{s}$ is a finite cardinal number). The reader is invited to think of such simplices (hint: consider the base of the Archimedean vector lattice $C[0,1]$ with the pointwise order).}.
\end{definition}

%Of course this definition might differ slightly from the readers intuition of an $n$-simplex in the Euclidean space $\reals^n$, ignoring compactness. Thus we introduce a weaker notion of compactness.
%
%\begin{definition}[line-compactness]\label{def4}
%Let $V$ be a linear space and $C\subset V$. Then $V$ is called \emph{line-compact} if for all lines $l\subset V$ (one-dimensional subspace) the set $C\intersect l$ is compact\footnote{We may replace $l$ in this definition by a finite-dimensional subspace if $C$ is convex (using the Euclidean topology of this space).} (in the topology of the line - i.e. the default topology of $\reals$).
%\end{definition}

Now it is a routine matter to check that a finite-dimensional compact simplex in the sense of these two definition coincides with the notion of an $n$-simplex (that is a convex hull of $n+1$ affinely independent points).

\begin{remark}\label{rem3}
The reader might ask why we use such general definition instead of just the version for finite-dimensional spaces. The reason for this is that from this definition one should directly see that the intersection of a chain of compact simplices is itself a simplex. This property will be of importance.
\end{remark}

\subsection{The idea of the proof}

The following simple fact about simplices and finite-dimensional polytopes shall be mentioned at this point as they were the initial idea for our proof of Perron-Frobenius theorem.

\begin{lemma}[Intersections of polytopes and simplices]\label{lem0}
Let $(S_m)_{m\in\nats}$ be a chain of sequentially compact\footnote{This condition is needed - otherwise any based Archimedian lattice space would have suprema for any totally ordered non-decreasing bounded net $x_\alpha\uparrow$ which is obviously false for $C[0,1]$.} simplices in a linear Hausdorff space $V$ (that is $S_m\supset S_n$ for $m\leq n$). Then $\intersection_{m\in\nats}{S_m}$ is itself a simplex\footnote{This is what was mentioned in remark \ref{rem3}.}. The same holds for compact finite-dimensional polytopes with uniformly bounded number of vertices.
\end{lemma}

\begin{proof}
\textbf{\emph{For simplices.}} Consider the set $S:=\intersection_{m\in\nats}{S_m}$ where $(S_m)_{m\in\nats}$ is the given chain of simplices. Then for any non-degenerate homotheties $\alpha,\beta$ we have\footnote{For degenerate homotheties there is nothing to prove as the $\alpha(S)\intersect\beta(S)$ would be empty or a singleton.}
\begin{equation}
\alpha(S)\intersect\beta(S)=\intersection_{m\in\nats}{\alpha(S_m)\intersect\beta(S_m)}
\end{equation}
where this is due to injectivity of $\alpha,\beta$. However, this is again a chain of simplices and we find homotheties $\gamma_m$ such that
\begin{equation}
\alpha(S)\intersect\beta(S)=\intersection_{m\in\nats}{\gamma_m(S_m)}\text{.}
\end{equation}
The obvious goal is now to find a homothety $\gamma$ satisfying $\gamma(S)=\intersection_{m\in\nats}{\gamma_m(S_m)}$. Let us distinguish two cases. 

\emph{Case 1.} If $S$ is a singleton\footnote{Singleton means a set of the form $\{x\}$.} there is nothing to prove (as $\alpha(S)\intersect\beta(S)$ is either a singleton - in which case we choose the corresponding constant homothety - or empty).
The same holds if $\alpha(S)\intersect\beta(S)$ is a singleton or empty and $S$ does not behave trivial.

\emph{Case 2.} Thus the only interesting case is when $\dim\lin{S},\dim\lin[\alpha(S)\intersect\beta(S)]>0$.

In this case due to sequential compactness of $S_0, \gamma_0(S_0)$ and non-trivial behavior of $\alpha(S)\intersect\beta(S)$  and $S$ the real number 
\begin{equation}
b :=\sup\{\lambda\geq 0:\exists y\in V:\lambda(S)+y\subset \gamma_0(S_0)\}
\end{equation}
 must exist (otherwise the former sets $S_0, \gamma_0(S_0)$ would not be bounded\footnote{Von-Neumann boundedness: a set $B\subset V$ is called \emph{bounded} if for any neighborhood $N$ of $0$ there is $\xi>0$ such that $B\subset \xi N$.} and thus could not be sequentially compact, respectively) This last argument uses the Hausdorff property. (Assume $b=\infty$. Then for any $x\in S-S$ there exists sequences $(y_m)_{m\in\nats}$ and $(\lambda_m)_{m\in\nats}$ with $y_m\in S,\lambda_m>0,\lambda_m\to\infty$ as $ m\to\infty$ such that $\conv\{y_m,y_m+\lambda_m x\}\subset \gamma_0(S_0)$. But as $S$ is sequentially compact we find a convergent subsequence $(y_{m_j})_{j\in\nats}$ convergent to $y^*\in S$. It is clear that then $y^*+\reals^+ x\subset \gamma_0(S_0)$, but this contradicts sequential compactness of $\gamma_0(S_0)$ as it implies that $\reals^+ x\subset N$ for any neighborhood of 0 contradicting $V$ being Hausdorff.)  

Thus the scaling constants $\lambda_m$ of $\gamma_m$ (that means $\gamma_m:V\to V,x\mapsto \lambda_m x+y_m
$) are uniformly bounded by $b$ as for $y\in V,m\in\nats$ we have

\begin{equation}
\lambda(S_m)+y\subset \alpha(S)\intersect\beta(S) \implies \lambda(S)+y\subset \gamma_0(S_0)\text{.}
\end{equation}

Moreover, we may assume w.l.o.g. that $0\in S$\footnote{The property being simplex is translation invariant.}. We then notice that the translation parameters $y_m$ all must lie in the sequentially compact set $S_0$ for $m\in\nats$. Thus the parameters of the $\gamma_m$ lie in the sequentially compact space $(\lambda_m,x_m)\in[0,b]\times S_0$ and thus have an accumulation point. This shows actually that there is a subsequence $(\gamma_{m_j})_j$ of the homotheties which converges pointwise. Now assume any accumulation point $(\lambda^*,x^*)$ satisfies $\lambda^*=0$. But in this case, $\beta(S)\intersect\beta(S)$ shrinks down to a point (which was excluded) as the following identity shows

\begin{equation}
\intersection_{m\geq m_0}{\gamma_m(S_m)}\subset \intersection_{m\in\nats}{\gamma_m(S_0)} = \{x\}\text{ for some } x\in V\text{.}\label{eq0}
\end{equation} 

This identity can be proven by the following argument: Define for a sequentially compact set $B$ and a vector $x\in V$ the number 
\begin{equation}
L(B,x):=\sup\{\lambda\geq 0:\exists y\in B:\conv\{x,x+\lambda y\}\subset B\}
\end{equation}
(it is well-defined as $B$ is sequentially compact). From this definition one observers that in equation (\ref{eq0}) we must have 

\begin{equation}
L\left(\intersection_{m\in\nats}{\gamma_m(S_0)},x\right) \leq\limsup_{m\in\nats}{\lambda_m}L(S_0,x)=0
\end{equation}

which implies the set being singleton.

Thus we may assume $\limsup_{m\in\nats}{\lambda_m}>0$. This implies (by previous compactness argument) there exists a subsequence $((\lambda_{m_j},x_{m_j}))_{j\in\nats}$ converging to some pair $(\lambda^*,x^*)$ with $\lambda^*>0$. For any point $s\in S$ we now have $\gamma_{m_i}(s)\in\gamma_{m_j}(S_{m_j})$ for all $i,j\in\nats$ with $i\geq j$ which implies by sequential compactness for the limit homothety $\gamma^*$ of $\gamma_{m_j}$ that $\gamma^*(s)\in \intersection_{j\in\nats}{\gamma_{m_j}(S_{m_j})}=\alpha(S)\intersect\beta(S)$ (here we use the property of the $\gamma_m(S_m)$ being a chain with respect to inclusion). Thus $\gamma^*$ satisfies $\gamma^*(S)\subset\alpha(S)\intersect\beta(S)$. 

It is now clear that we may apply the exact same argument to the operators $\gamma^{-1}_{m_j}$ converging to $\gamma^{*-1}$. From symmetric argument (interchanging $S_m$ and $\gamma_m(S_m)$ as well as $\gamma_m$ and $\gamma^ {-1}_m$, respectively) we see that $\gamma^*$ is a bijection and

\begin{equation}
\alpha(S)\intersect\beta(S) = \gamma^*(S)\text{.}
\end{equation}
Thus we are done.

\textbf{\emph{For polytopes:}} For finite-dimensional compact polytopes the proof is slightly different. Consider such a chain of polytopes $(P_m)_{m\in\nats}$ and define $P:=\intersection_{m\in\nats}{P_m}$. Then let $\sup_{m\in\nats}|\partial_0{P_m}|=:k$ ($<\infty$ by condition). Consider the compact\footnote{This is due to Tychonoff's telling us that any product of compact spaces is compact. Indeed, in finite-dimensional case this is obvious. In infinite-dimensional case it is equivalent to the axiom of choice.} space $[0,1]^k\times P^k_0$ equipped with the product topology. In this space the map $\eta:[0,1]^k\times P^k_0\to P_0$ which assigns the linear combination to a given pair $(\mu,x)\mapsto \sum_{i=1}^k{\mu_i x_i}$ is $L$-continuous (where $\mu=(\mu_1,\ldots,\mu_k)\in [0,1]^k$ and $x=(x_1,\ldots,x_k)\in P^k_0$). This can be seen from the following equation (we may calculate with the product metric which arises from the norm in the spanned vectorspace at this point as the product topology is 
precisely the one induced by any norm in finite-dimensional space)
\begin{align}
\norm{\eta(\mu,x)-\eta(\nu,y)}_{\lin{P_0}} & =\norm{\sum_{i=1}^k{\mu_i x_i} -\sum_{i=1}^k{\nu_i y_i}}_{\lin{P_0}}\\
& \leq \sum_{i=1}^k{|\mu_i-\nu_i| \norm{x_i}_{\lin{P_0}}}+\sum_{i=1}^k{|\nu_i| \norm{x_i-y_i}_{\lin{P_0}}}\\
& \leq \sup_{p\in P_0}{\norm{p}_{\lin{P_0}}}\sum_{i=1}^k{|\mu_i-\nu_i|} +\sum_{i=1}^k{\norm{x_i-y_i}_{\lin{P_0}}}\\
& \leq \max\left\{1,\sup_{p\in P_0}{\norm{p}_{\lin{P_0}}}\right\}\norm{(\mu,x)-(\nu,y)}_{\lin{[0,1]^k\times P^k_0}}
\end{align}

Now consider a point $p\in P$. Then there exists a sequence $\left((\mu_m,x_m)\right)_{m\in\nats}$ of parameters of convex combinations of $p$ such that the entries of $x_m$ contain the vertices $P_m$ (this holds as $p\in P\subset P_m = \conv{\partial_0{P_m}}$). Obviously, this sequence $\left((\mu_m,x_m)\right)_{m\in\nats}$ has an accumulation point $(\mu^*,x^*)$ which is also a convex combination as its underlying space is compact. Due to the shown continuity of the evaluation function $\eta$ we have $\eta(\mu^*,x^*)=p$. 

We thus showed that if $X^*$ denotes the set of entries of $x^*$ then $\conv X^* \supset P$.

The other direction is much faster. Obviously, the sequence $(x_m)_{m\in\nats}$ is eventually in $P_{m_0}^k$ for $m_0\in\nats$. Thus $x^*\in P^k$ (as $P^k = \intersection_{m\in\nats}{P_m^k}$). Thus $X^*\subset P$ and we obtain the desired equation $\conv X^* = P$ (where $X^*$ is a finite set). It is clear that the polytope $P$ has at most $\limsup_{m\in\nats}|\partial_0{P}|$ vertices. 

\end{proof}

Another fact which shall be mentioned at this point (without proof) is the following

\begin{lemma}
A face of a finite-dimensional compact simplex is itself a simplex (the same for polytopes).
\end{lemma}

\subsection{Irreducible and primitive operators}

We give a definition on primitive operators and generalize the notion of a positive irreducible matrices or operator for our purposes

\begin{definition}[primitive operator]\label{def5}
Let $A:V\to V$ be a linear operator on a finite-dimensional ordered vectorspace $V$ admitting a closed cone. Then $A$ is called \emph{primitive} if $A\geq 0$ and there is some $m\in\nats$ such that $A^m[V^+\setminus\{0\}]\subset\int V^+$. 
\end{definition}

\begin{definition}[ideal irreducible operator]\label{def6}\footnote{This definition is mostly given in the form that the operator $A$ is ideal irreducible if $A[I]\not\subset I$ for any non-trivial ideal $I\neq\{0\}$. However, our equivalent definition emphasizes the geometric structure.}
Let $V$ be an ordered vector space of finite dimension $n$ with $V^+$ admitting polyhedral (compact) base $B$ and let $A:V\to V$ be a linear operator.
$A$ is called \emph{ideal irreducible}\footnote{this definition is newly introduced} if $A[F]\not\subset F$ for any $k$-dimensional face $F$ of $V^+$ ($k\leq n-1$).
\end{definition}

\begin{remark}\label{rem4}
Normally, definition \ref{def6} is given for $V=\reals^n$ and the standard cone $\reals^n_+$.
An \emph{irreducible matrix} is a matrix $A$ for which there is no permutation matrix $P$ such that $P^{-1}AP=\diag(A_1,A_2)+B$ where $A_1,A_2$ are non-empty square matrices and $B$ is an upper triangular matrix. The notion of these definitions arise from graph theory as we will see in the last sections.
\end{remark}

\section{The theorem}

The theorem of Perron-Frobenius can be stated for both types of operators (where we introduced the definition \ref{def6} because it is the most general definition which preserves the most facts of the theorem as originally stated for positive matrices).

\begin{theorem}[Perron-Frobenius theorem for primitive operators]\label{theo1}
Let $V$ be an $n$-dimensional Archimedean vector space admitting compact base $B$ with generating cone\footnote{This condition is needed, because otherwise we cannot state maximility in modulus.}  and $T$ be some primitive operator. Then the following facts hold
\begin{enumerate}
\item There is a unique (up to positive multiples) positive eigenvector $v\in V^+$ which lies in the interior of $V^+$.
\item The corresponding eigenvalue $\lambda$ of $v$ is simple\footnote{algebraic multiplicity one} and the strict maximum in modulus among all complex eigenvalues of $T$, thus $\lambda=\varrho(T)$.
\end{enumerate}
\end{theorem}

\begin{theorem}[Perron-Frobenius theorem for ideal irreducible operators]\label{theo2}
Let $V$ be an $n$-dimensional ordered vector space with $V^+$ admitting polyhedral generating base $B$ with $q$ vertices and $T$ a linear positive, ideal irreducible operator. Then the following facts hold
\begin{enumerate}
\item There is a unique (up to positive multiples) positive eigenvector $v\in V^+$ which lies in the interior of $V^+$.
\item The corresponding eigenvalue $\lambda$ of $v$ is simple and the greatest eigenvalue in modulus\footnote{it is not necessarily the strict maximum in modulus} among all complex eigenvalues of $T$, that is $\lambda=\varrho(T)$.
\item For any other complex eigenvalue $\lambda'$ of maximal modulus we have $\lambda'=\varrho(T)\zeta$ where $\zeta$ is some root of unity with $\ord{\zeta}\leq q$. 
In the case $B$ is a simplex\footnote{That is $V$ is a lattice.} than $\sigma(T)\intersect\{c\in\complex:|c|=\lambda\}=\langle \zeta_{q'}\rangle \lambda$ for unique $q'$ with $q'\leq q$.\footnote{here $\langle \zeta_{q'}\rangle$ means the multiplicative group generated by $\zeta_{q'}=\e^{2\pi\mi/q'}$} Moreover, for all these $\lambda'$ the algebraic and geometric multiplicity coincide.
\end{enumerate}
\end{theorem}

\begin{example}
Consider the operator (which is ideal irreducible)
\begin{equation}
A=\small\begin{pmatrix}
1 & 0 & 0 \\ 0 & \cos(2\pi/q) & -\sin(2\pi/q) \\ 0 & \sin(2\pi/q) & \cos(2\pi/q)
\end{pmatrix}
\end{equation} 
acting on $\reals^3$ with a cone 
\begin{equation} 
K:=\conv\left(\left\{
\small\begin{pmatrix}
1\\
\cos(2\pi j/q)\\
\sin(2\pi j/q)
\end{pmatrix}
:j=0,\ldots,q-1\right\}\cup\{0\}\right)\text{.}
\end{equation}
We see that the eigenvalues of the complexified operator are $1,\e^{\pm 2\pi\mi/q}$. The positive eigenvector $\tiny\begin{pmatrix}
1 \\
0 \\
0 
\end{pmatrix}$ is an internal point of $V^+$.
\end{example}

\section{Collecting some ideas}

\subsection{Proof of existence.}

The main goal of this section is to establish Perron-Frobenius using the argument presented in lemma \ref{lem0}.
The following lemma would directly follow from Brouwers fixed point theorem.

\begin{lemma}\label{lem1}
Let $V$ be an $n$-dimensional vector space ($n\in\nats$) with a cone $V^+$ admitting polyhedral base $B$ (i.e. $B$ is a compact polytope). Let $T$ be a strictly positive operator (i.e. $\forall v\in V^+\setminus\{0\}:T(v)>0$). Define the equivalence relation $\sim:=\{(u,v)\in V\setminus\{0\}:\exists \lambda>0:u=\lambda v\}$.  Then the map $T\restrict_{V^+\setminus\{0\}}/\sim=:\chi$ (that is $\chi([x]_{\sim})=[Tx]_{\sim}$) has a fixed point (i.e. $T$ has a eigenvector in $V^+$).  
\end{lemma}

\begin{remark}\label{rem5}
If $T$ is regular the space $V\setminus\{0\}/\sim$ is isomorphic to $S^{n-1}$. For $n$ even any continuous map from $S^{n-1}$ to $S^{n-1}$ admits a fixed point (corollary of Lefschetz fixed point theorem). 
\end{remark}

\begin{remark}
Since $T$ was defined to be strictly positive, it is an easy matter to verify that $\chi$ is well-defined. However, the statement about the eigenvector without this restriction since one may otherwise $T$ has a positive eigenvector corresponding to the eigenvalue $0$.
\end{remark}

\begin{remark}\label{rem6}
In our further considerations $B$ is a generating polyhedral base, specialized when needed to a simplicial (lattice) base.
\end{remark}

During the proof of this lemma we also want to establish some kind of characterization.
The (short) proof of the lemma will be a minor part of that and will be given later. At first we impose some more general considerations.

\subsection{The map $\rho$}

\begin{definition}
Introduce the map $\rho:B\to B$ by

\begin{equation}
\rho=\phi^{-1}\chi\phi
\end{equation}

where $\phi$ is the \emph{natural projection map} $\phi:B\to V^+/\sim$ with $\phi(b):=[b]_\sim$ for $b\in B$.
\end{definition}

\begin{remark}
By definition of $\rho$, $T$ admits a positive eigenvector if and only if $\rho$ has a fixed point.
\end{remark}

Now we are ready for the following important

\begin{lemma}\label{lem2}
The map $\rho$ has the following three properties
\begin{enumerate}
\item $\rho$ is continuous.
\item For any set $A\subset B$ we have $\conv\rho[A]=\rho[\conv{A}]$.\footnote{The analogue identity $\aff\rho[A]=\rho[\aff{A}]$ does not hold in general.}
\item If $P\subset B$ is a compact convex polytope such that $\dim\aff{P} = \dim\aff\rho[P]$ then $\rho$ preserves $\deg$, i.e. $\deg_P{p} = \deg_{\rho[P]}\rho(p)$ (for $p\in P$).
\end{enumerate}
\end{lemma}

\begin{proof}
\begin{enumerate}
\item Clearly, as $V$ is a finite-dimensional Euclidean space, we have $T$ being continuous. Now taking the quotient topology and conjugating with the natural projection map $\phi$ does not destroy anything. 
\item Let us pick a set $A\subset B$. By definition any point $a\in\conv{A}$ admits a representation $a=\sum_{i=1}^l{\alpha_i a_i}$ with $a_i\in A, \alpha_i>0$ for all $i=1,\ldots,l$ and $\sum_{i=1}^l{\alpha_i}=1$. Let $f$ be a strictly positive linear functional such that $f^{-1}\{1\}\intersect V^+=B$. We then have by 
\begin{equation}
\rho(a)=\frac{\sum_{i=1}^l{\alpha_i T(a_i)}}{f\left(\sum_{i=1}^l{\alpha_i T(a_i)}\right)}=\frac{\sum_{i=1}^l{\alpha_i\lambda_i\rho(a_i)}}{f\left(\sum_{i=1}^l{\alpha_i T(a_i)}\right)}\label{eq1}
\end{equation}
that $\rho(a)\in\conv\rho[A]$ as we may obtain a convex combination of $\rho(a)$ in $\rho(a_i)$ ($i=1,\ldots,l$) by considering the fact that there exist (unique) $\lambda_i>0$ such that $\lambda_i\rho(a_i)=T(a_i)$ ($i=1,\ldots,l$). Thus $\conv\rho[A]\supset\rho[\conv A]$.

For the other inclusion consider $\tilde{a}=\sum_{i=1}^l{\alpha_i \rho(a_i)}\in\conv\rho[A]$ for $a_i\in A, \alpha_i>0$ for all $i=1,\ldots,l$ and $\sum_{i=1}^l{\alpha_i}=1$. By taking the $\lambda_i$ ($i=1,\ldots,n$) as in the previous case, we obtain that
\begin{equation}
\tilde{a}=\sum_{i=1}^l{\frac{\alpha_i}{\lambda_i} T(a_i)}=T\left(\sum_{i=1}^l{\frac{\alpha_i}{\lambda_i} a_i}\right)=\rho\left(\frac{\sum_{i=1}^l{\frac{\alpha_i}{\lambda_i} a_i}}{f\left(\sum_{i=1}^l{\frac{\alpha_i}{\lambda_i} a_i}\right)}\right)\text{.}\label{eq2}
\end{equation}
Thus we obtain the desired equality $\conv\rho[A]=\rho[\conv A]$.
\item We will show now: a subset $C\subset P$ is linearly (affinely) independent if and only if $\rho[C]$ is linearly independent (linearly and affinely independence are equivalent for such $C$ as $0\not\in P$).

\textbf{$\impliedby$:} So let us suppose $C$ is linearly dependent, i.e. for some numbering of the elements of $C$ by $c_1,\ldots,c_p,\overline{c}_1,\ldots,\overline{c}_q$ we have $\delta_1,\ldots,\delta_p,\gamma_1,\ldots,\gamma_q\geq 0$ with $\sum_{i=1}^p{\delta_i}=\sum_{i=1}^q{\gamma_i}=1$ (this condition is no restriction as $f(0)=0$ and from $f(c_i)=f(\overline{c}_j)=1$ for all $i,j$) such that 
\begin{equation}
\sum_{i=1}^p{\delta_i c_i}=\sum_{i=1}^q{\gamma_i \overline{c}_i}\text{.}\label{eq3}
\end{equation}

Plugging the $c_i$ and $\overline{c}_i$ in the role of the $a_i$ in equation (\ref{eq1}) one obtains that $\rho[C]$ is affinely dependent.

\textbf{$\implies$:} For the other direction we will need the dimension condition. 

Let $k:=\dim\aff{P}$.
Suppose there is a set $C$ of linearly independent points such that $\rho[C]$ is linearly dependent. Clearly, $|C|\leq k+1$ as $C\subset P$ and $\dim\aff P=k$. We may extend $C$ to a $(k+1)$-element set $\tilde{C}$ such that $\rho[\tilde{C}]$ is still linearly dependent. If we now pick some point $p\in P$ we get a unique representation of it as $p = \sum_{i=1}^{k+1}{\epsilon_i \tilde{c}_i}$ (with $\sum_{i=1}^{k+1}{\epsilon_i}=1$). Now, shifting the negative $\epsilon_i$ to the side of $p$ and applying equation \ref{eq1} we get that $\rho(p)\in\aff\rho[\tilde{C}]$ a contradiction to the condition $\dim\aff\rho[P]=k$ (as $p\in P$ was arbitrary).

As we now have shown that affinely independence and lying in the interior of a convex set are preserved (to see this consider equations \ref{eq1} and \ref{eq2}), we may deduce the desired result (as the definition of $\deg$ only works with these two).
\end{enumerate}
\end{proof}

\begin{remark}
In the case where $T$ does not preserve the dimension of $P$, $\deg$ may increase and decrease. This can be seen from a tetrahedron which degrades to a triangle (i.e. the degree of one vertex increases) and a tetrahedron which degrades to a line (where two antipodal edges degrade to points - here the degree of the internal points of theses edges decreases).
\end{remark}

\subsection{The $\rho$-invariant polytope (or simplex) $\tilde{B}$}

The first thing we do is to simplify the problem a bit by the following consideration. Define

\begin{equation} 
\tilde{B}:=\intersection_{m\in\nats}{\rho^m[B]}\text{.}
\end{equation} 

Then from lemma \ref{lem0} we know that $\tilde{B}$ is a compact polytope (or simplex if $B$ was one, respectively).
Note that if $\rho$ has a fixed point in $B$ then it has to be in $\tilde{B}$.

\paragraph{The degraded case.} If $\tilde{B}$ has lower dimension than $B$ we may consider the 'projected' problem in the ordered subspace $\lin\tilde{B}\subset\intersection_{i\in\nats}\img{T^i}$ equipped with the order induced by the cone $V^+\intersect\lin\tilde{B}$ on which $T$ acts as a bijection. 
Maybe, we shall mention at this point that the 'initial' case where $V$ is one-dimensional (if $\dim V=0$ we had no strictly positive operator and no base $B$) is trivial. 

\subsection{The case $T$ being regular.}
Thus let us assume that $\dim\lin\tilde{B}=n\geq 1$ (actually this implies $T$ being regular). 
Then it is immediately clear that $\rho[\partial_0{\tilde{B}}]=\partial_0{\tilde{B}}$ by considering the fact that $\tilde{B}$ is fixed under $\rho$ and $\rho$ preserves $\deg$ as $T$ is regular (cf. lemma \ref{lem2} part 3). 
Thus $\rho$ acts on the set of vertices of $\tilde{B}$ as a finite permutation which has some representation of the form $\rho\restrict_{\partial_0{\tilde{B}}}=c_1\compose\cdots\compose c_l$ (where $c_i$ for $i=1,\ldots,l$ are disjoint cycles). 
Let us denote the vertices of $\tilde{B}$ by $v_1,\ldots,v_m$ and define $\tau$ implicitly by $\rho\restrict_{\partial_0{\tilde{B}}}(v_i)=v_{\tau(i)}$ for $i=1,\ldots,m$. 

From the following we will see that the number of fixed points of $\rho$ is at least $l\geq 1$ and thus establish lemma \ref{lem1}.

\paragraph{The case $\tau$ being itself a cycle.}

Let us now consider the case that $\rho\restrict_{\partial_0{\tilde{B}}}$ is itself a cycle (which can be trivial). 
We then have some positive $\lambda_1,\ldots,\lambda_m$ such that $T(v_i)=\lambda_iv_{\tau(i)}$ for ($i=1,\ldots,m$). 
It is now an easy matter to solve the arising equation (for an eigenvector of $T$ in $\tilde{B}$)

\begin{equation}
T\left(\sum_{i=1}^m\mu_iv_i\right)=\sum_{i=1}^m\mu_i\lambda_iv_{\tau(i)}=\lambda \sum_{i=1}^m\mu_iv_i
\end{equation}

from which one obtains $\frac{\mu_i}{\mu_{\tau(i)}}=\frac{\lambda}{\lambda_i}$ ($i=1,\ldots,m$), which obviously has the solution $\lambda = \sqrt[m]{\prod_{i=1}^m{\lambda_i}}$ 
and $\mu_{\tau^k(1)}=\frac{1}{\lambda^k}\prod_{i=0}^{k-1}{\lambda_{\tau^i(1)}}\mu_1$ (all solutions $(\mu_1,\ldots,\mu_m)$ are a multiple of each other). 
Thus $\rho$ in this special case has a unique fixed point in the interior of $\tilde{B}$ (and thus in the interior of $V^+$).

\begin{remark}\label{rem7}
In this case the operator $T^m$ coincides with the homothety $v\mapsto \lambda^m v$. From this it follows that any eigenvalue must be of the form $\lambda\zeta$ where $\zeta$ is an $m$-th root of unity.
If $m=n$ we may explicitly calculate the eigenvalues of $T$ by taking $\partial_0{\tilde{B}}$ as the representation base and setting w.l.o.g. the map $\tau$ to $i\mapsto i+1$ (mod $n$):
\begin{equation}
\mathcal{M}_{\partial_0{\tilde{B}}}^{\partial_0{\tilde{B}}}(T)=\begin{pmatrix}
0 & \lambda_1 & 0 & \cdots & 0\\
\vdots & \ddots & \ddots & \ddots & \cdots\\
\vdots &  & \ddots & \ddots & 0\\
0 & \cdots & \cdots & 0 & \lambda_{n-1}\\
\lambda_n & 0 & \cdots & \cdots & 0
\end{pmatrix}\text{.}
\end{equation}
This yields $\chi_T(\mu)=(-1)^n(\mu^n-\lambda^n)$ from which one deduces that the eigenvalues of $T$ are exactly those of the form $\lambda\zeta$ where $\zeta$ is any $n$-th root of unity.
\end{remark}

\paragraph{The general case.}

Let us now reconsider the case where $\tau=c_1\compose\cdots\compose c_l$. Then for each $c_i$ we may apply the observations from the last paragraph in the corresponding subspace. We thus may deduce that there exists a unique eigenvector $v$ in $V^+$ (up to positive multiples) if and only if $\tau$ is a single cycle.

%\subsection{A fact about $\tilde{B}$}
%
%For the proof of theorem \ref{theo1} we will need the following
%
%\begin{lemma}\label{lem4} Let $\tilde{B}$ not be a singleton and $v\in\partial_0{\tilde{B}}$ be a vertex such that $v$ is an eigenvector of $T^{\ord\tau}$ of the smallest (positive) eigenvalue $\lambda_{\min}$. Then $v$ is a vertex of $B$. 
%
%\end{lemma}
%
%\begin{proof}
%The proof is elementary computation. Let (as previously) $v_1,\ldots,v_m$ be the vertices of $\tilde{B}$ and $\tilde{\lambda}_1,\ldots,\tilde{\lambda}_m$ ($0<\tilde{\lambda}_i\leq\tilde{\lambda}_j$ for $i<j$) be the corresponding eigenvalues with respect to $T^{\ord\tau}$. Furthermore, w.l.o.g. let $v=v_1$. 
%
%Assume $v$ is not a vertex of $B$. Then there are some $\alpha_k\in\reals$ ($k=2,\ldots,m$) such that $v_1+\sum_{k=2}^m{\alpha_k v_k},v_1-\sum_{k=2}^m{\alpha_k v_k}\in B$ (that is $v=v_1$ lies in the interior of some nontrivial convex hull of two points which lies in $B$). Now we have two cases
%
%\begin{enumerate}
%\item If $\tilde{\lambda}_i=\tilde{\lambda}_j$ (for all $i,j=1,\ldots,m$) then the above vectors are themselves eigenvectors of $T^{\ord\tau}$ and thus lie in $\tilde{B}$ which leads to the contradiction that $v$ is not a vertex of $\tilde{B}$.
%\item If $\tilde{\lambda}_m=\cdots=\tilde{\lambda}_{j+1}>\tilde{\lambda}_j$ (for some $j\geq 1$) then applying $T$ repeatedly to the above two vectors leads to the fact that $\sum_{i=j+1}^m{\alpha_iv_i},-\sum_{i=j+1}^m{\alpha_iv_i}\in B$ which contradicts the condition that $B$ is a cone (this argument is due to the compactness of $B$ and due to the fact that the largest eigenvalue in modulus dominates the behavior of $T^l(v_1+\sum_{k=2}^m{\alpha_k v_k})$ and $T^l(v_1-\sum_{k=2}^m{\alpha_k v_k})$ ($l\rightarrow\infty$), respectively).
%\end{enumerate}
%
%Thus $v$ must be a vertex of $B$.
%\end{proof}

\section{Some final lemmas}

We need to recall the following elementary results from linear algebra

\begin{lemma}[characterization of uniform boundedness of the powers of an operator]\label{lem4}
Let $A:V\to V$ some linear operator where $\dim{V}=n<\infty$. Then the powers $A^m$ ($m\in\nats$) are uniformly bounded if and only if all complex eigenvalues $\mu$ of $A$ satisfy $\module{\mu}\leq 1$ and if $\module{\mu}=1$ then the algebraic multiplicity and the geometric multiplicity coincide.
\end{lemma}

\begin{proof}
By an appropriate choice of the (complex) base we may assume $A$ can be represented as some Jordan matrix

\begin{equation}
\mathcal{M}_B^B(A)=\diag(J_1,\ldots,J_l)
\end{equation}

such that $J_i$ ($i=1,\ldots,l$) are Jordan blocks. We than have

\begin{equation}
\mathcal{M}_B^B(A^k)=\diag(J^k_1,\ldots,J^k_l)
\end{equation}

for $k\in\nats$ from which we see that no complex eigenvalue can have larger modulus than one (otherwise $A^m$ would not be bounded). Moreover if some eigenvalue $\mu$ with $\module{\mu}=1$ has a non-trivial Jordan block $J$ then by

\begin{equation}
J^k=(\mu I + N)^k =\sum_{i=0}^k{{k \choose i}\mu^i N^{k-i}}
\end{equation}

we see that $A^m$ would also not be uniformly bounded (here $I$ denotes the identity matrix and $N$ is the nilpotent matrix with ones on the secondary diagonal). Thus we are done (the other direction is trivial).
\end{proof}

\begin{lemma}[internal eigenvector]\label{lem5}
Suppose $T$ has an internal point $v$ of $V^+$ as an eigenvector. Then the corresponding eigenvalue $\lambda>0$ is the largest eigenvalue of $T$ in modulus among all complex eigenvalues of $T$. Moreover, if $\lambda'$ is a complex eigenvalue of $T$ such that $\module{\lambda'}=\module{\lambda}$ then the algebraic and geometric multiplicity of $\lambda'$ coincide.
\end{lemma}

\begin{proof}
Consider the strictly positive operator $S:=T/\lambda$. We prove that the powers of $S$ are uniformly bounded and apply the previous lemma.
Let $u=u_1-u_2\in V$ where $u_1,u_2\in V^+$ (this is possible because $V^+$ is generating). From this representation it is clear that it suffices to prove that the powers of $S$ are uniformly bounded on $V^+$. But this follows from the fact that $S[0,v]\subset [0,S(v)]=[0,v]$ (as $S$ is positive operator) because $[0,v]$ is also generating (in the sense that for any $u\in V$ there are $u_1,u_2\in[0,v],\mu_1,\mu_2\geq 0$ such that $v=\mu_1u_1-\mu_2u_2$) as $v$ is an internal point of $V$ (and thus an order unit). 
\end{proof}

\begin{remark}
However, this lemma implies that any eigenvector in the interior of $V^+$ must have corresponding eigenvalue $\lambda=\varrho(A)$.
\end{remark}

\begin{lemma}[symmetry property of the spectrum $\sigma(T)$]\label{lem6}
Let $V$ have polyhedral cone. Suppose $T$ has an internal point $v$ of $V^+$ as eigenvector (corresponding eigenvalue $\lambda$). Then any other eigenvalue $\lambda'$ of maximal modulus (i.e. by previous lemma $\module{\lambda'}=\module{\lambda}$) is of the form $\lambda\zeta$ where $\zeta$ is a root of unity of order smaller than the number $q$ of vertices of the cone.  
\end{lemma}

\begin{proof}
We already proved a stronger fact for simplicial (lattice cones) which is stated in remark \ref{rem7}. 

By the previous lemma we know that all Jordan chains of such $\lambda'$ behave trivial. W.l.o.g. we may assume that $\lambda=1$ (otherwise let us define $S:=T/\lambda$ as previously and do the argument on $S$). Then, for some $\varepsilon>0$ we obviously find a number $m\in\nats$ such that $\module{\lambda'^m-1}<\varepsilon$ holds for all $\lambda'$ of modulus one. From this we see that there is some $m\in\nats$ such that the pair of vectors $(v_1,v_2)=(\Re{\tilde{v}},\Im{\tilde{v}})$ (where $\tilde{v},\overline{\tilde{v}}$ are the associated complex eigenvectors of $\lambda',\overline{\lambda'}$) is nearly left invariant under $S^m$, i.e. for some $\varepsilon>0$ there is $m\in\nats$ such that $\norm{v_i-S^m(v_i)}<\varepsilon$ ($i=1,2$). 

Let 
\begin{equation}
U:=\lin(\{v_1,v_2:v_1\pm\mi v_2\text{ associated complex eigenvectors of } \lambda',\overline{\lambda'} \text{ with } |\lambda'|=1\}\unify \Eig_\lambda(T))\text{.}
\end{equation}

 We then have $\tilde{B}\supset U\intersect B$ from the last property. But from this (and from the fact that $v$ is an internal point of $V^+$) we get that $S^{\ord\tau}$ acts as the identity on $U$ and thus all $\lambda'$ must be roots of unity of order smaller than the number of vertices $q$ of $V^+$ (from remark \ref{rem7}).
\end{proof}

\begin{remark}
This does not hold for non polyhedral cones. To see this consider the ordered vector space $\reals^3$ with the cone $K:=\{x\in\reals^3:x_1\geq \sqrt{x_2^2+x_3^2}\}$ and the operator given by the matrix
\begin{equation}
M:=\begin{pmatrix}
1 & 0 & 0 \\
0 & a & b \\
0 & -b & a 
\end{pmatrix}
\end{equation}
where $a+b\i=e^{2\pi\mi r}$ for some $r\in\reals\setminus\rats$.
\end{remark}

\section{The proof}

We have collected all facts for the proof of the two theorems. Let us reassemble them.

The first thing which is of importance is that in both cases there cannot be an eigenvector on the boundary of the cone (simply by the properties of $T$ in the theorems). Thus we know by lemma \ref{lem1} that there exists a positive eigenvector in the interior of $V^+$. 

For non-polyhedral cone in theorem \ref{theo1} needs some easy approximation argument. Assume $B$ is the base of a non-polyhedral cone. We then observe that in the case where $T$ is primitive there is a number $m\in\nats$ such that for the induced map $\rho$ we have 
\begin{equation}
\dist(\rho^m[B],\partial_{n-2}{B})=:\varepsilon>0
\end{equation}
as the sets are compact\footnote{Where $\dist(A,B):=\inf_{a\in A,b\in B}{\norm{a-b}}$ for compact sets $A,B$.}. We then may build a polytope $P:=\conv{S}$ ($S\subset\partial_{n-2}{B}$ finite) such that any $(n-2)$-simplex in $\partial_{n-2}{P}$ has diameter at most $\varepsilon/2$ (this is not done explicitly here, but is based on precompactness arguments).

Then it is evident that $\dist(\rho^m[B],\partial_{n-2}{P})>\varepsilon/2$ using triangle inequality. To see this, consider a point $p\in\partial_{n-2}{P}$ which lies in the $(n-2)$-dimensional face $F$ of $P$. Moreover, let $b\in\rho^m[B]$ and $v\in\partial_0{B}$ be arbitrary. Then $\norm{p-n}\geq|\norm{p-v}-\norm{v-b}|>\varepsilon/2$ as $v\in \partial_{n-2}{B}$ and $\diam(F)<\varepsilon/2$. We thus see that $\dist(\rho^m[B],\partial_{n-2}{P})>\varepsilon/2$ is true implying $\dist(\rho^m[P],\partial_{n-2}{P})$ (as $\rho^m[P]\subset\rho^m[B]$). This implies that $T$ is primitive with respect to the order induced by the polyhedral cone generated by $P$. Thus we also happy and done in this case. 

\paragraph{Algebraic multiplicity one and unicity.} Now, let us prove that the algebraic multiplicity of the corresponding eigenvalue $\lambda$ is one. Suppose it would be larger than one. Then by lemma \ref{lem5} we woulds have $\dim\Eig_\lambda(T)>1$. But then $\Eig_\lambda(T)$ has non-empty intersection with the boundary of $B$. To see this take another eigenvector $v'\in\Eig_\lambda(T)$ which is linearly independent from $v$ the line $v+\reals v'$ must intersect with the boundary of $V^+$ otherwise $V$ would not be Archimedean and would thus not admit a base). We thus may also deduce that there is no other positive eigenvector as all internal eigenvectors $v'$ (with respect to $V^+$) must satisfy $Tv'=\lambda v'$.

\paragraph{Strict maximum.} Let us prove that $\lambda$ is the strict maximum in modulus in $\sigma(T)$ in the case $T$ is primitive (theorem \ref{theo1}). Otherwise, from lemma \ref{lem6} we know that there is a power of $T^m$ which has no other eigenvalues of modulus $\lambda^m$ (e.g. take $T^{q!}$). But this power is also primitive. Thus $\lambda^m$ must have algebraic multiplicity one. From this it follows that the number of eigenvalues of modulus $\lambda$ of $T$ is also one.

\paragraph{Symmetry properties of $\sigma(T)$.} We are left with the 3rd point of theorem \ref{theo2}. But this was already proven by lemma \ref{lem6} and remark \ref{rem7}.

\section{Connection to weighted adjacency matrices of graphs}

The theorem of Perron-Frobenius has many applications. One which is of significant importance is its usage in graph theory.

\begin{definition}[weighted adjacency matrix, induced graph]
Let $G=(V,E)$ a directed (or undirected) graph. Then a weight on the edges of $G$ is some map $\omega:E\to \reals^+$. The weighted adjacency matrix of $G$ is then a non-negative matrix $\reals^{n\times n}$ ($n=\module{V}$) which has as entries the values of $\omega$ and zeros where no edges exist (via some numbering of the vertices).
On the other hand, given any non-negative $A$ one defines the induced weighted graph (up to isomorphism) as the pair $(G,\omega)$ such that $A$ is a weighted adjacency matrix.
\end{definition}

We then have the following connections

\begin{itemize}
\item A non-negative matrix $A$ is periodic with period $p$ (that is $p=\gcd\{n\in\nats:\exists i\in[n]:(A^n)_{ii}>0\}$) if and only if for the induced graph $G$ we have $p=\gcd\{|C|:C \text{ cycle in } G\}$.
\item A non-negative matrix $A$ is irreducible (that is there is no permutation matrix $P$ and non-vanishing square matrices $A,B$ such that $P^{-1}AP=\begin{pmatrix} A & C \\ 0 & B\end{pmatrix}$\footnote{irreducible is defined for general matrices}) if and only if the induced graph $G$ is strongly connected (that is any two vertices are joined by a path - or topologically: the graph as topological (even metric) space has only one path component).
\item A non-negative matrix $A$ is primitive if and only if $A$ is aperiodic (that is $p=1$) and irreducible (the induced graph is aperiodic and strongly connected).
\item A graph $G$ is periodic with period $p$ if and only if there $p$ is the maximal natural number such that there exists some partition $\{V_1,\ldots,V_p\}=:\mathcal{P}$ of its vertices such  that any edge starting in $V_i$ ends in $V_{i+1}$ (mod $p$)\footnote{such partition is unique up to cyclic permutation}.
\end{itemize}

Note the following lemma on periodic matrices, which is very similar (but even stronger) to the spectral properties of ideal irreducible operators outlined in our proof (whereas these are satisfied in a more general situation as they do not only work for lattice cones). For lattice cones, the notion of an ideal irreducible and an irreducible matrix coincide. 

\begin{lemma}
Let $A$ be a periodic matrix with the representation 
\begin{equation}
A = 
\begin{pmatrix}
0 & A_1 & 0 & \cdots & 0\\
\vdots & \ddots & \ddots & \ddots & \cdots\\
\vdots &  & \ddots & \ddots & 0\\
0 & \cdots & \cdots & 0 & A_{p-1}\\
A_p & 0 & \cdots & \cdots & 0
\end{pmatrix}
\end{equation}
where $A_i\in\complex^{n_i\times n_{i+1}}$ ($i$ mod $p$) (where $n_i$ can be interpreted as the cardinalities of $V_i$ where $\{V_i:i=1,\ldots,p\}$ is the maximal periodic partition). Then the set of pairs of eigenvalues and corresponding eigenvectors $(\lambda,v)$ are invariant under the following actions (let $v^{\trans}=(v_1,\ldots,v_p)$ with $v_i\in\complex^{n_i}$):
\begin{equation}
\phi_{\zeta}(\lambda,v)=\left(\zeta\lambda,\begin{pmatrix}
v_1 \\ \zeta v_2 \\ \cdots \\ \zeta^{p-1} v_p
\end{pmatrix}
\right)\text{.}
\end{equation}
Here $\zeta\in\complex$ satisfies $\zeta^p=1$. Moreover, the Jordan submatrices corresponding to the eigenvalues $\zeta\lambda$ ($\zeta$ is $p$-th root of unity) all have the same structure.
\end{lemma}

\begin{proof}
As one directly verifies we have for $v\in V,\lambda\in\complex$ with $Av=\lambda v$ that

\begin{equation}
A\phi_{\zeta,2}(\lambda,v)=\zeta\lambda \phi_{\zeta,2}(\lambda,v)=\phi_{\zeta,1}(\lambda,v) \phi_{\zeta,2}(\lambda,v)
\end{equation}

where $\phi_{\zeta,i}$ denote the components of $\phi_\zeta$.

We obtain a similar fact if $u,v\in V$ are some consecutive vectors in a Jordan chain of $A$ of the eigenvalue $\lambda$ (that is $Au = \lambda u+v$):

\begin{equation}
A\phi_{\zeta,2}(\lambda,u)=\zeta(\lambda \phi_{\zeta,2}(\lambda,u) + \phi_{\zeta,2}(\lambda,v))
\end{equation}

From this one obtains that some Jordan chain $v_1,\ldots,v_k$ satisfying
$Av_1 = \lambda v_1$ and $Av_j = \lambda v_j + v_{j-1}$ (for $1<j\leq k$) is mapped to the chain $\tilde{v}_j = \zeta^{k-j}\phi_{\zeta,2}(\lambda,v_j)$ ($1\leq j \leq k$).
\end{proof}

\begin{remark}
Thus the spectrum $\sigma(A)$ is invariant under multiplication with $\zeta$ ($p$-th root of unity).
\end{remark}

\paragraph{A little exercise.} At this point, we want the reader to prove the following simple fact about graphs using Perron-Frobenius theorem.

\subparagraph{Task.} Let $G=(V,E)$ be a graph of period $p$ such that $V=\union_{i=1}^p{V_i}$ ($V_i$,$V_j$ disjoint for $i\neq j$) and for any $v\in V_i$ we have $d(v)=c_i$ ($i=1,\ldots,p$) where $d$ denotes the degree\footnote{that is the number of outgoing edges of $v$} of $v$. Prove that $\rho(A)=\sqrt[p]{c_1\cdots c_p}$ (where $A$ denotes the adjacency matrix of $G$)\footnote{One often calls this the spectral radius of the graph $G$}.

\section{An appetizer at the end: The spectral radius of the infinite $d$-regular tree}

At the end we want to prove a familiar fact in graph theory which makes a statement about the Perron-Frobenius eigenvalue of trees of uniformly bounded degree. 

\paragraph{The example.} It is a well known theorem due to Nilli that any $d$-regular connected graph $G$ of diameter $m$ has its biggest non-trivial eigenvalue greater than $2\sqrt{d-1}(1-1/m)$. We want to prove another well known fancy result.

\begin{lemma}
Let $G=(V,E)$ be a connected tree with $\max_{v\in V} d(v)\leq d$. Then for the adjacency matrix $M$ of $G$ we have $\varrho(M)\leq 2\sqrt{d-1}$. 
\end{lemma}

The proof of this result can be done elementary (or using extremal graph theory). 
We will give an elementary proof.

\begin{proof}
Let us extend $G$ to the infinite $d$-regular tree $G'$. Then the adjacency matrix $M'$ of $G'$ can be interpreted as an operator on $\boldsymbol{l}^2$. As the operators $M$ and $M'$ are self-adjoint it holds that $\norm{M}_2=\varrho(M)$ and $\norm{M'}_2=\varrho(M')$. 

Moreover, as $M$ is an initial square submatrix of $M'$ we have $\varrho(M)=\norm{M}_2\leq\norm{M'}_2=\varrho(M')$. Hence, we just need to show that $\rho(M')=2\sqrt{d-1}$.

To see this, consider the powers of $M$ and its behavior on the main diagonal.
Basically, $(M^m)_{ii}$ counts the paths of length $m$ starting and ending at vertex $i$. Let us denote this number by $\beta_m$ ($m\in\nats$).

We will calculate $\beta_m$ with some help of combinatorics.

Let us denote by $\beta'_m$ the number of paths of length $m$ from vertix $i$ to $i$ which do not return to $i$ in between.

Such paths can be parameterized by two sequences $(a_j)_{j=1}^l$ and $(b_j)_{j=1}^l$ satisfying $\sum_{j=1}^m{a_j}>\sum_{j=1}^m{b_j}$ for $m<l$ and $\sum_{j=1}^l{a_j}=\sum_{j=1}^l{b_j}$.

At first we walk from $i$ along some path of length $a_1$. Then we backtrack this path by $b_1$ steps. Again we start a new path of length $a_2$ (which does not backtrack) etc.

During each step in the sequence $(a_j)$ we may choose from $d-1$ possible directions (disregarding the beginning where we may choose from $d$ directions).

It is clear that for $m$ odd the number of circles starting and ending at $i$ is zero. So let $m=2k$ ($k\in\nats$).

We already constructed a (not very formal) bijection between the set of such sequences $(a_j)$ and $(b_j)$ times some set of the size $d(d-1)^{k-1}$ to the number of such paths $\beta'_{2k}$. Let us denote the former number by $C'_k$. It is easy to see that this number is exactly the same as the number $C_{k-1}$ of Dyck-paths of length $2(k-1)$ (that are paths  on the discrete positive axis of length $2(k-1)$ starting and ending at $0$).
But thus number is well known by its recurrence relation
\begin{equation}
C_{k+1}=\sum_{i=0}^k{C_iC_{k-i}}\text{.}
\end{equation}
This shows us that $C_k$ are actually the Catalan numbers, thus $C_n=\frac{1}{n+1}{2n \choose n}$.
From this we get for $k>0$

\begin{equation}
\beta'_{2k}=d(d-1)^{k-1}\frac{1}{k}{2(k-1) \choose k-1}
\end{equation}

Now we may compute $\beta_{2k}$ by the fact that any returning path from vertex $i$ to $i$ can be uniquely partitioned in the subpaths described by $\beta'_{2k}$.

\begin{equation}
\beta_{2k} = \sum_{k_1+\ldots+k_l=k, k_i>0}{\beta'_{2k_1}\ldots\beta'_{2k_l}}\label{eq4}
\end{equation}

(Here the sum is taken over all positive ordered partitions, i.e. permuted partitions are not considered equal.)
In this case we are lucky to know $z$-transformation.
We thus may compute the generating sequence (from the recursion formula of $C_n$) of $\beta'_{2k}$ which we will denote by $\alpha'_k$:

\begin{equation}
A'(z) = \sum_{k=0}^{\infty} \alpha'_kz^k =d\left(\frac{1-\sqrt{1-4(d-1)z}}{2(d-1)}\right)
\end{equation}

(this can easily be derived from the generating function of the Catalan numbers $C_n$ which is $C(z)={\small\frac{1-\sqrt{1-4z}}{2z}}$). Now from this last equation we get the convergence radius of $A'(z)$ in 0 which is $\varrho_{A'}=\frac{1}{4(d-1)}$. Simple argument shows that this is also the convergence radius of $A(z)=\sum_{k\in\nats}{a_kz^k}$ which is $\frac{1}{\limsup_{k\in\nats\sqrt[k]{a_k}}}$.

Let us now consider any initial square submatrix $\tilde{M}\in\reals^{n\times n}$ of $M'$. We then have $\lim_{n\to\infty}\varrho(\tilde{M})=\varrho(M')$ (which is true as we are in a separable Hilbert space). Moreover, the diagonal entries of $\tilde{M}^{2k}$ can at most be $\beta_{2k}$. Thus we get

\begin{equation}
\varrho(\tilde{M}) =\limsup_{k\to\infty} \sqrt[2k]{|\trace{\tilde{M}^{2k}}|}\leq \limsup_{k\to\infty}{\sqrt[2k]{\beta_{2k}}}=2\sqrt{d-1}
\end{equation}
This last we already have computed as $\varrho_{A} = \frac{1}{\limsup_{k\in\nats\sqrt[k]{a_k}}}$. This finishes the proof. 
\end{proof}

\paragraph{Remarks.} Some papers that inspired me are given in the references.

%\bibliographystyle{alphadin}
%\nocite{SeTanSchTy}
%\nocite{wikiPerFrob}
\nocite{*}
\bibliographystyle{acm} 
\bibliography{mybib}
%\printbibliography

\end{document}
