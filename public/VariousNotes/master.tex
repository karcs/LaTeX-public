\documentclass[8pt,a4paper]{article}
% commutative diagramms
%\usepackage{tikz-cd}
%\usepackage
%\usepackage[utf8]{inputenc} % utf8 char set as input (e.g. for " and umlauts)
%\usepackage[ngerman]{babel} % this is needed for umlauts
%\usepackage[T1]{fontenc}    % this is needed for correct output of umlauts in pdf
\usepackage{etoolbox}
\usepackage{makeidx}
\usepackage{xparse}
\usepackage{blindtext}
% \usepackage{nath}
%\usepackage{mathtools}
% for mathclap
% and other useful stuff
%\mathtoolsset{showonlyrefs}
\makeindex
%My personal maths package
%\bibliography{Bibliography.bib}

%%%%%%%%%%%%%%%%%%%%%%%%%%%%%%%%%%%%%%%%%%%%%%%%%%%%%%%%%%%%%%%%%%%%%%%%%%%%%%
%%%%% MATH PACKAGES %%%%%%%%%%%%%%%%%%%%%%%%%%%%%%%%%%%%%%%%%%%%%%%%%%%%%%%%%%
%%%%%%%%%%%%%%%%%%%%%%%%%%%%%%%%%%%%%%%%%%%%%%%%%%%%%%%%%%%%%%%%%%%%%%%%%%%%%%

% very good package
%\usepackage{mathtools}

%%% font stuff
\usepackage[T1]{fontenc}        % for capitals in section /paragraph etc.
\usepackage[utf8]{inputenc}     % use utf8 symbols in code

\usepackage{amssymb,amsmath,amsfonts} % amsthm not needed -- use my own envs
%\usepackage{mathtools}
%\mathtoolsset{showonlyrefs}
% further alternative math packages: unicode-math, abx-math
\usepackage{bm}
\usepackage{mathrsfs} % used for: fraktal math letters
\usepackage[bigsqcap]{stmaryrd} % used for: big square cap symbol
\usepackage{xargs}

% standard packages
%\usepackage{color} % for color

%%
%%index

\newcommand*{\keyword}[2][\empty]{\emph{#2}\ifx#1\empty\index{#2}\else\index{#1}\fi}
\newcommand*{\person}[1]{\textsc{#1}}

%%%%%%%%%%%%%%%%%%%%%%%%%%%%%%%%%%%%%%%%%%%%%%%%%%%%%%%%%%%%%%%%%%%%%%%%%%%%%%%%%%%%%%%%%%%%%%%%%%%%%%%%%%%%
%%%%% MATH ALPHABETS & SYMBOLS %%%%%%%%%%%%%%%%%%%%%%%%%%%%%%%%%%%%%%%%%%%%%%%%%%%%%%%%%%%%%%%%%%%%%%%%%%%%%
%%%%%%%%%%%%%%%%%%%%%%%%%%%%%%%%%%%%%%%%%%%%%%%%%%%%%%%%%%%%%%%%%%%%%%%%%%%%%%%%%%%%%%%%%%%%%%%%%%%%%%%%%%%%

% w: http://milde.users.sourceforge.net/LUCR/Math/math-font-selection.xhtml

% ===== Set quick commands for math letters ================================================================
% calagraphic letters (only upper case available; standard)
\newcommand{\cA}{\mathcal{A}}
\newcommand{\cB}{\mathcal{B}}
\newcommand{\cC}{\mathcal{C}}
\newcommand{\cD}{\mathcal{D}}
\newcommand{\cE}{\mathcal{E}}
\newcommand{\cF}{\mathcal{F}}
\newcommand{\cG}{\mathcal{G}}
\newcommand{\cH}{\mathcal{H}}
\newcommand{\cI}{\mathcal{I}}
\newcommand{\cJ}{\mathcal{J}}
\newcommand{\cK}{\mathcal{K}}
\newcommand{\cL}{\mathcal{L}}
\newcommand{\cM}{\mathcal{M}}
\newcommand{\cN}{\mathcal{N}}
\newcommand{\cO}{\mathcal{O}}
\newcommand{\cP}{\mathcal{P}}
\newcommand{\cQ}{\mathcal{Q}}
\newcommand{\cR}{\mathcal{R}}
\newcommand{\cS}{\mathcal{S}}
\newcommand{\cT}{\mathcal{T}}
\newcommand{\cU}{\mathcal{U}}
\newcommand{\cV}{\mathcal{V}}
\newcommand{\cW}{\mathcal{W}}
\newcommand{\cX}{\mathcal{X}}
\newcommand{\cY}{\mathcal{Y}}
\newcommand{\cZ}{\mathcal{Z}}

% bold math letters (standard)
\newcommand{\bfA}{\mathbf{A}}
\newcommand{\bfB}{\mathbf{B}}
\newcommand{\bfC}{\mathbf{C}}
\newcommand{\bfD}{\mathbf{D}}
\newcommand{\bfE}{\mathbf{E}}
\newcommand{\bfF}{\mathbf{F}}
\newcommand{\bfG}{\mathbf{G}}
\newcommand{\bfH}{\mathbf{H}}
\newcommand{\bfI}{\mathbf{I}}
\newcommand{\bfJ}{\mathbf{J}}
\newcommand{\bfK}{\mathbf{K}}
\newcommand{\bfL}{\mathbf{L}}
\newcommand{\bfM}{\mathbf{M}}
\newcommand{\bfN}{\mathbf{N}}
\newcommand{\bfO}{\mathbf{O}}
\newcommand{\bfP}{\mathbf{P}}
\newcommand{\bfQ}{\mathbf{Q}}
\newcommand{\bfR}{\mathbf{R}}
\newcommand{\bfS}{\mathbf{S}}
\newcommand{\bfT}{\mathbf{T}}
\newcommand{\bfU}{\mathbf{U}}
\newcommand{\bfV}{\mathbf{V}}
\newcommand{\bfW}{\mathbf{W}}
\newcommand{\bfX}{\mathbf{X}}
\newcommand{\bfY}{\mathbf{Y}}
\newcommand{\bfZ}{\mathbf{Z}}
\newcommand{\bfa}{\mathbf{a}}
\newcommand{\bfb}{\mathbf{b}}
\newcommand{\bfc}{\mathbf{c}}
\newcommand{\bfd}{\mathbf{d}}
\newcommand{\bfe}{\mathbf{e}}
\newcommand{\bff}{\mathbf{f}}
\newcommand{\bfg}{\mathbf{g}}
\newcommand{\bfh}{\mathbf{h}}
\newcommand{\bfi}{\mathbf{i}}
\newcommand{\bfj}{\mathbf{j}}
\newcommand{\bfk}{\mathbf{k}}
\newcommand{\bfl}{\mathbf{l}}
\newcommand{\bfm}{\mathbf{m}}
\newcommand{\bfn}{\mathbf{n}}
\newcommand{\bfo}{\mathbf{o}}
\newcommand{\bfp}{\mathbf{p}}
\newcommand{\bfq}{\mathbf{q}}
\newcommand{\bfr}{\mathbf{r}}
\newcommand{\bfs}{\mathbf{s}}
\newcommand{\bft}{\mathbf{t}}
\newcommand{\bfu}{\mathbf{u}}
\newcommand{\bfv}{\mathbf{v}}
\newcommand{\bfw}{\mathbf{w}}
\newcommand{\bfx}{\mathbf{x}}
\newcommand{\bfy}{\mathbf{y}}
\newcommand{\bfz}{\mathbf{z}}

% fractal math letters (standard)
\newcommand{\fkA}{\mathfrak{A}}
\newcommand{\fkB}{\mathfrak{B}}
\newcommand{\fkC}{\mathfrak{C}}
\newcommand{\fkD}{\mathfrak{D}}
\newcommand{\fkE}{\mathfrak{E}}
\newcommand{\fkF}{\mathfrak{F}}
\newcommand{\fkG}{\mathfrak{G}}
\newcommand{\fkH}{\mathfrak{H}}
\newcommand{\fkI}{\mathfrak{I}}
\newcommand{\fkJ}{\mathfrak{J}}
\newcommand{\fkK}{\mathfrak{K}}
\newcommand{\fkL}{\mathfrak{L}}
\newcommand{\fkM}{\mathfrak{M}}
\newcommand{\fkN}{\mathfrak{N}}
\newcommand{\fkO}{\mathfrak{O}}
\newcommand{\fkP}{\mathfrak{P}}
\newcommand{\fkQ}{\mathfrak{Q}}
\newcommand{\fkR}{\mathfrak{R}}
\newcommand{\fkS}{\mathfrak{S}}
\newcommand{\fkT}{\mathfrak{T}}
\newcommand{\fkU}{\mathfrak{U}}
\newcommand{\fkV}{\mathfrak{V}}
\newcommand{\fkW}{\mathfrak{W}}
\newcommand{\fkX}{\mathfrak{X}}
\newcommand{\fkY}{\mathfrak{Y}}
\newcommand{\fkZ}{\mathfrak{Z}}
\newcommand{\fka}{\mathfrak{a}}
\newcommand{\fkb}{\mathfrak{b}}
\newcommand{\fkc}{\mathfrak{c}}
\newcommand{\fkd}{\mathfrak{d}}
\newcommand{\fke}{\mathfrak{e}}
\newcommand{\fkf}{\mathfrak{f}}
\newcommand{\fkg}{\mathfrak{g}}
\newcommand{\fkh}{\mathfrak{h}}
\newcommand{\fki}{\mathfrak{i}}
\newcommand{\fkj}{\mathfrak{j}}
\newcommand{\fkk}{\mathfrak{k}}
\newcommand{\fkl}{\mathfrak{l}}
\newcommand{\fkm}{\mathfrak{m}}
\newcommand{\fkn}{\mathfrak{n}}
\newcommand{\fko}{\mathfrak{o}}
\newcommand{\fkp}{\mathfrak{p}}
\newcommand{\fkq}{\mathfrak{q}}
\newcommand{\fkr}{\mathfrak{r}}
\newcommand{\fks}{\mathfrak{s}}
\newcommand{\fkt}{\mathfrak{t}}
\newcommand{\fku}{\mathfrak{u}}
\newcommand{\fkv}{\mathfrak{v}}
\newcommand{\fkw}{\mathfrak{w}}
\newcommand{\fkx}{\mathfrak{x}}
\newcommand{\fky}{\mathfrak{y}}
\newcommand{\fkz}{\mathfrak{z}}

% script math symbols (only uppercase; package: mathrsfs)
\newcommand{\sA}{\mathscr{A}}
\newcommand{\sB}{\mathscr{B}}
\newcommand{\sC}{\mathscr{C}}
\newcommand{\sD}{\mathscr{D}}
\newcommand{\sE}{\mathscr{E}}
\newcommand{\sF}{\mathscr{F}}
\newcommand{\sG}{\mathscr{G}}
\newcommand{\sH}{\mathscr{H}}
\newcommand{\sI}{\mathscr{I}}
\newcommand{\sJ}{\mathscr{J}}
\newcommand{\sK}{\mathscr{K}}
\newcommand{\sL}{\mathscr{L}}
\newcommand{\sM}{\mathscr{M}}
\newcommand{\sN}{\mathscr{N}}
\newcommand{\sO}{\mathscr{O}}
\newcommand{\sP}{\mathscr{P}}
\newcommand{\sQ}{\mathscr{Q}}
\newcommand{\sR}{\mathscr{R}}
\newcommand{\sS}{\mathscr{S}}
\newcommand{\sT}{\mathscr{T}}
\newcommand{\sU}{\mathscr{U}}
\newcommand{\sV}{\mathscr{V}}
\newcommand{\sW}{\mathscr{W}}
\newcommand{\sX}{\mathscr{X}}
\newcommand{\sY}{\mathscr{Y}}
\newcommand{\sZ}{\mathscr{Z}}

%%%%%%%%%%%%%%%%%%%%%%%%%%%%%%%%%%%%%%%%%%%%%%%%%%%%%%%%%%%%%%%%%%%%%%%%%%%%%%%%%%%%%%
%%%% BOLD MATH IN BOLD TEXT ENVIRONMENT %%%%%%%%%%%%%%%%%%%%%%%%%%%%%%%%%%%%%%%%%%%%%%
%%%%%%%%%%%%%%%%%%%%%%%%%%%%%%%%%%%%%%%%%%%%%%%%%%%%%%%%%%%%%%%%%%%%%%%%%%%%%%%%%%%%%%

% for bold math in bold text (e.g. sections)
\makeatletter
\g@addto@macro\bfseries{\boldmath}
\makeatother

\def\brackets#1{\ifx#1\empty\else\left(#1\right)\fi}

%%%%%%%%%%%%%%%%%%%%%%%%%%%%%%%%%%%%%%%%%%%%%%%%%%%%%%%%%%%%%%%%%%%%%%%%%%%%%%%%%%%%%%
%%%%% CATEGORY THEORY %%%%%%%%%%%%%%%%%%%%%%%%%%%%%%%%%%%%%%%%%%%%%%%%%%%%%%%%%%%%%%%%
%%%%%%%%%%%%%%%%%%%%%%%%%%%%%%%%%%%%%%%%%%%%%%%%%%%%%%%%%%%%%%%%%%%%%%%%%%%%%%%%%%%%%%

% ===== Category theory concepts =====================================================

\newcommand{\Ob}{\mathop\mathrm{Ob}}
\newcommand{\Mor}{\mathop\mathrm{Mor}}

\newcommand{\ccoprod}{\bigsqcup}
\newcommand{\cprod}{\bigsqcap}
\newcommand{\cincl}{\mathop\mathrm{incl}}
\newcommand{\cproj}{\mathop\mathrm{pr}}


% ===== Define standard categories ===================================================

% Define sets
\newcommand{\Set}{\mathbf{Set}}
% Define set-builder operator (equivalent to gen for algebras)
\newcommand{\set}[1]{\left\{#1\right\}}
% define interval operator: o - open, c - closed
\newcommand{\intervalcc}[2]{\left[#1,#2\right]}
\newcommand{\intervalco}[2]{\left[#1,#2\right)}
\newcommand{\intervaloc}[2]{\left(#1,#2\right]}
\newcommand{\intervaloo}[2]{\left(#1,#2\right)}

\newcommand{\inter}{\mathop\mathrm{int}}
\newcommand{\face}{\mathop\mathrm{F}}
\newcommand{\Pol}{\mathop\mathrm{Pol}}
\newcommand{\Inv}{\mathop\mathrm{Inv}}
\def\struct#1{\gen{#1}}

\let\originaltimes\times%
\renewcommand{\times}{\mathbin{\sqcap}}
\newcommand\settimes{\originaltimes}
\newcommand{\setleq}{\subseteq}
\newcommand{\setgeq}{\supseteq}

\newcommand{\pderive}[2]{\frac{\partial{#1}}{\partial{#2}}}
\renewcommand{\div}{\mathop\mathrm{div}}

%% Diffgeo
\def\Ric{\mathop\mathrm{Ric}}
\def\ric{\mathop\mathrm{ric}}
\def\tr{\mathop\mathrm{tr}}

\def\cotimes{\mathbin{\sqcup}}

\def\@rightopen#1{\ifx#1]{\right]}\else{\interval@errmessage}\fi}
\def\@leftclosed[#1){\left[#1\right)}
\makeatother

% finite
\newcommand{\fin}{\mathrm{fin}}

% Define groups (optarg: properties such as -> abelian, noetherian (acc), artinian (dcc) etc.)
\newcommand{\Grp}[1][\empty]{\if\empty{#1}{\mathbf{Grp}}\else{\mathbf{Grp}_{#1}}}
\def\PGL{\mathrm{PGL}}
\def\PGammaL{\mathrm{P\Gamma L}}
\def\GL{\mathrm{GL}}
% Define rings
\newcommand{\rg}{\mathrm{rg}} %rank of a matrix
\newcommand{\Rg}[1][\empty]{\if\empty{#1}{\mathbf{Rg}}\else{\mathbf{Rg}_{#1}}}
\edef\units#1{#1^{\settimes}}
\def\dual#1{#1^{\ast}}

%% redefine the command \P to produce the projective functor in math mode
\let\parsymb\P%
\def\P{\ifmmode\mathrm{P}\else\parsymb\fi}
\renewcommand{\iff}{\ifmmode\equival\else{if and only if}\fi}
\newcommand{\quotring}{\mathop{\mathrm{Q}}}
\newcommand{\rad}{\mathrm{rad}}
% Standard rings
% integral domains
\newcommand{\ID}{\mathbf{ID}}
% unique factorization domains
\newcommand{\UFD}{\mathbf{UFD}}
% principal ideal domains
\newcommand{\PID}{\mathbf{PID}}

% Define modules over a group or ring
\newcommand{\Mod}[1]{\mathbf{Mod}_{#1}}
% Define vector space over a field
\renewcommand{\Vec}[1]{\mathbf{Vec}_{#1}}%

% when cases
\def\otherwise{\textrm{otherwise}}

% new concepts
\newcommand{\new}[1]{\emph{#1}}

\usepackage{xifthen,xstring}

% replace the bar command by overline when argument just one character (shorter and better)
%$\let\oldbar\bar
%\renewcommand{\bar}[1]{\StrLen{#1}[\length]\ifthenelse{\length > 1}{\overline{#1}}{\oldbar{#1}}}
\def\bar{\overline}

%% argument in equatoin
\def\arg{\bullet}

% groups and algebras
\newcommand{\Con}{\mathop\mathrm{Con}}
\newcommand{\Sub}{\mathop\mathrm{Sub}}
\newcommand{\Hom}{\mathop\mathrm{Hom}}
\newcommand{\Aut}{\mathop\mathrm{Aut}}
\newcommand{\Out}{\mathop\mathrm{Out}}
\newcommand{\End}{\mathop\mathrm{End}}
\newcommand{\id}{\mathop\mathrm{id}}
\newcommand{\rk}{\mathop\mathrm{rk}} % rank of a group module/ lattice
\newcommandx{\con}[1][1=\empty]{\ifx#1\empty{\mathop{\mathrm{con}}}\else{\mathop{\mathrm{con}}\left(#1\right)}\fi}
\newcommand{\leftsemidirprod}[1][]{\mathbin{\ifx&#1&\ltimes\else{\ltimes_{#1}}\fi}}
\newcommand{\rightsemidirprod}[1][]{\ifx#1\empty\rtimes\else{\rtimes_{#1}}\fi}
\newcommand{\normalisor}[2][]{\ifx#1\empty{\mathrm{N}\left(#2\right)}\else{\mathrm{N}_{#1} \left(#2\right)}\fi}
% support
\newcommand{\spt}{\mathop{\mathrm{spt}}}
% commutator
\newcommand{\gcom}[2]{\left[#1,#2\right]}

% physics stuff
\newcommand{\float}[3][\empty]{\ifx#1\empty{{#2}\cdot{10^{#3}}}\else{{#2}\cdot{{#1}^{#3}}}\fi}
\makeatletter
\def\newunit#1{\@namedef{#1}{\mathrm{#1}}}
\def\mum{\mathrm{\mu m}}
\def\ohm{\Omega}
\newunit{V}
\newunit{mV}
\newunit{kV}
\newunit{s}
\newunit{ms}
\def\mus{\mathrm{\mu s}}
\newunit{m}
\newunit{nm}
\newunit{cm}
\newunit{mm}
\newunit{fF}
\newunit{A}

\newunit{fA}
\newunit{C}

% elements
\def\newelement#1{\@namedef{#1}{\mathrm{#1}}}
\newelement{Si}
\makeatother


% groups
\newcommand{\ord}{\mathop\mathrm{ord}}
\newcommand{\divides}{|}

\newcommand{\conleq}{\trianglelefteq}
\newcommand{\congeq}{\trianglerighteq}

% common algebraic objects
\newcommand{\reals}{\mathbb{R}} 			% real numbers
\newcommand{\nats}{\mathbb{N}} 				% natural numbers
\newcommand{\ints}{\mathbb{Z}} 				% integers
\newcommand{\rats}{\mathbb{Q}}				% rationals
\newcommand{\complex}{\mathbb{C}}			% complex numbers
\newcommand{\field}[1]{\mathbb{F}_{#1}}  		% finite field
\newcommand{\cards}{\boldsymbol{Cn}}                     % The cardinal numbers
\newcommand{\ords}{\boldsymbol{On}}                      % The ordinal numbers

% graphs
\def\KG{\mathop\mathrm{KG}}                     % Knesergraph

\newcommand{\uvect}{\boldsymbol{e}}

% new operators and relations

%%%%%%%%%%%%%%%%%%%
% complex numbers %
%%%%%%%%%%%%%%%%%%%

\renewcommand{\Re}{\mathop\mathrm{Re}}		% real part
\renewcommand{\Im}{\mathop\mathrm{Im}}		% imaginary part
\newcommand{\sgn}{\mathop\mathrm{sgn}}				% the sign operator (0 for 0)

%%%%%%%%%%%%%%%%
% reel numbers %
%%%%%%%%%%%%%%%%

\newcommand{\floor}[1]{\left\lfloor#1\right\rfloor}
\newcommand{\ceil}[1]{\left\lceil#1\right\rceil}

%%%%%%%%%%%%%%%%%%
% set operations %
%%%%%%%%%%%%%%%%%%

\newcommand{\intersect}{\cap}			% intersect to sets
\newcommand{\setjoin}{\cup}				% join two sets
\newcommand{\setmeet}{\cap}                     % intersect to sets
\newcommand{\bigsetjoin}{\bigcup}			% the union of sets ... subscripts to be added
\newcommand{\distunion}{\dot{\bigcup}}	% disjoint union of sets ... subscripts to be added
\newcommand{\bigsetmeet}{\bigcap}		% intersection of sets
\newcommand{\powerset}[1][]{\ifx&#1&\mathcal{P}\else\mathcal{P}_{#1}\fi}		% powerset ... to be customized
\newcommand{\card}[1]{\left|#1\right|}

%%%%%%%%%%%%%%%%%%%%%%%%%%%%%%%%%%%%%%%%
% composition operations of structures %
%%%%%%%%%%%%%%%%%%%%%%%%%%%%%%%%%%%%%%%%

%\newcommand{\setprod}{\bigtimes}			% setproduct - needed
\newcommand{\dirprod}{\bigotimes} 			% direct product for groups and spaces
\newcommand{\dirtimes}{\otimes}				% direct multiply for groups and spaces
\newcommand{\dirsum}{\bigoplus} 			% direct sum for groups and spaces
\newcommand{\dirplus}{\oplus}				% direct add for groups and spaces
\newcommand{\inprod}[2]{\left\langle #1,#2 \right\rangle}

\newcommand{\tuple}{\meet}
\newcommand{\cotuple}{\join}

%%%%%%%%%%%%%%
% categories %
%%%%%%%%%%%%%%

% combinatorics
%%

\renewcommand{\binom}[3][\empty]{\if\empty{#1}{{#2 \choose #3}}\else{{#2 \choose #3}_{#1}}}

%%%%%%%%%%%%%%%%%%%%%%%%%%%%%%%%%%%%%%%%%%%%%%%%%%%%%
% metric spaces and normed spaces and vector spaces %
%%%%%%%%%%%%%%%%%%%%%%%%%%%%%%%%%%%%%%%%%%%%%%%%%%%%%

\newcommand{\dist}{\mathop\mathrm{dist}}				% distance operator ... dist(A,b), where A is a set and b a point
\newcommand{\diam}{\mathop\mathrm{diam}}	% diameter operator for sets
\newcommand{\norm}[1]{\left\Vert #1 \right\Vert}	% norm in a normed space ... subscript to be added
\newcommand{\conv}{\mathop\mathrm{conv}} 			% convex hull - vectorspaces
\newcommand{\lin}{\mathop\mathrm{lin}} 				% linear hull - vectorspaces
\newcommand{\aff}{\mathop\mathrm{aff}}				% affine hull - vectorspaces

%%%%%%%%%%%%%%%%%%%%%%
% operators in rings %
%%%%%%%%%%%%%%%%%%%%%%

\newcommand{\lcm}{\mathop\mathrm{lcm}}				% least common multiple - in euclidean rings
\renewcommand{\gcd}{\mathop\mathrm{gcd}}				% greatest command devisor - in euclidean rings
\newcommand{\res}{\mathop\mathrm{res}}				% residue of p mod q is res(p,q)
\renewcommand{\mod}{\textrm{ mod }}

%%%%%%%%%%%%%%%%%%%
% logical symbols %
%%%%%%%%%%%%%%%%%%%

%\newcommand{\impliedby}{\Leftarrow}				% reverse implicatoin arrow
%\newcommand{\implies}{\Rightarrow}				% implication
\newcommand{\equival}{\Leftrightarrow}				% equivalence

%%%%%%%%%%%%%%%%%%%%%%%%%%%%%%%%%%
% functions - elementary symbols %
%%%%%%%%%%%%%%%%%%%%%%%%%%%%%%%%%%

\newcommand{\rest}[1]{\left. #1\right\vert}		% restriction of a function to a set / also used as restriction in other terms like differential expressions / evaluation of a function
\newcommand{\rto}[3][]{#2\ifx&#1&\rightarrow\else\stackrel{#1}{\rightarrow}\fi#3}
\renewcommand{\to}{\rightarrow}							% arrow between domain and image
\newcommand{\dom}{\mathop\mathrm{dom}}					% domain of a function
\newcommand{\im}{\mathop\mathrm{im}}						% image of a function
\newcommand{\compose}{\circ}							% compose two functions
\newcommand{\cont}{\mathop\mathrm{C}}					% continuous functions from a domain into the reels or complex numbers 

%%%%%%%%%%%%%%%%%%%%
% groups - symbols %
%%%%%%%%%%%%%%%%%%%%

\newcommand{\stab}[1][]{\if&#1{\mathop\mathrm{stab}}&\else{\mathop\mathrm{stab}_{#1}}\fi}					% the stabilizer ... subscripts to be added
\newcommand{\orb}[1][]{\ifx&#1&\mathrm{orb}\else\mathrm{orb}_{#1}\fi} 					% orbit ... subscrit to be added (group)
\newcommand{\gen}[2][\empty]{\ifx#1\empty{\left\langle#2\right\rangle}\else{\left\langle#2\right\rangle_{#1}}\fi}					% generate ... kind of hull operator ---- to be thought of !!!!!!!

\def\Clo{\mathrm{Clo}}
\def\Loc{\mathrm{Loc}}
%% nets
\newcommand{\net}[2][\empty]{\ifx#1\empty{\left(#2\right)}\else{{\left(#2\right)}_{#1}}\fi}

%% open half ray
\newcommand{\ray}[2]{R_{#1}(#2)}

%% new
\let\oldcong\cong%
\newcommand{\iso}{\oldcong}

\def\cong{\equiv}
\newcommand{\base}[2]{\left[#2\right]_{#1}}                                   % base n expansion of some number
%%%%%%%%%%%%%%%%%%%%%%%%%%%%%%%%%
% matrices and linear operators %
%%%%%%%%%%%%%%%%%%%%%%%%%%%%%%%%%

\newcommand{\diag}{\mathop\mathrm{diag}}				% diagonal matrix or operator
\newcommand{\Eig}[1]{\mathop\mathrm{Eig}_{#1}}		% eigenspace for a certain eigenvalie		
\newcommand{\trace}{\mathop\mathrm{tr}}				% trace of a matrix
\newcommand{\trans}{\top} 							% transponse matrix

%%%%%%%%%%%%%
% constants %
%%%%%%%%%%%%%

\renewcommand{\i}{\boldsymbol{i}}			% imaginary unit
\newcommand{\e}{\boldsymbol{e}}				% the Eulerian constant

%%%%%%%%%%%%%%%%%%%%%%%%%%%%%%
% limit operators and arrows %
%%%%%%%%%%%%%%%%%%%%%%%%%%%%%%

\newcommand{\upto}{\uparrow}				% convergence from above
\newcommand{\downto}{\downarrow}			% convergence from below

%%
% other
%%
\newcommand{\cind}{\mathop\mathrm{Ind}}		% Cauchy index
\newcommand{\sgnc}{\sigma}					% sign changes
\newcommand{\wnumb}{\omega}					% winding number
\newcommand{\cfunc}{\mathop\mathrm{Cf}}		% Cauchy function of a compact curve in complex\setminus\{0\}


% evaluation of a function as a difference or single value

\newcommand{\abs}[1]{\left|#1\right|}
\newcommand{\conj}[1]{\overline{#1}}
\newcommand{\diff}{\mathop\mathrm{d}}

%%%%% test
\newcommand{\distjoin}{\mathaccent\cdot\cup}	% to be modified (name)
\newcommand{\cl}{\mathop\mathrm{cl}}				% topological closure
\newcommand{\sphere}{\mathbb{S}} % n-sphere
\newcommand{\ball}{\mathbb{B}} % n-ball
\newcommand{\bound}{\partial}
\newcommand{\bigmeet}{\mathop\mathrm{\bigwedge}}
\newcommand{\bigjoin}{\mathop\mathrm{\bigvee}}
\newcommandx{\rchar}[1][1=\empty]{\mathop\mathrm{char}\brackets{#1}}              % characteristic of a ring
\newcommand{\lgor}{\vee}                               % logical
\newcommand{\lgand}{\wedge}
\newcommand{\codim}{\mathop\mathrm{codim}}
\newcommand{\row}{\mathop\mathrm{row}}
\newcommand{\cone}{\mathop\mathrm{cone}}
\newcommand{\comp}{\mathop\mathrm{comp}}
\newcommand{\proj}{\mathrm{P}}
\def\PG{\mathrm{PG}}           % projective space
\newcommand{\meet}{\wedge}
\newcommand{\join}{\vee}
\newcommand{\col}{\mathop\mathrm{col}}
\newcommand{\vol}{\mathop\mathrm{vol}\nolimits}
%arrangements
\newcommand{\tpert}{\mathop\mathrm{tpert}}
\renewcommand{\epsilon}{\varepsilon}
% groups
\newcommand{\symgr}{\mathop\mathrm{Sym}}
\newcommand{\symalg}{\mathop\mathrm{S}}
\newcommand{\extalg}{\mathop\mathrm{\Lambda}}
\newcommand{\extpow}[1]{\mathop\mathrm{\Lambda}^{#1}}

%% set hulloperators -> define



\newcommandx{\homl}[3][1=1,2=2,3=3]{\ifx#1\empty{\mathrm{Hom}}\else{\mathrm{Hom}_{#1}}\fi(#2,#3)}
\makeatletter
\newenvironment{myproofof}[1]{\par
  \pushQED{\qed}%
  \normalfont \topsep6\p@\@plus6\p@\relax
  \trivlist
  \item[\hskip\labelsep
        \bfseries
    Proof of #1\@addpunct{.}]\ignorespaces
}{%
  \popQED\endtrivlist\@endpefalse
}
\makeatother



%%% environment test with enumerates

    

% Local variables:
% mode: tex
% End:


\usepackage{mathenv}
%\usepackage{nath}
\begin{document}
\title{Various Notes}
\author{Jakob Schneider}
\date{\today}
\maketitle
\label{a}
%\autoref{a}%

\makeatletter
\def\com#1{\@ifnextchar[{\a#1}{\b#1}}
\def\a#1[#2]{Hallo}
\def\b#1{Tschüss #1}
%
$$
\set{sdfsdf}
$$
\section{Complex analysis}
\def\sectionn#1. #2\par{\medskip\noindent {\bf\LARGE #1. #2\par}}
\sectionn 2.3. Hallo
\def\Sum#1\from#2\to#3{\sum_{#2}^{#3}{#1}}
%\def\Sum#1\over#2{\sum_{#2}{#1}}
$$
\Sum 3 \from i=1 \to 4
$$

\begin{lemma}
Let $\seq{a_n}_{n\in\nats}$ be a sequences of non-negative numbers such that
$$
\rho_a:=\frac 1{\limsup_{n\to\infty}{\sqrt[n]{a_n}}}
$$
prove that the sequence $b_n:=\sum_{m\geq n}{\binom m n a_m}$ satisfies
$$
\rho_b\geq \rho_a-1.
$$
\end{lemma}

\begin{proof}
    For $N$ sufficiently large we have that $a_n\leq{(\rho_a-\epsilon)}^{-n}$ for $n\geq N$.
    Thus
    $$
    b_n\leq \sum_{m\geq n}{\binom m n {(\rho_a-\epsilon)}^{-m}}={\left(\frac 1 {1-{(\rho_a-\epsilon)}^{-1}}\right)}^{n+1}{(\rho_a-\epsilon)}^{-n}
    $$
    by the formula ${\left(\frac 1 {1-x}\right)}^n=\sum_{i=0}^\infty{\binom{n-1+i}{n-1}x^i}$.
    Thus
    $$
    \limsup_{n\to\infty}\sqrt[n]{b_n}\leq \frac 1 {(\rho_a-\epsilon)-1}
    $$
    for all $\epsilon>0$ so $\rho_b\geq \rho_a-1$.    
\end{proof}

\begin{remark}
    The transformation $a_n=\sum_{m\geq n}{\binom{m}{n}b_m}$ has inverse ${(-1)}^n b_n=\sum_{m\geq n}{\binom{m}{n}{(-1)}^m a_m}$.
\end{remark}

\begin{lemma}
    Let $\seq{a_n}_{n\in\nats}$ be a sequence of complex numbers such that $\sum_{n\in\nats}{a_n x^n}$ has radius of convergence $\rho_a$. Then the sequence $\seq{b_n}_{n\in\nats}$ defined by the formal equality $\sum_{n\in\nats}{a_n x^n}=\sum_{n\in\nats}{b_n {(x-x_0)}^n}$ has radius of convergence $\rho_b$ with $\abs{\rho_a-\rho_b}\leq\abs{x_0}$.
\end{lemma}

\begin{proof}
    It is easy to check that $b_n$ is defined by
    $$
    a_n=\sum_{m\geq n}{\binom{m}{n}b_m{(-x_0)}^{m-n}}={(-x_0)}^{-n}\sum_{m\geq n}{\binom{m}{n}{\left(\frac{b_m}{-x_0}\right)}^m}.
    $$
    Thus the previous lemma gives that the radii of convergence $\rho'_a$ and $\rho'_b$ of the sequences
    $\seq{\frac{a_n}{-x_0}}_{i\in \nats}$ and $\seq{\frac{b_n}{-x_0}}_{i\in \nats}$ satisfy
    $$
    \rho'_a\leq \rho'_b-1
    $$
    and of course the symmetric identity does also hold (by a change of coordinates). Thus $\abs{\rho'_a-\rho'_b}=\frac{\abs{\rho_a-\rho_b}}{\abs{x_0}}\leq 1$ proving the claim.
\end{proof}


\section{Filtered spaces}
%%%
$$
\Mor[a]{\arg_1}{\arg_2}
\End[A]A
$$

\begin{definition}[antitone \person{Galois} connection of convergent nets and filters]
    Let $X$ be a space with convergence structure $C$ (i.e.~a bunch of convergent nets in $X$). Then define
    $$
    \N_x \C\defeq\set{N\in\Sub X}[\forall \seq{x_i}_{i\in I}\in\C_x:\exists i\in I:\forall j\geq i:x_i\in N]
    $$
    and analogously for a given filter system $\N$ define
    $$
    \C_x \N\defeq\set{\seq{x_i}_{i\in I}\in\Mor{I}{X}}[\forall N\in \N_x:\exists i\in I:\forall j\geq i:x_i\in N].
    $$
    Then the above operators form an antintone \person{Galois} connection.
\end{definition}

\begin{definition}
    A filtered space $X$ is a structure with operators $\N$ such that each $x\in X$ gets assigned to a filter $\rest{\N}_x X$ which is a subset of $\Sub X$, where any filter element contains $x$.
\end{definition}

\begin{definition}[ordered sets as filtered spaces]
    Ordered sets are filtered spaces in a natural way. For $x\in X$ the filter $\rest{N}_x X$ is defined as the filter generated by all sets of the form $X\setminus[y)$ where $y\not\leq x$.
\end{definition}

\section{Filters and convergence}

A filtered space can be discribed either by its filters or its convergent nets.

\section{Lattices and posets}

\begin{definition}[topology of complete lattices]
    A complete lattice admits a natural filter stucture which is generated by the convergence of monotone nets. This filter structure is said to be the filter structure induced by the order of $L$.
\end{definition}

\begin{lemma}
    For $x\in X$ by the previous definition the filter in $x$ is
    $$
    \N_x X\defeq \N^{\join}_x X\setjoin \N^{\meet}_x X$$
    where
    $$
    \N^{\join}_x X\defeq
    \set{U\in\Sub X}[\forall\seq{x_i}_{i\in I}:\left(\Join_{i\in I}{x_i}=x\lgand \forall i,j:i\leq j\implies x_i\leq x_j\right)\implies\exists i\in I:\forall j\geq i:x_j\in U]
    $$
    and
    $$
    \N^{\meet}_x X\defeq
    \set{U\in\Sub X}[\forall\seq{x_i}_{i\in I}:\left(\Meet_{i\in I}{x_i}=x\lgand \forall i,j:i\leq j\implies x_i\geq x_j\right)\implies\exists i\in I:\forall j\geq i:x_j\in U].
    $$
    Moreover, $\N^{\join}_x$ is generated by the complements of lower-closed $\Join$-closed sets containing $x$ and dually, $\N^{\meet}_x$ is generated by the complements of upper-closed $\Meet$-closed sets. 
\end{lemma}

\begin{proof}
    \paragraph{Lower-open $\Join$-open sets are in $\N^{\join}_x X$.}
    Let $\seq{x_i}_{i\in I}$ be a non-decreasing net with $\Join_{i\in I}{x_i}=x$ and $C$ be a lower-closed $\Join$-closed set not containing $x$. Assuming that $x_i\in C$ for all $i\in I$ leads to $\Join_{i\in I}{x_i}=x\in C$ since $C$ is $\Join$-closed, a contradiction. Thus $x_i\in X\setminus C$ for some $i\in I$. Since $C$ is lower-closed it follows that $x_j\in X\setminus C$ for all $j\geq i$ (since otherwise $x_i\in C$, a contradiction).

    \paragraph{Lower-open $\Join$-open sets generate $\N^{\join}_x X$.}
    To show that these sets generate $\N^{\join}_x X$ we pick a non-decreasing net $\seq{x_i}_{i\in I}$ with $\Join_{i\in I}{x_i}=x$. We have to show that the filter generated by these sets says that $\seq{x_i}_{i\in I}$ converges. For each $y\not\geq x$ the set $(y]$ is lower-closed and $\Join$-closed. Thus a limit $x'$ of $\seq{x_i}_{i\in I}$ with respect to the filter generated by the lower-open $\Join$-open sets then must also lie outside $(y]$. Then $x'\not\leq y$ for $y\not\geq x$.
But then $x'\geq x$ since otherwise $x'\not\geq x$ and $x\leq x$ a contradiction.

Conversely, we deduce that any $x'\in x^{\meet}$ is a limit in the second sense.

Dually, we are done.
\end{proof}

\begin{definition}[completion lattice of a poset]
    
\end{definition}%%

%
\begin{definition}[upper and lower bounds for posets]
    For a set $A\setleq P$ write
    $$A^{\join}:=\set{p\in P: \forall a\in A: a\leq p}$$
    for the set of upper bounds of $A$ and
    $$A^{\meet}:=\set{p\in P}[\forall a\in A: p\leq a]$$
    for the set of lower bounds. These operators correspond to the operators $\Meet$ and $\Join$ of the completion lattice of $P$ in the sense that when $\iota:P\to \bar{P}$ is the canonical map to the completion lattice then $\Join\Sub(\iota)(A^{\meet})=\Meet\Sub(\iota)(A)$. Moreover, the operators $\meet$ and $\join$ form an antitone \person{Galois} connections $\struct{\struct{\Sub(P),\meet},\struct{\Sub(P),\join}}$ on the subsets of the poset $P$ similarly as $\Meet$ and $\Join$ form such connection on the subsets of a complete lattice. 
\end{definition}

\begin{definition}[topology of posets]
    A poset $P$ is equipped with a natural topology which is induced by the \person{Galois} connection $\struct{\struct{\Sub(P),\meet},\struct{\Sub(P),\join}}$ when defining the closed sets to be the \person{Galois} closed sets. It can be descibed by $$\tau_P:=\gen{P\setminus A^{\meet},P\setminus A^{\join}: A\in\Sub(P)}.$$
\end{definition}
%
\begin{remark}
    Here closed intervals are already closed under intersection since it holds for antitone \person{Galois}-connections that
    $$\Setmeet_{i\in I}{A_i^{\meet}}={\left(\Setjoin_{i\in I}{A_i}\right)}^{\meet} \textrm{ and } \Setmeet_{i\in I}{A_i^{\join}}={\left(\Setjoin_{i\in I}{A_i}\right)}^{\join}.$$
\end{remark}

// analogue between complete lattice and PID

\begin{lemma}[bounds for the limit point of a sequence]
    Let $N:=\net[i\in I]{a_i}$ be a net in a poset $P$. Then any accumulation point $a$ of $N$ it holds that
    $$
    a\in {\left(\Setjoin_{i\in I}{\set{a_j}[j \geq i]^{\meet}}\right)}^{\join}\setmeet{\left(\Setmeet_{i\in I}\set{a_j}[j \geq i]^{\join}\right)}^{\meet}
    $$
    and if $P$ is a complete lattice, this can be written as
    $$
    \Join_{i\in I}\Meet_{j\geq i}{a_j}\leq a\leq \Meet_{i\in I}\Join_{j\geq i}{a_j}.
    $$
\end{lemma}%

\begin{proof}
    It is clear that for $i\in I$ we have $a\in\set{a_j}[j \geq i]^{\meet\join}\setmeet\set{a_j}[j \geq i]^{\join\meet}$ or equivalently for a lattice $\Meet\set[j \geq i]{a_j}\leq a\leq \Join\set[j \geq i]{a_j}$ since otherwise
    $P\setminus\set[j \geq i]{a_j}^{\join\meet}$ or $P\setminus\set[j \geq i]{a_j}^{\join\meet}$ would be an open neighbourhood of $a$ not containing a point of $\set[j\geq i]{a_j}$ (contradicting the assumption that $a$ is an accumulation point of $N$).
    Thus we have that
    $$
    a\in \Setmeet_{i\in I}[j \geq i]{\set{a_j}^{\meet\join}}\setmeet\Setmeet_{i\in I}\set[j \geq i]{a_j}^{\join\meet}=
    {\left(\Setjoin_{i\in I}{\set[j \geq i]{a_j}^{\meet}}\right)}^{\join}\setmeet{\left(\Setmeet_{i\in I}\set[j \geq i]{a_j}^{\join}\right)}^{\meet},
    $$
    or respectively for the complete lattice case
    $$
    \Join_{i\in I}\Meet_{j\geq i}{a_j}\leq a\leq \Meet_{i\in I}\Join_{j\geq i}{a_j}.
    $$
\end{proof}

\begin{definition}[limes superior and inferior]
    Let $N:=\net[i\in I]{a_i}$ be a net in a poset $P$. When there exists elements $\check{N},\hat{N}\in P$ such that
    $$
    \sup{\left(\Setjoin_{i\in I}{\set[j \geq i]{a_j}^{\meet}}\right)}=\hat{N}
    $$
    or
    $$
    \inf{\left(\Setjoin_{i\in I}{\set[j \geq i]{a_j}^{\meet}}\right)}=\check{N}
    $$
    then $\hat{N}$ is called \keyword{limes superior} and denoted by $\limsup{N}$ and $\check{N}$ is called \keyword{limes inferior} and denoted by $\liminf{N}$.
\end{definition}

Now we introduce the notion of a topological lattice

\begin{definition}[topological lattice]
    Let $L$ be a structure the form $\struct{L',\tau}$ where $L'=\struct{L'',\meet,\join,\leq}$ is a lattice and $\tau$ is a topology on $L'$ under which $\meet, \join:L\times L \to L$ are continuous.
\end{definition}

\begin{lemma}
    Let $L=\struct{L',\tau}$ be topological lattice with an identifying, separating topology ($T_1$ space) then $\tau$ contains the \person{Galois} topology of $L'$. The converse is also true.
\end{lemma}

\begin{proof}
    Let $x\in L$ then since $\tau$ is $T_1$ the set $\set{x}$ is closed. Define $f:=(c_x\tuple\id_L)\join$ where $c_x$ is the constant map then $f$ is continuous. Thus, $f^{-1}{\set{x}}=\set{x}^{\join}$ is closed. This implies that for all $A\setleq L$ the set $A^{\join}=\Setmeet_{a\in A}{\set{a}^{\join}}$ is closed. For $\meet$ the dual argument works.
    For the other direction, note that $\set{x}^\join\setmeet\set{x}^{\meet}=\set{x}$ is closed in the \person{Galois} topology, so $L$ is $T_1$.
\end{proof}

\begin{lemma}[monotone nets in posets]
    Let $P$ be a poset equipped with \person{Galois} topology and $N:=\net[i\in I]{a_i}$ a monotone net, i.e. $i\leq j$ implies $a_i\leq a_j$. Then it holds that
    \begin{statements}
            \item If $N$ is non-decreasing $\sup{N}=\lim{N}$.
            \item If $N$ is non-increasing $\inf{N}=\lim{N}$.
    \end{statements}
    Moreover, $N$ has at most one accumulation point.
\end{lemma}

\begin{proof}
    We only prove the first statement, the second following by duality.
    
    Let $a$ an accumulation point of $N$. Then $a$ is an upper bound of $\set[i\in I]{a_i}$ for otherwise there would exist an $i\in I$ such that $a_i\not\leq a$ implying that $P\setminus \set{a_i}^{\join}$ is a neighbourhood of $a$ not containing a point of $\net[j\geq i]{a_j}$ contradicting the assumption that $a$ is an accumulation point of $N$. Moreover, if there exists an $a'<a$ which is also an upper bound of $\set[i\in I]{a_i}$ then $P\setminus\set{a'}^{\meet}$ is an open neighbourhood of $a$ not containing a point of $N$. Thus, if $\lim{N}$ exists it must be the supremum of $N$. Conversely, if the supremum of $N$ exists, it must be its limit. This can be seen from as follows. Assume there is an open neighbourhood $U$ of $\sup{N}$ which seperates it from $N$. Then by reducing to the base of topology of the \person{Galois} topology we can assume that $U=P\setminus(\Setjoin_{i=1}^n{A_i^{\meet}}\setjoin\Setjoin_{i=1}^m{B_i^{\join}})$. When there is an $i\in I$ and $j\in\set{1,\ldots,m}$ such that $a_i\in B_j^{\join}$ then $\sup{N}\in U$. So we may assume that $m=0$. On the other hand, it is easy to see that the sets $A_i^{\meet}\setmeet\set[i\in I]{a_i}$ are totally ordered by inclusion since $N$ is non-decreasing. Thus there is an $i\in\set{1,\ldots,n}$ such that $\set[i\in I]{a_i}\setleq A_i^{\meet}$. But then $\sup{N}\in A_i^{\meet}$ by definition of the supremum.
\end{proof}

\begin{lemma}[topology of posets]
    A lattice $L$ is a topological space when equipped with the topology
    $\tau:=\gen{\set{c\in L: a\not\leq c\not\leq b}:a,b\in L}$ such that lattice convergence and topological convergence coincide.
\end{lemma}

\begin{proof}
    Let a net $\net[i\in I]{a_i}$ converge in $P$ to $a$. Then from $a=\limsup_{i\in I}{a_i}$ we know that for all $a'<a$ there exists an $i\in I$ such that for all $j\geq i$ one of the sets $\set[k\geq j]{a_k}$ has an upper bound $u$ which is smaller than $a'$.
\end{proof}



\subsection{Rank}

\begin{definition}
    Let $P:\struct{P',0,\leq}$ be a poset with a distinguished element 0 and $L:\struct{L',\meet,\join,0}$ be a complete lattice with a distinguished element 0.
    A map $\rk:P\to L$ is called a rank function if
    \begin{properties}
            \item $\rk{0}=0$,
            \item $\forall p\in P: \rk{p}=\Meet\set{q\in P:\forall r<p: q>\rk{r}}=\Join\set{q\in P:\forall r>p: q<\rk{r}}$.
    \end{properties}
\end{definition}

\begin{remark}
    Mostly, $P$ will itself be a lattice. We will however see that rank functions occur in many areas of classical algebra as well as measure theory.
\end{remark}



\section{Measures}



\section{Preliminaries}

\index{bla}
Here, I assemble some notes on various interesting subjects.

\begin{definition}[preorder]
    A preorder on a set $S$ is a relation $\leq\setleq S^{\setprod2}$ satisfying the following properties.
    \begin{statements}
            \item \emph{Transitivity} If $a\leq b\leq c$ then $a\leq c$.
            \item \emph{Reflexivity} It holds that $\id_S\setleq \leq$. 
    \end{statements}
\end{definition}

\begin{definition}[Directedness]
    A preorder $\leq$ on a set $S$ is called directed if it for all $a,b\in S$ there exists a $c$ such that $a,b\leq c$ (existence of upper bounds for finite sets).
\end{definition}


\begin{definition}[order]
    An \keyword{order} is a identifying or seperating preorder, that is, if $a\leq b$ if and only if $a'\leq b$, and $a\geq b$ if and only if $a'\geq b$ then $a=a'$. More simply, one can weaken this to $a\leq a'$ and $a'\leq a$ implies that $a=a'$.
\end{definition}

\begin{remark}
    Think of $T_1$ spaces.
\end{remark}


\begin{definition}[net]
    A net $N$ is a structure $\struct{I,\leq}$ where $\leq$ a directed preorder.
\end{definition}



\begin{definition}[substructures]
    For a structure $S$ define the lattice of substructures as $\Sub(S)$ (every lattice is a net).
\end{definition}


\begin{definition}[topological summation]
    Let $A$ be a topological abelian group. Then define the sum of a set $B\setleq A$ as
    $$
    \sum{B}:=\lim_{B'\in\Sub_{\fin}(B)}{\sum{B'}}.
    $$
\end{definition}

\begin{lemma}[properties of topological summation]
    
\end{lemma}

\begin{proof}
    Let $\net[i\in I]{B_i}$ be a net in $\Sub(A)$ such that $\lim_{i\in I}{B_i}=B$.
\end{proof}




\begin{lemma}[Riesz' rising sun lemma]
Let $f:[a,b]\to\reals$ be a continuous function. Define the \emph{set
of shadowed points} $S$ by
\begin{equation}
S:=\{x\in [a,b]:\exists y\in[a,b]:y>x\lgand f(y)<f(x)\}\textrm{.}
\end{equation}
Then it holds that
\begin{enumerate}
\item $S$ is open within $[a,b)$. 
\item If $(a',b')$ is a maximal open interval within $S$ then 
\begin{enumerate}
\item$f(a')=f(b')$ if $a'\neq a$.
\item $f(a')\leq f(b')$ if $a'=a$.
\item $\forall x\in(a',b'):f(x)<f(b')$. 
\end{enumerate}
\end{enumerate}
\end{lemma}

\begin{proof}
The set $S$ can be written as
\begin{eqnarray*}
    S
    &=& \Setjoin_{y\in[a,b]}\set{x\in[a,b]:x<y \lgand f(x)<f(y)}\\
    &=& \Setjoin_{y\in[a,b]}(\set{x\in[a,b):x<y\}\setmeet\{x\in[a,b):f(x)<f(y)})
\end{eqnarray*}
which is open as a union of finite intersection of open sets.
To prove the second claim, let $a',b'\in [a,b]$ such that
$(a',b')\subseteq S$ is a maximal open interval (it may happen that also
$a'\in S$ if $a'=a$, but $b'\not\in S$ as $S$ is open within $[a,b)$). Now, define
\begin{equation}
b'':=\min\set{x\in[b',b]:f(x)=\max_{y\in[a',b]}{f(y)}}\textrm{.}
\end{equation}
(Due to the continuity of $f$ and as $[b',b]$ is compact this minimum exists and equals the
infimum of the same set.) We show that $b''=b'$. Observe that for
$x\in[b',b'')$ it holds that $x<b''$ and $f(x)<f(b'')$ from which we
deduce that $x\in S$ by the definition of $S$.
Thus it follows that $(a',b')\setjoin[b',b'')=(a',b'')\subseteq S$
which implies $b''=b'$ as $(a',b')$ is a maximal open interval within
$S$.

Let $x\in(a',b')$. By the definition of $S$ there exists $y\in(x,b]$
with $f(x)<f(y)$. Now, using the last fact we have that
$f(b')=f(b'')>f(y)$ if $y>b'$. Thus, there exists $y\in(x,b']$ such
that $f(x)<f(y)$. This shows, that $\rest{f}_{[x,b']}$ does not attain its
maximum in $x$. Now let $z\in[x,b']$ such that
$f(z)=\max_{y\in[x,b']}{f(y)}$. Then $f(z)=\max_{y\in[z,b']}{f(y)}$
from which we deduce that $z=b'$ for if $z\in(a',b')$ would lead to a
contradiction by the last fact. Thus $f(x)<f(b')$.

As $f$ is continuous, it follows that $f(a')\leq f(b')$. If $a<a'$
then by the maximality of $(a',b')$ we have that $a'\notin S$ from
which we deduce that $f(a')\geq f(b')$ from the above, which gives $f(a')=f(b')$. 
\end{proof}

\section{Functions and relations}

\begin{definition}[binary relation]
Let $A, B\in\Set$. Then we define the \new{Cartesian product} of $A$
and $B$ as $A\times B = \{(a,b):a\in A\lgand b\in B\}$.
A \new{binary relation} $R$ between $A$ and $B$ is a subset of $A\times B$.
\end{definition}

\paragraph{Remarks.}
\begin{enumerate}
\item The Cartesian product $A\times B$ is a direct product of $A$ and $B$
in the categorical sense (thus the notation coincides 'up to
isomorphism' with the notation for the direct product).
\item Here, the
tuple $(a,b)$ is defined as $\{\{a\},\{a,b\}\}$.
\item For convenience, we abbreviate $(a,b)\in R$ by $aRb$, $\{a\in A:
  aRb\}$ by $Rb$ and $\{b\in B: aRb\}$ by $aR$.
\end{enumerate}

\begin{definition}[Composition of binary relations]
Let $R\subseteq A\times B$ and $S\subseteq B\times C$. Then the
composition of $R$ and $S$ is defined as $R\compose S = \{(a,c)\in
A\times C: \exists b\in B: aRbSc\}$.
\end{definition}

\begin{definition}[antisymmetric]
A binary relation $R\subset A\times A$ is called \new{antisymmetric}
if for $a,b\in A$ we have that $aRb$ and $bRa$ imply $a=b$.
For more general relations, we note these relational sets by $R_a$
(stabilizer notation).
\end{definition}

\begin{definition}[identifying or separating relation]
A binary relation $R\subset \prod_{i\in I}{A_i}$ is called \new{separating}
or \new{identifying} in $A_j$ if an object of $A_j$ is identified by its relational
behavior. That is, for $a,b\in A_j$ the equality
$\pi_j^{-1}(a)\setmeet R=\pi_j^{-1}(b)\setmeet R$ implies $a=b$
where $\pi_j$ is the projection from the product $\prod_{i\in
  I}{A_i}$ onto the $j$-th factor.
\end{definition}

\paragraph{Remarks}
\begin{enumerate}
\item The notion of separating relations appears in many areas
  (e.g. Hausdorff-spaces, dual pairings etc.).
\end{enumerate}



Functions $(f(x),x)$ for the correctness of composition.

\section{Primitive roots of unity and the Carmichael function}

\begin{definition}[Exponent of a group]
    The exponent $\exp(G)$ of some group $G$ is defined as the smallest number $n\in\nats$ such that for all $g\in G$ we have $g^n=\id_G$. If such number does not exist, say the exponent of $G$ is infinity ($\exp(G)=\infty$).
\end{definition}

\begin{lemma}
    If $G$ is an abelian group then $\{\ord(g):g\in G, \ord(G)<\infty\}$ (here $g$ runs over the torsion module of the $\ints$-module $G$) is a sublattice of $\ints$ (equipped with the devided-by relation). Moreover, $\exp(G)=\Join_{g\in G}{\ord(g)}$.
\end{lemma}

\begin{proof}
    Let $a,b\in G$. We have to show that there exists an element $c\in G$ with $\ord(c)=a\join b$. As $\ints$ is a PID we have numbers $a',b'\in\ints$ such that $\ord(a)\meet\ord(b)=a'\ord(a)+b'\ord(b)$. Thus $c:=b'a+a'b\in G$ (interpreting $G$ as a $\ints$-module) is an element of the desired order which can be seen from the homomorphism $\phi:\gen{c}\to \gen{a}\times\gen{b}$ via $x\mapsto (\ord(b)x,\ord(a)x)$ and the first homomorphism theorem.
    At first we see that $\ord(b)c=b'ord(b)a\in\gen{a}$ and $\ord(a)c=a'\ord(a)b\in\gen{b}$ showing that the domains are correct.
    Moreover, we have that $\phi(kc)=k\phi(c)=k(b'\ord(b)a,a'\ord(a)b)=0$ if and only if $k\ord(b)=0$ mod $\ord(a)$ and $k\ord(a)=0$ mod $\ord(b)$. But this equivalent to $k=0$ mod $\ord(a)\join\ord(b)$. Thus we see that $\gen{c} \iso  \ints_{\frac{\ord(a)}{\ord(a)\meet\ord(b)}}\times\ints_{\frac{\ord(a)}{\ord(a)\meet\ord(b)}} \iso \ints_{\ord(a)\join\ord(b)}$.
\end{proof}

\begin{lemma}
    Let $p$ be prime, $k\geq 1$ and $a\in\ints^*_{p^{k+1}}$ be such that its projection $\bar{a}\in\ints^*_{p^k}$ is an element of maximum order. Then the $\ord_{\ints^*_{p^{k+1}}}(a)\in\set{\ord_{\ints^*_{p^k}}(\bar{a}),p\ord_{\ints^*_{p^k}}(\bar{a})}$.
\end{lemma}

\begin{proof}
    It is clear that $\ord_{\ints_{p^{k-1}}}(\bar{a})\divides\ord_{\ints_{p^k}}(a)$ as $a^n=1\implies \bar{a}^n=\bar{1}$. Moreover, since $\ints_p$ is a field $\ints^*_p$ is cyclic and thus $p-1\divides\ord(\bar{a})\divides\ord(a)$ (as $\ints_p$ is a quotient ring of $\ints_{p^{k-1}}$). Thus we have that $\ord(\bar{a})=(p-1)p^j$ for some $j\leq k$ as $\ord(\bar{a})\divides\phi(p^k)=(p-1)p^{k-1}$. Finally, we have that $a^{p\ord(\bar{a})}\in (\gen{p^{k-1}}_{\Con}+1)^p=\set{1}$ which completes the proof.
\end{proof}

\begin{proof}
    Let $a\in \ints$ such that $\ord(a)$ mod $\ints^*_{p^k}$ is maximal. ...
\end{proof}

\section{fields}

\begin{lemma}
    Let $K$ be a connected ordered field (where the topology is the induced order topology). Then $K\iso \reals$. 
\end{lemma}

\begin{proof}
    Consider $R\defeq\cl \rats$. Then $R$ is closed by definition. On the other hand, $R$ is closed, since any element $k\in K$ satisfying $\rats \leq k$ or $k \leq \rats$ has the open neighbourhood $N\defeq (-1+k,k+1)$ satisfying $\rats < N$ or $N < \rats$ implying that $k\not\in R$. So any $r\in R$ must lie between two rationals. But then $R=\Setjoin_{n\in\nats}{(-n,n)}$ is open. But $K$ is connected and $R\neq\emptyset$ so $K=R$. On the other hand, it is clear that the supremum of any open interval must exist in $R$ since otherwise one could find a non-trivial decomposition of $R$. Thus $R\iso \reals$. 
\end{proof}

\begin{exercise}
    Prove that the following two are equivalent in some field $F$.
    \begin{statements}
        \item Every polynomial function is surjective.
        \item $F$ is algebraically closed.
    \end{statements}
\end{exercise}

\begin{solution}
    Trivial.
\end{solution}

\begin{exercise} Let $A$ be an algebra and $\End{A}$ its endomorphisms. Prove that if $\Out{\Aut{A}}\iso 1$ then $\Aut{\End{A}}=\Aut{A}$.
\end{exercise}
\def\iso{\equiv}
\begin{solution}
    Clearly, $\Aut(A)$ embeds in $\Aut{\End{A}}$ via the map $\iota:\Aut{A}\to\Aut{\End{A}}$ by $\iota(\alpha)(\phi)=\alpha^{-1}\phi\alpha$. On the other hand, for any $\beta\in\Aut{\End{A}}$ we have that $\rest{\beta}_{\Aut{A}}\in\Aut{\Aut{A}}$ and as $\Out{\Aut{A}}\iso 1$ any automorphism is a conjugation.

??
\end{solution}

\begin{lemma}[subfields of fractional field] Let $\alpha=\frac{P}{Q}\in K(X)$ for some field $K$ and polynomials $P,Q\in K[X]$ then $[K(X):K(\alpha)]=\max\set{\deg_X{P},\deg_X{Q}}$.
\end{lemma}

\begin{proof}
    Set $R(Y)=P(Y)-\alpha Q(Y)$ then $R(X)=0$ and $R$ is irreducible in $K(\alpha)[Y]$ since it is irreducible in $K[\alpha][Y]=K[Y][\alpha]$ as a linear polynomial (Gauss Lemma).
\end{proof}
    
\begin{exercise}
    Prove that $\Aut(K(X)/K)\iso \GL[K^2]$
\end{exercise}

\begin{solution}    
    Let us first notice that all $\alpha=\frac{aX+b}{dX+c}$ with $ac-bd\neq 0$ induce field automorphisms since $$\alpha\compose\alpha'=\frac{a\frac{a'X+b'}{c'X+d'}+b}{c\frac{a'X+b'}{c'X+d'}+d}=\frac{(aa'+bc')X+(ab'+bd')}{(ca'+dc')X+(cb'+dd')}$$for appropriate $\alpha,\alpha'$ showing that these $\alpha$ form a group isomorphic to $\GL[K^2]$.
    Any $K$-endomorphism $\alpha$ of $K(X)$ is uniquely determined by its image on $X$ thus we may assume that
    $\alpha(X)$ is some rational function such that there is a rational function $\beta$ with $\beta\compose\alpha(X)=X$ if $\alpha$ is left invertible (and an automorphism has both left and right inverse). This implies that $\alpha$ interpreted as a map on an algebraic closure $\bar{K}\setjoin\set{\infty}$ of $K$ is injective.
    Let $\alpha=\frac{P}{Q}$. W.l.o.g. we may assume that $\deg_X{P}\geq\deg_X{Q}$ and that $P,Q$ are monic since if $\alpha$ has left inverse $\beta$ then $1/\alpha$ has left inverse $1/\beta$ and $c\alpha(X)$ has left inverse $\beta(c^{-1}X)$.
    Injectivity of $\alpha$ means that $\alpha(z)=c$ has only one solution $z\in \bar{K}\setjoin\set{\infty}$ for $c\in\bar{K}\setjoin\set{\infty}$ or equivalently $P-cQ$ is completely inseparable. For all but one $c=-1$ if $\deg_X{P}=\deg_X{Q}$ all these polynomials are monic and have degree $d:=\deg_X{P}$.
    Assume first that $\deg_X{P}>\deg_X{Q}$. Thus plugging in two of these $c$ (they exist since $\bar{K}$ is infinite) with difference one gives that
    $$Q={(X-\xi)}^d-{(X-\xi')}^d=\sum_{i=1}^d{{(-1)}^i\binom{d}{i}(\xi^i+{\xi'}^i)X^{n-i}}$$
    But on the other hand $Q$ must itself be inseparable so
    $$\sum_{i=1}^d{{(-1)}^i\binom{d}{i}(\xi^i+{\xi'}^i)X^{n-i}}=c_Q{(X-{\xi''})}^{d-1}$$
    from which we get
    $${(-1)}^i\binom{d}{i}(\xi^i+{\xi'}^i)=c_Q\binom{d-1}{i-1}{(-1)}^{i-1}{\xi''}^{i-1}$$

   
Similarly, since if $\alpha$ is left invertible then also $1/\alpha$ (inverse is $1/\beta$). Thus $P=c_P(X-\xi_P)^n$, $Q=c_Q(X-\xi_Q)^m$ for some $c_P,c_Q\in K$, $\xi_P,\xi_Q\in\bar{K}$, $m,n\in\nats$. 
\end{solution}

\begin{exercise}\label{cauchy-f-eq}
    Let $f:K\to K$ be a mapping where $K$ is a field with subfield $L\leq K$. Assume that
    \[
    f(\alpha x+\beta y)=\alpha f(x)+\beta f(y)
    \]
    for all $\alpha,\beta\in L$ and $x,y\in K$ (that is $f$ is $L$-linear) and additionaly that
    \[
    f(x)f(1/x)=1
    \]
    for $x\in K^{\ast}$ and $f(1)=1$.
    Show that $f\compose F=F\compose f$ for any $L$-rational function $F\in L(X)$. Moreover, show that $f\in\End(K/L)$ if $\rchar[K]\neq 2$.
    Deduce that $f\in\Aut(K/L)$ in the case where $K/L$ is algebraic and that $f=\id$ in the case $K=\reals$, $L=\rats$.
\end{exercise}

\begin{solution}[\ref{cauchy-f-eq}]
    At first we show that $f(x^n)=f(x)^n$ for all $n\in\nats$. This can be seen by induction from $f(x^1)=f(x)^1$ and the induction step
    \begin{eqnarray*}
        f\left(\frac{1}{x}-\sum\limits_{i=1}^n{\frac{1}{{(1+x)}^n}}\right)
        &=& f\left(\frac{1}{x{(1+x)}^n}\right)\\
        &=& \frac{1}{f(x){(1+f(x))}^n-{f(x)}^{n+1}+f(x^{n+1})}\\
        &=& \frac{1}{f(x)}-\sum\limits_{i=1}^n{\frac{1}{{(1+f(x))}^n}}
\end{eqnarray*}
where we deduce that $f(x^{n+1})={f(x)}^{n+1}$ for $x\in K\setminus\set{0,-1}$ from $f(x^k)={f(x)}^k$ for all $1\leq k\leq n$.
But obviously $f(0)=0$ and $f(-1)=-1$ can be derived from the additivity of $f$ and from $f(1)=1$.
If $\rchar{K}\neq 2$ we then get $f({(x+y)}^2)={f(x)}^2+{f(y)}^2+2f(x)f(y)=f(x^2)+f(y^2)+2f(xy)$ implying that $f(xy)=f(x)f(y)$. Thus $f\in\End(K/L)$ as $f(x)=x$ for $x\in L$ (by $L$-linearity).
If $K$ is algebraic over $L$ then $\Aut(K/L)=\End(K/L)$ proving the second fact in the case $\rchar{K}\neq 2$.
On the other hand, the multiplicativity of $f$ follows as well for $\rchar{K}=2$ and $L/K$ finite.
Thus it also holds for the inductive limit of finite field-extensions.
In the case $K=\reals$ we get that $f(x)^2=f(x^2)$ showing that $f$ is continuous and thus $f(x)=x$.
\end{solution}

\begin{exercise}\label{3-cycl-an}
    Determine the minimum number of $3$-cylces necessary to generate $A_n$.
\end{exercise}

\begin{solution}[\ref{3-cycl-an}]
    It is clear that for $n\geq 3$ as $A_n$ acts transitively. Let $\gen{c_i:i\in I}=A_n$ where $c_i$ are the $3$-cycles. Then for $n\geq 4$ the following condition must be satisfied by the $c_i$ ($i\in I$) for otherwise $\gen{c_i:i\in I}$ would not be transitive.

    For all subsets $I'\leq I$ such that $M:=\Setjoin_{i\in I'}{\spt{c_i}}\subset n$ there exists $j\in I\setminus I'$ such that $\spt{c_j}\setmeet M\neq \emptyset$ and $\spt{c_j}\not\subseteq M$.

    This implies inductively that $\card{\Setmeet_{i\in I'}{\spt{c_i}}}\leq 3+2(i-1)$. Thus it follows that $1+2\card{I}\geq n$ must hold. We claim that any set $\set{c_i:i\in I}$ satisfying the above condition necessary for transitity of the action of $\gen{c_i:i\in I}$ generates $A_n$.
    \paragraph{Case 1.}
    To prove this we first assume that there are two $3$-cycles intersecting in only one point. That is w.l.o.g. $\alpha:=c_1=(1\ 2\ 3)$ and $\beta:=c_2=(3\ 4\ 5)$. Then one computes that $\gamma:=\gcom{\alpha}{\beta^{-1}}=(2\ 3\ 4)$. Thus the induction hypothesis gives that $\gen{c_i:i\in I}\geq \gen{c_i,\gamma:i\in I\setminus\set{1}}\geq\stab[A_n](1)\iso A_{n-1}$.
    But the point stabilizer of a primitive action is a maximal subgroup and $\gen{c_i:i\in I}$ contains $c_1$ not fixing $1$. Thus in this case we are done.
    \paragraph{Case 2.} In this case we may assume that all cycles $c_i$ are of the form $c_i=(1\ 2\ (i+2))$ for $i=1,\ldots,n-2$.
    But again we see that $\gen{c_i:i=1,\ldots,n-2}\geq \gen{c_i:i=1,\ldots,n-3}=\stab[A_n](n)\iso A_{n-1}$ by induction hypothesis.
    Again $\gen{c_i:i\in I}$ contains a point stabilizer and an element $c_n$ not in this stabilizer. Thus it is $A_n$ again.    
\end{solution}    

\section{Group theory}

\subsection{The $p$-Sylow theorem}

\begin{theorem}[number of $p^e$-subgroups]
    Let $G$ be a finite group of order $\card{G}=n=mp^e$ and let $p^e\divides n$ for some prime $p$ and $m,e\in \nats$. Then the set of subgroups $S_{p^e}:=\set{P\leq G:\card{P}=p^e}$ satisfies
    \begin{eqnarray}
        \card{S_{p^e}(G)} &=\frac{1}{m}\binom{n}{p^e}=\binom{n-1}{p^e-1} \mod p
    \end{eqnarray}
\end{theorem}

\begin{proof}
    $G$ acts on the set $\cP$ of $p^e$-element subsets of $G$ by right multiplication. One then obtains for some $P\in\cP$ that $\Setjoin{P[\stab[G]{P}]}=P$ is a disjoint union of right cosets of $\stab[G]{P}$ yielding that $\stab[G]{P}\divides\card{P}=p^e$. Moreover, $P\in \cP$ is a subgroup of $G$ if and only if $\stab[G]{P}$ has order $p^e$.
    This can be shown by the following.

    If $P$ is a subgroup then $\stab[G](P)=P$ since by the previous argument $\stab[P]\subseteq P$ and $P$ stabilizes itself.
    On the other hand, if $\stab[G]{P}$ has the desired cardinality it follows that $\stab[G]{P}=P$ since both have cardinality $p^e$ and the first is contained in the latter. Thus $P$ is a subgroup of $G$ in this case.
    
    We now obtain by orbit-stabiliser theorem
    \begin{eqnarray}
        \card{\cP}=\binom{n}{p^e} &= \sum_{k}{\card{\orb[G]{P_k}}} = \sum_{k}{\card{G/\stab[G]{P_k}}}\\
        &= m(pl+\card{\cP}\setmeet\Sub{G})
    \end{eqnarray}
    for some number $l\in\nats$ as only the $G$-orbits of subgroups $P\in\cP$ have cardinality not divisible by $p$ and the cardinality of any $G$-orbit is divisible by $m$ (as we have seen above).
    Thus we get that
    \begin{equation}
        S_{p^e}(G)\cong\frac{1}{m}\binom{n}{p^e}=\binom{n-1}{p^e-1} \mod p.
    \end{equation}
\end{proof}

Another interesting general result concerning congruences modulo $p$ of the last type is the following

\begin{theorem}[Lucas]
    Let $m,n^j\in\ints$, $n={(n^j)}_{j=1}^l$ be a multiindex and $p$ be a prime. Morover let $\base{p}{m}=m_k\cdots n_0$, $\base{p}{n^j}=n^j_k\cdots n^j_0$ be the base $p$ expansions of $m$ and $n^j$, respectively. Then
    \begin{equation}
        \binom{m}{n}\cong \prod_{i=0}^k{\binom{m_i}{n_i}} \mod p
    \end{equation}
\end{theorem}

\begin{proof}
    We have that
    \[
    {\sum_{i=1}^l{X_j}}^m={\left(\sum_{j=1}^l{X_j}\right)}^{\sum_{i=0}^k{m_i p^i}}=\prod_{i=0}^k{{\left(\sum_{j=1}^l{X_j^{p^i}}\right)}^{m_i}}
    \]
    in $\ints/{p\ints}$ since the frobenius homomorphism $x\mapsto x^p$ is the identity (the $X_j$ are variables). Comparing the coefficients before $X^n=\prod_{j=1}^l{X^{n^j}_j}$ leeds to the desired result.
\end{proof}

\begin{theorem}[Sylow subgroups]
    Let $G$ be a finite group of order $n=mp^e$ with $p$ prime $e,m\in\nats$ and $p\not\divides m$.
    Then the number of $p$-Sylow-subgroups $S_{p^e}(G)$ satisfies
    $$
    \card{S_{p^e}(G)} \cong 1 \mod p\label{p-sylow-cong}
    $$
    $$
    \card{S_{p^e}(G)} \divides m\label{p-sylow-div}
    $$
    $$
    \card{S_{p^e}(G)} = \card{G/\normalisor[G]{P}}\label{p-sylow-nor},
    $$
    where $P\in S_{p^e}$.    
    Moreover, they are all conjugate to each other.
\end{theorem}

\begin{proof}
    \begin{description}
        \item[\ref{p-sylow-cong}:]
    By \textsc{Lucas}' theorem we have that $\binom{mp^e}{p^e}\cong\binom{m}{1} \mod p$ proving the first fact.
    \end{description}
\end{proof}
%

    
\subsection{Group actions}

\begin{definition}
    Let $G$ be a group and $\Omega$ be an algebra (or a set). We call a homomorphism $h:G\to\Aut(\Omega)$ a $G$-action on $\Omega$.
    Moreover, we call it a leftaction if $\omega^{h(g)}$ is denoted by $\omega g$ and a rightaction if it is denoted by $g \omega$.
\end{definition}

\begin{lemma}[orbit-stabiliser for transitive actions] Let $G$ act on $\Omega$ transitively und let $H\leq G$ such that $H$ contains the stabiliser of a point $\omega\in\Omega$ and acts transitively on $\Omega$. Then $H=G$.
\end{lemma}

\begin{proof}
    Let $g\in G$ then there exists $h\in H$ such that $\omega g=\omega h$ and thus $\omega gh^{-1}=\omega$ whence $gh^{-1}\in\stab[G](\omega)\leq H$. Thus $g\in H$.
\end{proof}
    

\begin{lemma}[orbit-stabiliser for primitive actions]\label{or-stab-ff-trans} Let $G$ act on $\Omega$ transitively.
    The the following are equivalent:
    \begin{statements}
            \item\label{prim-act} $G$ acts primitively on $\Omega$.
            \item\label{p-stab-max-subgr} When $S:=\stab[G](\omega)$ for some point $\omega\in\Omega$. Then $S$ is a maximal subgroup of $G$.
    \end{statements}
\end{lemma}

\begin{proof}
    \begin{implications}
        \item[$\ref{prim-act}\implies\ref{p-stab-max-subgr}$:]
    Assume that $H\geq S$. Then $\set{\omega Hg:g\in G}$ is a $G$-invariant partition of $\Omega$ since $\omega Hg\setmeet \omega Hg'$ implies that there is a $h,h'\in H$ such that $h'g'{(hg)}^{-1}\in \stab[G](\omega)=S\leq H$. This implies that $g'g^{-1}\in H$ or in other words $Hg=Hg'$. It thus follows that either $\omega Hg$ is a singleton or $\omega Hg=\Omega$ as $G$ acts primitively on $\Omega$ (for all $g\in G$).
    In the first case we have that $\omega H=\set{\omega}$ and thus $H\leq S$ implying that $H=S$.
    In the second case we have that $H$ acts transitively on $\Omega$ and thus $G=\gen{H,S}=HS=SH$.
    However, since $H\geq S$ we have that $HS=H=G$.
        \item[$\ref{p-stab-max-subgr}\implies\ref{prim-act}$:]
    If $S$ is maximal and $\sim\subseteq \Omega^2$ is a non-trivial congruence relation ($G$-invariant) then for we have that $H:=\stab[G](\omega/\sim)\neq 1$, since $\omega/\sim$ is not a singleton and $G$ acts transitively. Thus it follows that $H\not\leq S$ and thus $\gen{H,S}=G$ by maximality of $S$ in $G$.
    But since $\omega\in\omega/\sim H\setmeet \omega/\sim$ it follows that $S\leq H$ and thus $H=G$. But then $\orb[H](\omega)=\omega/\sim = \Omega$ since $G=H$ acts transitively. Thus $G$ must act primitively on $\Omega$. 
\end{implications}
\end{proof}
%
\begin{lemma} Let $G$ act on $\Omega$ via $\alpha:G\to\Aut(\Omega)$ and $N\conleq G$. Then $G/N$ acts on $\Omega/N$ via $\bar{\alpha}:G/N\to\Aut(\Omega/N)$ where $(\omega N)(gN):=(\omega g)N$.
\end{lemma}

\begin{proof}\label{bew}
    We have to show that the action is welldefined. This follows from $NgN=gN^gN=gN$.
\end{proof}

False:
\begin{lemma}[characterisation of primitive actions]
    Let $G$ act faithfully on $\Omega$.
    The following are equivalent
    \begin{statements}
            \item\label{nor-act-tr} Any normal subgroup $N\neq 1$ of $G$ acts transitively on $\Omega$.
            \item\label{min-nor-act-tr} Any minimal normal subgroup of $G$ acts transitively on $\Omega$.
            \item\label{sact-prim} $G$ acts primitively on $\Omega$.
    \end{statements}
\end{lemma}

\begin{proof}
    \begin{implications}
            \item[$\ref{sact-prim}\equival\ref{nor-act-tr}$:]
        Let $N\conleq G$ and let $G$ act on $\Omega$ primitively via $\phi:G\to\Aut(\Omega)$. Then $G$ acts on $\Omega/N$ via $G\to G/N\to \Aut(\Omega/N)$. Thus it follows that $\Omega/N=\Omega$ or $\Omega/N=\set{\set{\omega}}[\omega\in \Omega]$ as $G$ acts primitively on $\Omega$. In the first case we have that $N$ acts transitively on $\Omega$ and in the second case we have that $N=1$ since for all $n\in N$ it then holds that $\alpha(n)=\id_{\Omega}$ which implies by primitivity of the action of $G$ (i.e. injectivity of $\alpha$) that $N=1$.
        Conversely, if $\Omega/\sim\subseteq \Sub{\Omega}$ is a system of imprimitivity, then we have $G\to \Aut(\Omega)\to\Aut(\Omega/\sim)$ where the last arrow is induced by the natural projection map $\pi:\Omega\to\Omega/\sim$ with $\omega\mapsto \omega/\sim$. But as $\sim$ is not a trivial congruence we have that $\Aut(\pi)$ has as a kernel a normal subgroup $N\conleq \Aut(\Omega)$ with $1< N< \Aut(\Omega)$. Thus $1<N\meet \im(\alpha)\conleq \im(\alpha)$.
        To be continued.

        
            \item[$\ref{nor-act-tr}\equival\ref{min-nor-act-tr}$:]
        If any normal subgroup $N$ of $G$ acts transitively then especially every minimal one does. Conversely, any normal $N$ subgroup contains a minimal normal subgroup $M\conleq G$. Thus $N$ must act transitively if any minimal normal subgroup does.
\end{implications}
\end{proof}
%\ref[Proof]{bew}
%\ref[Lemma]{or-stab}


\begin{corollary} Let $G$ act primitively on $\Omega$ and let $N\conleq G$. Then $N$ acts transitively on $\Omega$.
\end{corollary}



\begin{lemma}[Dedekind's medular law] Let $J,K,L\leq G$ be groups and $J\leq L$. Then it holds that
    \begin{equation}
        J(K\meet L)=JK\meet L.
    \end{equation}
\end{lemma}

\begin{proof}

    \begin{implications}
            \item[$\supseteq$:] We have that $J(K\meet L)\leq J(JK\meet JL)\leq JK\meet L$.
            \item[$\subseteq$:] If $j\in J, k\in K$ such that $jk\in L$, then $k\in j^{-1}L=L$ (since $J\leq L$). Thus
        $jk\in J(K\meet L)$.
    \end{implications}
\end{proof}

\begin{lemma}[join with normal subgroup]
    Let $H,N\leq G$, $N\conleq G$. Then $H\join N=HN=NH$.
    If $H\meet N=1$ then $H\meet N\iso H\leftsemidirprod N$.
\end{lemma}

\begin{proof}
    Of course we must have $HN\leq H\meet N$.
    As $N$ is normal we have $nh=h(h^{-1}nh)=hn^h\in HN$ which shows that $NH=HN$ and thus $(HN)^2=H^2N^2=HN$ and $(HN)^{-1}=N^{-1}H^{-1}=NH=HN$ showing that $HN$ is already a group, thus $HN=NH=H\meet N$.

    If $H\meet N=1$, it follows that any element $g=hn\in HN$ with $h\in H$, $n\in N$ is unique in this representation as $hn=h'n'$ implies ${h'}^{-1}h=n'n^{-1}\in H\meet N=1$.
    Moreover, $H$ acts on $N$ via $\alpha:H\to\Aut(N)$ with $n^{\alpha(h)}:=h^{-1}nh$. Thus $\phi:H\leftsemidirprod[\alpha] N\to HN$ with $(h,n)\mapsto hn$ is an isomorphism of groups since $\phi((h,n)(h',n'))=\phi((hh',n^{h'}n'))=hh'n^{h'}n'=hnh'n'=\phi(h,n)\phi(h',n')$ and $\phi$ is bijective (as outlined before).
\end{proof}

\begin{lemma}[third isomorphism theorem, projection law]
    Let $H,N\leq G$ be groups and $N\conleq H\join N$ (that is $H\leq \normalisor[G](N)$). Then $H\meet N\conleq H$ and
    \begin{equation}
        (H\join N)/N=HN/N\iso H/(H\meet N).
    \end{equation}
\end{lemma}

\begin{proof}
We have that $N^h=N$ for all $h\in H$ thus $(N\meet H)^h=N\meet H$ for all $h\in H$. Thus $H/(H\meet N)$ is well defined and one easily verifies that $h(H\meet N)\mapsto hN$ is an isomorphism with inverse $hN\mapsto hN\setmeet H$.
\end{proof}

\begin{remark}
    The third isomorphism theorem for groups is basically a consequence of the more general version in universal algebra.
\end{remark}

%% exercises
\begin{lemma}[Gauss]
    Let $q$ be an odd prime power and $a\in{\field{q}}^{\ast}$. Show that $x^2=a$ has a solution in $x$ if and only if $a^{\frac{q-1}{2}}=1$.
\end{lemma}

\begin{proof}
    If $\xi$ is generator of ${(\field{q})}^{\ast}$ then $a=\xi^n$ for some $n\in\set{0,\ldots,q-2}$ we have $a=\xi^n$ and $q-1\divides n\frac{q-1}{2}$. Thus $2\divides n$ and we find the solutions $\pm\xi^{n/2}$. Conversely, when $a=x^2$ then $a^{\frac{q-2}{2}}=x^{q-1}=1$.
\end{proof}

\begin{exercise}
    Let $p,q$ odd primes with $p\neq q$. How many solutions $x$ has the equation $x^2=a\neq 0$ which are themselves a square if $a$ is a square.
\end{exercise}

\begin{solution}
    When $\xi$ is a generator of ${(\field{q})}^{\ast}$ then some solution $\mod p$ is $\xi^n$ and the other is $\xi^{n+\frac{p-1}{2}}$. If $p\cong 1\mod 4$ then both exponents are even or odd. Otherwise, one is even and one is odd. The same holding for $q$ one sees that there is exactly one square $x$ satisfying the equation if $p\cong q\cong 3\mod 4$, two or zero such $x$ if $p\not\cong q \mod 4$ and  zero or four such $x$ if $p\cong q\cong 1\mod 4$.
\end{solution}


\section{Ring theory}

\begin{lemma}[quotient prime ideal]
    Let $R$ be a commutative ring and $\fkp\in\con[R]$. Then the following two are equivalent
    \begin{statements}
            \item\label{quot-int-dom} $R/\fkp$ is an integral domain.
            \item\label{prim-id} $\fkp$ is a prime ideal.
    \end{statements}
\end{lemma}

\begin{proof}
    \begin{implications}
        \item[$\ref{prim-id}\implies\ref{quot-int-dom}$:]
    Let $\bar{a},\bar{b}\in R/\fkp$ such that $\bar{a}\bar{b}=0$. Then it follows that $ab\in\fkp$ and thus as $\fkp$ is a prime ideal that $a\in\fkp$ or $b\in\fkp$. But this implies that $\bar{a}=0$ or $\bar{b}=0$.
        \item[$\ref{quot-int-dom}\implies\ref{prim-id}$:]
    When $R/\fkp$ is an integral domain and $\bar{a},\bar{b}\in\con(R)$ such that $\bar{a}\bar{b}=0$ then $\bar{a}=0$ or $\bar{b}=0$ or equivalently $a\in\fkp$ or $b\in\fkp$.
\end{implications}
\end{proof}

\begin{lemma}[quotient maximal ideal]
    Let $R$ be a commutative ring and $\fkm$ an ideal. Then the following two are equivalent
    \begin{statements}
            \item\label{quot-field} $R/\fkm$ is a field.
            \item\label{max-id} $\fkm$ is a maximal ideal.
    \end{statements}
\end{lemma}

\begin{proof}
    \begin{implications}
            \item[$\ref{quot-field}\implies\ref{max-id}$:]
        When $R/\fkm$ is a division ring and $a\not\in\fkm$ then $\bar{a}\neq 0$ so it is a unit. Thus $\gen{\bar{a}}=\bar{a}R/\fkm=R/\fkm$ implying that $\gen{\fkm,a}=R$ so $\fkm$ is maximal.
            \item[$\ref{max-id}\implies\ref{quot-field}$:]
        When $\fkm$ is maximal and $0\neq \bar{a}\in R/\fkm$ then $a\not\in \fkm$ meaning that $\gen{a,\fkm}=aR+\fkm=R$ as $\fkm$ is maximal. This implies that there exist $b\in R$ such that $ab+\fkm=1+\fkm$. Thus $\bar{b}$ is the inverse of $\bar{a}$ in $R/\fkm$.
    \end{implications}
\end{proof}
    
    
\begin{lemma}
    Let $\fkm\in\con(R)$ be a maximal ideal of a commutative ring $R$. Then $\fkm$ is a prime ideal.
\end{lemma}

\begin{proof}
    Let $a,b\in R$ with $ab\in \fkm$. Assume that $a,b\not\in\fkm$ then $aR+\fkm=bR+\fkm=R$ so $abR+a\fkm+b\fkm+\fkm\fkm=R\leq \fkm$ a contradiction. 
\end{proof}

\section{Partial fractions}

\begin{lemma}[partial fraction representation, principal ideal domain] Let $R$ be a principal ideal domain and let $Q:=\quotring(R)$ its quotient field. Then any element of $f=f_0{(\prod_{i=1}^n{p_i^{\nu_i}})}^{-1}\in Q$ ($f_0,p_i\in R$, $f_0\meet p_i=1$, $\nu_i>0$, $i=1,\ldots,n$) can be written as
    $$
    f = \sum_{i=1}^n{\frac{a_i}{p_i^{\nu_i}}}
    $$
    where $a_0,a_i\in R$ and $a_i\meet p_i=1$ ($i=1,\ldots,n$).
\end{lemma}

\begin{proof}
    The proof is given by the fact that
    $$
    \Join_{i=1}^n{\gen[\con]{\prod_{j\neq i}{p_j^{\nu_j}}}}=R=\gen[\con]{1}
    $$
    since $R$ is a factorial domain whence prime elements $p_i$ are also irreducible.
    Interpreting this equation in the fractional field and multiplying by $\prod_{i=1}^n{p_i^{\nu_i}}$ we get the desired fact since in a PID
        $$
        \Join_{i=1}^n{\fka_i}=\sum_{i=1}^n{\fka_i}.
        $$
\end{proof}

\begin{remark}[uniqueness]
    Note that there is no statement about uniqueness of the partial fraction representation in PIDs. In fact, it is immediately clear that without an additional restriction the above representation is not unique in most cases.
    But the image $\bar{a_i}$ under the natural map $\pi_i:R\to R/\gen[\con]{p_i^{\nu_i}}$ is unique for $i=1,\ldots,n$.
This can be seen by multiplying the partial fraction representation by $\prod_{i=1}^n{p_i^{\nu_i}}$ and then taking the image under the above map (modulo $p_i^{\nu_i}$). This yields $\pi_i(\prod_{j\neq i}{p_j^{\nu_j}}a_i)$. But since $R$ is an integral domain and $p_j, p_i$ coprime for $i\neq j$, this already uniquely determines $\pi_i(a_i)=\bar{a_i}$.
\end{remark}


In most cases we deal with polynomials when talking about partial fractions.

\begin{lemma}
    Let $p,q\in K(X)$ be coprime for an algebraically closed field $K$.
    $$
    p/q=
    $$
    and the $a_{ij}$ are given by the formula
\end{lemma}

\section{Ordered monoids and metric spaces}

\begin{definition}[ordered monoid]
    A structure $A=\struct{M,1,\cdot,\leq}$ where $\struct{M,1,\cdot}$ is a monoid and $\leq\setleq M\setprod M$ is an order being compatible with the multiplication, i.e.
    $$
    \leq \in \Inv(\cdot)
    $$
    or explicitely
    $$
    (a\leq b) \lgand (c\leq d) \implies (ac\leq bd)\lgand (ca\leq db).
    $$
    If 1 is minimal in $M$ we say that $M$ is an \keyword{upright monoid}.
\end{definition}

\begin{definition}[ideal in an upright monoid]
    Let $A$ be an upright monoid. Then a subset $\fkb\setleq A$ is called an order ideal if $A\fkb=\fkb A=\fkb$ (absorbative), $a\fkb=\fkb a$ (normal) and it holds that $a\leq b \implies a\in \fkb$ (absorbative in order sense) for $b\in \fkb$, $a\in A$.
\end{definition}

\begin{lemma}[representation of the generated ideal]
    Let $B\setleq A$ a set. Then the upright ideal generated by $B$ is $\Setjoin_{n\in\nats}{{(AB)}^n}$.
\end{lemma}

\begin{lemma}[first homomorphism theorem for upright monoids]
    Let $h:A\to B$ be a homomorphism of ordered monoids then
    $\ker{h}=h^{-1}[0]$ is an ideal. Moreover, $A/\ker{h}\iso \im{h}$
\end{lemma}

\begin{proof}
    Clearly, when $a\in A$, $b\in\ker{h}$ we have that $ab,ba\in\ker{h}$ since $h(ab)=h(ba)=h(b)=1$. Moreover, when $a\leq b$ (for the same $a$ and $b$), then $h(a)\leq h(b)=1$ implying that $h(a)=1$ since $1$ is minimal. The rest follows by general isomorphism theorem.
\end{proof}

\begin{definition}[unit in upright monoid]
    Let $A$ be an upright monoid and $a\in A$. Then $a$ is called a \keyword{order unit} if $\gen[\con]{a}=A$ (unit if also unit in the monoid sense). We write $\units{A}$ for all these elements of $A$.
\end{definition}

\begin{remark}
    The units $\units{A}$ form a monoid as for $a,b\in\units{A}$ we have $ab=aAb=$ ...
\end{remark}

\begin{definition}[$A$-metric space]
    An $A$-metric space where $A$ is an upright abelian monoid is a structure $M=\struct{S,d}$ where $S$ is a structure which bases on $\Set$ and $d:S\setprod S\to A$ satisfies
    \begin{align}
        d(a,b) &\leq d(a,c)+d(c,b),\\
        d(a,b) &=d(b,a),\\
        d(a,a) &=0.
    \end{align}
    for $a,b,c\in M$.
\end{definition}

\begin{lemma}[Lebesgue]
    Let $K$ be a compact $A$-metric space and $\cU$ be an open cover of $K$, i.e. $\Setjoin\cU=K$. Then there exists an $\epsilon>0$ such that for any $k\in K$ the ball $\ball_{\epsilon}(k)$ lies entirely in some set $U\in \cU$.
\end{lemma}

\begin{proof}
    Define $\cU':=\set{\ball_{\epsilon}(k):\epsilon>0\lgand\exists U\in\cU:\ball_{2\epsilon}(k)\setleq U}$. Then $\cU'$ also covers $K$ since for $k\in K$ there is a $U\in\cU$ with $k\in U$ and $U$ is open in $k$. Now, choose a finite subcover $\cF$ of $\cU'$. Then there exists a minimal radius $\epsilon$ among the balls in $\cF$. Now, for all balls $\ball_{\epsilon}(k)$ in $\cU$ we find a ball $\ball_{\epsilon}(k')\in\cF$ containing $k$. This means that $\ball_{\epsilon}(k)\setleq\ball_{2\epsilon}(k')$. But for this last ball we know by the definition of $\cU$ that it is enterily contained in some set $U\in\cU$.
    This shows that the desired for this $\epsilon>0$.
\end{proof}

\begin{lemma}[sequencial compactness]

\end{lemma}

\begin{proof}
    Let $\cU$ be an open cover of $M$ such that $\cU$ has no finite subcovers. By the previous lemma, we know that there is a number $\epsilon>0$ such that the cover $\cU':=\set{\ball_{\epsilon}(k):k\in K}$ has the property that for any ball $B\in \cU'$ there exists $U\in\cU$ such that $B\setleq U$. Thus $\cU'$ cannot have a finite subcover, too, since otherwise $\cU$ would admit one. From this we see that there is a sequence of balls $\net[n\in\nats]{\ball_{\epsilon}(k_n)}$ of $\cU'$ such that $k_n\not\in \ball_{\epsilon}(k_i)$ for $i< n$. But then the sequence $\net[n\in\nats]{k_n}$ cannot have a convergent subsequence since the distance between any two distinct points of it is at least $\epsilon$. Thus $M$ is not sequencially compact.
\end{proof}



\section{Euclidean geometry}

\paragraph{Isoperimetric inequality in two dimensions.}

\begin{lemma}
    Among all Jordan curves $C$ in the plane of a given length circles enclose the largest area.
\end{lemma}

\begin{proof}
    Let $C$ be a jordan curve of finite length then $\bound{\conv{C}}$ is a curve of at most the length of $C$ prescribing at least the area inside $C$.
    Thus we may assume that $C$ is the boundary of convex set (and thus differenciable nearly everywhere). For point $x,y \in C$ we then may choose points $w,z$ such that $xwyz$ is a kite. We can than modify that kite leaving its side length fixed such that its area is maximal moving the bows of $C$ above the sides of $xwyz$ with them. This clearly happens when $\angle zxw=\angle wyz=\pi/2$. ... From this one gets that any three points of $C$ lie on a circle.
\end{proof}


\begin{lemma}[\person{Steiner}'s equations]
    Let $P\setleq \reals^n$ be a convex polytope. Then it holds that
    $$
    \vol_{\reals^n}{(P+\epsilon\ball^n)}=\vol_{\reals^n}(P)+\sum_{i=1}^n{\epsilon^i\sum_{f\in F_i(P)}{\vol_{\reals^{n-i}}(f)\frac{\vol_{\sphere^i}(\alpha_f)}{i}}}
    $$
    and
    $$
    \vol_{\reals^n}{(\bound(P+\epsilon\ball^n))}=\sum_{i=1}^{n}{\epsilon^{i-1}\sum_{f\in F_i(P)}{\vol_{\reals^{n-i}}(f)\vol_{\sphere^i}(\alpha_f)}}=\rest{\pderive{\vol_{\reals^n}(P+\ball_r)}{r}}_{\epsilon},
    $$
    where $F_i(P)$ means the $i$-dimensional faces of $P$ and $\alpha_f$ means the spherical polytope associated to the face $f$.
\end{lemma}

\begin{lemma}[sum of exerior angles of convex polytope]
    For a convex polytope $P\setleq\reals^n$ it holds that
    $$
    \sum_{f\in F_{n-1}(P)}\alpha_f=\sphere^n
    $$
    (in the sense of elementary geometric addition).
\end{lemma}


\begin{corollary}[isoperimetric inequality for convex polytopes]
    Let $P$ be a convex polytope then
    $$
    \frac{{(\vol_{\reals^{n-1}}(\bound{P}))}^n}{{(\vol_{\reals^n}(P))}^{n-1}}\geq \frac{{(\vol_{\reals^{n-1}}(\sphere^{n-1}))}^n}{{(\vol_{\reals^n}(\ball^n))}^{n-1}}.
    $$
\end{corollary}

\begin{proof}
    We may assume
    $$
    \frac{\vol_{\reals^{n-1}}(\bound{P})}{\vol_{\reals^n}(P)}=\frac{\vol_{\reals^{n-1}}(\sphere^{n-1})}{\vol_{\reals^n}(\ball^n)}=n.
    $$
    as the inequality is invariant under scaling $P$ by scalars.

    Interpolation between $P$ and the $n$-ball know leads to
    $$
    \frac{\vol_{\reals^n}{(\bound(P+\epsilon\ball^n))}}{\vol_{\reals^n}{(P+\epsilon\sphere^{n-1})}}=
    $$
\end{proof}


\begin{lemma}
    Let $X$ be a topological space and $\cC(X)$ be the topological space defined by $\cC(X):=\set{Y\setleq X:Y\text{ is compact}}$ equipped with the topology $\tau_{\cC(X)}:=\set{\set{x\in\cC(X):x\setleq U}}[U\in\tau_X]$.
\end{lemma}

\textbf{TEST:}

%\begin{tikzcd}
%    \dirsum\limits_{i\in I}{A_i}\arrow[r,shift left,"\cproj_i"]\arrow[d,dashed,shift left]&
%    A_i\arrow[l,shift left,"\cincl_i"]\arrow[ld,shift left,"\iota_i"]\\
%    A'\arrow[ru,shift left,"\pi_i"]\arrow[u,dashed,shift left,"\exists !"]&
%    \\
%\end{tikzcd}

where $\cproj_i\compose\cincl\nolimits_j=\pi_i\compose\iota_j=\delta_{ij}$


%\def\interval#1#2#3{#1.#2.#3}
%$\intervaloo{a}{n}\intervaloo{.}{.}\interval$
\printindex

\makeatletter
\newcommand{\newdecl}[2]{\csgdef{decl@#1}{#2}}% Creates a declaration
\newcommand{\csvdel}{}% Delimiter used in CSV representation
\newcommand{\newusedecl}[2][,]{% Use a declaration
  \renewcommand{\csvdel}{\renewcommand{\csvdel}{#1\,}}% Delay \csvdel one cycle.
  \csname decl@#2\endcsname(\checknextarg}
\newcommand{\checknextarg}{\@ifnextchar\bgroup{\gobblenext}{}}% Check if another "argument" exists
\newcommand{\gobblenext}[1]{\csvdel#1aaa\@ifnextchar\bgroup{\gobblenext}{)}}% Gobble next "argument"
\makeatother
% declare some interface routines
\newdecl{foo}{FOO}
\newdecl{bar}{BAR}


\newusedecl{foo}{p1}{p2}{p3}\par
\newusedecl{bar}{p1}{p2}{p3}{p4}{p1}{p2}{p3}{p4}{p1}{p2}{p3}{p4}{p1}{p2}{p3}{p4}
  {p1}{p2}{p3}{p4}\par
\newusedecl[;]{foo}{p1}{p2}{p3}{p4}{p1}{p2}{p3}{p4}{p1}{p2}{p3}{p4}{p1}{p2}{p3}{p4}
\DeclareDocumentCommand \foo { O{} m }{$Hallo^{#1}_{#2}$}
\foo{fdg}{a1}
sd

%\test[s]
\def\arg#1{\cdot^{\small #1}}
$\small\homl[\arg{1}] {(345+354)}^{435}$

\paragraph{Test.} \blindtext
\end{document}


%%% Local Variables
%%% mode: latex
%%% End: