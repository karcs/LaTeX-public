\documentclass[10pt,a4paper]{article}
%My personal maths package
%\bibliography{Bibliography.bib}

%%%%%%%%%%%%%%%%%%%%%%%%%%%%%%%%%%%%%%%%%%%%%%%%%%%%%%%%%%%%%%%%%%%%%%%%%%%%%%
%%%%% MATH PACKAGES %%%%%%%%%%%%%%%%%%%%%%%%%%%%%%%%%%%%%%%%%%%%%%%%%%%%%%%%%%
%%%%%%%%%%%%%%%%%%%%%%%%%%%%%%%%%%%%%%%%%%%%%%%%%%%%%%%%%%%%%%%%%%%%%%%%%%%%%%

% very good package
%\usepackage{mathtools}

%%% font stuff
\usepackage[T1]{fontenc}        % for capitals in section /paragraph etc.
\usepackage[utf8]{inputenc}     % use utf8 symbols in code

\usepackage{amssymb,amsmath,amsfonts} % amsthm not needed -- use my own envs
%\usepackage{mathtools}
%\mathtoolsset{showonlyrefs}
% further alternative math packages: unicode-math, abx-math
\usepackage{bm}
\usepackage{mathrsfs} % used for: fraktal math letters
\usepackage[bigsqcap]{stmaryrd} % used for: big square cap symbol
\usepackage{xargs}

% standard packages
%\usepackage{color} % for color

%%
%%index

\newcommand*{\keyword}[2][\empty]{\emph{#2}\ifx#1\empty\index{#2}\else\index{#1}\fi}
\newcommand*{\person}[1]{\textsc{#1}}

%%%%%%%%%%%%%%%%%%%%%%%%%%%%%%%%%%%%%%%%%%%%%%%%%%%%%%%%%%%%%%%%%%%%%%%%%%%%%%%%%%%%%%%%%%%%%%%%%%%%%%%%%%%%
%%%%% MATH ALPHABETS & SYMBOLS %%%%%%%%%%%%%%%%%%%%%%%%%%%%%%%%%%%%%%%%%%%%%%%%%%%%%%%%%%%%%%%%%%%%%%%%%%%%%
%%%%%%%%%%%%%%%%%%%%%%%%%%%%%%%%%%%%%%%%%%%%%%%%%%%%%%%%%%%%%%%%%%%%%%%%%%%%%%%%%%%%%%%%%%%%%%%%%%%%%%%%%%%%

% w: http://milde.users.sourceforge.net/LUCR/Math/math-font-selection.xhtml

% ===== Set quick commands for math letters ================================================================
% calagraphic letters (only upper case available; standard)
\newcommand{\cA}{\mathcal{A}}
\newcommand{\cB}{\mathcal{B}}
\newcommand{\cC}{\mathcal{C}}
\newcommand{\cD}{\mathcal{D}}
\newcommand{\cE}{\mathcal{E}}
\newcommand{\cF}{\mathcal{F}}
\newcommand{\cG}{\mathcal{G}}
\newcommand{\cH}{\mathcal{H}}
\newcommand{\cI}{\mathcal{I}}
\newcommand{\cJ}{\mathcal{J}}
\newcommand{\cK}{\mathcal{K}}
\newcommand{\cL}{\mathcal{L}}
\newcommand{\cM}{\mathcal{M}}
\newcommand{\cN}{\mathcal{N}}
\newcommand{\cO}{\mathcal{O}}
\newcommand{\cP}{\mathcal{P}}
\newcommand{\cQ}{\mathcal{Q}}
\newcommand{\cR}{\mathcal{R}}
\newcommand{\cS}{\mathcal{S}}
\newcommand{\cT}{\mathcal{T}}
\newcommand{\cU}{\mathcal{U}}
\newcommand{\cV}{\mathcal{V}}
\newcommand{\cW}{\mathcal{W}}
\newcommand{\cX}{\mathcal{X}}
\newcommand{\cY}{\mathcal{Y}}
\newcommand{\cZ}{\mathcal{Z}}

% bold math letters (standard)
\newcommand{\bfA}{\mathbf{A}}
\newcommand{\bfB}{\mathbf{B}}
\newcommand{\bfC}{\mathbf{C}}
\newcommand{\bfD}{\mathbf{D}}
\newcommand{\bfE}{\mathbf{E}}
\newcommand{\bfF}{\mathbf{F}}
\newcommand{\bfG}{\mathbf{G}}
\newcommand{\bfH}{\mathbf{H}}
\newcommand{\bfI}{\mathbf{I}}
\newcommand{\bfJ}{\mathbf{J}}
\newcommand{\bfK}{\mathbf{K}}
\newcommand{\bfL}{\mathbf{L}}
\newcommand{\bfM}{\mathbf{M}}
\newcommand{\bfN}{\mathbf{N}}
\newcommand{\bfO}{\mathbf{O}}
\newcommand{\bfP}{\mathbf{P}}
\newcommand{\bfQ}{\mathbf{Q}}
\newcommand{\bfR}{\mathbf{R}}
\newcommand{\bfS}{\mathbf{S}}
\newcommand{\bfT}{\mathbf{T}}
\newcommand{\bfU}{\mathbf{U}}
\newcommand{\bfV}{\mathbf{V}}
\newcommand{\bfW}{\mathbf{W}}
\newcommand{\bfX}{\mathbf{X}}
\newcommand{\bfY}{\mathbf{Y}}
\newcommand{\bfZ}{\mathbf{Z}}
\newcommand{\bfa}{\mathbf{a}}
\newcommand{\bfb}{\mathbf{b}}
\newcommand{\bfc}{\mathbf{c}}
\newcommand{\bfd}{\mathbf{d}}
\newcommand{\bfe}{\mathbf{e}}
\newcommand{\bff}{\mathbf{f}}
\newcommand{\bfg}{\mathbf{g}}
\newcommand{\bfh}{\mathbf{h}}
\newcommand{\bfi}{\mathbf{i}}
\newcommand{\bfj}{\mathbf{j}}
\newcommand{\bfk}{\mathbf{k}}
\newcommand{\bfl}{\mathbf{l}}
\newcommand{\bfm}{\mathbf{m}}
\newcommand{\bfn}{\mathbf{n}}
\newcommand{\bfo}{\mathbf{o}}
\newcommand{\bfp}{\mathbf{p}}
\newcommand{\bfq}{\mathbf{q}}
\newcommand{\bfr}{\mathbf{r}}
\newcommand{\bfs}{\mathbf{s}}
\newcommand{\bft}{\mathbf{t}}
\newcommand{\bfu}{\mathbf{u}}
\newcommand{\bfv}{\mathbf{v}}
\newcommand{\bfw}{\mathbf{w}}
\newcommand{\bfx}{\mathbf{x}}
\newcommand{\bfy}{\mathbf{y}}
\newcommand{\bfz}{\mathbf{z}}

% fractal math letters (standard)
\newcommand{\fkA}{\mathfrak{A}}
\newcommand{\fkB}{\mathfrak{B}}
\newcommand{\fkC}{\mathfrak{C}}
\newcommand{\fkD}{\mathfrak{D}}
\newcommand{\fkE}{\mathfrak{E}}
\newcommand{\fkF}{\mathfrak{F}}
\newcommand{\fkG}{\mathfrak{G}}
\newcommand{\fkH}{\mathfrak{H}}
\newcommand{\fkI}{\mathfrak{I}}
\newcommand{\fkJ}{\mathfrak{J}}
\newcommand{\fkK}{\mathfrak{K}}
\newcommand{\fkL}{\mathfrak{L}}
\newcommand{\fkM}{\mathfrak{M}}
\newcommand{\fkN}{\mathfrak{N}}
\newcommand{\fkO}{\mathfrak{O}}
\newcommand{\fkP}{\mathfrak{P}}
\newcommand{\fkQ}{\mathfrak{Q}}
\newcommand{\fkR}{\mathfrak{R}}
\newcommand{\fkS}{\mathfrak{S}}
\newcommand{\fkT}{\mathfrak{T}}
\newcommand{\fkU}{\mathfrak{U}}
\newcommand{\fkV}{\mathfrak{V}}
\newcommand{\fkW}{\mathfrak{W}}
\newcommand{\fkX}{\mathfrak{X}}
\newcommand{\fkY}{\mathfrak{Y}}
\newcommand{\fkZ}{\mathfrak{Z}}
\newcommand{\fka}{\mathfrak{a}}
\newcommand{\fkb}{\mathfrak{b}}
\newcommand{\fkc}{\mathfrak{c}}
\newcommand{\fkd}{\mathfrak{d}}
\newcommand{\fke}{\mathfrak{e}}
\newcommand{\fkf}{\mathfrak{f}}
\newcommand{\fkg}{\mathfrak{g}}
\newcommand{\fkh}{\mathfrak{h}}
\newcommand{\fki}{\mathfrak{i}}
\newcommand{\fkj}{\mathfrak{j}}
\newcommand{\fkk}{\mathfrak{k}}
\newcommand{\fkl}{\mathfrak{l}}
\newcommand{\fkm}{\mathfrak{m}}
\newcommand{\fkn}{\mathfrak{n}}
\newcommand{\fko}{\mathfrak{o}}
\newcommand{\fkp}{\mathfrak{p}}
\newcommand{\fkq}{\mathfrak{q}}
\newcommand{\fkr}{\mathfrak{r}}
\newcommand{\fks}{\mathfrak{s}}
\newcommand{\fkt}{\mathfrak{t}}
\newcommand{\fku}{\mathfrak{u}}
\newcommand{\fkv}{\mathfrak{v}}
\newcommand{\fkw}{\mathfrak{w}}
\newcommand{\fkx}{\mathfrak{x}}
\newcommand{\fky}{\mathfrak{y}}
\newcommand{\fkz}{\mathfrak{z}}

% script math symbols (only uppercase; package: mathrsfs)
\newcommand{\sA}{\mathscr{A}}
\newcommand{\sB}{\mathscr{B}}
\newcommand{\sC}{\mathscr{C}}
\newcommand{\sD}{\mathscr{D}}
\newcommand{\sE}{\mathscr{E}}
\newcommand{\sF}{\mathscr{F}}
\newcommand{\sG}{\mathscr{G}}
\newcommand{\sH}{\mathscr{H}}
\newcommand{\sI}{\mathscr{I}}
\newcommand{\sJ}{\mathscr{J}}
\newcommand{\sK}{\mathscr{K}}
\newcommand{\sL}{\mathscr{L}}
\newcommand{\sM}{\mathscr{M}}
\newcommand{\sN}{\mathscr{N}}
\newcommand{\sO}{\mathscr{O}}
\newcommand{\sP}{\mathscr{P}}
\newcommand{\sQ}{\mathscr{Q}}
\newcommand{\sR}{\mathscr{R}}
\newcommand{\sS}{\mathscr{S}}
\newcommand{\sT}{\mathscr{T}}
\newcommand{\sU}{\mathscr{U}}
\newcommand{\sV}{\mathscr{V}}
\newcommand{\sW}{\mathscr{W}}
\newcommand{\sX}{\mathscr{X}}
\newcommand{\sY}{\mathscr{Y}}
\newcommand{\sZ}{\mathscr{Z}}

%%%%%%%%%%%%%%%%%%%%%%%%%%%%%%%%%%%%%%%%%%%%%%%%%%%%%%%%%%%%%%%%%%%%%%%%%%%%%%%%%%%%%%
%%%% BOLD MATH IN BOLD TEXT ENVIRONMENT %%%%%%%%%%%%%%%%%%%%%%%%%%%%%%%%%%%%%%%%%%%%%%
%%%%%%%%%%%%%%%%%%%%%%%%%%%%%%%%%%%%%%%%%%%%%%%%%%%%%%%%%%%%%%%%%%%%%%%%%%%%%%%%%%%%%%

% for bold math in bold text (e.g. sections)
\makeatletter
\g@addto@macro\bfseries{\boldmath}
\makeatother

\def\brackets#1{\ifx#1\empty\else\left(#1\right)\fi}

%%%%%%%%%%%%%%%%%%%%%%%%%%%%%%%%%%%%%%%%%%%%%%%%%%%%%%%%%%%%%%%%%%%%%%%%%%%%%%%%%%%%%%
%%%%% CATEGORY THEORY %%%%%%%%%%%%%%%%%%%%%%%%%%%%%%%%%%%%%%%%%%%%%%%%%%%%%%%%%%%%%%%%
%%%%%%%%%%%%%%%%%%%%%%%%%%%%%%%%%%%%%%%%%%%%%%%%%%%%%%%%%%%%%%%%%%%%%%%%%%%%%%%%%%%%%%

% ===== Category theory concepts =====================================================

\newcommand{\Ob}{\mathop\mathrm{Ob}}
\newcommand{\Mor}{\mathop\mathrm{Mor}}

\newcommand{\ccoprod}{\bigsqcup}
\newcommand{\cprod}{\bigsqcap}
\newcommand{\cincl}{\mathop\mathrm{incl}}
\newcommand{\cproj}{\mathop\mathrm{pr}}


% ===== Define standard categories ===================================================

% Define sets
\newcommand{\Set}{\mathbf{Set}}
% Define set-builder operator (equivalent to gen for algebras)
\newcommand{\set}[1]{\left\{#1\right\}}
% define interval operator: o - open, c - closed
\newcommand{\intervalcc}[2]{\left[#1,#2\right]}
\newcommand{\intervalco}[2]{\left[#1,#2\right)}
\newcommand{\intervaloc}[2]{\left(#1,#2\right]}
\newcommand{\intervaloo}[2]{\left(#1,#2\right)}

\newcommand{\inter}{\mathop\mathrm{int}}
\newcommand{\face}{\mathop\mathrm{F}}
\newcommand{\Pol}{\mathop\mathrm{Pol}}
\newcommand{\Inv}{\mathop\mathrm{Inv}}
\def\struct#1{\gen{#1}}

\let\originaltimes\times%
\renewcommand{\times}{\mathbin{\sqcap}}
\newcommand\settimes{\originaltimes}
\newcommand{\setleq}{\subseteq}
\newcommand{\setgeq}{\supseteq}

\newcommand{\pderive}[2]{\frac{\partial{#1}}{\partial{#2}}}
\renewcommand{\div}{\mathop\mathrm{div}}

%% Diffgeo
\def\Ric{\mathop\mathrm{Ric}}
\def\ric{\mathop\mathrm{ric}}
\def\tr{\mathop\mathrm{tr}}

\def\cotimes{\mathbin{\sqcup}}

\def\@rightopen#1{\ifx#1]{\right]}\else{\interval@errmessage}\fi}
\def\@leftclosed[#1){\left[#1\right)}
\makeatother

% finite
\newcommand{\fin}{\mathrm{fin}}

% Define groups (optarg: properties such as -> abelian, noetherian (acc), artinian (dcc) etc.)
\newcommand{\Grp}[1][\empty]{\if\empty{#1}{\mathbf{Grp}}\else{\mathbf{Grp}_{#1}}}
\def\PGL{\mathrm{PGL}}
\def\PGammaL{\mathrm{P\Gamma L}}
\def\GL{\mathrm{GL}}
% Define rings
\newcommand{\rg}{\mathrm{rg}} %rank of a matrix
\newcommand{\Rg}[1][\empty]{\if\empty{#1}{\mathbf{Rg}}\else{\mathbf{Rg}_{#1}}}
\edef\units#1{#1^{\settimes}}
\def\dual#1{#1^{\ast}}

%% redefine the command \P to produce the projective functor in math mode
\let\parsymb\P%
\def\P{\ifmmode\mathrm{P}\else\parsymb\fi}
\renewcommand{\iff}{\ifmmode\equival\else{if and only if}\fi}
\newcommand{\quotring}{\mathop{\mathrm{Q}}}
\newcommand{\rad}{\mathrm{rad}}
% Standard rings
% integral domains
\newcommand{\ID}{\mathbf{ID}}
% unique factorization domains
\newcommand{\UFD}{\mathbf{UFD}}
% principal ideal domains
\newcommand{\PID}{\mathbf{PID}}

% Define modules over a group or ring
\newcommand{\Mod}[1]{\mathbf{Mod}_{#1}}
% Define vector space over a field
\renewcommand{\Vec}[1]{\mathbf{Vec}_{#1}}%

% when cases
\def\otherwise{\textrm{otherwise}}

% new concepts
\newcommand{\new}[1]{\emph{#1}}

\usepackage{xifthen,xstring}

% replace the bar command by overline when argument just one character (shorter and better)
%$\let\oldbar\bar
%\renewcommand{\bar}[1]{\StrLen{#1}[\length]\ifthenelse{\length > 1}{\overline{#1}}{\oldbar{#1}}}
\def\bar{\overline}

%% argument in equatoin
\def\arg{\bullet}

% groups and algebras
\newcommand{\Con}{\mathop\mathrm{Con}}
\newcommand{\Sub}{\mathop\mathrm{Sub}}
\newcommand{\Hom}{\mathop\mathrm{Hom}}
\newcommand{\Aut}{\mathop\mathrm{Aut}}
\newcommand{\Out}{\mathop\mathrm{Out}}
\newcommand{\End}{\mathop\mathrm{End}}
\newcommand{\id}{\mathop\mathrm{id}}
\newcommand{\rk}{\mathop\mathrm{rk}} % rank of a group module/ lattice
\newcommandx{\con}[1][1=\empty]{\ifx#1\empty{\mathop{\mathrm{con}}}\else{\mathop{\mathrm{con}}\left(#1\right)}\fi}
\newcommand{\leftsemidirprod}[1][]{\mathbin{\ifx&#1&\ltimes\else{\ltimes_{#1}}\fi}}
\newcommand{\rightsemidirprod}[1][]{\ifx#1\empty\rtimes\else{\rtimes_{#1}}\fi}
\newcommand{\normalisor}[2][]{\ifx#1\empty{\mathrm{N}\left(#2\right)}\else{\mathrm{N}_{#1} \left(#2\right)}\fi}
% support
\newcommand{\spt}{\mathop{\mathrm{spt}}}
% commutator
\newcommand{\gcom}[2]{\left[#1,#2\right]}

% physics stuff
\newcommand{\float}[3][\empty]{\ifx#1\empty{{#2}\cdot{10^{#3}}}\else{{#2}\cdot{{#1}^{#3}}}\fi}
\makeatletter
\def\newunit#1{\@namedef{#1}{\mathrm{#1}}}
\def\mum{\mathrm{\mu m}}
\def\ohm{\Omega}
\newunit{V}
\newunit{mV}
\newunit{kV}
\newunit{s}
\newunit{ms}
\def\mus{\mathrm{\mu s}}
\newunit{m}
\newunit{nm}
\newunit{cm}
\newunit{mm}
\newunit{fF}
\newunit{A}

\newunit{fA}
\newunit{C}

% elements
\def\newelement#1{\@namedef{#1}{\mathrm{#1}}}
\newelement{Si}
\makeatother


% groups
\newcommand{\ord}{\mathop\mathrm{ord}}
\newcommand{\divides}{|}

\newcommand{\conleq}{\trianglelefteq}
\newcommand{\congeq}{\trianglerighteq}

% common algebraic objects
\newcommand{\reals}{\mathbb{R}} 			% real numbers
\newcommand{\nats}{\mathbb{N}} 				% natural numbers
\newcommand{\ints}{\mathbb{Z}} 				% integers
\newcommand{\rats}{\mathbb{Q}}				% rationals
\newcommand{\complex}{\mathbb{C}}			% complex numbers
\newcommand{\field}[1]{\mathbb{F}_{#1}}  		% finite field
\newcommand{\cards}{\boldsymbol{Cn}}                     % The cardinal numbers
\newcommand{\ords}{\boldsymbol{On}}                      % The ordinal numbers

% graphs
\def\KG{\mathop\mathrm{KG}}                     % Knesergraph

\newcommand{\uvect}{\boldsymbol{e}}

% new operators and relations

%%%%%%%%%%%%%%%%%%%
% complex numbers %
%%%%%%%%%%%%%%%%%%%

\renewcommand{\Re}{\mathop\mathrm{Re}}		% real part
\renewcommand{\Im}{\mathop\mathrm{Im}}		% imaginary part
\newcommand{\sgn}{\mathop\mathrm{sgn}}				% the sign operator (0 for 0)

%%%%%%%%%%%%%%%%
% reel numbers %
%%%%%%%%%%%%%%%%

\newcommand{\floor}[1]{\left\lfloor#1\right\rfloor}
\newcommand{\ceil}[1]{\left\lceil#1\right\rceil}

%%%%%%%%%%%%%%%%%%
% set operations %
%%%%%%%%%%%%%%%%%%

\newcommand{\intersect}{\cap}			% intersect to sets
\newcommand{\setjoin}{\cup}				% join two sets
\newcommand{\setmeet}{\cap}                     % intersect to sets
\newcommand{\bigsetjoin}{\bigcup}			% the union of sets ... subscripts to be added
\newcommand{\distunion}{\dot{\bigcup}}	% disjoint union of sets ... subscripts to be added
\newcommand{\bigsetmeet}{\bigcap}		% intersection of sets
\newcommand{\powerset}[1][]{\ifx&#1&\mathcal{P}\else\mathcal{P}_{#1}\fi}		% powerset ... to be customized
\newcommand{\card}[1]{\left|#1\right|}

%%%%%%%%%%%%%%%%%%%%%%%%%%%%%%%%%%%%%%%%
% composition operations of structures %
%%%%%%%%%%%%%%%%%%%%%%%%%%%%%%%%%%%%%%%%

%\newcommand{\setprod}{\bigtimes}			% setproduct - needed
\newcommand{\dirprod}{\bigotimes} 			% direct product for groups and spaces
\newcommand{\dirtimes}{\otimes}				% direct multiply for groups and spaces
\newcommand{\dirsum}{\bigoplus} 			% direct sum for groups and spaces
\newcommand{\dirplus}{\oplus}				% direct add for groups and spaces
\newcommand{\inprod}[2]{\left\langle #1,#2 \right\rangle}

\newcommand{\tuple}{\meet}
\newcommand{\cotuple}{\join}

%%%%%%%%%%%%%%
% categories %
%%%%%%%%%%%%%%

% combinatorics
%%

\renewcommand{\binom}[3][\empty]{\if\empty{#1}{{#2 \choose #3}}\else{{#2 \choose #3}_{#1}}}

%%%%%%%%%%%%%%%%%%%%%%%%%%%%%%%%%%%%%%%%%%%%%%%%%%%%%
% metric spaces and normed spaces and vector spaces %
%%%%%%%%%%%%%%%%%%%%%%%%%%%%%%%%%%%%%%%%%%%%%%%%%%%%%

\newcommand{\dist}{\mathop\mathrm{dist}}				% distance operator ... dist(A,b), where A is a set and b a point
\newcommand{\diam}{\mathop\mathrm{diam}}	% diameter operator for sets
\newcommand{\norm}[1]{\left\Vert #1 \right\Vert}	% norm in a normed space ... subscript to be added
\newcommand{\conv}{\mathop\mathrm{conv}} 			% convex hull - vectorspaces
\newcommand{\lin}{\mathop\mathrm{lin}} 				% linear hull - vectorspaces
\newcommand{\aff}{\mathop\mathrm{aff}}				% affine hull - vectorspaces

%%%%%%%%%%%%%%%%%%%%%%
% operators in rings %
%%%%%%%%%%%%%%%%%%%%%%

\newcommand{\lcm}{\mathop\mathrm{lcm}}				% least common multiple - in euclidean rings
\renewcommand{\gcd}{\mathop\mathrm{gcd}}				% greatest command devisor - in euclidean rings
\newcommand{\res}{\mathop\mathrm{res}}				% residue of p mod q is res(p,q)
\renewcommand{\mod}{\textrm{ mod }}

%%%%%%%%%%%%%%%%%%%
% logical symbols %
%%%%%%%%%%%%%%%%%%%

%\newcommand{\impliedby}{\Leftarrow}				% reverse implicatoin arrow
%\newcommand{\implies}{\Rightarrow}				% implication
\newcommand{\equival}{\Leftrightarrow}				% equivalence

%%%%%%%%%%%%%%%%%%%%%%%%%%%%%%%%%%
% functions - elementary symbols %
%%%%%%%%%%%%%%%%%%%%%%%%%%%%%%%%%%

\newcommand{\rest}[1]{\left. #1\right\vert}		% restriction of a function to a set / also used as restriction in other terms like differential expressions / evaluation of a function
\newcommand{\rto}[3][]{#2\ifx&#1&\rightarrow\else\stackrel{#1}{\rightarrow}\fi#3}
\renewcommand{\to}{\rightarrow}							% arrow between domain and image
\newcommand{\dom}{\mathop\mathrm{dom}}					% domain of a function
\newcommand{\im}{\mathop\mathrm{im}}						% image of a function
\newcommand{\compose}{\circ}							% compose two functions
\newcommand{\cont}{\mathop\mathrm{C}}					% continuous functions from a domain into the reels or complex numbers 

%%%%%%%%%%%%%%%%%%%%
% groups - symbols %
%%%%%%%%%%%%%%%%%%%%

\newcommand{\stab}[1][]{\if&#1{\mathop\mathrm{stab}}&\else{\mathop\mathrm{stab}_{#1}}\fi}					% the stabilizer ... subscripts to be added
\newcommand{\orb}[1][]{\ifx&#1&\mathrm{orb}\else\mathrm{orb}_{#1}\fi} 					% orbit ... subscrit to be added (group)
\newcommand{\gen}[2][\empty]{\ifx#1\empty{\left\langle#2\right\rangle}\else{\left\langle#2\right\rangle_{#1}}\fi}					% generate ... kind of hull operator ---- to be thought of !!!!!!!

\def\Clo{\mathrm{Clo}}
\def\Loc{\mathrm{Loc}}
%% nets
\newcommand{\net}[2][\empty]{\ifx#1\empty{\left(#2\right)}\else{{\left(#2\right)}_{#1}}\fi}

%% open half ray
\newcommand{\ray}[2]{R_{#1}(#2)}

%% new
\let\oldcong\cong%
\newcommand{\iso}{\oldcong}

\def\cong{\equiv}
\newcommand{\base}[2]{\left[#2\right]_{#1}}                                   % base n expansion of some number
%%%%%%%%%%%%%%%%%%%%%%%%%%%%%%%%%
% matrices and linear operators %
%%%%%%%%%%%%%%%%%%%%%%%%%%%%%%%%%

\newcommand{\diag}{\mathop\mathrm{diag}}				% diagonal matrix or operator
\newcommand{\Eig}[1]{\mathop\mathrm{Eig}_{#1}}		% eigenspace for a certain eigenvalie		
\newcommand{\trace}{\mathop\mathrm{tr}}				% trace of a matrix
\newcommand{\trans}{\top} 							% transponse matrix

%%%%%%%%%%%%%
% constants %
%%%%%%%%%%%%%

\renewcommand{\i}{\boldsymbol{i}}			% imaginary unit
\newcommand{\e}{\boldsymbol{e}}				% the Eulerian constant

%%%%%%%%%%%%%%%%%%%%%%%%%%%%%%
% limit operators and arrows %
%%%%%%%%%%%%%%%%%%%%%%%%%%%%%%

\newcommand{\upto}{\uparrow}				% convergence from above
\newcommand{\downto}{\downarrow}			% convergence from below

%%
% other
%%
\newcommand{\cind}{\mathop\mathrm{Ind}}		% Cauchy index
\newcommand{\sgnc}{\sigma}					% sign changes
\newcommand{\wnumb}{\omega}					% winding number
\newcommand{\cfunc}{\mathop\mathrm{Cf}}		% Cauchy function of a compact curve in complex\setminus\{0\}


% evaluation of a function as a difference or single value

\newcommand{\abs}[1]{\left|#1\right|}
\newcommand{\conj}[1]{\overline{#1}}
\newcommand{\diff}{\mathop\mathrm{d}}

%%%%% test
\newcommand{\distjoin}{\mathaccent\cdot\cup}	% to be modified (name)
\newcommand{\cl}{\mathop\mathrm{cl}}				% topological closure
\newcommand{\sphere}{\mathbb{S}} % n-sphere
\newcommand{\ball}{\mathbb{B}} % n-ball
\newcommand{\bound}{\partial}
\newcommand{\bigmeet}{\mathop\mathrm{\bigwedge}}
\newcommand{\bigjoin}{\mathop\mathrm{\bigvee}}
\newcommandx{\rchar}[1][1=\empty]{\mathop\mathrm{char}\brackets{#1}}              % characteristic of a ring
\newcommand{\lgor}{\vee}                               % logical
\newcommand{\lgand}{\wedge}
\newcommand{\codim}{\mathop\mathrm{codim}}
\newcommand{\row}{\mathop\mathrm{row}}
\newcommand{\cone}{\mathop\mathrm{cone}}
\newcommand{\comp}{\mathop\mathrm{comp}}
\newcommand{\proj}{\mathrm{P}}
\def\PG{\mathrm{PG}}           % projective space
\newcommand{\meet}{\wedge}
\newcommand{\join}{\vee}
\newcommand{\col}{\mathop\mathrm{col}}
\newcommand{\vol}{\mathop\mathrm{vol}\nolimits}
%arrangements
\newcommand{\tpert}{\mathop\mathrm{tpert}}
\renewcommand{\epsilon}{\varepsilon}
% groups
\newcommand{\symgr}{\mathop\mathrm{Sym}}
\newcommand{\symalg}{\mathop\mathrm{S}}
\newcommand{\extalg}{\mathop\mathrm{\Lambda}}
\newcommand{\extpow}[1]{\mathop\mathrm{\Lambda}^{#1}}

%% set hulloperators -> define



\newcommandx{\homl}[3][1=1,2=2,3=3]{\ifx#1\empty{\mathrm{Hom}}\else{\mathrm{Hom}_{#1}}\fi(#2,#3)}
\makeatletter
\newenvironment{myproofof}[1]{\par
  \pushQED{\qed}%
  \normalfont \topsep6\p@\@plus6\p@\relax
  \trivlist
  \item[\hskip\labelsep
        \bfseries
    Proof of #1\@addpunct{.}]\ignorespaces
}{%
  \popQED\endtrivlist\@endpefalse
}
\makeatother



%%% environment test with enumerates

    

% Local variables:
% mode: tex
% End:


% for use of german language in a document
\usepackage[utf8]{inputenc} % this is needed for umlauts
\usepackage[ngerman]{babel} % this is needed for umlauts
\usepackage[T1]{fontenc}    % this is needed for correct output of umlauts in pdf

% my package
\usepackage[de]{mathenv}
%\usepackage[bold-style=ISO]{unicode-math}

\begin{document}
\author{Jakob Schneider\\ Seminarleiter: Carsten Schultz}
\date{3. April, 2014}
\title{Der Satz von Borsuk-Ulam}
\maketitle

\section{Das Theorem}

Wir formulieren zunächst einige Versionen des Theorems, deren Äquivalenz leicht einzusehen ist:

\begin{theorem}[Borsuk-Ulam]\label{thm:1}
    Sei $n\in\nats$. Dann sind die folgenden Aussagen wahr und äquivalent:
    \begin{statements}
            \item\label{stm:1} Sei $f:\sphere^n\to \reals^n$ stetig. Dann existiert ein $x\in\sphere^n$, sodass $f(x)=f(-x)$.\label{0}
            \item\label{stm:2}Sei $f:\sphere^n\to \reals^n$ stetig und antipodenerhaltend, d.h. $f(-x)=-f(x)$ für alle $x\in\sphere^n$. Dann hat $f$ eine Nullstelle.
            \item\label{stm:3} Es gibt keine antipodenerhaltende stetige Abbildung $f:\sphere^n\to\sphere^{n-1}$.
            \item\label{stm:4} Es gibt keine stetige Abbildung $f:\ball^n\to\sphere^{n-1}$, welche antipodenerhaltend auf $\bound{\ball^n}$ ist.
            \item\label{stm:5} Sei $\{A_k\}_{k=0}^n$ eine Überdeckung der $\sphere^n$, derart, dass jedes $A_k$ andipodenerhaltend gleichermaßen offen oder abgeschlossen ist, d.h. für jedes $x\in A_k$ gibt es eine Umgebung $U\in\mathcal{U}(x)\intersect\mathcal{U}(-x)$, sodass $A_k\intersect U$ offen oder abgeschlossen ist ($\mathcal{U}(x)$ ist hier der Filter aller Umgebungen um $x$). Dann enthält eines der $A_j$ ein antipodales Paar. 
            \item\label{stm:6} Seien $\{B_k\}_{k=0}^n$ eine Überdeckung der $\sphere^n$, dann hat eine der Mengen $B_j$ Durchmesser 2, d.h. $\cl{B_j}$ enthält ein antipodales Paar.
    \end{statements}
\end{theorem}

\begin{remark}
    Die \fref{stm:5}
    in \fref{thm:1} gilt insbesondere, wenn jede der Mengen offen oder abgeschlossen ist.
\end{remark}

\rref[Lemma]{thm:1}
Wir beweisen zunächst die Äquivalenz. Zum eigentlichen Beweis kommen wir erst später.

\begin{proof}[\fref{thm:1}]
    Wir beweisen die Aussage durch zwei Ringschlüsse.
    \begin{implications}  
            \item[$\fref{stm:1}\implies\fref{stm:2}$:]
        Sei $f:\sphere^n\to\reals^n$ wie in \fref{stm:2}, dann gibt es gemäß \fref{stm:2} einen Punkt $x\in\sphere^n$ mit $f(x)=f(-x)=-f(x)=0$.
        
            \item[$\fref{stm:2}\implies\fref{stm:3}$:]
        Haben wir eine antipodenerhaltende stetige Abbildung $f:\sphere^n\to\sphere^{n-1}$, dann ist isbesondere $\cincl\compose f:\sphere^n\to\reals^n$ stetig ohne Nullstelle im Widerspruch zu \fref{stm:2}.

            \item[$\fref{stm:3}\implies\fref{stm:4}$:]
        Gibt es eine Abbildung $f$ wie in \fref{stm:4} dann können wir diese vermittels $\tilde f:\sphere^n\to\sphere^{n-1}$ mit $x \mapsto f(x)$ falls $x\in\ball^n_+$ und $x\mapsto -f(-x)$ falls $x\in\ball^n_-$ zu einer stetigen Abbildung wie in \fref{stm:3} fortsetzen (hierbei identifizieren wir $\ball^n$ mit der oberen abgeschlossenen Hemisphäre $\ball^n_+$ von $\sphere^n$).

            \item[$\fref{stm:1}\implies\fref{stm:5}$:]
        Man definiert wieder $f:\sphere^n\to\reals^n$ durch
        \[f(x):=(\dist(x,A_1),\ldots,\dist(x,A_n)).\]
        Dann erhalten wir nach Voraussetzung ein $x\in\sphere^n$ sodass $f(x)=f(-x)$. Gilt $x,-x\in A_0$ gelten, so sind wir fertig.
        Anderenfalls gibt es ein $i\in \{1,\ldots,n\}$, sodass $f_i(x)=f_i(-x)=0$ und $x\in A_i$.
        Im dem Falle, dass es eine Umgebung $V$ von $x$ und $-x$ gibt, sodass $A_i\setmeet V\ni x$ offen ist, muss es ein $\epsilon>0$ geben, sodass $U_{\epsilon}(x)\subseteq A_i$ und da $\dist(-x,A_i)=0$ gibt es dann ein $x'\in U_{\epsilon}(x)$, sodass $-x'\in A_i$.
        Damit ist $x',-x'\in A_i$. Im anderen Falle, gibt es eine Umgebung $V$ von $x,-x$, sodass $A_i\setmeet V$ abgeschlossen ist.
        Dann gilt aber, dass $\dist(-x,A_i\setmeet V)=0$, da $V$ Umgebung von $-x$ ist und $\dist(-x,A_i)=0$.
        Daraus folgt jedoch schon $-x\in A_i\setmeet V$ (da diese Menge abgeschlossen ist). Dies zeigt in wieder $x,-x\in A_i$.

            \item[$\fref{stm:5}\implies\fref{stm:6}$:]
        Das System ${\{\cl{B_i}\}}_{i=0}^n$ genügt der \fref{stm:5}.
        Also gibt es ein $i\in\{0,\ldots,n\}$ und $x\in\sphere^n$, sodass $x,-x\in \cl{B_i}$.

            \item[$\fref{stm:6}\implies\fref{stm:3}$:]
        Nehmen wir die Existenz einer stetigen, antipodenerhaltenden Abbildung $f:\sphere^n\to\sphere^{n-1}$ an, so können wir eine Überdeckung der $\sphere^n$ durch $n$ abgeschlossene Mengen, die keine Antipoden enthalten, angeben (z.B. die Zentralprojektion der Facetten eines regulären $n$-Simplexes, deren Eckpunkte auf $\sphere^{n-1}$ liegen).
        Damit würden die Urbilder unter $f$ dieser Mengen jedoch auch keine Antipoden enthalten im Widerspruch zu \fref{stm:6}.

            \item[$\fref{stm:3}\implies\fref{stm:1}$:]
        Gilt $f(x)\neq f(-x)$ für alle $x\in\sphere^n$, dann ist $x\mapsto \frac{f(x)-f(-x)}{\norm{f(x)-f(-x)}}$ eine stetige Abbildung, die \fref{stm:3} widerspricht.
    \end{implications}
    Dies beendet den Beweis.
\end{proof}

\section{Zum eigentlichen Vortrag - Das kombinatorische Äquivalent zum BU-Theorem}

Sobald man sich einmal das Theorem klar gemacht hat, ist es natürlich, sich ein äquivalentes diskretes Resultat (reiner kombinatorischer Natur) zu überlegen (ähnlich wie beim Brouwer'schen Fixpunktsatz und Sperner's Lemma - vgl. G. Ziegler, M. Aigner, Buch der Beweise).

Wir benötigen zunächst folgende

\begin{definition}
Die $n$-Sphäre in der $L^1$-Norm ist gegeben durch $\sphere^n_1:=\{x\in\reals^{n+1}:\norm{x}_1=1\}$.
\end{definition}

Anders als die $\sphere^n$ bildet die $\sphere^n_1$ auch ein kombinatorisches Gebilde, nämlich den Rand des $n$-dimensionalen Kreuzpolytops (Dualkörper zum Würfel, welcher als die konvexe Hülle von den $n+1$ Einheitsvektoren und ihren Inversen aufgefasst werden kann).

Naturlüch sind die beiden Strukturen vom topologischen Standpunkt aber isomorph (denn $x\mapsto \frac{1}{\norm{x}_1}x$ ist Homöomorphismus).

Weiterhin besitzt $\sphere^n_1$ eine kanonische Triangulierung. Entsprechend dieser definieren wir den induzierten Simplizialkomplex $\Sigma_n$ als
\[\Sigma_n:=\{A\subseteq\{\pm e_1,\ldots,\pm e_{n+1}\}, a\in A\implies -a\notin A\}.\]

In analoger Weise definieren wir die kanonischerweise eine Triangulierung des $\ball^n_1:=\conv\sphere^n_1$ (Einheitsball in der $L_1$-Norm) durch $\Gamma_n$ durch $\Gamma_n:=\{A\subseteq\{\pm e_1,\ldots,\pm e_n,0\}, a\in A\implies -a\notin A\}$.
Dann ist $\Sigma_{n-1}$ in $\Gamma_n$ eingebettet durch die simpliziale Abbildung $i:\Sigma_{n-1}\to\Gamma_n$.

Nun wollen wir aus einer antipodenerhaltenden Abbildung $f^{(0)}:T^{(0)}\to\{\pm e_i:i=1,\ldots,n\}$ ein kombinatorisches Resultat ableiten, wobei $T$ eine antipodalsymmetrische Verfeinerung von $\Sigma_n$ sei (antipodalsymmetrisch heißt, dass die Abbildung $x\mapsto -x$ auf $T$ ein simplizialer Isomorphismus ist).

Ist diese auf den Eckpunkten $T^{(0)}$ definiert, so kann man sie stückweise affin auf den Simplizes fortzusetzen (siehe Eingangsvortrag von Karsten) zu einer Abbildung $f:\abs{T}\to\reals^n$. Wenden wir das BU-Theorem an, so sehen wir ein, dass es ein $x\in \abs{T}$ gibt mit $f(x)=0$. Dieses $x$ muss jedoch im inneren eines $k$-Simplizes $\abs{\sigma}$ liegen. Man siehr nun leicht ein, dass dann zwei der Eckpunkte dieses Simplizes auf diametral gegenüberliegende Punkte unter $f^{(0)}$ abgebildet werden (Einheitsvektoren). 

Substituieren wir also gedacht die Einheitsvektoren $\{\pm e_1,\ldots, \pm e_n\}$ im Wertebereich durch die Farben $\{\pm1,\ldots,\pm n\}$ und betrachten die Abbildung $f$ nur auf den Eckpunkten $T^{(0)}$, dann ergibt sich aus \fref{lem:tuck} und \fref{thm:1}.

\begin{lemma}[Tucker]\label{lem:tuck}
Sei $T$ eine antipodalsymmetrische Verfeinerung von $\Sigma^n$ und $\lambda(=f^{(0)}):T^{(0)}\to\{\pm1,\ldots,\pm n\}$ eine antipodenerhaltende Färbung (d.h. $\lambda(-x)=-\lambda(x)$ für $x\in T^{(0)}$). Dann gibt es eine komplementäre Kante in $T$, d.h. ein 1-Simplex $\{u,v\}\in T$, sodass $\lambda(u)=-\lambda(v)$. 
\end{lemma}

Dieses Lemma würde natürlich nicht so schön sein, wenn es schwächer als \fref{thm:1} wäre.
Daher noch eine kurze Demonstration der Rückrichtung.

Sei $f:\sphere^n_1\to\reals^n$ stetig und antipodenerhaltend. Dann können wir einen beliebig feinen Simplizialkomplex $T$ von der Art in \fref{lem:tuck} definieren (z.B. durch baryzentrische Unterteilung von $\Sigma^n$). Wir können dann die Färbung $\lambda:T^{(0)}\to \{\pm1,\ldots,\pm n\}$ definieren durch
\begin{equation}
\lambda(v):=\sgn(f_i(v))\cdot i, 
\end{equation}
wobei $i$ minimal ist mit $\abs{f_i(v)}=\max_{j\in\{1,\ldots,n\}}{|f_j(v)|}$.

Dann prüft man leicht nach, dass $\lambda$ auch antipodenerhaltend ist und wir erhalten eine komplementäre Kante $\{u,v\}$ in $T$. Lässt man die Feinheiten einer Folge solcher Triangulationen $\{T_m\}_{m\in\nats}$ gegen null gehen (d.h. $\sup_{\sigma\in T_m}{\diam{\abs{\sigma}}}\to 0$ für $m\to \infty$), so erhält man eine Folge komplementärer Kanten ${\{\{u_m,v_m\}\}}_{m\in\nats}$, die aufgrund der Kompaktheit von $\sphere^n_1$ einen Häufungspunkt hat. Sei ${\{\{u'_m,v'_m\}\}}_{m\in\nats}$ nun also eine konvergente Teilfolge der vorherigen, dann kann man o.E. voraussetzen, dass alle $u'_m$ mit demselben $k\in\{1,\ldots,n\}$ gefärbt sind (da ein $k$ unendlich oft vorkommen muss). Damit folgt für den Grenzwert $u$ der Folge, dass $f_i(u) = 0$, denn $f_i$ ist stetig und $f_i(u'_m)\geq 0$, sowie $f_i(v'_m)\leq 0$ (o.B.d.A. seien $u'_m$ und $v'_m$ immer so hingetauscht). Weiterhin folgt daraus aber schon $f(u)=0$, denn in den gegen $u$ konvergenten Folgen sind alle Komponenten $f_j(u_m),f_j(v_m)$ ($j\neq i$) betragsmäßig kleiner als $f_i(u_m),f_i(v_m)$.

\section{Zum eigentlichen Beweis.}

Nun wollen wir also Tucker's Lemma beweisen und damit das BU-Theorem in der Luft zerreißen.
Es stellt sich jedoch heraus, dass es sogar leichter ist, ein stärkeres aber ähnliches Resultat zu benutzen:

\begin{lemma}[Ky Fan]\label{lem:kyfan}
Sei $T$ eine antipodalsymmetrische Verfeinerung von $\Sigma^n$ und $\lambda:T^{(0)}\to\{\pm1,\ldots,\pm m\}$ eine antipodenerhaltende Färbung (d.h. $\lambda(-x)=-\lambda(x)$ für $x\in T^{(0)}$) ohne komplementäre Kanten. Dann ist die Anzahl der $+$-alternierenden $n$-Simplizes ungerade. 
\end{lemma}

Oops. Wir haben müssen noch definieren, was $+$- bzw. $-$-alternierend bedeutet.
Weil wir es später (bzw. gleich) noch für den Beweis von \fref{lem:kyfan} benötigen, definieren wir noch etwas mehr.

\begin{definition}[$\epsilon$-alternierend und fast $\epsilon$-alternierende Simplizes]
Ein $d$-Simplex heißt $\epsilon$-alternierend ($\epsilon=\pm1$), falls sich seine Ecken $v_0,\ldots,v_d$ so anordnen lassen, dass $|\lambda(v_0)|<\cdots<|\lambda(v_d)|$ und $\sgn(\lambda(v_i))=\epsilon(-1)^i$. 

Ein $d$-Simplex heißt fast $\epsilon$- alternierend, wenn es nicht alternierend ist, aber eine $\epsilon$-alternierende Facette (also $(d-1)$-Seite hat).
\end{definition}

\begin{remark}
Man überlegt sich leicht, dass $\sigma$ dann genau zwei solche Facetten hat, falls $\sigma$ keine komplementäre Kante enthält und genau eine solche wenn doch (in unserem Falle also immer zwei).
\end{remark}

Der Beweis wird stark auf dem Prinzip dieser Simplizes in $T$ beruhen, zunächst möchten wir aber noch festhalten, warum aus Ky Fan direkt \fref{lem:tuck} folgt. Dies liegt schlicht daran, dass es für $m<n+1$ keine alternierenden Simplizes geben kann. Damit muss im Umkehrschluss dann $T$ eine komplementäre Kante enthalten.

Nachdem das geklärt wäre, nun zum Beweis.

\begin{proof}[\fref{lem:kyfan}]
Das Konzept des ersten (und sehr wahrscheinlich einzigen präsentierten) Beweises, ist es sich einen Graphen über die Färbung der Triangulierung zu definieren und über dessen Gestalt eine Aussage über die Anzahl der entsprechenden Simplizes treffen zu können (Graphentheoretiker mögen das Handshaking-Lemma).

Um dies tun zu können, müssen wir für ein Simplex $\sigma\in T$ aber noch seine sogenannte Trägerhemisphäre definieren.
Dazu definieren wir zunächst die folgenden Mengen (wieder $\epsilon\in\{\pm1\}$, $d\geq 0$):
\begin{equation}
H^{\epsilon}_d:=\{(x_0,\ldots,x_n)\in\sphere^n_1:\epsilon x_d\geq 0,x_{d+1}=\cdots=x_n=0\}. 
\end{equation}
Aufgrund der Art der Triangulierung $T$ (als Verfeinerung von $\Sigma_n$), lässt sich jedem (nichtleeren) Simplex $\sigma\in T$ nun eine eindeutige Hülle in dem System der obgien Mengen zuordnen (nämlich die kleinste der Mengen $H^{\epsilon}_d$, in der $\sigma$ enthalten ist). Diese nennen wir \emph{Trägerhemisphäre} von $\sigma$.

Nun also können wir anfangen den Graphen zu definieren: Wir definieren uns eine Graphen $G$, wobei
\begin{itemize}
\item die Ecken $V$ von $G$ genau diejenigen Simplizes $\sigma$ seien, welche eine der folgenden Eigenschaften haben:
\begin{itemize}
\item $\sigma$ ist $n$-dimensional und alternierend. 
\item $\sigma$ ist $(d-1)$-dimensional und $\epsilon$-alternierend und hat $H^\epsilon_d$ als Trägerhemisphäre (ACHTUNG: die $\epsilon$'s und $d$'s sind gleich).
\item $\sigma$ ist $d$-dimensional und und fast $\epsilon$-alternierend und seine Trägerhemisphäre ist $H^\epsilon_d$.
\end{itemize}
\item die Kanten von $G$ sind nun so zu wählen, dass jede Ecke nach Möglichkeit immer Grad 1 oder 2 hat.
Daher definieren wir, dass zwei Simplizes $\sigma,\tau\in V$ genau dann benachbart sind, wenn folgende Punkte beide erfüllt sind:
\begin{itemize}
\item $\sigma$ Facette von $\tau$ ist (bis auf Vertauschung von $\sigma, \tau$)
\item falls $\tau$ als Trägerhemisphäre $H^\epsilon_d$ hat, dann ist $\sigma$ $\epsilon$-alternierend (gleiches $\epsilon$).
\end{itemize}
\end{itemize}

Nun sieht man schnell ein, dass sämtlichen Ecken in $G$ Eckengrade 1 oder 2 haben.
\begin{itemize}
\item Wenn $\sigma$ Dimension $d-1$ hat und Trägerhemisphäre $H^\epsilon_d$ (also $\epsilon$-alternierend ist), so ist es Facette von genau zwei $d$-Simplizes, welche beide auch in $H^\epsilon_d$ liegen müssen und beide $\epsilon$-alternierend oder fast $\epsilon$-alternierend sind. Also $\deg(\sigma)=2$
\item Wenn $\sigma$ Dimension $d$ hat und Trägerhemispäre $H^\epsilon_d$, sowie fast $\epsilon$-alternierend ist, dann hat $\sigma$ genau zwei $\epsilon$-alternierende Facetten (denn es enthält keine komplementäre Kante). Also $\deg(\sigma)=2$.
\item Wenn nun zuletzt $\sigma$ wieder $d$-dimensional ist und alternierend (egal ob $+$ oder $-$) und als Trägerhemisphäre $H^\epsilon_d$ hat, dann hat im Falle $0<d<n$ das Simplex $\sigma$ auch wieder zwei Nachbarn, den einen als $\epsilon$-alternierende Facette (Ecke mit betragsmäßig größter oder kleinster Farbe löschen) und den anderen durch weiterlaufen vom Inneren von $\sigma$ aus in Richtung $\epsilon e_{d+1}$.
Im Falle $d=0$ oder $d=n$ fällt jeweils einer dieser Nachbarn weg.
\end{itemize}

Damit besteht der Graph $G$ nur aus Pfaden und Kreisen. Wobei eine Ecke $\sigma$ genau dann Eckengrad 1 hat, wenn $\sigma$ ein alternierendes $n$-Simplex ist, oder $\sigma$ ein $0$-Siplex ist. Die einzigen $0$-Simplizes, die tatsächlich in unserem Graphen $G$ auftauchen sind aber $\{e_0\}$ und $\{-e_0\}$.

Über jeden Pfad hinweg bleibt das $\epsilon$ 'jedoch konstant'. Damit können die beiden $0$-Simplizes nicht durch einen Pfad verbunden sein ($-1\neq 1$).

Damit folgt jedoch dass die Anzahl der $+$-alternierenden $n$-Simplizes ungerade ist, denn durch den in $\{e_0\}$ startenden Pfad erhält man eines und durch jeden weiteren Pfad, deren Simplizes der die 'Signatur' + trägt immer jweils zwei.

Bleibt noch zu erwähnen, dass die Anzahl der $-$-alternierenden $n$-Simplizes in $G$ dieselbe ist (denn unter $x\mapsto-x$ werden $+$-alternierende Simplizes auf $-$-alternierende abgebildet (1-1).
\end{proof}

\begin{remark}
Der Graph funktioniert auch mit komplementären Kanten. Man hat dann als Ecken vom Grade 1 genau die $0$-Simplizes und alternierenden $n$-Simplizes, sowie fast alternierende Simplizes mit komplementärer Kante.
\end{remark}

\section{Wenn noch etwas Zeit bleibt ... }

Den eben geführten Beweis kann man auch induktiv gestalten. Dazu betrachtet man diejenigen Pfade im Graphen $G$, die mit der in $\sphere^n_1$ eingebetteten niederdimensionalen Späre $S_{n-1}:=H^+_{n-1}\cup H^-_{n-1}$ nichtleeren Schnitt haben (Kreise gibt es nicht - denn jede Komponente von $G$ hat ein alternierendes $n-$-Simplex in sich). Der Schnitt eines solchen Pfades $p$ mit $S_{n-1}$ ist ein entsprechender Pfad des entsprechenden Graphen $G'$ in der Einschränkung von $T$ auf $S_{n-1}$.

Damit sieht man, dass alle $+$- oder $-$-alternierenden $n$-Simplexe in $G$, die mit einem Simplex in $G'$ verbunden sind (über einen Pfad in $G$) jeweils eindeutig mit einem $(n-1)$-Simplex in $G'$ identifiziert werden können (nämlich mit dem, zu welchem sie in $G$ den kleinsten Abstand haben). Der Rest der hinzukommenden Simplizes lässt sich aber in Vierergruppen derart einteilen, dass je zwei Endpunkt eines Pfades, und die anderen beiden Endpunkt des anderen Pfades sind. Damit ist die Summe der alternierenden $n$-Simplizes in jeder Dimension modulo 4 dieselbe und zwar 2 (denn für 0 ist sie 2).

Den eben wenig formell geäußerten Gedanken kann man auch mithilfe von Kettenkomplexen noch eleganter hinschreiben:
Dazu zählt man einfach in $\ints_2$ die $+$- und $-$-alternierenden $n$-Simplizes in der Halbsphäre $H^+_n$. Dazu definiert man zunächst die Homomorphismen
\begin{eqnarray*}
    & \alpha:C_n(T,\ints_2)\to\ints_2,& \\
    &\sum_{\sigma\in S}{\sigma}\mapsto\abs{\{\sigma\in S:\sigma \textrm{alternierend}\}}&
\end{eqnarray*}
und $\beta:C_{n-1}(T,\ints_2)\to\ints_2$,  $\sum_{\sigma\in S}{\sigma}\mapsto\abs{\{\sigma\in S:\sigma +\textrm{-alternierend}\}}$. Dann realisiert man schnell, dass $\alpha=\beta\bound_n$, denn für die Erzeuger prüft man diese Identität leicht nach, indem man realisiert, dass ein alternierendes $n$-Simplex genau eine $+$-alternierende Facette hat, womit die Identität für ein eingesetztes alternierendes $n$-Simplex gilt. Anderenfalls ist $\alpha$ immer 0 und $\beta$ ebenso, denn der einzige Fall, in dem $\beta\bound_n$ noch nicht 0 sein kann ist, wenn man ein fast alternierendes $n$-Simplex einsetzt. Dieses hat jedoch nach Voraussetzung keine komplementären Kanten, also entweder 2 oder kein $+$-alternierendes $(n-1)$-Simplex als Facette. 

Damit ergibt sich nun $\alpha(c)=\beta\bound_n(c)=\beta(b)$ (wobei $c$ die Summe aus allen $n$-Simplexen in $H^+_n$ sei und $b$ die Summe aller $(n-1)$-Simplexe in $S_{n-1}=\bound H^+_n$). Daraus folgt, dass die Anzahl der alternierenden $n$-Simplexe in der oberen Hemisphäre, welche aufgrund der antipodalen Symmetrie der Anzahl der $+$-alternierenden $n$-Simplexe der ganzen Sphäre entspricht, gleich der Anzahl der $+$-alternierenden $(n-1)$-Simplexe in einer Dimension niedriger ist (mod 2). Induktion beendet den Beweis und diesen Vortrag (wenn er nicht schon zu Ende ist).

\end{document}

%%% Local Variables: 
%%% mode: latex
%%% End: 