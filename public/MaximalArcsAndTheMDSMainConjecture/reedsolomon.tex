\section{Extended \person{Reed-Solomon}-Codes and normal rational curves}\label{sec-reedsolomon}

In this section we discuss the most popular (probably) maximal examples of the MDS main conjecture in the case where $\cA\setleq \P(K^n)$ for $n\geq 3$ and $K$ a field with $n\leq\card K$.

\subsection{Normal rational curves}

\begin{definition}[normal rational curve]\label{norratcurve}
    A \keyword{normal rational curve} is the image of the projective map $\proj{f}:\proj{K^2}\to\proj{K^n}$ where $f$ is given by
    $$
    (X_1,X_2) \mapsto (X_1^{n-1},X_1^{n-1}X_2,\ldots,X_1X_2^{n-1},X_2^{n-1})
    $$
    modulo $\PGammaL$.
\end{definition}

A natural question is weather a normal rational curve is a complete arc.

\begin{definition}[notation for polynomial coefficients]
    For a polynomial $p\in R[X_1,\ldots,X_m]$ write $[X_1^{e_1}\cdots X_m^{e_m}]p$ for the coefficient which is in front of the monomial $X_1^{e_1}\cdots X_n^{e_n}$ in $p$.
\end{definition}

\begin{definition}[projective zeros]
    Let $p\in K[X_1,X_2]$ be a homogeneous polynomial. We say that $z\in\proj{K^2}$ is a projective zero of $p$ if $z=\gen{(\alpha,\beta)}$ such that $p(\alpha,\beta)=0$. 
\end{definition}

\begin{lemma}[description of $(n-2)$-secant hyperplanes of a normal rational curve]\label{norrattandesc}
    The hyperplanes $\cH$ intersecting a normal rational curve $\cA=\im\proj{f}\setleq \proj{K^n}$ in the representation of \autoref{norratcurve} in exactly the points of an $(n-2)$-set $\proj{f}(A)$ ($A$ is the corresponding $(n-2)$-set of preimages in $\proj K^2$) are given as the kernels of the linear forms 
    $$
    l_{A,\alpha}:\doteq\sum_{i=0}^{n-1}{[X_1^i X_2^{n-1-i}]\left(\det(X,\alpha)\prod_{a\in A}{\det(X,a)}\right)\dual e_i}
    $$
    where $\alpha\in A$ and ${(e_i)}_{i=1}^n$ denotes the standard basis of $K^n$ and ${(\dual e_i)}_{i=1}^n$ its dual basis.  
\end{lemma}

\begin{proof}
    It is clear that the $l_{A,\alpha}(f(a_1,a_2))=0$ if and only if $\gen{(a_1,a_2)}\in A$. On the other hand, any linear form $l$ which has as a kernel a hyperplane $H$ with $H\setmeet\cA=A$ must be of the form
    $$
    l=\sum_{j=0}^{n-1}{[X_1^i X_2^{n-1-i}]\left(\prod_{a\in A}{\det(X,a)}\det(X,\alpha')\right)x_i^{\ast}}
    $$
    since its coefficients in $\dual e_i$ can be interpreted as the corresponding coefficients of a homogeneous polynomial in two variables of degree $n-1$. But this polynomial must have as zeros all elements of $A$ (as the assumption on the form $l$ delivers) which are $n-2$. Hence, there is another zero $\alpha'=\gen{(\alpha'_1,\alpha'_2)}\in\proj{K^2}$ since the polynomial factors into linear polynomials. Since then $\gen{f(\alpha'_1,\alpha'_2)}\in H\setmeet\cA=\proj f(A)$ we must have $\alpha'\in A$ (clearly $\proj f$ is injective).
    
\end{proof}

In the sense of the proof of this lemma, we make the following definition.
Henceforth, we assume that $\cA$ is an incomplete normal rational curve.

\begin{definition}[associated polynomials and linear forms for incomplete normal rational curves]
    Assume the normal rational curve $\cA\setleq \proj{K^n}=\proj{V}$ can be extended to a $(q+2)$-arc $\hat{\cA}:=\cA\setjoin\set{\hat{a}}$ by a point $\hat a$, then for each $(n-2)$-set $A\setleq\proj{K^2}$ there exists a hyperplane $H_A:=\gen{(\proj f)A,\hat{a}}$ and $H_A\setmeet \hat{\cA}=\proj f(A)\setjoin\set{\hat{a}}$. Choose $l_A\in\dual V$ such that $\ker{l_A}=H_A$ and call it the \keyword{associated linear form of $A'$}.
    Define $p_A:=\sum_{i=0}^{n-1}{l_A(e_i)X_1^{n-1-i}X_2^i}$ and call it the \keyword{associated polynomial of $A$}.
\end{definition}

\begin{remark}
    The polynomials $p_A$ are precisely the polynomials which are used to construct the forms $l_{A,\alpha}$ in the proof of \autoref{norrattandesc}. Thus $p_A$ factors completely into $n-1$ linear factors and has the elements of $A$ as a projective zeros (with one zero of order two).
\end{remark}

\begin{definition}
    Define $P^k$ as the subspace of $K[X_1,X_2]$ of homogeneous polynomials of degree $k$.
\end{definition}

\begin{lemma}
    The space $P:=\gen{p_A\in P^{n-1}:A\setleq \proj{K^2},\card A=n-2}_{\lin}$ is of dimension $n-1$.
\end{lemma}

\begin{proof}
    The map $\phi:V^{\ast}\to P^{n-1}$ mapping associated linear forms to associated polynomials given by $l\mapsto \sum_{i=0}^{n-1}{l(e_i)X_1^{n-1-i}X_2^i}$ is clearly linear and an isomorphism (the inverse is $p\mapsto\sum_{i=0}^{n-1}{([X_1^i X_2^{n-1-i}]p)\dual e_i}$ as used in the proof of \autoref{norrattandesc}). Thus,
    $$
\dim{P}=\dim\gen[\lin]{l_A\in V^{\ast}:A\setleq \proj{K^2},\card A=n-2}.
$$
But the intersection of the kernels of the $l_A$ contains the extending point $\hat{a}\neq 0$. Hence, 
$$
\dim\gen[\lin]{l_A\in V^{\ast}:A\setleq \proj{K^2},\card A=n-2}=n-\dim\Meet_{\substack{A\setleq\proj{K^2}\\ \card A=n-2}}{\ker{l_A}}\leq n-1.
$$
On the other hand, we can choose an $(n-1)$-set $B\setleq \proj{K^2}$ and directly verify that the system of polynomials $p_A$ where $A\setleq B$, $\card{A'}=n-2$ is linearly independent by the interpolation formula
$$
\left(\sum_{\substack{A\setleq B\\ \card{A}=n-2}}{\lambda_A p_A}\right)(a)=\lambda_{B\setminus\set{a}}p_{B\setminus\set{a}}(a)
$$
showing that the linear combination must be trivial if it evaluates to zero (since $p_{B\setminus\set{a}}(a)\neq 0$).
Thus,
$$
\dim{P}=\dim\gen[\lin]{l_{A'}\in V^{\ast}:A'\setleq \proj{K^2},\card{A'}=n-2}=n-1.
$$
\end{proof}

\begin{lemma}
    The space $P$ does not contain a non-zero separable polynomial which splits in $\proj(K^2)$.
\end{lemma}

\begin{proof}
    Assume there is such polynomial $p$ with a set of $n-1$ zeros $B$.
    As we have seen in the proof of the previous lemma, the set $\set{p_A:A\setleq B,\card A=n-2}$ forms a basis of $P$. Thus we can interpolate $p$ by
    $$
    p=\sum_{a\in A}{p(a)\frac{p_{A\setminus\set{a}}}{p_{A\setminus\set{a}}(a)}}=0.
    $$
    A contradiction.
\end{proof}

We thus arrive at a much more `algebraic' version of the assumption. 

\begin{lemma}\label{char-incomp-nor-rat-curve}
    The following two are equivalent
    \begin{statements}
            \item A normal rational curve in $\proj (K^n)$ is incomplete.\label{norrat-incomp}
            \item There is an $(n-1)$-dimensional subspace of the space $P_{n-1}$ of homogeneous polynomials of degree $n-1$ containing no splitting separable polynomials.\label{subspace-without-sep-spl-poly}
    \end{statements}
\end{lemma}

\begin{proof}
    We have just shown that \autoref{norrat-incomp} implies \autoref{subspace-without-sep-spl-poly}. Conversely, if there is a subspace $P$ as described in \autoref{subspace-without-sep-spl-poly}, then it intersects in a one-dimensional subspace with each subspace $P_A:=P_1\prod_{a\in A}{\det(X,a)}$ (for $A\setleq P(K^2)$ an $(n-2)$-set) --- the intersection must be at least one-dimensional as $\dim P +\dim P_A=n$, and has dimension at most one since otherwise there would be a splitting separable polynomial in $P\setmeet P_A=P_A$ since $\card A + 1 = n-1\leq \card K+1$.
\end{proof}


The next fact we want to prove is that a normal rational curve is complete in more than half of the expected cases.
For this we need the following elementary facts

\begin{lemma}
    Let $Q\leq P_2$ be of codimension one. Then if $Q$ is not of the form $Q=P_1\det(X,\alpha)$ for some $\alpha\in\P(K^2)$ the following statements hold.
    \begin{statements}
            \item If $\rchar K = 2$, then either $Q=P_1^2$ (i.e. $Q$ consists entirely of squares) or $Q$ contains one square, $\frac{\card K} 2$ splitting separable polynomials and the same number of irreducible polynomials (up to scalar factor).
        \item If $\rchar K \neq 2$ then $Q$ contains $1\pm 1$ squares, $\frac{\card K \mp 1} 2$ splitting separable polynomials and the same number of irreducible polynomials.
    \end{statements}
\end{lemma}

\begin{proof}
    \begin{statements}
            \item If $\P(Q)$ contains two distinct squares, then it is clearly $P_1^2$ (since they already span a two-dimensional space). Otherwise, the kernel of the map $[X_1X_2]:P_2\to K$ intersects with $Q$ in a one-dimensional space (showing that $Q$ contains a square).
            \item Assume $Q$ contains two squares. Choosing appropriate coordinates, we may assume that these are $X_1^2$ and $X_2^2$. But then any linear combination $\mu X_1^2+\lambda X_2^2$ ($\lambda,\mu\in\units K$) cannot be a square since $(\mu X_1^2+\lambda X_2^2)\meet (2\mu X_1+2\lambda X_2)\divides X_2(X_2-X_1),X_1(X_1-X_2),X_1X_2$, so it is associated to $1$ (since the greatest common divisor of the last three is one).
        Moreover, if $Q$ contains one square, say $X_1^2$ and is not of the form $P_1\det(C,\alpha)$, then it contains a polynomial $aX_1^2+bX_1X_2+X_2^2$ which can be completed to a second square by adding $\left(\frac{b^2} 4 -a\right)X_1^2$.
    \end{statements}
\end{proof}

\begin{remark}
    It is easy to see that for $K=\field{q}$ and $q$ even there exists exactly $q + 1$ subspaces of codimension one in $P_2$ of type $P_1\det(X,\alpha)$, one subspace of type $P_1^2$ and $q^2-1$ subspaces containing only one square.
    When $q$ is odd, there are $q+1$ such subspaces of type $P_1\det(X,\alpha)$, $\binom {q+1} 2$ subspaces with two squares and $\binom q 2$ subspaces without square.  
\end{remark}

\begin{lemma}
    If a normal rational curve is incomplete in $P(K^n)$ then
    $$
    \ceil{\frac{\card K + 5} 2}\leq n.
    $$
\end{lemma}

\begin{proof}
    Let $\rchar K\neq 2$. Pick two $(n-2)$-sets $A,B\setleq \P(K^2)$ such that $A\setmeet B =C$ is an $(n-3)$-set and the zeros $z_A$ and $z_B$ of order two of $p_A$ and $p_B$ are distinct (for this we need $n\geq 4$, e.g.~we can then choose $A$ first and then $B\setleq \P(K^2)\setminus\set{z_A}$).
    The space $\P\gen{p_A,p_B}$ contains exactly one polynomial with a set of zeros $C\setjoin\set c$ for all $c\in\P(K^2)\setminus C$ --- that is $\card K + 1 - (n-3)$ polynomials of this kind (the uniqueness follows from \autoref{char-incomp-nor-rat-curve}). Thus it contains exactly $\card K + 1 - (\card K + 1 - (n-3))=n-3$ polynomials which have exactly $C$ as their zeros (the rest must be of that latter kind by \autoref{char-incomp-nor-rat-curve}). In the case where $\rchar K\neq 2$ we then immediately deduce that the space
    $$
    \P\gen{p_A/(p_A\meet p_B),p_B/(p_A\meet p_B)}
    $$
    contains $1\pm 1$ squares, $\frac{\card K \mp 1} 2$ splitting separable polynomials, and thus $\frac{\card K \mp 1} 2$ irreducible polynomials.
    From this we get the inequality
    $$
    \frac{\card K - 1} 2\leq n-3
    $$
    since all irreducible polynomials $p$ of the above space lead to a polynomial of the second kind by multiplying it by $\prod_{c\in C}{\det(X,c)}$. Thus
    $$
    \frac{\card K + 5} 2\leq n
    $$
    if $\rchar K\neq 2$.

    If $\rchar K = 2$ we need to pick $A$ and $B$ such that the cases 
    $z_A=z_B$ and $\set{z_A}=A\setminus B$, $\set{z_B}=B\setminus A$ do not occur. So assume the second case occurs, then choose an $(n-2)$-set $D\setleq\P(K^2)$ such that $D\setmeet B$ is an $(n-3)$-set which contains $z_B$ (i.e.~it cannot be $C$; we use that $n\leq q+1$). Then the sets $A':=B$ and $B':=D$, then $A'$ and $B'$ cannot be of the second case by the choice of $D$. If these sets are of the first case, then $p_B$ and $p_D$ have the same zero of order two $z_B$ and so $\gen{p_B,p_D}=P_1(p_B\meet p_D)$ contains a polynomial $p_{B\setmeet D\setjoin\set{z_A}}$ (with the same zero of order two). Now, $A'':=A$ and $B'':=B\setmeet D\setjoin\set{z_A}$ are sets avoiding the two bad cases.
    Assuming that $A$ and $B$ are appropriately chosen it is simple to verify that the space $\P\gen{p_A/(p_A\meet p_B),p_B/(p_A\meet p_B)}$ (of homogeneous quadratic polynomials) contains one square, $\frac{\card K} 2$ splitting separable polynomials and $\frac{\card K} 2$ irreducible polynomials.
    Hence, we get the inequality
    $$
    \frac{\card K} 2 \leq n-3, \text{ i.e. } \frac{\card K+6} 2\leq n.
    $$
\end{proof}

TODO: continue
\begin{align*}
    &\binom{q+1} 2\binom{\frac{q+1} 2}{q+1-(n-3)} 2^{q+1-(n-3)}\\
    &??+\binom q 2\left(\binom{\frac{q-1} 2}{q+1-(n-3)}2^{q+1-(n-3)}+2\binom{\frac{q-1} 2}{q-(n-3)}2^{q-(n-3)}+\binom{\frac{q-1} 2}{q-1-(n-3)}2^{q-1-(n-3)}\right)\\
    &??\geq \frac 1 2\binom{q+1}{n-2}
\end{align*}

\subsection{Related matrices}

\paragraph{Extended \person{Vandermonde}-matrices.}
Denote the elements of $\field{q}$ by $a_i$ ($i=1,\ldots,q$) and define a matrix $V$ by
$$
V :=
\begin{pmatrix}
    1      & \cdots & 1      & 0      \\
    a_1    & \cdots & a_q    & 0      \\
    \vdots & \ddots & \vdots & \vdots \\
    a_1^n  & \cdots & a_q^n  & 1
\end{pmatrix}\textrm{.}
$$
One then checks easily that $V$ has the property that any $n$ distinct column vectors are linearly independent since such a matrix

$$
\tilde V :=
\begin{pmatrix}
    1             & \cdots & 1       \\
    a_{k_1}       & \cdots & a_{k_n} \\
    \vdots        & \ddots & \vdots  \\
    a_{k_1}^{n-1} & \cdots & a_{k_n}^{n-1} 
\end{pmatrix}
$$
is a \keyword{\person{Vandermonde}-matrix} in the case it does not contain the last vector having determinant
$$
    \det{\tilde V} = \prod_{i<j}{(a_{k_i}-a_{k_j})}
$$
which is not zero as all $a_k$ are distinct. In the other case, the determinant is also non-zero, as one notes by applying \person{Laplace}'s formula to the last column.

That same fact can also be seen via the observation that any polynomial $P\in\field{q}[X]$ of degree at most $n-1$ has at most $n-1$ zeros, so any non-zero linear form $f:\field{q}^n\to\field{q}$ has at most $n-1$ of the first $q$ columns of $V$ in its kernel.
And if the last column is in $\ker{f}$ then the corresponding coefficient $f_n$ in the coordinate representation $f=\sum_{i=1}^n{f_i e_i^{\ast}}$ is zero, so $f$ contains at most $n-2$ of the first $q$ column vectors of $V$.

We call this representation of a classical arc introduced in this section the \keyword{\person{Vandermonde}-representation}.

\paragraph{Extended \person{Cauchy}-matrices.}% good
%
In the last paragraph we saw that the arcs corresponding to extended \person{Reed-Solomon}-codes can be represented as a \keyword{normal rational curve} admitting a representation
$$ C = \set{\begin{pmatrix} 1 \\
            \vdots            \\
            z^{n-1}\end{pmatrix}\in\field{q}^n}[z\in \field{q}]\setjoin
    \set{\begin{pmatrix} 0    \\
            \vdots            \\
            1\end{pmatrix}}\textrm{.} $$
We want to construct another type of matrix representation for these codes.
The first step is to realize that the curve $C$ is $\PGL$-equivalent to some curve $\tilde{C}$ with representation
$$ C' = \set{
    \begin{pmatrix} P_1(z)             \\
        \vdots                         \\
        P_n(z)\end{pmatrix}\in\field{q}^n:z\in \field{q}}
\setjoin\set{\begin{pmatrix} p_1^{n-1} \\
        \vdots                         \\
        p_n^{n-1}\end{pmatrix}} $$
where $P_i=\sum_{j=0}^{n-1}{p_i^j X^j}$ ($i=1,\ldots,n$) form a basis of the polynomials of degree $k\leq n-1$. Now we pick $n$ distinct elements of $\field{q}$ and set
$$ P_i := \prod_{\substack{j=1\\ j\neq i}}^n{\frac{X-\nu_j}{\nu_i-\nu_j}}\in\field{q}[X]\textrm{.} $$

Then it is immediately clear that the $P_i$ are linearly independent since they form a \person{Lagrange}-basis with respect to the points $\nu_i$ ($i=1,\ldots,n$). The point $\gen{\hat{e}}$ corresponding to the preimage $\infty$ now evaluates to
$$ e = 
    \begin{pmatrix}
        \prod\limits_{j\neq 1}{\frac{1}{\nu_1-\nu_j}}\\
        \vdots               \\
        \prod\limits_{j\neq n}{\frac{1}{\nu_n-\nu_j}}
    \end{pmatrix}\textrm{.} $$

Now we may consider the curve with representation
$$ \set{\begin{pmatrix} P_1(z)\\ \vdots\\ P_n(z)\end{pmatrix}:z\in\field{q}}\setjoin\set{v} $$
and its image under the linear mapping given by matrix representation $A:={\diag(v_1,\ldots,v_n)}^{-1}$.
To abbreviate notation we now write $P(z)$ for
$$ \begin{pmatrix} P_1(z)\\ \vdots\\ P_n(z)\end{pmatrix}\textrm{.} $$

Scaling all vectors $AP(\nu_i)$ ($i=1,\ldots,n$) by
$$ \prod_{\substack{j=1\\ j\neq i}}^n{\frac{1}{\nu_i-\nu_j}}\textrm{,} $$
the vectors $AP(z)$ ($z\not\in\set{\nu_i:i=1,\ldots,n}$) by 
$$ \prod_{j=1}^n{\frac{1}{z-\nu_j}} $$
and leaving $Av$ fixed we get an representation
$$ C'' = \set{e_1,\ldots,e_n,\sum_{i=1}^n{e_i}}\setjoin\set{\begin{pmatrix} \frac{1}{z-\nu_1} \\ \vdots\\ \frac{1}{z-\nu_n}\end{pmatrix}:z\in\field{q}\setminus\set{\nu_i:i=1,\ldots,n}} $$

of a curve being $\PGL$-equivalent to the initial representation.
However, writing down the generator matrix of the corresponding MDS-code we obtain
$$ G =
    \begin{pmatrix}
        1      & 0      & \cdots & 0      & 1      & \frac{1}{\nu_{n+1}-\nu_1} & \cdots & \frac{1}{\nu_q-\nu_1} \\
        0      & \ddots & \ddots & \vdots & \vdots & \vdots                & \ddots & \vdots            \\
        \vdots & \ddots & \ddots & 0      & \vdots & \vdots                & \ddots & \vdots            \\
        0      & \cdots & 0      & 1      & 1      & \frac{1}{\nu_{n+1}-\nu_n} & \cdots & \frac{1}{\nu_q-\nu_n}
    \end{pmatrix} $$

which contains a \keyword{\person{Cauchy}-matrix}. Hence, we call this representation of a classical arc the \keyword{\person{Cauchy}-representation}. The matrix consisting of the columns of index $n+1$ till $q+1$ is called \keyword{extended \person{Cauchy}-matrix}\label{cauchy-rep}
