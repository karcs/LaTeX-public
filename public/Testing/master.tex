\documentclass{article}
% for use of german language in a document
\usepackage[utf8]{inputenc} % this is needed for umlauts
\usepackage[ngerman]{babel} % this is needed for umlauts
\usepackage[T1]{fontenc}    % this is needed for correct output of umlauts in pdf

\usepackage{layer}
%\usepackage{MnSymbol}
\usepackage[cal=dutchcal,
calscaled=.94,
bb=ams,
frak=mma,
frakscaled=.97,
scr=rsfs]{mathalfa}

\begin{document}
%%%
\chapter{Hallo}\par
%%%%%%%%%%%%%%
\section{Hallo}


$$ A \leftarrow \leq \underbrace{B + SDFSDFFS}_{abc}
\langle
%\begin{matrix}
%    1 & 1 & 1 \\
%    0 & 0 & 0
%\end{matrix}
\rangle \langle A++ \rangle$$
%
Die weiße Leinwand, welche er zusammen mit zwei anderen mühsam aus der Stadt in seine Wohnung getragen hatte, lag in beängstigender Größe vor ihm. Er hatte sich zwar über die Konstruktion seines ersten Werkes schon ein wenig Gedanken gemacht, war jedoch nun ein wenig zögerlich, da an jener großen leeren Fläche noch keinen rechten Anhaltspunkt finden konnte. Auf einem alten Teller, der direkt daneben lag, begann S nun einen gelben Kleks mit einer der Tuben zu machen. Er tränkte einen dünnen Pinsel mit jener gelblich-grünen Mischung und malte einen unsichtbaren Strich in die Mitte des Tuches --- dabei setzte er den Pinsel nicht auf. Einige weitere solcher Striche folgten, bevor sich S schließlich überwandt und nun doch eine zugegebenermaßen ein wenig lustlos aufgetragen wirkende gelbe Linie in die weiße Leere hineinsetzte. Aller Anfang ist schwer --- wie wahr dieses Sprichwort doch in diesem Momente erschien. Doch auch die wenigen nun folgenden Linien schienen nicht der Natur zu sein, die endgültige Wirkung des Werkes einmal zu prägen.
Früher war er für seine Zeichnungen oft gelobt worden. Jedoch hatten die Betrachter oftmals nicht den teiferen Sinn, der sich dahinter verbarg, nicht die Stimme, welche S durch das Gebilde aus Linien, Punkten, Flächen und Verläufen, zu ihnen sprechen ließ, erfasst --- ja viele erahnten diesen noch nicht einmal.
Jedoch war S stets der Auffassung gewesen, dass die Erklärung eines Bildes dessen Wirkung zerstören, ja mehr noch, das Zwischenspiel zwischen Betrachter und Komposition unterbinden würde. Außerdem stellte er immer wieder fest --- so sollte es nun auch bei diesem Gemälde geschehen --- dass seine Bilder nach nur wenigen Linien und Elementen begannen in seltsamerweise eine eigene Dynamik zu entwickeln. Ja manchmal drohte ihm sogar der rote Faden, an welchem er das zukünftige Werk festgemacht hatte, gänzlich aus der Hand gerissen zu werden und die Darstellung entglitt in eine Richtung, welche er hätte selbst oft nicht voraussehen können.
%%
Langsam begannen nun die ersten Linien eine gewisse Form erkennbar zu machen. Ein felsiger Pfad entfaltete sich in der linken unteren Ecke der Leinwand.
S schien von seiner bisherigen Arbeit noch nicht sonderlich überzeugt zu sein --- obschon er wusste, dass in diesem Stadium des Entstehens eine Überzeugung ohnehin nicht engebracht gewesen wäre --- jedoch mühte er sich, den Pinsel nicht aus der Hand zu legen und eiferte weiter den ersten Formen nach.
Am Ende des Pfades entstand nun eine Baum --- eine kräftige alte Eiche, so schien es, die die Jahrhunderte überdauert und den Wettern getrotzt hatte. Einige ihrer Äste waren schon kahl und abgestorben und die felsige Umgebung, welche allmählich um ihren Stamm erwuchs, schien eher untypisch für den Standort eines solch imposanten Baumes zu sein.

S hatte beim Malen oft die Angewohnheit, die unangenehmen Stellen bis zum Schluss aufzusparen und zunächst --- nach und nach --- die restlichen Elemente zumindest grob anzudeuten. Er wusste wohl, dass diese Attitude kaum eine sehr effektive war, denn misslangen ihm diese Dinge zum Schluss, so waren auch die anderen, weniger Konzentration erfordernden Bemühungen vergebens.
Er konnte sich jedoch nie überwinden, jene Spannung schon gleich von Anfang an herauszunehmen.

Diesem Zwang erlegen fuhr er nun also mit der umliegenden Landschaft fort, deren Gestalt er sich auch schon ein wenig vor seinem geistigen Auge zurecht gelegt hatte. Wolken --- der steinige Pfad stellte sich nun zunehmend mehr als eine Treppe in den Himmel, die schließlich jäh endedete, heraus.

Das Meer aus Wolken, was sich nun unterhalb der alten Eiche entfaltete wirkte wie ein sanfter Teppich, der sich über den Abgrund ausbreitete.

Schließlich trat S zurück und betrachtete seine Arbeit. Noch konnte er selbst nicht mit Bestimmtheit sagen, dass diese schlussendlich in ein gelungenes Werk übergehen würde, im Ansatz jedoch, war er damit zufrieden, was recht selten auftrat. Überhaupt gab es nicht ein einziges seiner Bilder, an dem er nicht selbst noch etwas auszusetzen oder hinzuzufügen hätte.
Jedoch existierten dennoch einige seiner Werke, die ihm selbst ein wenig ans Herz gewachsen waren und mit denen er sich auf irgendeine eigenartige Weise verbunden fühlte.

In diesem Moment klopfte es an sein Zimmer. Einen kurzen Moment lang starrte S noch auf sein Bild, dann drehte er es auf der Staffelei mit dem Gesicht zur Wand und trat zur Tür um diese zu öffnen. Es war die Haushälterin, die ihm ein wenig heiße Suppe gekocht hatte. ``Es sind zwar noch einige Reste von gestern darin verwertet, aber das sollte dem Wohlgeschmack keine Abbruch tun'', erklärte sich diese. S bedankte sich und schloss die Tür wieder. Er mochte es nicht besonders bei seiner Arbeit unterbrochen zu werden, abschon dies immer wieder eine willkommene Ablenkung darstellte, die aber nicht gerade zur Produktivität beitrug.

S stellte den Teller heißer, wohlriechender Suppe beiseite und wandte sich wieder seinem unfertigen Bilde zu. Nachdem er noch einige Momente in die Betrachtung desselben investiert hatte fuhr er nun damit fort, die bereits angedeuteten auftretenden Motive ein wenig genauer zu modellieren.

Grundsätzlich war S der Auffassung, dass jedes Bild in sich seine eigene Realität barg, sodass er es zumeist ablehnte sich im Bezug auf diese, übermäßrig an der von ihm wahrgenommenen zu orientieren. Gerade was die farblichen Eigenschaften von Objekten betraf, ließ er große Freizügigkeit walten. Auf diese Weise ließen sich einfach viele immaterielle Dinge auf sehr subtile Art und Weise transportieren.

Er fuhr nun damit fort, den Dingen Farben zuzuordnen. 

\chapter{Betrachtung}

T betrachtete das Bild eine Weile. Es schien sie an irgendetwas zu erinnern, jedoch konnte sie dieses Etwas nicht genauer identifizieren.
Das letzte Bild in der Gallerie --- ohne eine Signatur des Urhebers.


\chapter{Der Spaziergang}

Nachdem S einige Stunden so dahinvegetiert hatte ...
\end{document}%