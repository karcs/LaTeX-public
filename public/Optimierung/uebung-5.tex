23
\begin{exercise}
    Gegeben seien Vektoren $a\in\reals^{m_1}, b\in\reals^{m_2}, c\in\reals^{n_1}, d\in\reals^{n_2}$ und Matrizen $A\in\reals^{m_1\settimes n_1}$, $B\in\reals^{m_1\settimes n_2}$, $C\in\reals^{m_2\settimes n_1}$ $D\in\reals^{m_2\settimes n_2}$. Betrachtet werde die lineare Optimierungsaufgabe
    $$
    z=c^\top x + d^\top y\to\min
    $$
    bei
    $$
    Ax+By\leq a,\ Cx+Dy=b,\ x\in\reals^{n_1}_+, y\in\reals^{n_2}.
    $$
    \begin{tasks}
            \item Gib die duale Aufgabe an.
            \item Formuliere eine angepasste Version des Charakterisierungssatzes der primalen und dualen Optimierungsaufgaben.
    \end{tasks}
\end{exercise}

\begin{solution}
    \begin{tasks}
        \item  
    Es soll $z=c^\top x+d^\top y\to\min$ bei
    $$
    Ax+By\leq a,
    $$
    $$
    Cx+Dy = b,
    $$
    $$
    x\geq 0,\ y \text{ frei}
    $$
    erreicht werden.
    Idee ist es die Aufgabe in die duale Form zu überführen:
    $z=\tilde c^\top \tilde x\to\min$ bei
    $$
    \tilde Ax\geq  \tilde b,\ \tilde x\geq 0\eqno{(P)}
    $$
    Dafür ist dann die duale Aufgabe bekannt:
    $$
    \tilde b^\top \tilde u\to\max\text{ bei } \tilde A^\top \tilde u\leq \tilde c,\ \tilde u\leq 0.\eqno{(D)} 
    $$
    Dazu führen wir Variablen $y^+,y^-\geq 0$ ein mit $y=y^+-y^-$, dadurch ist $z=c^\top x+d^\top y^+-d^\top y^-$.
    Weiter kann man mithin $Ax+By\leq a$ nach $-Ax-By^+ +By^-\geq a$ überführen. $Cx+Dy=b$ schreibt man als die zwei Ungleichungen $Cx+Dy\geq b$ und $Cx+Dy\leq b$, also mit $y^+, y^-$ dann $-Cx-Dy^++Dy^-\geq,\leq -b$.
    Es ergibt sich also
    $$
    \tilde c =
    \begin{pmatrix}
        -c \\
        d \\
        -d
    \end{pmatrix},\
    \tilde x =
    \begin{pmatrix}
        x \\
        y^+\\
        y^-
    \end{pmatrix},\
    \tilde b =
    \begin{pmatrix}
        -a \\
        b \\
        -b
    \end{pmatrix},\
    A =
    \begin{pmatrix}
        -A & -B & B \\
        C & D & -D \\
        -C & -D & D
    \end{pmatrix}.
    $$
    mit der Zerlegung
    $$
    \tilde u =
    \begin{pmatrix}
        u^1 \\ u^2 \\ u^3 
    \end{pmatrix}
    $$
    lässt sich schreiben $(D)$ schreiben als
    $$
    -a^\top u^1 b^\top u^2-b^\top u^3\to\max
    $$
    bei
    $$
    -A^\top u^1+C^\top u^2-C^\top u^3\leq c
    $$
        $$
    -B^\top u^1+D^\top u^2-D^\top u^3\leq d
    $$
        $$
    B^\top u^1-D^\top u^2+D^\top u^3\leq -d
    $$
    bei $u^i\geq 0$. Man kann $v:=u^2-u^3$ als eine freie Variable schreiben und die letzten beiden Ungleichungen zusammenfassen zu $-B^\top u^1+D^\top v=d$.
    Setze $u:=-u^1$ ($u\leq 0$). Dann ergibt sich
    $$
    a^\top u+b^\top\to\max
    $$
    bei
    $$
    A^\top u+C^\top v\leq c,
    $$
    $$
    B^\top u+ D^\top v=d
    $$
    und $u\leq 0$, $v$ frei, als duales Problem.
        \item Für $(P)$/$(D)$ ist der Charakterisierungssatz bekannt. $\tilde x^\ast$ ist genau dann optimale Lösung von $(P)$, wenn $\tilde u^\ast$ existiert, sodass
    \begin{itemize}
            \item $\tilde x^\ast$ für $(P)$, $\tilde u^\ast$ zulässig für $(D)$
        \item $(\tilde u^\ast)^\top(\tilde A\tilde x^\ast-\tilde b)=0$, $(\tilde x^\ast)(\tilde A^\top \tilde u^\ast-\tilde c)=0$.
\end{itemize}
Bei uns gilt:
$$
(\tilde A\tilde x-\tilde b)=
\begin{pmatrix}
    -Ax-By^++By^-+a\\
    Cx+Dy^+-Dy^--b\\
    -Cx-Dy^++Dy^-+b
\end{pmatrix},\
\tilde u =
\begin{pmatrix}
    u^1\\
    u^2\\
    u^3
\end{pmatrix}.
$$
Also
\begin{align*}
    \tilde u^\top(\tilde A\tilde x-\tilde b)& = {(u^1)}^\top(-Ax-By^++By^-+a)\\
    & \quad {(u^2)}^\top(Cx+Dy^+-Dy^--b)\\
    & \quad {(u^3)}^\top(-Cx-Dy^++Dy^-+b)\\
    & = u^\top(Ax+By-a)\\
    & \quad +v^\top\underbrace{(Cx+Cy-b)}_{=0\text{ für zulässiges $(x,y)$}}\\
    & = u^\top(Ax+By-a)
\end{align*}
analog: ${(\tilde x)}^\top(\tilde A^\top\tilde u-\tilde c)=x^\top(A^\top u+C^\top-c)$ (falls $(u,v)$ zulässig für $(D)$).
Bedingungen im angepassten Charakterisierungssatz:
\begin{itemize}
        \item $(x^\ast,y^\ast)$ zulässig für $(P)$, $(u^\ast,v^\ast)$ zulässig für $D$.
    \item ${(u^\ast)}^\top(Ax^\ast+By^\ast-a)=0$, ${(x^\ast)}^\top(A^\top u^\ast+C^\top v^\ast-c)=0$.
\end{itemize}
\end{tasks}
\end{solution}
% 24
\begin{solution}
    \begin{tasks}
    \item
    $z=10x_1-x_2\to\max$ bei
    $$
    \begin{pmatrix}
        2 & 1 \\
        1 & 2 \\
        1 & -1 
    \end{pmatrix}
    \begin{pmatrix}
        x_1 \\
        x_2
    \end{pmatrix}
    \leq
    \begin{pmatrix}
        7 \\
        10 \\
        12
    \end{pmatrix}
    $$
    und $x_1\geq 0$, $x_2$ frei. Optimaler Zielfunktionswert ist $z^\ast=69$. Drei $\leq$-Ungleichungen führen auf $\geq 0$-Variablen $u_1,u_2,u_3$.
    Die Bedingung $x_1\geq 0$ führt auf $\geq$-Ungleichung (erste Spalte und Zielfunktionskoeffizient)
    $$
    2u_1+u_2+u_3\geq 10,
    $$
    weil $x_2$ frei ist erhalten wir die Gleichung (zweite Spalte und Zielfunktionskoeffizient)
    $$
    u_1+2u_2-u_3=-1
    $$
    Zielfunktion des dualen Problems ist $z_D=7u_1+10u_2+12u_3\to\min$ (da $(P)$ ein Maximierungsproblem war, die Koeffizienten sind die rechte Seite von $(P)$).
    Es ist daher $u^\ast=(3,0,4)$ eine Lösung des dualen Problems, denn $z_D(u^\ast)=7\cdot 3+12\cdot 4=69=z^\ast$. Aufgrund der Dualität ist $u^\ast$ optimale Lösung der dualen Aufgabe.
        \item Charakterisierungssatz liefert: Optimale Lösung des Ausgangspbolems muss erfüllen:
    $$
    u_1^\ast(2x_1+x_2-7)=0,\ u_2^\ast(x_1+2x_2-10)=0,\\
    u_3^\ast(x_1-x_2-12)=0,\ x_1(2u_1^\ast+u_2^\ast-10)=0
    $$
    erfüllt für $u^\ast$ sind die Gleichung mit $u_2^\ast$ und $x_1$. Es muss noch gelten $2x_1+x_2-7=0$, $x_1-x_2-12=0$ also $x_1^\ast=19/3$, $x_2^\ast=-17/3$.
\end{tasks}
\end{solution}



















