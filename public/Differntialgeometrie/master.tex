\documentclass{article}
\usepackage[de]{mathenv}
%My personal maths package
%\bibliography{Bibliography.bib}

%%%%%%%%%%%%%%%%%%%%%%%%%%%%%%%%%%%%%%%%%%%%%%%%%%%%%%%%%%%%%%%%%%%%%%%%%%%%%%
%%%%% MATH PACKAGES %%%%%%%%%%%%%%%%%%%%%%%%%%%%%%%%%%%%%%%%%%%%%%%%%%%%%%%%%%
%%%%%%%%%%%%%%%%%%%%%%%%%%%%%%%%%%%%%%%%%%%%%%%%%%%%%%%%%%%%%%%%%%%%%%%%%%%%%%

% very good package
%\usepackage{mathtools}

%%% font stuff
\usepackage[T1]{fontenc}        % for capitals in section /paragraph etc.
\usepackage[utf8]{inputenc}     % use utf8 symbols in code

\usepackage{amssymb,amsmath,amsfonts} % amsthm not needed -- use my own envs
%\usepackage{mathtools}
%\mathtoolsset{showonlyrefs}
% further alternative math packages: unicode-math, abx-math
\usepackage{bm}
\usepackage{mathrsfs} % used for: fraktal math letters
\usepackage[bigsqcap]{stmaryrd} % used for: big square cap symbol
\usepackage{xargs}

% standard packages
%\usepackage{color} % for color

%%
%%index

\newcommand*{\keyword}[2][\empty]{\emph{#2}\ifx#1\empty\index{#2}\else\index{#1}\fi}
\newcommand*{\person}[1]{\textsc{#1}}

%%%%%%%%%%%%%%%%%%%%%%%%%%%%%%%%%%%%%%%%%%%%%%%%%%%%%%%%%%%%%%%%%%%%%%%%%%%%%%%%%%%%%%%%%%%%%%%%%%%%%%%%%%%%
%%%%% MATH ALPHABETS & SYMBOLS %%%%%%%%%%%%%%%%%%%%%%%%%%%%%%%%%%%%%%%%%%%%%%%%%%%%%%%%%%%%%%%%%%%%%%%%%%%%%
%%%%%%%%%%%%%%%%%%%%%%%%%%%%%%%%%%%%%%%%%%%%%%%%%%%%%%%%%%%%%%%%%%%%%%%%%%%%%%%%%%%%%%%%%%%%%%%%%%%%%%%%%%%%

% w: http://milde.users.sourceforge.net/LUCR/Math/math-font-selection.xhtml

% ===== Set quick commands for math letters ================================================================
% calagraphic letters (only upper case available; standard)
\newcommand{\cA}{\mathcal{A}}
\newcommand{\cB}{\mathcal{B}}
\newcommand{\cC}{\mathcal{C}}
\newcommand{\cD}{\mathcal{D}}
\newcommand{\cE}{\mathcal{E}}
\newcommand{\cF}{\mathcal{F}}
\newcommand{\cG}{\mathcal{G}}
\newcommand{\cH}{\mathcal{H}}
\newcommand{\cI}{\mathcal{I}}
\newcommand{\cJ}{\mathcal{J}}
\newcommand{\cK}{\mathcal{K}}
\newcommand{\cL}{\mathcal{L}}
\newcommand{\cM}{\mathcal{M}}
\newcommand{\cN}{\mathcal{N}}
\newcommand{\cO}{\mathcal{O}}
\newcommand{\cP}{\mathcal{P}}
\newcommand{\cQ}{\mathcal{Q}}
\newcommand{\cR}{\mathcal{R}}
\newcommand{\cS}{\mathcal{S}}
\newcommand{\cT}{\mathcal{T}}
\newcommand{\cU}{\mathcal{U}}
\newcommand{\cV}{\mathcal{V}}
\newcommand{\cW}{\mathcal{W}}
\newcommand{\cX}{\mathcal{X}}
\newcommand{\cY}{\mathcal{Y}}
\newcommand{\cZ}{\mathcal{Z}}

% bold math letters (standard)
\newcommand{\bfA}{\mathbf{A}}
\newcommand{\bfB}{\mathbf{B}}
\newcommand{\bfC}{\mathbf{C}}
\newcommand{\bfD}{\mathbf{D}}
\newcommand{\bfE}{\mathbf{E}}
\newcommand{\bfF}{\mathbf{F}}
\newcommand{\bfG}{\mathbf{G}}
\newcommand{\bfH}{\mathbf{H}}
\newcommand{\bfI}{\mathbf{I}}
\newcommand{\bfJ}{\mathbf{J}}
\newcommand{\bfK}{\mathbf{K}}
\newcommand{\bfL}{\mathbf{L}}
\newcommand{\bfM}{\mathbf{M}}
\newcommand{\bfN}{\mathbf{N}}
\newcommand{\bfO}{\mathbf{O}}
\newcommand{\bfP}{\mathbf{P}}
\newcommand{\bfQ}{\mathbf{Q}}
\newcommand{\bfR}{\mathbf{R}}
\newcommand{\bfS}{\mathbf{S}}
\newcommand{\bfT}{\mathbf{T}}
\newcommand{\bfU}{\mathbf{U}}
\newcommand{\bfV}{\mathbf{V}}
\newcommand{\bfW}{\mathbf{W}}
\newcommand{\bfX}{\mathbf{X}}
\newcommand{\bfY}{\mathbf{Y}}
\newcommand{\bfZ}{\mathbf{Z}}
\newcommand{\bfa}{\mathbf{a}}
\newcommand{\bfb}{\mathbf{b}}
\newcommand{\bfc}{\mathbf{c}}
\newcommand{\bfd}{\mathbf{d}}
\newcommand{\bfe}{\mathbf{e}}
\newcommand{\bff}{\mathbf{f}}
\newcommand{\bfg}{\mathbf{g}}
\newcommand{\bfh}{\mathbf{h}}
\newcommand{\bfi}{\mathbf{i}}
\newcommand{\bfj}{\mathbf{j}}
\newcommand{\bfk}{\mathbf{k}}
\newcommand{\bfl}{\mathbf{l}}
\newcommand{\bfm}{\mathbf{m}}
\newcommand{\bfn}{\mathbf{n}}
\newcommand{\bfo}{\mathbf{o}}
\newcommand{\bfp}{\mathbf{p}}
\newcommand{\bfq}{\mathbf{q}}
\newcommand{\bfr}{\mathbf{r}}
\newcommand{\bfs}{\mathbf{s}}
\newcommand{\bft}{\mathbf{t}}
\newcommand{\bfu}{\mathbf{u}}
\newcommand{\bfv}{\mathbf{v}}
\newcommand{\bfw}{\mathbf{w}}
\newcommand{\bfx}{\mathbf{x}}
\newcommand{\bfy}{\mathbf{y}}
\newcommand{\bfz}{\mathbf{z}}

% fractal math letters (standard)
\newcommand{\fkA}{\mathfrak{A}}
\newcommand{\fkB}{\mathfrak{B}}
\newcommand{\fkC}{\mathfrak{C}}
\newcommand{\fkD}{\mathfrak{D}}
\newcommand{\fkE}{\mathfrak{E}}
\newcommand{\fkF}{\mathfrak{F}}
\newcommand{\fkG}{\mathfrak{G}}
\newcommand{\fkH}{\mathfrak{H}}
\newcommand{\fkI}{\mathfrak{I}}
\newcommand{\fkJ}{\mathfrak{J}}
\newcommand{\fkK}{\mathfrak{K}}
\newcommand{\fkL}{\mathfrak{L}}
\newcommand{\fkM}{\mathfrak{M}}
\newcommand{\fkN}{\mathfrak{N}}
\newcommand{\fkO}{\mathfrak{O}}
\newcommand{\fkP}{\mathfrak{P}}
\newcommand{\fkQ}{\mathfrak{Q}}
\newcommand{\fkR}{\mathfrak{R}}
\newcommand{\fkS}{\mathfrak{S}}
\newcommand{\fkT}{\mathfrak{T}}
\newcommand{\fkU}{\mathfrak{U}}
\newcommand{\fkV}{\mathfrak{V}}
\newcommand{\fkW}{\mathfrak{W}}
\newcommand{\fkX}{\mathfrak{X}}
\newcommand{\fkY}{\mathfrak{Y}}
\newcommand{\fkZ}{\mathfrak{Z}}
\newcommand{\fka}{\mathfrak{a}}
\newcommand{\fkb}{\mathfrak{b}}
\newcommand{\fkc}{\mathfrak{c}}
\newcommand{\fkd}{\mathfrak{d}}
\newcommand{\fke}{\mathfrak{e}}
\newcommand{\fkf}{\mathfrak{f}}
\newcommand{\fkg}{\mathfrak{g}}
\newcommand{\fkh}{\mathfrak{h}}
\newcommand{\fki}{\mathfrak{i}}
\newcommand{\fkj}{\mathfrak{j}}
\newcommand{\fkk}{\mathfrak{k}}
\newcommand{\fkl}{\mathfrak{l}}
\newcommand{\fkm}{\mathfrak{m}}
\newcommand{\fkn}{\mathfrak{n}}
\newcommand{\fko}{\mathfrak{o}}
\newcommand{\fkp}{\mathfrak{p}}
\newcommand{\fkq}{\mathfrak{q}}
\newcommand{\fkr}{\mathfrak{r}}
\newcommand{\fks}{\mathfrak{s}}
\newcommand{\fkt}{\mathfrak{t}}
\newcommand{\fku}{\mathfrak{u}}
\newcommand{\fkv}{\mathfrak{v}}
\newcommand{\fkw}{\mathfrak{w}}
\newcommand{\fkx}{\mathfrak{x}}
\newcommand{\fky}{\mathfrak{y}}
\newcommand{\fkz}{\mathfrak{z}}

% script math symbols (only uppercase; package: mathrsfs)
\newcommand{\sA}{\mathscr{A}}
\newcommand{\sB}{\mathscr{B}}
\newcommand{\sC}{\mathscr{C}}
\newcommand{\sD}{\mathscr{D}}
\newcommand{\sE}{\mathscr{E}}
\newcommand{\sF}{\mathscr{F}}
\newcommand{\sG}{\mathscr{G}}
\newcommand{\sH}{\mathscr{H}}
\newcommand{\sI}{\mathscr{I}}
\newcommand{\sJ}{\mathscr{J}}
\newcommand{\sK}{\mathscr{K}}
\newcommand{\sL}{\mathscr{L}}
\newcommand{\sM}{\mathscr{M}}
\newcommand{\sN}{\mathscr{N}}
\newcommand{\sO}{\mathscr{O}}
\newcommand{\sP}{\mathscr{P}}
\newcommand{\sQ}{\mathscr{Q}}
\newcommand{\sR}{\mathscr{R}}
\newcommand{\sS}{\mathscr{S}}
\newcommand{\sT}{\mathscr{T}}
\newcommand{\sU}{\mathscr{U}}
\newcommand{\sV}{\mathscr{V}}
\newcommand{\sW}{\mathscr{W}}
\newcommand{\sX}{\mathscr{X}}
\newcommand{\sY}{\mathscr{Y}}
\newcommand{\sZ}{\mathscr{Z}}

%%%%%%%%%%%%%%%%%%%%%%%%%%%%%%%%%%%%%%%%%%%%%%%%%%%%%%%%%%%%%%%%%%%%%%%%%%%%%%%%%%%%%%
%%%% BOLD MATH IN BOLD TEXT ENVIRONMENT %%%%%%%%%%%%%%%%%%%%%%%%%%%%%%%%%%%%%%%%%%%%%%
%%%%%%%%%%%%%%%%%%%%%%%%%%%%%%%%%%%%%%%%%%%%%%%%%%%%%%%%%%%%%%%%%%%%%%%%%%%%%%%%%%%%%%

% for bold math in bold text (e.g. sections)
\makeatletter
\g@addto@macro\bfseries{\boldmath}
\makeatother

\def\brackets#1{\ifx#1\empty\else\left(#1\right)\fi}

%%%%%%%%%%%%%%%%%%%%%%%%%%%%%%%%%%%%%%%%%%%%%%%%%%%%%%%%%%%%%%%%%%%%%%%%%%%%%%%%%%%%%%
%%%%% CATEGORY THEORY %%%%%%%%%%%%%%%%%%%%%%%%%%%%%%%%%%%%%%%%%%%%%%%%%%%%%%%%%%%%%%%%
%%%%%%%%%%%%%%%%%%%%%%%%%%%%%%%%%%%%%%%%%%%%%%%%%%%%%%%%%%%%%%%%%%%%%%%%%%%%%%%%%%%%%%

% ===== Category theory concepts =====================================================

\newcommand{\Ob}{\mathop\mathrm{Ob}}
\newcommand{\Mor}{\mathop\mathrm{Mor}}

\newcommand{\ccoprod}{\bigsqcup}
\newcommand{\cprod}{\bigsqcap}
\newcommand{\cincl}{\mathop\mathrm{incl}}
\newcommand{\cproj}{\mathop\mathrm{pr}}


% ===== Define standard categories ===================================================

% Define sets
\newcommand{\Set}{\mathbf{Set}}
% Define set-builder operator (equivalent to gen for algebras)
\newcommand{\set}[1]{\left\{#1\right\}}
% define interval operator: o - open, c - closed
\newcommand{\intervalcc}[2]{\left[#1,#2\right]}
\newcommand{\intervalco}[2]{\left[#1,#2\right)}
\newcommand{\intervaloc}[2]{\left(#1,#2\right]}
\newcommand{\intervaloo}[2]{\left(#1,#2\right)}

\newcommand{\inter}{\mathop\mathrm{int}}
\newcommand{\face}{\mathop\mathrm{F}}
\newcommand{\Pol}{\mathop\mathrm{Pol}}
\newcommand{\Inv}{\mathop\mathrm{Inv}}
\def\struct#1{\gen{#1}}

\let\originaltimes\times%
\renewcommand{\times}{\mathbin{\sqcap}}
\newcommand\settimes{\originaltimes}
\newcommand{\setleq}{\subseteq}
\newcommand{\setgeq}{\supseteq}

\newcommand{\pderive}[2]{\frac{\partial{#1}}{\partial{#2}}}
\renewcommand{\div}{\mathop\mathrm{div}}

%% Diffgeo
\def\Ric{\mathop\mathrm{Ric}}
\def\ric{\mathop\mathrm{ric}}
\def\tr{\mathop\mathrm{tr}}

\def\cotimes{\mathbin{\sqcup}}

\def\@rightopen#1{\ifx#1]{\right]}\else{\interval@errmessage}\fi}
\def\@leftclosed[#1){\left[#1\right)}
\makeatother

% finite
\newcommand{\fin}{\mathrm{fin}}

% Define groups (optarg: properties such as -> abelian, noetherian (acc), artinian (dcc) etc.)
\newcommand{\Grp}[1][\empty]{\if\empty{#1}{\mathbf{Grp}}\else{\mathbf{Grp}_{#1}}}
\def\PGL{\mathrm{PGL}}
\def\PGammaL{\mathrm{P\Gamma L}}
\def\GL{\mathrm{GL}}
% Define rings
\newcommand{\rg}{\mathrm{rg}} %rank of a matrix
\newcommand{\Rg}[1][\empty]{\if\empty{#1}{\mathbf{Rg}}\else{\mathbf{Rg}_{#1}}}
\edef\units#1{#1^{\settimes}}
\def\dual#1{#1^{\ast}}

%% redefine the command \P to produce the projective functor in math mode
\let\parsymb\P%
\def\P{\ifmmode\mathrm{P}\else\parsymb\fi}
\renewcommand{\iff}{\ifmmode\equival\else{if and only if}\fi}
\newcommand{\quotring}{\mathop{\mathrm{Q}}}
\newcommand{\rad}{\mathrm{rad}}
% Standard rings
% integral domains
\newcommand{\ID}{\mathbf{ID}}
% unique factorization domains
\newcommand{\UFD}{\mathbf{UFD}}
% principal ideal domains
\newcommand{\PID}{\mathbf{PID}}

% Define modules over a group or ring
\newcommand{\Mod}[1]{\mathbf{Mod}_{#1}}
% Define vector space over a field
\renewcommand{\Vec}[1]{\mathbf{Vec}_{#1}}%

% when cases
\def\otherwise{\textrm{otherwise}}

% new concepts
\newcommand{\new}[1]{\emph{#1}}

\usepackage{xifthen,xstring}

% replace the bar command by overline when argument just one character (shorter and better)
%$\let\oldbar\bar
%\renewcommand{\bar}[1]{\StrLen{#1}[\length]\ifthenelse{\length > 1}{\overline{#1}}{\oldbar{#1}}}
\def\bar{\overline}

%% argument in equatoin
\def\arg{\bullet}

% groups and algebras
\newcommand{\Con}{\mathop\mathrm{Con}}
\newcommand{\Sub}{\mathop\mathrm{Sub}}
\newcommand{\Hom}{\mathop\mathrm{Hom}}
\newcommand{\Aut}{\mathop\mathrm{Aut}}
\newcommand{\Out}{\mathop\mathrm{Out}}
\newcommand{\End}{\mathop\mathrm{End}}
\newcommand{\id}{\mathop\mathrm{id}}
\newcommand{\rk}{\mathop\mathrm{rk}} % rank of a group module/ lattice
\newcommandx{\con}[1][1=\empty]{\ifx#1\empty{\mathop{\mathrm{con}}}\else{\mathop{\mathrm{con}}\left(#1\right)}\fi}
\newcommand{\leftsemidirprod}[1][]{\mathbin{\ifx&#1&\ltimes\else{\ltimes_{#1}}\fi}}
\newcommand{\rightsemidirprod}[1][]{\ifx#1\empty\rtimes\else{\rtimes_{#1}}\fi}
\newcommand{\normalisor}[2][]{\ifx#1\empty{\mathrm{N}\left(#2\right)}\else{\mathrm{N}_{#1} \left(#2\right)}\fi}
% support
\newcommand{\spt}{\mathop{\mathrm{spt}}}
% commutator
\newcommand{\gcom}[2]{\left[#1,#2\right]}

% physics stuff
\newcommand{\float}[3][\empty]{\ifx#1\empty{{#2}\cdot{10^{#3}}}\else{{#2}\cdot{{#1}^{#3}}}\fi}
\makeatletter
\def\newunit#1{\@namedef{#1}{\mathrm{#1}}}
\def\mum{\mathrm{\mu m}}
\def\ohm{\Omega}
\newunit{V}
\newunit{mV}
\newunit{kV}
\newunit{s}
\newunit{ms}
\def\mus{\mathrm{\mu s}}
\newunit{m}
\newunit{nm}
\newunit{cm}
\newunit{mm}
\newunit{fF}
\newunit{A}

\newunit{fA}
\newunit{C}

% elements
\def\newelement#1{\@namedef{#1}{\mathrm{#1}}}
\newelement{Si}
\makeatother


% groups
\newcommand{\ord}{\mathop\mathrm{ord}}
\newcommand{\divides}{|}

\newcommand{\conleq}{\trianglelefteq}
\newcommand{\congeq}{\trianglerighteq}

% common algebraic objects
\newcommand{\reals}{\mathbb{R}} 			% real numbers
\newcommand{\nats}{\mathbb{N}} 				% natural numbers
\newcommand{\ints}{\mathbb{Z}} 				% integers
\newcommand{\rats}{\mathbb{Q}}				% rationals
\newcommand{\complex}{\mathbb{C}}			% complex numbers
\newcommand{\field}[1]{\mathbb{F}_{#1}}  		% finite field
\newcommand{\cards}{\boldsymbol{Cn}}                     % The cardinal numbers
\newcommand{\ords}{\boldsymbol{On}}                      % The ordinal numbers

% graphs
\def\KG{\mathop\mathrm{KG}}                     % Knesergraph

\newcommand{\uvect}{\boldsymbol{e}}

% new operators and relations

%%%%%%%%%%%%%%%%%%%
% complex numbers %
%%%%%%%%%%%%%%%%%%%

\renewcommand{\Re}{\mathop\mathrm{Re}}		% real part
\renewcommand{\Im}{\mathop\mathrm{Im}}		% imaginary part
\newcommand{\sgn}{\mathop\mathrm{sgn}}				% the sign operator (0 for 0)

%%%%%%%%%%%%%%%%
% reel numbers %
%%%%%%%%%%%%%%%%

\newcommand{\floor}[1]{\left\lfloor#1\right\rfloor}
\newcommand{\ceil}[1]{\left\lceil#1\right\rceil}

%%%%%%%%%%%%%%%%%%
% set operations %
%%%%%%%%%%%%%%%%%%

\newcommand{\intersect}{\cap}			% intersect to sets
\newcommand{\setjoin}{\cup}				% join two sets
\newcommand{\setmeet}{\cap}                     % intersect to sets
\newcommand{\bigsetjoin}{\bigcup}			% the union of sets ... subscripts to be added
\newcommand{\distunion}{\dot{\bigcup}}	% disjoint union of sets ... subscripts to be added
\newcommand{\bigsetmeet}{\bigcap}		% intersection of sets
\newcommand{\powerset}[1][]{\ifx&#1&\mathcal{P}\else\mathcal{P}_{#1}\fi}		% powerset ... to be customized
\newcommand{\card}[1]{\left|#1\right|}

%%%%%%%%%%%%%%%%%%%%%%%%%%%%%%%%%%%%%%%%
% composition operations of structures %
%%%%%%%%%%%%%%%%%%%%%%%%%%%%%%%%%%%%%%%%

%\newcommand{\setprod}{\bigtimes}			% setproduct - needed
\newcommand{\dirprod}{\bigotimes} 			% direct product for groups and spaces
\newcommand{\dirtimes}{\otimes}				% direct multiply for groups and spaces
\newcommand{\dirsum}{\bigoplus} 			% direct sum for groups and spaces
\newcommand{\dirplus}{\oplus}				% direct add for groups and spaces
\newcommand{\inprod}[2]{\left\langle #1,#2 \right\rangle}

\newcommand{\tuple}{\meet}
\newcommand{\cotuple}{\join}

%%%%%%%%%%%%%%
% categories %
%%%%%%%%%%%%%%

% combinatorics
%%

\renewcommand{\binom}[3][\empty]{\if\empty{#1}{{#2 \choose #3}}\else{{#2 \choose #3}_{#1}}}

%%%%%%%%%%%%%%%%%%%%%%%%%%%%%%%%%%%%%%%%%%%%%%%%%%%%%
% metric spaces and normed spaces and vector spaces %
%%%%%%%%%%%%%%%%%%%%%%%%%%%%%%%%%%%%%%%%%%%%%%%%%%%%%

\newcommand{\dist}{\mathop\mathrm{dist}}				% distance operator ... dist(A,b), where A is a set and b a point
\newcommand{\diam}{\mathop\mathrm{diam}}	% diameter operator for sets
\newcommand{\norm}[1]{\left\Vert #1 \right\Vert}	% norm in a normed space ... subscript to be added
\newcommand{\conv}{\mathop\mathrm{conv}} 			% convex hull - vectorspaces
\newcommand{\lin}{\mathop\mathrm{lin}} 				% linear hull - vectorspaces
\newcommand{\aff}{\mathop\mathrm{aff}}				% affine hull - vectorspaces

%%%%%%%%%%%%%%%%%%%%%%
% operators in rings %
%%%%%%%%%%%%%%%%%%%%%%

\newcommand{\lcm}{\mathop\mathrm{lcm}}				% least common multiple - in euclidean rings
\renewcommand{\gcd}{\mathop\mathrm{gcd}}				% greatest command devisor - in euclidean rings
\newcommand{\res}{\mathop\mathrm{res}}				% residue of p mod q is res(p,q)
\renewcommand{\mod}{\textrm{ mod }}

%%%%%%%%%%%%%%%%%%%
% logical symbols %
%%%%%%%%%%%%%%%%%%%

%\newcommand{\impliedby}{\Leftarrow}				% reverse implicatoin arrow
%\newcommand{\implies}{\Rightarrow}				% implication
\newcommand{\equival}{\Leftrightarrow}				% equivalence

%%%%%%%%%%%%%%%%%%%%%%%%%%%%%%%%%%
% functions - elementary symbols %
%%%%%%%%%%%%%%%%%%%%%%%%%%%%%%%%%%

\newcommand{\rest}[1]{\left. #1\right\vert}		% restriction of a function to a set / also used as restriction in other terms like differential expressions / evaluation of a function
\newcommand{\rto}[3][]{#2\ifx&#1&\rightarrow\else\stackrel{#1}{\rightarrow}\fi#3}
\renewcommand{\to}{\rightarrow}							% arrow between domain and image
\newcommand{\dom}{\mathop\mathrm{dom}}					% domain of a function
\newcommand{\im}{\mathop\mathrm{im}}						% image of a function
\newcommand{\compose}{\circ}							% compose two functions
\newcommand{\cont}{\mathop\mathrm{C}}					% continuous functions from a domain into the reels or complex numbers 

%%%%%%%%%%%%%%%%%%%%
% groups - symbols %
%%%%%%%%%%%%%%%%%%%%

\newcommand{\stab}[1][]{\if&#1{\mathop\mathrm{stab}}&\else{\mathop\mathrm{stab}_{#1}}\fi}					% the stabilizer ... subscripts to be added
\newcommand{\orb}[1][]{\ifx&#1&\mathrm{orb}\else\mathrm{orb}_{#1}\fi} 					% orbit ... subscrit to be added (group)
\newcommand{\gen}[2][\empty]{\ifx#1\empty{\left\langle#2\right\rangle}\else{\left\langle#2\right\rangle_{#1}}\fi}					% generate ... kind of hull operator ---- to be thought of !!!!!!!

\def\Clo{\mathrm{Clo}}
\def\Loc{\mathrm{Loc}}
%% nets
\newcommand{\net}[2][\empty]{\ifx#1\empty{\left(#2\right)}\else{{\left(#2\right)}_{#1}}\fi}

%% open half ray
\newcommand{\ray}[2]{R_{#1}(#2)}

%% new
\let\oldcong\cong%
\newcommand{\iso}{\oldcong}

\def\cong{\equiv}
\newcommand{\base}[2]{\left[#2\right]_{#1}}                                   % base n expansion of some number
%%%%%%%%%%%%%%%%%%%%%%%%%%%%%%%%%
% matrices and linear operators %
%%%%%%%%%%%%%%%%%%%%%%%%%%%%%%%%%

\newcommand{\diag}{\mathop\mathrm{diag}}				% diagonal matrix or operator
\newcommand{\Eig}[1]{\mathop\mathrm{Eig}_{#1}}		% eigenspace for a certain eigenvalie		
\newcommand{\trace}{\mathop\mathrm{tr}}				% trace of a matrix
\newcommand{\trans}{\top} 							% transponse matrix

%%%%%%%%%%%%%
% constants %
%%%%%%%%%%%%%

\renewcommand{\i}{\boldsymbol{i}}			% imaginary unit
\newcommand{\e}{\boldsymbol{e}}				% the Eulerian constant

%%%%%%%%%%%%%%%%%%%%%%%%%%%%%%
% limit operators and arrows %
%%%%%%%%%%%%%%%%%%%%%%%%%%%%%%

\newcommand{\upto}{\uparrow}				% convergence from above
\newcommand{\downto}{\downarrow}			% convergence from below

%%
% other
%%
\newcommand{\cind}{\mathop\mathrm{Ind}}		% Cauchy index
\newcommand{\sgnc}{\sigma}					% sign changes
\newcommand{\wnumb}{\omega}					% winding number
\newcommand{\cfunc}{\mathop\mathrm{Cf}}		% Cauchy function of a compact curve in complex\setminus\{0\}


% evaluation of a function as a difference or single value

\newcommand{\abs}[1]{\left|#1\right|}
\newcommand{\conj}[1]{\overline{#1}}
\newcommand{\diff}{\mathop\mathrm{d}}

%%%%% test
\newcommand{\distjoin}{\mathaccent\cdot\cup}	% to be modified (name)
\newcommand{\cl}{\mathop\mathrm{cl}}				% topological closure
\newcommand{\sphere}{\mathbb{S}} % n-sphere
\newcommand{\ball}{\mathbb{B}} % n-ball
\newcommand{\bound}{\partial}
\newcommand{\bigmeet}{\mathop\mathrm{\bigwedge}}
\newcommand{\bigjoin}{\mathop\mathrm{\bigvee}}
\newcommandx{\rchar}[1][1=\empty]{\mathop\mathrm{char}\brackets{#1}}              % characteristic of a ring
\newcommand{\lgor}{\vee}                               % logical
\newcommand{\lgand}{\wedge}
\newcommand{\codim}{\mathop\mathrm{codim}}
\newcommand{\row}{\mathop\mathrm{row}}
\newcommand{\cone}{\mathop\mathrm{cone}}
\newcommand{\comp}{\mathop\mathrm{comp}}
\newcommand{\proj}{\mathrm{P}}
\def\PG{\mathrm{PG}}           % projective space
\newcommand{\meet}{\wedge}
\newcommand{\join}{\vee}
\newcommand{\col}{\mathop\mathrm{col}}
\newcommand{\vol}{\mathop\mathrm{vol}\nolimits}
%arrangements
\newcommand{\tpert}{\mathop\mathrm{tpert}}
\renewcommand{\epsilon}{\varepsilon}
% groups
\newcommand{\symgr}{\mathop\mathrm{Sym}}
\newcommand{\symalg}{\mathop\mathrm{S}}
\newcommand{\extalg}{\mathop\mathrm{\Lambda}}
\newcommand{\extpow}[1]{\mathop\mathrm{\Lambda}^{#1}}

%% set hulloperators -> define



\newcommandx{\homl}[3][1=1,2=2,3=3]{\ifx#1\empty{\mathrm{Hom}}\else{\mathrm{Hom}_{#1}}\fi(#2,#3)}
\makeatletter
\newenvironment{myproofof}[1]{\par
  \pushQED{\qed}%
  \normalfont \topsep6\p@\@plus6\p@\relax
  \trivlist
  \item[\hskip\labelsep
        \bfseries
    Proof of #1\@addpunct{.}]\ignorespaces
}{%
  \popQED\endtrivlist\@endpefalse
}
\makeatother



%%% environment test with enumerates

    

% Local variables:
% mode: tex
% End:


% for use of german language in a document
\usepackage[utf8]{inputenc} % this is needed for umlauts
\usepackage[ngerman]{babel} % this is needed for umlauts
\usepackage[T1]{fontenc}    % this is needed for correct output of umlauts in pdf

\usepackage{framed}
\usepackage{blindtext}

\begin{document}
\hfill{30.05.2014}

\begin{definition} Eine Riemannsche Mannigfaltigkeit $M$ heißt ein Raum konstanter (Schnitt-)Krümmung, wenn es eine Zahl $\kappa$ mit $K_{\sigma}=\kappa$ for alle $2$-dimensionalen Unterräume $\sigma$ von Tangentialräumen von $M$. Eine solche Mannigfaltigkeit $M$ nennt man \keyword{elliptisch}, falls $\kappa>0$, \keyword{flach}, falls $\kappa=0$ bzw. \keyword{hyperbolisch} falls $\kappa<0$. 
\end{definition}%%%%%%%%%%%%%%

\begin{example}
    Der euklidische Raum $\reals^n$ ist ein flacher Raum, die sphäre $\sphere^n$ ist ein elliptischer Raum konstanter Krümmung.
\end{example}

Zur Erinnerung:
Bezeichnungsmäßig galt $$
R_1(X,Y)Z:=g(Y,Z)X-g(X,Z)Y,
$$
und
$$
\eqalign{
    k_1(X,Y)
    &:=g(r_1(X,Y)Y,X)\cr
    &=g(g(Y,Y)X-g(X,Y)Y,X)=g(X,X)g(Y,Y)-{g(X,Y)}^2.
}
$$

\begin{fact}\label{f39}
    Falls die Schnittkrümmung $K_{\sigma}$ nur von $p$ aber nicht von $\sigma\setleq T_p M$ abhängt, also eine funktion $K:M\to\reals$ ist, dann gilt $R=K R_1$.
\end{fact}

\begin{proof}
    Es galt $k(X,Y)=K k_1(X,Y)$. Die Formel zur Bestimmung von $R$ aus $k$ im Beweis von Lemma 3.7 ist eine Linearkombination von Termen der Form $k(aX+bY, cX+dY)$, ferner erfüllt $R_1(X,Y)Z$ dieselben Rechenregeln (1)--(4) aus Lemma 3.6 wie $R(X,Y)Z$, also $R=K R_1$ (da man $R$ aus $k$ in gleicher Weise extrahiert wie $R_1$ aus $k_1$).
\end{proof}

Insbesondere im Falle $n=2$ ist die Voraussetzung erfüllt, da es nur einen $2$-dimensionalen Unterraum von $T_p M$ gibt, also $R = K R_1$.

für $n\geq 3$ gitl aber

\begin{theorem}[\person{F. Schur}, 1886]\label{thmschur} Wenn die Schnittkrümmung $K_\sigma$ einer zusammenhängenden Mannigfaltigkeit der Dimension $n\geq 3$ nicht von der Richtung $\sigma$, sondern nur von dme Punkt $p$ abhängt, dann ist sie konstant (hängt auch nicht von dem Punkt ab).
\end{theorem}

Für den Beweis benötigen wir noch ein paar Vorbereitungen.

\begin{definition}[Erweiterung des linearen Zusammenhangs auf $\fkF(M)$]
    Für $f\in\fkF(M)$ bezeichne $\nabla_X f:= Xf\in \fkF(M)$.
\end{definition}

\begin{definition}[Kovariante Ableitung von Tensorfeldern]
    Sie $\nabla$ ein linearer Zusammenhang und $A:\fkX_r(M)\to \fkX_s(M)$ ein Tensorfeld auf $M$ vom Typ $(r,s)$ mit $r\geq 1, s\in\set{0,1}$. Dann definieren wir die \keyword{kovariante Ableitung} von $A$ in Richtung $X$ durch
    $$
    \eqalign{
        (\nabla_X A)(Y_1,\ldots,Y_r)
        &:=\nabla(A(Y_1,\ldots,Y_r))\cr
        &\qquad-\sum_{i=1}^r{A(Y_1,\ldots,Y_{i-1},\nabla_X Y_i,Y_{i+1},\ldots,Y_r)}.
    }
    $$
    Im Falle $r=0$, $\nabla_X = Xf$.
\end{definition}

\begin{remark}
    Es ist dann $\nabla_X A$ ein Tensorfeld desselben Typs wie $A$. Ferner ist $\nabla A$ mit $(\nabla A)(X,Y_1,\ldots,Y_r)=(\nabla_X a)(Y_1,\ldots,Y_r)$ ein Tensorfeld des Typs $(r+1,s)$.
\end{remark}

\begin{proof}
    Wir müssen die $\fkF(M)$-Linearität von $\nabla A(X,Y_1,\ldots,Y_r)$. Additivität ist unmittelbar ersichtlich, da $A$ $r$-multilinear ist und $\nabla_X(Y+Z)=\nabla_X Y + \nabla_X Z$.
    Zu zeigen ist $\fkF(M)$-Homogenität in jeder Komponente
    $$
    (\nabla_X A)(Y_1,\ldots,Y_{i-1},fY_i,Y_{i+1},\ldots,Y_r) = f (\nabla_X A)(Y_1,\ldots,Y_r).
    $$
    Es gilt
    $$
    \eqalign{
        (\nabla_X A)(Y_1,\ldots,Y_{i-1}, f Y_i, Y_{i+1},\ldots,Y_r)
        & = \nabla_X (fA(Y_1,\ldots,Y_r))\cr
        &\qquad-\sum_{i=1}^r{fA(Y_1,\ldots,Y_{i-1}\nabla_X Y_i,Y_{i+1},\ldots, Y_r)}\cr
        &\qquad-(Xf)(A(Y_1,\ldots,Y_r))\cr
        & = (Xf) A(Y_1,\ldots,Y_r)\cr
        &\qquad+f\nabla_X A(Y_1,\ldots,Y_r)\cr
        &\qquad-f(\sum_{i=1}^r{A(Y_1,\ldots,Y_{i-1},\nabla_X Y_i, Y_{i+1},\ldots, Y_r)})\cr
        &\qquad-(Xf)A(Y_1,\ldots,Y_r),
    }
    $$
        wie gewünscht.
        Fenrer ist
        $$(\nabla A)(f X, Y_1,\ldots, Y_r) = (\nabla_{fX} A)(Y_1,\ldots,Y_r)=f(\nabla_X A)(Y_1,\ldots,Y_r)$$
        und damit der zweite Teil der Bemerkung erfüllt.
\end{proof}

Nach lemma 3.5 hängt $\rest{(\nabla_X A)}_p(Y_1,\ldots,Y_r)$ nur von $\rest{Y_1}_p,\ldots,\rest{Y_r}_p$ ab.

Daher kann $\rest{(\nabla_X A)}_p$ als multilineare Abbildung $T_p M \settimes \cdots \settimes T_p M \to {(T_p M)}^s$ aufgefasst werden.

Wir führen die Schreibweise $(\nabla_X R)(Y,Z)V = \nabla_X(RY,Z)V)-R(\nabla_X Y,Z)-R(Y,\nabla_X Z)-R(Y,Z)\nabla_X V$. Weitere Rechenregeln für $R$.

\begin{lemma}\label{lem312} Sei $\nabla$ ein torsionsfreier linearer Zusammenhang und $R$ der zugehörige Krümmungstensor. Dann gilt $(\nabla_X R)(Y,Z)V+(\nabla_Y R)(Z,X)V+(\nabla_Z R)(X,Y) V=0$ (zweite Biorchi Identität).
\end{lemma}

\begin{proof}
    Der Beweis wird eine Übungsaufgabe sein \textbf{TODO} (Hinweis: Argumentieren Sie wie im Beweis von Lemma 3.6, aber ausführlich).
\end{proof}

\begin{fact}[Kovariante Ableitung von $g$]
    für die (semi-)\person{Riemann}'sche Metrik $g$ und den zugehörigen \person{Levi-Civita}-Zusammenhang gilt:
    \begin{align*}
        (\nabla g)(X,Y,Z) &= \nabla_X(g(X,Y))-g(\nabla_X Y,Z)-g(Y,\nabla_X Z)\\
        &= Xg(Y,Z)-g(\nabla_X Y,Z)-g(Y,\nabla_X Z)=0,
    \end{align*}
    da dies genau die Verträglichkeitsbedingung (6) aus Satz 3.2 darstellt ($X,Y,Z\in\fkX(M)$, also $\nabla g = 0$).
\end{fact}

\begin{fact}\label{f314}
    Es gilt $\nabla_X R_1= 0$, wobei $R_1$ wie gehabt definiert sei ($R_1(X,Y)Z = g(Y,Z)X-g(X,Z)Y$). Beweis ist Aufgabe.
\end{fact}

Nun können wir den Beweis von \autoref{thmschur} führen.

\begin{proof}[\autoref{thmschur}] Nach \autoref{f39} folgt $R = K R_1$ mit $K\in\fkF(M)$. Also
    $$
    (\nabla_X R)(Y,Z) V = K(\nabla_X R_1)(Y,Z) V + X(K) R_1(Y,Z) V \ \textrm{(nach \autoref{f314})}
    $$
    Wir zeigen nun $X(K)=0$ für alle $X\in\fkX(M)$. Für alle $X,Y,Z\in\fkX(M)$ gilt
    \begin{align*}
        (\nabla_X R)(Y,Z) V &= X(K)(g(Z,V)Y-g(Y,V)Z)\\
        (\nabla_Y R)(Z,X) V &= Y(K)(g(X,V)Z-g(Z,V)X)\\
        (\nabla_Z R)(X,Y) V &= X(K)(g(Y,V)X-g(X,V)Y).
    \end{align*}
    Nach \autoref{lem312} ist die Summe der drei Ausdrücke gleich 0, also
    \begin{align*}
        0 &= (Z(K)g(Y,V)-Y(K)g(Z,V))X+(X(K)g(Z,V)-Z(K)g(X,V))Y\\
        &\quad +(Y(K)g(X,V)-X(K)g(Y,V))Z.
    \end{align*}
    Sie $X\in\fkX(M)$ beliebig. Da nach Voraussetzung die Dimension $n\geq 3$ ist, gibt es lokal (bzgl. $g$ orthogonale Vektorfelder $Y,Z$. Für $V=Y$ folgt $0=Z(K)X-X(K)Z$ also $X(K)=0$. Damit ist $K$ lokal konstant und wegen $M$ zusammenhängend global konstant.   
\end{proof}

\hfill{05.06.2014}

\begin{definition}
    Seien $(M,g)$, $(\tilde M, \tilde g)$ \person{Riemann}'sche Mannigfaltigkeit und $F:M\to\tilde M$ eine differenzierbar Abbildung. $F$ heißt eine \keyword{lokale Isometrie} (oder isometrisch), falls für alle $p\in M, X,Y\in T_p M$ gilt: $\tilde g_{F(p)}(\rest{DF}_p(X),\rest{DF}_p(Y))=g_p(X,Y)$. Eine \keyword{isometrischer Diffeomorphismus} heißt auch eine \keyword{Isometrie}. $F$ heißt \keyword{konform}, falls es eine Funktion $\lambda:M\to \reals^+$ gibt, sodass $\lambda(p)\neq 0$ und
    $$
    \rest{\tilde g}_p(\rest{DF}_p(X),\rest{DF}_p(Y))=\lambda(p)\rest{g}_p(X,Y)
    $$
    für alle $p\in M, X,Y\in T_p M$.
\end{definition}

Es gilt sogar:

\begin{theorem}
    Für je zwei \person{Riemann}'sche Mannigfaltigkeiten $(M,g)$, $(\tilde M, \tilde g)$ derselben Dimension mit derselben konstanten Krümmung gilt, dass es zu je zwei $p\in M, \tilde p\in\tilde M$ Umgebungen $O\in\cU(p), \tilde O\in\cU(\tilde p)$ gibt und eine Isometrie
    $$
    F:O\to \tilde O.
    $$
\end{theorem}

\begin{definition}
    Für einen linearen Zusammenhang $\nabla$ und den zugehörigen Krümmungstensor definiert man den \keyword{Ricci}-Tensor durch $\Ric(V,W):=\tr(R(\bullet,V)W)$ für $U,V,W\in T_p M$, $p\in M$.
    $$
    \rest{\Ric(V,W)}_p=\tr(U_p\mapsto R(U_p,V_p)W_p)
    $$
    für $U,V,W\in \fkX(M)$.
\end{definition}

Da die Spur eine lineare Abbildung vom Raum der Endomorphismen nach $\reals$ ist, ist $\Ric$ ein Tensor vom Type $(2,0)$ ($\tr(\lambda A+\mu B)=\lambda\tr A+\mu \tr B$.

\begin{remark}
    Aus einem Tensorfeld vom Typ $(r,1)$ erhält man somit durch Bildung der Spur ein Tensorfeld (Kontraktion) vom Typ $(r-1,0)$.
\end{remark}

Sei $M$ nun eine semi-\person{Riemann}'sche Mannigfaltigkeit und $E_1,\ldots, E_n$ eine Orthogonalbasis von $T_p M$ bzgl.~$g$, d.h. $g(E_i,E_j)=\epsilon_i\delta_{ij}$ mit $\epsilon_i\in\set{-1,1}$. Ein Vektor $X_p\in T_p M$ hat also bzgl.~der Basis $E_1,\ldots,E_n$ die Darstellung
$$
X_p=\sum_{i=1}^n\epsilon_ig_p(X,E_i)E_i.
$$
Auf der Diagonalen der Matrix der linearen Abbildung $U_p\mapsto R(U_p,V_p)W_p$ bzgl.~der Basis $E_1,\ldots,E_n$ stehen also die Elemente $\epsilon_i g_p(R(E_i,V_p)W_p,E_i)$. Also
$$
{\Ric(V,W)}_p=\sum_{i=1}^n{\epsilon_i g_p(R(E_i,V_p)W_p,E_i)}.
$$
Wir würden eigentlich gerne noch einmal die Spur bilden, aber Achtung (!) die Spur einer Bilinearform ist apriori nicht sinnvoll (basisabhängig). Aber man definiert die Spur einer Biliniearform bzgl.~einer nicht entarteten symmetrischen Bilinearform $g$ als $\tr A$, wobei $A$ die eindeutig bestimmte lineare Abbildung ist mit $\beta(X,Y)=g(X,AY)$.

Für eine Pseudo-ON-Basis $E_1,\ldots,E_n$ bzgl.~$g$, g.h. $g(E_i,E_j)=\epsilon\delta_{ij}$ ergibt sich
$$
\tr A = \sum_{i=1}^n{\epsilon_i g(E_i,A E_i)} = \sum_{i=1}^n{\epsilon_i\beta(E_i,E_i)}
$$ 
Die Skalarkrümmung $s$ einer semi-\person{Riemann}'schen Mannigfaltigkeit ist definiert als die Spur bzgl.~$g$, also
$$
s(p):=\tr(A_p)
$$
mit
$$
\Ric(X_p,Y_p)=g(X_p,AY_p),
$$
also für eine Pseudo-ON-Basis $E_1,\ldots,E_n$ von $T_p M$ bzgl. $g$:
\begin{framed}
$$
s(p)=\sum_{i=1}^n{\epsilon_i\Ric(E_i,E_i)}=\sum_{i=1}^n{\epsilon_i\epsilon_j g(R(E_i,E_j)E_j,E_i)}.
$$
\end{framed}

\begin{definition}
    Die \keyword{\person{Ricci}-Krümmung} von $M$ in Richtung $V$ mit $g(V,V)\neq 0$ ist definiert als
    $$
    \ric(V):=\frac{\Ric(V,V)}{g(V,V)}.
    $$
    Es gilt also für eine Pseudo-ON-Basis $E_1,\ldots,E_n$ bzgl.~$g$, dass
    $$
    s=\sum_{i=1}^n{\ric(E_i)}.
    $$
\end{definition}

Im Zusammenhang mit Differenzialgeometrie I:

Sei $X_1,\ldots,X_n$ eine ON-Basis von Hauptkrümmungsrichtungen zu Hauptkrümmungen $k_1,\ldots,k_n$ einer parametrisierten Hyperfläche. Dann ist $g(X_i,X_j)=\delta_{ij}$, $LX_i=k_i X_i$ ($L$ \person{Weingarten}-Abbildung).
$$
g(R(X_i,x_j)X_k,X_l)=g(g(LX_j,X_k)LX_i-g(LX_i,X_k)LX_j,X_l)=k_ik_j\delta_{jk}\delta_{il}-k_ik_j\delta_{ik}\delta_{jl},
$$
wobei wir die \person{Gauß}-Gleichungen in koordinatenfreier Form benutzt haben.
Also
$$
g(R(X_i,X_j)X_k,X_l)=
\begin{cases}
    k_i k_j&: \textrm{falls $j=k$ und $i=l$ und $i\neq j$}\\
    -k_i k_j&: \textrm{falls $i=k$ und $j=l$ und $i\neq j$}\\
        0&:\textrm{sonst}
\end{cases}
$$
Und weiterhin
$$
K(X_i,X_j)=k_i k_j \textrm{für $i\neq j$}. 
$$
Zum \person{Ricci}-Tensor ergibgt sich
$$
\Ric(X_i,X_j)=
\begin{cases}
    k_i\sum_{l=1, l\neq i}^n{k_l}\ \textrm{falls $i=j$}\\
    0\ \textrm{sonst}
\end{cases}
$$
und damit
$$
\ric(X_i)=k_i\sum_{j=1,j\neq i}^n{k_j}.
$$
Für die Skalarkrümmung ergibt sich weiterhin
$$
s=\sum_{i=1}^n\sum_{\substack{j=1\\ i\neq j}}^n{k_i k_j}=n(n-1)K_2,
$$
wobei $K_2$ die zweite elementarsymmetrische Krümmung ist.

\hfill{13.06.2014}

\section{Übung --- Teil 3}

\begin{exercise}
    Seien $\nabla, \tilde\nabla$ lineare Zusammenhänge auf einer Mannigfaltigkeit $M$. Zeige:
    \begin{tasks}
            \item Es ist $\nabla - \tilde\nabla$ ein Tensorfeld vom Typ $(2,1)$.
        \item Es ist $\tau:\fkX(M)\settimes\fkX(M)\to\fkX(M)$ mit $\tau(X,Y):=\nabla_X Y - \nabla_Y X - [X,Y]$ ein Tensorfeld vom Typ $(2,1)$.
    \end{tasks}
\end{exercise}

\begin{solution}
    \begin{tasks}
        \item
        Die Additivität in beiden Komponenten von $\nabla - \tilde\nabla$ folgt sogleich aus der Additivität von $\nabla$ und $\tilde \nabla$ in diesen. In der ersten Komponente sind $\nabla$ und $\tilde \nabla$ sogar $\fkF(M)$-linear, damit auch ihre Differenz. Für die zweite Komponente berechnet man
        $$
        {(\nabla-\tilde\nabla)}_X {fY}=f\nabla_X Y+X(f)Y-f\tilde\nabla_X Y -X(f)Y.
        $$
        Also ist der Ausdruck auch in dieser Komponente $\fkF(M)$-homogen und damit linear.
            \item
        Es ist offensichtlich, dass $\tau(X,Y)=-\tau(Y,X)$, also $\tau$ schiefsymmetrisch (was wegen $\rchar\reals\neq 2$ äquivalent zu alternierend ist). Wir zeigen $\fkF(M)$-Homogenität in einer Komponente (aufgrund der Schiefsymmetrie gilt sie dann auch in der anderen). Additivität in beiden Komponenten ist trivial, da alle Ausdrücke aus denen $\tau$ kombiniert wird biadditiv sind.
        $$
        \eqalign{
            &\nabla_{fX} Y -\nabla_Y {fX} - [fX,Y]=f\nabla_X Y - f\nabla_Y X - Y(f)X - (fX)Y+Y(fX)\cr
            & =f\tau(X,Y)-Y(f)X+Y(f)X  
        }
        $$
        wie gewünscht.
    \end{tasks}
\end{solution}

\begin{exercise}
    Sei $\nabla$ ein linearer Zusammenhang auf der Mannigfaltigkeit $M$.
    Setze $R:{\fkX(M)}^3\to\fkX(M)$ mit $R(X,Y)Z:=\nabla_X\nabla_Y Z - \nabla_Y\nabla_X Z -\nabla{[X,Y]} Z$. Zeige, dass $R$ ein Tensorfeld ist (\person{Riemann}'scher Krümmungstensor).
\end{exercise}

\begin{solution}
    Es ist $R$ offensichtlich antisymmetrisch in $X$ und $Y$. Additivität in jeder Komponente ist wieder trivial, da alle auftretende Ausdrücke in der Definition von $R$ diese Eigenschaft haben.
    Für $\fkF(M)$-Homogenität in $X$ (und damit $Y$) rechnen wir:
    \begin{align*}
        R(fX,Y) & = f\nabla_X\nabla_Y-\nabla_Y(f\nabla_X)-\nabla_{f[X,Y]-Y(f)X}\\
        & = f(\nabla_X\nabla_Y-\nabla_Y\nabla_X-\nabla_{[X,Y]})-Y(f)\nabla_X+Y(f)\nabla_X,
    \end{align*}
    was das gewünschte Resultat ist.
    Für die $\fkF(M)$-Homogenität in $Z$ berechnet man:
    \begin{align*}
        R(X,Y)(fZ) & = f\nabla_X\nabla_Y Z+X(f)\nabla_Y Z+Y(f)\nabla_X Z+XY(f) Z\\
        & \quad -f\nabla_Y\nabla_X Z-Y(f)\nabla_X Z-X(f)\nabla_Y Z-YX(f) Z\\
        & \quad -f\nabla_{[X,Y]} Z-[X,Y](f) Z = fR(X,Y) Z.
    \end{align*}
    Alle Terme, welche `unerwünscht' sind, heben sich hinweg.
\end{solution}

\begin{exercise}
    Gib ein Beispiel einer $4$-dimensionalen semi-\person{Riemann}'schen Mannigfaltigkeit $M$ mit Signatur $(3,1)$ an und eine $3$-dimensionale Untermannigfaltigkeit, die mit der induzierten Biliniearform keine semi-\person{Riemann}'sche Mannigfaltigkeit ist.
\end{exercise}

\begin{solution}%% falsch
    Wir wählen den $\reals^4$ mit den Karten $\phi:U\to U$, $u\mapsto u$ ($U\setleq \reals^4$ offen). Als Untermannigfaltigkeit wählen wir den linearen Unterraum $\ker(\dual x_1+\dual x_2+\dual x_3 - \sqrt 3\dual x_4)$. Es enthält dann dieser Raum den bzgl. $g$ isotropen Vektor $(1,1,1,\sqrt 3)$. Damit ist die induzierte Biliniearform auf dem Unterraum jedoch in jedem Punkt entartet.
\end{solution}

\begin{exercise}
    Definiere das Produkt $M_1\settimes M_2$ in natürlicher Weise für $M_i$ differenzierbare Mannigfaltigkeit ($i=1,2$). Zeige, dass $M_1\settimes M_2$ eine differenzierbare Mannigfaltigkeit ist und $\pi_i:M_1\settimes M_2\to M_i$ Submersionen sind.
\end{exercise}

\begin{solution}
    Wir setzen die zugrundeliegende Menge von $M_1\settimes M_2$ gleich mit dem kartesischen Produkt der Mengen von $M_1$ und $M_2$. Gleichzeitig setzen wir auch die Kartenabbildungen $\phi_{\alpha,\beta}:=\phi^1_{\alpha}\settimes\phi^2_{\beta}$, wobei $\phi^1_{\alpha},\phi^2_{\beta}$ Karten von $M_1$ bzw. $M_2$ sind.
    Es ist dann klar, dass
    $$
    \bigsetjoin_{\alpha,\beta}{\dom{\phi_{\alpha,\beta}}}=\bigsetjoin_{\alpha,\beta}{\dom\phi^1_{\alpha}\settimes\dom\phi^2_{\beta}}=M_1\settimes M_2.
    $$
    Andererseits gilt auch, dass
    $$
    \im\phi_{\alpha,\beta}\setmeet\im\phi_{\alpha',\beta'}=\im\phi^1_{\alpha}\setmeet\im\phi^1_{\alpha'}\settimes\im\phi^1_{\beta}\setmeet\im\phi^1_{\beta'},
    $$
    was als Produkt zweier offener Mengen offen in der Produkttopologie ist ($\reals^n$ sind so definiert, dass die darauf erklärte Topologie genau die $n$-te Potenz der Topologie auf $\reals$ ist).
    Weiter ist die Komposition $\phi_{\alpha,\beta}\compose\phi_{\alpha',\beta'}$ auch differenzierbare Abbildung, da sie gleich dem Produkt der Abbildungen $\phi^1_{\alpha}\compose{(\phi^1_{\alpha'})}^{-1}$ und $\phi^2_{\beta}\compose{(\phi^2_{\beta'})}^{-1}$ ist und ein solches Produkt wieder differenzierbar ist (die Ableitung ist gegeben durch
    $$
    \begin{pmatrix}
        J_1 & 0\\
        0 & J_2
    \end{pmatrix},
    $$ wobei $J_i$ die entsprechenden Ableitungen der ersten bzw.\ zweiten Funktion sind.
    Nun muss noch gezeigt werden, dass $\pi_i$ Submersionen sind.
    Dafür benötigen wir, dass $\phi^i_{\alpha'}\compose\pi_i\compose\phi_{\alpha,\beta}^{-1}$ überall surjektives Differential hat. Diese Abbildung ist jedoch einfach nur die Projektion $(x_1,x_2)\mapsto x_i$, welche klarerweise surjektives Differential in jedem Punkte hat (denn sie ist schon linear und `surjektiv').
\end{solution}

\begin{exercise}
    Seien $M,N$ glatte Mannigfaltigkeiten und $F:M\to N$ Submersion. Zeige, dass für alle $q\in N$ gilt: $F^{-1}(q)$ ist Untermannigfaltigkeit von $M$.
\end{exercise}

\begin{solution}
    TODO
\end{solution}

\begin{exercise}
    Gegeben sei die Abbildung $\Phi:\sphere^2\to\reals^6$ mit
    $$
    \Phi(X_1,X_2,X_3)=(X_1^2,X_2^2,X_3^2,\sqrt 2X_1X_2,\sqrt 2X_1X_3,\sqrt 2X_2X_3).
    $$
    \begin{tasks}
            \item Zeige, dass $\im\Phi\setleq \sphere^5$ und $\im\Phi$ in einer $4$-dimensionalen Sphäre vom Radius $\sqrt{\frac 2 3}$ liegt.
            \item Zeige, $\Phi$ ist Immersion.
            \item Bestimme die von $\Phi$ induzierte \person{Riemann}'sche Metrik auf $\sphere^2$.
    \end{tasks}
\end{exercise}
%%% to much space here%%% --- reason is enumerate spacing%%%%

\begin{solution}
    \begin{tasks}
            \item Es gilt
        $$
        X_1^4+X_2^4+X_3^4+2 X_1^2X_2^2+2 X_1^2X_3^2+2X_2^2X_3^2={(X_1^2+X_2^2+X_3^2)}^2.
        $$
        Damit liegt $\im\Phi$ in $\sphere^5$, da $(X_1,X_2,X_3)\in\sphere^2$.
        Weiterhin ist klar, dass die Hyperebene $\ker (\dual x_1+\dual x_2+\dual x_3 -1)$ ebenfalls $\im\Phi$ enthält. Der Schnitt dieser mit der $\sphere^5$ ist eine $4$-dimensionale Sphäre mit Mittelpunkt
        $$(1/3,1/3,1/3, 0, 0, 0)$$
        und Radius $\sqrt{{(1-\frac 1 3)}^2+\frac 1 {3^2}+\frac 1 {3^2}}=\sqrt{\frac 2 3}$.
            \item Wir können die obere Halbsphäre (und auch jede andere Halbsphäre nach Koordinatentransformation) parametrisieren durch
        $$
        \phi^{-1}(X_1,X_2)=(X_1,X_2,\sqrt{1-X_1^2-X_2^2}).
        $$
        Damit ist
        $$
        \Psi\compose\phi^{-1}(X_1,X_2)=(X_1^2,X_2^2,1-X_1^2-X_2^2,\sqrt 2X_1 X_2,\sqrt 2X_1\sqrt{1-X_1^2-X_2^2},\sqrt 2 X_2\sqrt{1-X_1^2-X_2^2}).
        $$
        Das Differential dieser  Abbildung ist offensichtlich injektiv, da die Ableitung gegeben ist durch
        $$
        \begin{pmatrix}
            2X_1 & 0\\
            0 & 2X_2\\
            -2X_1 & -2X_2\\
            \sqrt 2X_2 & \sqrt 2X_1\\
            \sqrt 2 \frac{1-2X_1^2-X_2^2}{\sqrt{1-X_1^2-X_2^2}} &-\sqrt 2\frac{X_1X_2}{\sqrt{1-X_1^2-X_2^2}}\\
            -\sqrt 2\frac{X_1X_2}{\sqrt{1-X_1^2-X_2^2}} & \sqrt 2 \frac{1-X_1^2-2X_2^2}{\sqrt{1-X_1^2-X_2^2}}
        \end{pmatrix}.
        $$
        Von den Untermatrizen aus erster und dritter, bzw.\ zweiter und vierter Zeile ist immer eine regulär außer $X_1=X_2=0$.
        In diesem Falle ist jedoch die Matrix aus fünfter und sechster Zeile regulär. Damit ist $\Phi$ Immersion.
        \item TODO
    \end{tasks}
\end{solution}

\hfill{19.06.2014}
Sei $(M,g)$ zusammenghängend (semi-)\person{Riemann}'sche Mannigfaltigkeit.

\begin{definition}
    Sei $c:I\to M$ eine differenzierbare Kurve. Eine differenzierbare Abbildung $X:I\to TM$ heißt Vektorfeld längs $c$, falls $x(t)\in T_{c(t)}M$ (d.h. $\pi\compose X=c$).
    Weiter sei $\cX$ der Vektorraum der Vektorfelder.
\end{definition}

\begin{theorem}
    Sei $(M,g)$ (semi-)\person{Riemann}'sche Mannigfaltigkeit mit \person{Levi-Civita}-Zusammenhang $\nabla$, $C:I\to M$ Kurve. Dann gibt es genau eine $\reals$-lineare Abbildung $\frac\nabla{\diff t}:\cX_c\to\cX_c$ mit $x\mapsto \frac{\nabla x}{\d t}$ mit
    \begin{properties}
            \item $$
        \frac{\nabla}{\d t}(f\cdot x)=f' x+\frac{\nabla x}{\diff t} \text{ ($\forall x\in\cX_c, f\in \cF_c$) }
        $$
            \item $$
        \rest{\frac{\nabla Y\compose c}{\diff t}}_{t=t_0} = \nabla_{c'(t_0)} \text{ $Y\in\cX_c$ } 
        $$
    \end{properties}
\end{theorem}

\begin{proof}
    Eindeutigkeit: lokal um $c_0(t)\in M$ wähle Karte $X(t)=\sum_{i=1}^n{\alpha_i(t)\rest{\partial_i}_{c(t)}}$, also $\frac{\nabla X}{\diff{t}}(t_0)=\sum_{i=1}^n{(\alpha_i'\rest{\partial_i}_{c(t_0)}+\nabla_{c'(t_0)}{\partial_i})}$. Zur Existenz nimmt man selbiges als Definition.
\end{proof}

\begin{lemma}
    Es gilt folgende Produktregel für $\frac{\nabla}{\diff{t}}:\cX_c\to\cX_c$. Für alle $X,Y\in\cX_c$ haben wir
    $$
    \rest{\frac{\diff}{\diff{t}}}_{c(t_0)}{g(X(t),Y(t))}=g_{c(t_0)}(\frac{\nabla X}{\diff t}X(t),Y(t))
    $$
\end{lemma}

\begin{proof}
    lokal: $X=\sum{\alpha_i(\partial_i\compose c)}$, $Y=\sum{\beta_j(\partial_j\compose c)}$.
        Dann $\rest{\frac{\diff}{\diff t}}_{t=t_0}{g(X(t),Y(t))}=$
\end{proof}
\end{document}

















