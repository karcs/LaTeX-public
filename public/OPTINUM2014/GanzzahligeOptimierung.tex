\section{Ganzzahlige Optimierung}

\subsection{Die Methode `Branch and Bound' (BaB)}

\paragraph{Grundprinzipien:}

Aufgabenstellung ist es $f\to \min$ bei $x\in G$ mit $G$ endlich. Zum Beispiel betrachte das Rucksackproblem $\sum_{i=1}^n{c_ix_i}\to\max$, $G:=\set{x\in\reals^n_{\geq 0}:\sum_{i=1}^n{x_i}\leq l, x_i\in\nats}$.

\subparagraph{Vorgehensweise} (dabei sei ein `aktuell bestes' $\bar x\in G$ bekannt) ist die schrittweise Zerlegung in disjunkte Teilmengen $G',G''$ mit $G'\setjoin G''=G$, $G'\setmeet G''=\emptyset$.
Verzweigung (branch): Gewinnung von unteren Schranken auf den erzeugten Teilmengen durch einfache Hilfsaufgaben (d.h $d(G')\leq f(x)$ für alle $x\in G'$).

Danach verfährt man rekursiv (man erhält einen `Aufsplittungsbaum').

Im weiteren Verlauf braucht \emph{nicht} weiter verzweigt zu werden, wenn einer der folgenden Fälle eintritt:
\begin{casebycase}
        \item $G'=\emptyset$.
        \item $d(G')>f(\bar x)$
        \item $d (G'')=f(\bar x)$ und $\bar x\in G''$
\end{casebycase}

\begin{definition}
    Das Problem $z(x)\to\min$ bei $x\in Q$ (1.1), heißt Relaxation zum Problem
    $f(x)\to\min$, bei $x\in G$ (1.2), falls gilt $G\setleq Q$, $z(x)\leq f(x)$ für alle $x\in G$.
\end{definition}

\begin{lemma}[hinreichendes Optimalitätskriterium]
    Es sei $\hat x\in Q$ Lösung von (1.1). Gilt $\tilde x\in G$ und $z(\tilde x)=f(\tilde x)$, dann ist $\tilde x$ Lösung von (1.2).
\end{lemma}

Möglich ist
\begin{itemize}
        \item lineare Optimierungsaufgabe ohne Ganzzahligkeitsforderung 
        \item Straf-Ersatzprobleme
        \item duale Probleme 
\end{itemize}

\paragraph{Anwendung von `Branch and Bound' in der ganzzahligen Optimierung}

\emph{Der Algorithmus von Land/Doig/Dakin:} Gegeben sei die Aufgabe
$$
z(x)=c^{\top}x\to\min \text{ bei } Ax=b,x\geq 0 \text{ und }x\in\ints \eqno{(1.3)}
$$

Sei ein optimales Tableau für (1) erreicht.
$$
\begin{matrix}
    & x_N(=0) & \\
    \midrule
    x_B & P & p \\
    z & q^{\top} & q_0    
\end{matrix}\quad\text{ bei } (x_b,x_n)=(p,0)\eqno{(1.4)}
$$
$p\geq 0$, $q\geq 0$. Falls $p\in\nats_+^m$, dann löst $(p,0)$ die Aufgabe (2). $p_i\not\in\nats$, dann Verzweigung in $G'$:$x_i\leq \floor{p_i}$, $x\in G$ und $G''$:$x_i\geq \floor{p_i}+1$, Danach

($1_1$) (1) mit $x_i\leq\floor{p_i}$

($1_2$) (1) mit $x_i\geq\floor{p_i}+1$,

$x_i+s_i=\floor{p_i}$.










