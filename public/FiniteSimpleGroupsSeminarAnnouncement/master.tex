\documentclass[14pt]{memoir}
\usepackage{graphicx}
\usepackage{tikz-cd}
% for use of german language in a document
\usepackage[utf8]{inputenc} % this is needed for umlauts
\usepackage[ngerman]{babel} % this is needed for umlauts
%\usepackage[T1]{fontenc}    % this is needed for correct output of umlauts in pdf


\usepackage{geometry}
\geometry{a4paper,left=10mm,right=10mm, top=5mm, bottom=5mm}
\usepackage{wallpaper}
\pagestyle{empty}
\usepackage{titlesec}

\definecolor{darkblue}{rgb}{0.103, 0.105, 0.500}
\definecolor{mygreen}{rgb}{0.404, 0.718, 0.216}

\begin{document}

\tikz[remember picture,overlay] \node[opacity=1,inner sep=0pt] at (current page.center){\includegraphics[width=\paperwidth,height=\paperheight]{bg.pdf}};
\pgfdeclarelayer{bg}    % declare background layer
\pgfsetlayers{bg,main}  % set the order of the layers (main is the standard layer)

\section*{\color{darkblue}\Huge Arbeitsgruppe: Einfache endliche Gruppen}

Im kommenden Semester soll eine Arbeitsgruppe zum Thema \emph{einfache endliche Gruppen} angeboten werden. Ziel ist es, an interessante Gruppen des CFSG\footnote{Classification Theorem for Finite Simple Groups} heranzuführen und die damit verbundenen algebraischen Strukturen zu studieren.

\begin{centering}
    \begin{tikzpicture}[scale=0.85]%
        \color{darkblue}
        \draw (8.5,0) node (0) {$A\triangleright B$};
        \draw (8.5,-1.3) node (1) {$C\triangleright D$};
    \draw (0,0) node (A) {$B\cap C$};
    \draw (17,0) node (A') {$A\cap D$};
    \draw (8.5,-3.5) node (B){$(B\cap C)(A\cap D)$};
    \draw (1.5,-3) node (C) {$B$};
    \draw (15.5,-3) node (C') {$D$};%
    \draw (4,-4.6) node (D) {$B(A\cap D)$};
    \draw (13,-4.6) node (D') {$(B\cap C)D$};
    \draw (8.5,-7) node (E) {$A\cap C$};
    \draw (4.5,-10) node (F) {$B(A\cap C)$};
    \draw (12.5,-10) node (F') {$(A\cap C)D$};
    \draw (4,-11.8) node (G) {$A$};
    \draw (13,-11.8) node (G') {$C$};

    \draw (0,0) node (a) {};
    \draw (17,0) node (a') {};
    \draw (8.5,-3.5) node (b) {};
    \draw (2.5,-4.5) node (c) {};
    \draw (14.5,-4.5) node (c') {};
    \draw (4,-4.6) node (d) {};
    \draw (13,-4.6) node (d') {};
    \draw (8.5,-7) node (e) {};
    \draw (4.5,-10) node (f) {};
    \draw (12.5,-10) node (f') {};
    \draw (4,-11.8) node (g) {};
    \draw (13,-11.8) node (g') {};
%
    \begin{pgfonlayer}{bg}
        \shade [shading=axis, right color = orange, left color = yellow, shading angle = 0, opacity=0.65] (a) to [out=10,in=140] (b) to [out=-170,in=15] (d) to [out=180,in=-10] (c) to [out=140,in=-65] (a);
        \shade [shading=axis, right color = orange, left color = yellow, shading angle = 0, opacity=0.65] (a') to [out=170,in=40] (b) to [out=-10,in=165] (d') to [out=0,in=-170] (c') to [out=40,in=-115] (a');
        \shade [shading=axis, right color = orange, left color = yellow, shading angle = 0, opacity=0.65] (b) to [out=-170,in=15] (d) to [out=-130,in=110] (f) to [out=-100,in=80] (g) to [out=40,in=-115] (f) to [out=20,in=-110] (e) [out=90, in=-90] (b);
        \shade [shading=axis, right color = orange, left color = yellow, shading angle = 0, opacity=0.65] (b) to [out=-10,in=165] (d') to [out=-50,in=70] (f') to [out=-80,in=100] (g') to [out=140,in=-65] (f') to [out=160,in=-70] (e) [out=90, in=-90] (b);
    \end{pgfonlayer}
    \path
    [out=10] (A) edge [in=140] (B)
    [out=170] (A') edge [in=40] (B)
    [out=-65] (A) edge [in=125] (C)
    [out=-115] (A') edge [in=55] (C')
    [out=-50] (C) edge [in=160] (D)
    [out=-130] (C') edge [in=20] (D')
    [out=-170] (B) edge [in=15] (D)
    [out=-10] (B) edge [in=165] (D')
    [out=-90] (B) edge [in=90] (E)
    [out=-130] (D) edge [in=110] (F)
    [out=-50] (D') edge [in=70] (F')
    [out=-110] (E) edge [in=20] (F)
    [out=-70] (E) edge [in=160] (F')
    [out=-100] (F) edge [in=80] (G)
    [out=-80] (F') edge [in=100] (G');
\end{tikzpicture}
\footnote{das \textsc{Zassenhaus}-Lemma wird oft auch als `Butterfly Lemma' bezeichnet (hat jedoch mit dem Thema nur mäßig vi$_{\textrm{e}}$l zu tun)}
{\Huge\color{darkblue}
$$%
\frac{(A\cap C)B}{(A\cap D)B}\cong\frac{(A\cap C)D}{(B\cap C)D}$$}
\end{centering}

\paragraph{\color{darkblue}\Large Inhalt:}

Zunächst wird es am Anfang des Semesters eine kleine Auffrischung zu sehr grundlegenden Aussagen und Sätzen der Gruppentheorie geben.
Im Wesentlichen soll das Seminar dann auf dem Buch `The Finite Simple Groups' von Robert Wilson
basieren (aus dem wir wahrscheinlich nur einen Bruchteil schaffen werden).

\paragraph{\color{darkblue}\Large Gestaltung:} Optimal wäre es, wenn die Teilnehmer die Arbeitsgruppe selbst
durch eigene Vorträge beleben würden. Grundsätzlich soll die Vermittlung des Stoffes dabei derart konzipiert sein, dass dieser nahezu ohne Vorwissen verständlich ist. Ich stehe bei Rückfragen gerne per Email zur Verfügung (\texttt{jakob.schneider@tu-dresden.de}).%

\begin{flushright}
    \today,\\
    Jakob Schneider  
\end{flushright}

\end{document}











