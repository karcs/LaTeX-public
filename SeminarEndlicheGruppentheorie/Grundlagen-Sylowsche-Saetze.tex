\part{Grundlagen}

\section{Symbole, Wörter, Ausdrücke, Formeln, Typen}

\subsection{Formeln}

Die folgenden Begriffe sind von essentieller Natur.

\begin{definition}[Wort]
    Ein Wort ist eine Kette von Symbolen eines gewissen Alphabets.
\end{definition}

\begin{example}
    Es sind beispielsweise $XYZ$, $a\instanceof b$, $a\to b$ Wörter.
\end{example}

\begin{definition}[Ausdruck]
    Ein Ausdruck ist ein gewissen syntaktischen Regeln genügendes Wort (d.h.~aus einer
    entsprechenden Grammatik ableitbar).
\end{definition}

\begin{example}
    Es ist beispielsweise $a\instanceof b$ ein Ausdruck, $a::b$ jedoch nicht.
\end{example}

Auf die im Folgenden verwendeten syntaktischen Regeln wollen wir nicht im Detail eingehen.

\begin{definition}[Formel]
    Eine Formel ist ein gewissen semantischen Regeln genügender Ausdruck (d.h.~aus Typaxiomen ableitbar).
\end{definition}

Die verwendeten semantischen Regeln wollen wir hier ebenfalls nicht im Detail besprechen.

\begin{convention}[impliziter Kontext]
    Oftmals werden die Typen der in einer Formel auftretenden
Variablen nicht explizit notiert, sondern entweder vorher erwähnt oder implizit als bekannt
vorausgesetzt.
\end{convention}

\begin{example}[Assoziativität]
    Es gilt zum Beispiel die Formel
    $$
    (ab)c=a(bc)
    $$
    mit dem impliziten Kontext, dass $a$, $b$ und $c$ Instanzen einer Gruppe sind (d.h.~    $a,b,c\instanceof G\instanceof \group$).
\end{example}

\begin{convention}[Interpretation von Formeln]
    In Formeln werden Variablen immer von links nach rechts
interpretiert. Bei stärker bindenden Terminalen werden diese zuerst interpretiert.
\end{convention}

\begin{example}
    $$
    ab+cd
    $$
    wird interpretiert als $(ab)+(cd)$, da das implizite Terminal der Aneinanderreihung stärker
    bindet als das Terminal $+$. Hierbei können $a$, $b$ und $c$ zum Beispiel Elemente eines
    Ringes sein (d.h.~$a,b,c:R:\ring$).
\end{example}

Im Rahmen jedes sinnvollen Mathematikbuches werden also Formeln studiert und hergeleitet, welche
auf einer gewissen Axiomatik basieren.

Nun wenden wir uns den Symbolen genauer zu.

\subsection{Variablen und Terminale}

Ein jedes Symbol in einer Formel ist entweder eine Variable, d.h.~ein Symbol, dessen
Substitution durch ein beliebiges anderes Symbol, welches dem gleichen Kontext genügt, wieder zu einer gültigen Formel führt, oder ein
Terminal, d.h.~ein Symbol, das sich nicht weiter ersetzen lässt (möglicherweise führt die Substitution
auch zu einer gültigen Formel, dies ist aber apriori nicht gewährleistet).

\begin{notation}[Variablen und Terminale]
    Variablen werden im Rahmen dieses Skriptes ausschließlich als \textit{nicht fette} Buchstaben notiert. 
    Alle Zahlen, Relationensymbole, Operatorsymbole sind Terminale. Ist ein Buchstabe oder Wort
    \textbf{fett} notiert, so handelt es sich ebenfalls ein Terminal. Wörter zeichnen sich dabei
    als solche aus, indem sie aufrecht notiert werden.
    Das \keyword{implizite Terminal}\as{{implizites Terminal}} ist gegeben durch die
    Aneinanderreihung zweier Ausdrücke.
\end{notation}

\begin{example}[implizites Terminal]
    Der Ausdruck
    $$
    ab
    $$
    beinhaltet das implizite Terminal (der Kontext kann hier z.B.~$a,b:R:\ring$ sein).
\end{example}

\subsection{Typen und Instanzen}

Formal ist ein Typ schlicht ein Symbol. Andererseits kann ein Symbol einen Typ haben und
definiert gleichermaßen einen selbigen (ob es
Objekte dieses Types tatsächlich gibt, ist eine Frage, die man sich gemeinhin nicht stellt).

\begin{notation}
    Wir notieren `$A$ ist vom Typ $B$' oder genauer `$A$ ist Instanz des Typs $B$' durch
$A\instanceof B$.
\end{notation}

Ein Typ zeichnet sich nun aus durch eine gewissen Satz an Regeln.

\begin{example}[Gruppen]
    Der Typ der Gruppe hat folgende Eigenschaften:
    $$
    a,b\instanceof G\instanceof\group\implies ab\instanceof G
    $$
    $$
    a,b,c\instanceof G\instanceof \group\implies (ab)c\interpret=\in G a(bc)
    $$
    $$
    a\instanceof G\instanceof\group\implies \power a\operation\exponent{-1}\instanceof G
    $$
    $$
    G\instanceof\group\implies \interpret 1\in G\instanceof G,\interpret 1\in G a=a
    $$
    $$
    \interpret 1\in G a\equals a
    $$
    $$
    a\power a\operation\exponent{-1}=
    $$
    definiert für
    $G\instanceof\group$, $a,b,c\instanceof G$, dann ist $ab,a^{-1}\instanceof G$ und beispielsweise $(ab)c=a(bc)$. 
\end{example}


Eines der wichtigsten Symbole ist das Gleichheitszeichen. Hier gibt es nun zwei verschiedene
Gleichheiten:

Typengleichheit zweier Typen $b$ und $c$ gilt, falls $a:b$ und $a:c$ äquivalent sind.
Formal
$$
b\equals c.
$$

Instanzengleichheit im Bezug auf einen Typ (d.h.~in einem gewissen Kontext) zweier Symbole
$a\instanceof c$
und $b\instanceof c$ heißt, dass $a$ und $b$ sich als Instanz des Typs $c$ gleich verhält, d.h.~wir
können in jedem Ausdruck, in dem $b$ bzw.~$c$ als Instanz von $c$ interpretiert werden beide
durcheinander ersetzen.

Notationsmäßig haben wir:
$$
a\as c\equals b
$$


\begin{definition}[Subinstanzen]
    Sei $a:c$ und $b:c$, und folgt aus $d:a$, dass $d:b$, dann nennen wir $a$ eine Unterinstanz
    von $b$ im Bezug auf den Typ $c$.
\end{definition}

\begin{definition}
    Die Größe eines Typs $a$ ist, falls sie existiert, die Anzahl seiner Instanzen. Sie wird mit
    $\size a$ bezeichnet.
\end{definition}

\subsection{Morphismen}

\begin{definition}[Morphismen]
    Seien $a,b\instanceof A$, dann definieren wir den Typ $a\to b$ der $A$-Morphismen von
    $a$ nach $b$ durch die folgenden Regeln.
    $$
    \alpha\instanceof a\to b, \beta\instanceof b\to c\implies \alpha\beta\instanceof a\to c
    $$
    $$
    (\alpha\beta)\gamma=\alpha(\beta\gamma)
    $$
    $$
    d\instanceof a\implies d\alpha\instanceof b
    $$
    Alle Regeln die den Typ $a$ definieren gelten ebenfalls, wenn darin jede Instanz $d$ des
    Typs $a$ durch $d\alpha$ ersetzt wird, und jedes $a$ durch $b$.
\end{definition}

\section{Geordnete Mengen und Verbände}

\begin{definition}
    Der Typ Verband $\lattice$ ist gegeben durch die folgenden Festsetzungen.
    Für $a,b,c:V:\lattice$ gilt
    $$
    a\join(b\join c)\equals (a\join b)\join c
    $$
    $$
    a\join b=b\join a
    $$
    $$
    a\join(a\meet b)=a, 
    $$
    $$
    a\join a=a
    $$
\end{definition}

\begin{definition}[modularer Verband]
    Ein Verband $V$ heißt modular, falls
    $$
    (a\meet b)\join c
    $$
\end{definition}

\begin{definition}[Kette und Reihe]
    Sei $P:\poset$ eine geordnete Menge. Ein Unterobjekt $K:\subobject V\in\poset$ heißt Kette, falls $K$ total geordnet
    ist. $K$ heißt Reihe, falls $K$ diskret ist, d.h.~die durch die Ordnung induzierte Topologie
    ist diskret (was bedeutet, dass für $x:K$, falls $x$ nicht maximal ist ein $y:K$ existiert
    mit $x\lessdot y$ und analog, falls $x$ nicht minimal ist ein $z:K$ mit $z\lessdot x$ existiert). 
\end{definition}

\begin{definition}[Maximums- und Minimumsbedingung]
    Sei $P:\poset$. Dann genügt $P$ der \keyword{Maximumsbedingung}\as{{aufsteigende
            Reihenbedingung}{\person{Noether}'sche Eigenschaft}}, wenn jedes nichtleere Unterobjekt von $P$ ein Maximum
    hat. Analog genügt $P$ der \keyword{Minimumsbedingung}\as{{absteigende
            Reihenbedingung}{\person{Artin}'sche Eigenschaft}}, wenn jedes Unterobjekt von $P$
    ein minimales Element hat.
\end{definition}

\begin{lemma}[Äquivelente Beschreibungen der Maximums- und Minimumsbedingung]
    Die folgenden Aussagen sind äquivalent
    \begin{statements}
            \item $P$ genügt der Maximumsbedingung
            \item Jede Kette von $P$ hat ein größtes Element.
            \item Jede Reihe von $P$ hat ein größtes Element.
    \end{statements}
\end{lemma}

\begin{proof}
    
\end{proof}

\begin{theorem}[Charakterisierung von endlicher Erzeugbarkeit]
    Sei $a:b$. Die folgenden Aussagen sind äquivalent
    \begin{statements}
            \item $a$ ist endlich erzeugt im Bezug auf $b$.
            \item $\subinstances a\in b$ genügt der aufsteigenden Reihenbedingung.
            \item Jede Unterinstanz von $a$ im Bezug auf $b$ liegt in einer maximalen Subinstanz
        von $a$ im Bezug auf $b$ enthalten.
    \end{statements}
\end{theorem}%%

\begin{proof}
    \begin{implications}
            \item Ist $a$ endlich erzeugt und $c_i$ mit $i:\naturalnumber$ eine
        aufsteigende Reihe in $\subinstances a\in
        b$. Dann gibt es $n:\naturalnumber$ und $a_1,\ldots,a_n:a$, sodass $\hull
        a_1,\ldots,a_n\in{\subinstances a\in b}=a$
            \item
        \item blub
    \end{implications}
\end{proof}%


\section{Kategorien}

In diesem Abschnitt werden die nötigen kategorientheoretischen Kenntnisse (bzw.~Terminologie) bereitgestellt, welche für die
Gruppentheorie (von einem modernen Standpunkt aus) unentbehrlich sind.

\subsection{Logische Kategorie}

\begin{definition}[Logische Kategorie]
    Eine Kategorie heißt \keyword{logisch}\as{{logische Kategorie}}, falls für jedes Paar von
    Objekten $a,b$ höchstens ein Morphismus $\alpha\instanceof a\interpret\to\in \bfL b$. 
\end{definition}



\subsection{Terminale und finale Objekte}

\begin{definition}[initiale und terminale Objekte]
    Sei $\bfA$ eine Kategorie. Dann heißt ein Objekt $T:\bfA$ \keyword{terminal}\as{{terminales Objekt}{koinitiales Objekt}}, falls es für jedes andere Objekt $O:\bfA$ genau
    einen Morphismus $\alpha:O\to T$ gibt. Mit anderen Worten: $T$ ist maximal bezüglich der transitiven Relation $\to$.
    In analoger Weise heißt $I$ ein \keyword{initiales}\as{{initiales Objekt}{koterminales Objekt}}, falls es für jedes $O:\bfA$
    genau einen Morphismus $\beta:I\to O$ gibt. Mit anderen Worten: $I$ ist minimal bezüglich der transitiven Relation $\to$. 
\end{definition}

\begin{definition}[Nullobjekt]
    Ein Objekt $0:\bfA$ heißt \keyword{Nullobjekt}\as{{Nullobjekt}}, falls $0$ sowohl initial als auch terminal ist.
\end{definition}

\subsection{Initiale und terminale Morphismen und Nullmorphismen}

\begin{definition}[Initiale und terminale Pfeile]
    Ein Pfeil $\gamma:A\to B$ heißt \keyword{terminaler Morphismus}\as{{terminaler Morphismus}{konstanter Morphismus}}, falls $\alpha\gamma=\beta\gamma$
    für alle $\alpha,\beta:\to A$. Analog heißt $\gamma$ \keyword{initialer Morphismus}\as{{initialer Morphismus}{kokonstanter Morphismus}}, falls
    $\gamma\alpha=\gamma\beta$ für $\alpha,\beta:B\to$ (d.h. $\dual\gamma$ ist konstant in $\dual\bfA$).
\end{definition}

\begin{remark}
    Initiale und terminale Pfeile sind genau die initialen und terminalen Objekte in der Morphismenkategorie von $\bfA$. 
\end{remark}

\begin{definition}[Nullpfeil]
    Ein Pfeil $0:A\to B$ heißt \keyword{Nullmorphismus}\as{{Nullmorphismus}}, falls er konstant und kokonstant zugleich ist.
\end{definition}

\begin{remark}
    Nullpfeile sind genau die Nullobjekte in der Morphismenkategorie von $\bfA$.
\end{remark}

\begin{remark}
    Gibt es ein Nullobjekt $0:\bfA$, so auch ein kanonischen Nullmorphismus zwischen Objekten $A,B:\bfA$ via $A\to 0\to B$, wobei
    die beiden Morphismen aufgrund der Nullobjekteigenschaft schon eindeutig sind. 
\end{remark}

\subsection{Kerne und Kokerne}

\begin{definition}[Kern und Kokern]
Sei $\alpha:A\to_\bfA B$ ein Morphismus. Dann wird ist der \keyword{Kern}\as{{Kern eines Morphismus}} $\ker\phi$ das Unterobjekt mit der universellen Eigenschaft, dass jedes
Unterobjekt $U$ von $A$ gilt, dass $U\alpha=\interpret 0\in{\subobject A\in\bfA}$, dann gilt $\ker\phi\leq U$.
In analoger Weise definieren wir den \keyword{Kokern}\as{{Kokern}} $\cokernel \alpha$ von $\alpha$ als das Quotientenbojekt $Q$ von $B$, welches die
Eigenschaft hat, dass $\alpha Q=\interpret 0\in{\cosubobjects A}$ gilt $\cokernel \alpha\leq Q$.
\end{definition}

\subsection{Unterobjekte und Quotientenobjekte}

\begin{definition}
    Sei $\bfA$ eine Kategorie. Der Verband der \keyword{Unterobjekte}\as{{Unterobjekt}} $\subobject A\in\bfA$ für jedes Objekt $A:\bfA$ als die
    Isomorphieklassen der Kategorie $\monoto_\bfA A$ (also $\monoto_\bfA A/\isoto$).
    Analog definieren wir den Verband der \keyword{Quotientenobjekte}\as{{Quotientenobjekt}{Kounterobjekt}} von $A$ als $\cosubobjects A$ durch die Isomorphieklassen von $A\epito_\bfA/\isoto$.
\end{definition}

\begin{remark}
    Die beiden Konzepte sind also genau dual zueinander.
\end{remark}

\subsection{Normale Unterobjekte und konormale Quotientenobjekte}

\begin{definition}[Normales Unterobjekt und konormales Quotientenobjekt]
        Ein Unterobjekt $N$ von $A$ heißt \keyword{normal}\as{{normales Unterobjekt}}, falls es einen Morphismus $\alpha:A\to$ gibt, sodass $N=\ker\alpha$.
        Ein Quotientenobjekt $Q$ heißt \keyword{konormal}\as{{konormales Quotientenobjekt}}, falls es einen Morphismus $\beta:\to\alpha$ gibt mit $\cokernel \beta=Q$.
\end{definition}

\subsection{Bilder}

\begin{definition}
    Sei $\alpha:A\to B$ ein Morphismus, dann bezeichnet $\image \alpha$ das induzierte Unterobjekt von $\alpha$.
\end{definition}

\subsection{Normale Morphismen}

\begin{definition}
    Ein Morphismus $\alpha:A\interpret\to\in\bfA B$ heißt \keyword{normal}\as{{normaler Morphismus}}, falls $\image\alpha$ ein
    normales Unterobjekt von $B$ ist.
\end{definition}

\subsection{Morphiesätze}

\begin{theorem}
    Sei $\bfA$ eine Kategorie mit Bildern und Kernen. Sei $\phi:A\interpret\to\in\bfA B$ ein Morphismus. Dann gibt es ein Objekt
    $C$ mit $\pi:A\epito C$, $\iota:C\monoto B$, sodass $\phi = \pi\iota$. 
\end{theorem}

%\begin{proof}
%    Blub
%\end{proof}

\section{Produkte und Koprodukte}

Sei $\bfB\monoto\bfA$ eine Unterkategorie. 


\section{Gruppenaxiome}

Unter einer Gruppe verstehen wir eine Struktur vom Typ $\group$, derart dass folgende Identitäten gelten
\begin{itemize}
        \item $(a\compose b)\compose c = a\compose (b\compose c)$ (Assoziativität)
        \item $a^{-1}\compose a = a\compose a^{-1} = 1$ (Inversenabblildung)
    \item $a\compose 1 = 1\compose a = a$ (neutrales Element)
\end{itemize}

\begin{theorem}
    Hallo
\end{theorem}

\section{Aufsteigende und absteigende Kettenbedingung}

\section{Der Satz von \person{Lagrange}}

\begin{theorem}[Satz von \person{Lagrange}]
    Sei $G$ eine Gruppe und $U$ eine Untergruppe. Dann definiert
    $
    a\sim b
    \define\equivalent
    a\cast U\to\set = b\cast U\to\set
    $
    eine reguläre Äquivalenzrelation, d.h.~jede Äquivalenzklasse ist gleich groß.
    Es gilt also
    $$
    \size G = \size U \size {\cosubobject G\over \sim}.
    $$
    Speziell gilt $\size G \epito \size U$. 
\end{theorem}

\section{Der Satz von \person{Lucas}}

\begin{theorem}
    Sei $p$ eine Primzahl. Dann gilt
    $$
    {m \choose n} = \product {{m_i} \choose {n_i}}
    $$
    wobei $m=\sum \power m_i p\operation\exponent i$
\end{theorem}

\section{Die \person{Sylow}'schnen Sätze}

Eine natürliche Frage, welche sich aus dem Theorem von \person{Lagrange} ergibt, welche Aussagen über die Anzahl und Art der
Untergruppen von Ordnung $n\instanceof\naturalnumber$ einer endlichen Gruppe $G$ getroffen werden können.
Falls $n$ die Zahl $\size G$ nicht teilt, ist selbige Anzahl nach dem Theorem von \person{Lagrange} (TODO: REF) gleich null. Ist $G$ zyklisch, so
ist jene Anzahl im entgegengesetzten Falle genau eins. Tatsächlich muss es aber für $n\divides \size G$ keine Untergruppen dieser
Ordnung geben, was man am leichtesten an der symmetrischen Gruppe $\auto m$ sieht, denn wählt man nun $n$ als eine Zyklizität
erzwingende Zahl, sodass $m<n\divides m!$, dann gibt es offensichtlich keine Untergruppen von $\auto m$ dieser Ordnung.
ist es ob bei einer endlichen Gruppe $G$ der

Tatsächlich lassen sich aber befriedigende Aussagen treffen, falls $n=p^e$ die Potenz einer Primzahl $p$ ist. Diese werden
gemeinhin als \person{Sylow}'sche Sätze bezeichnet.

\begin{definition}[Primärgruppe]
    Sei $G$ eine Gruppe derart, dass jedes Element $g\instanceof G$ eine Primzahlpotenz $p^{e_g}$ als Ordnung hat (wobei $p$ eine feste
    Primzahl sei). Dann heißt $G$ \keyword{Primärgruppe}\as{{$p$-Gruppe}{Primärgruppe}} zur Primzahl $p$.
\end{definition}

\begin{remark}
    Eine triviale Konsequenz der \person{Sylow}'schen Theoreme wird es sein, dass jede endliche Primärgruppe selbst von
    Primzahlpotenzordnung $p^e$ ist.
\end{remark}
%
\begin{theorem}[Existenz von Primäruntergruppen jeder Ordnung]
    Sei $G$ eine endliche Gruppe mit $\size G = p^e n$. Für die Anzahl $N_{p^e}\defeq \size{\hull{U\leq G:\size U=p^e}\in\set}$ gilt dann
    $$
    N_{p^e} = 1 \mod p
    $$
\end{theorem}
%
\begin{proof}
    Wir betrachten die Aktion von $G$ auf den $p^e$-elementigen Untermengen von $\cast G \to \set$ welche gegeben wird durch
    elementweise Rechtsmultiplikation. Die Bahnengleichung für diese Aktion wird dann zu
    $$
    \size {{\cast G \to \set} \choose {p^e}} = \sum_i {\size {G/\stab A_i}},
    $$
    wobei $A_i$ Repräsentanten der $G$-Bahnen sind.
    Für $A_i\monoto \cast G \to \set$, $\size {A_i}=p^e$ gilt allerdings dann $A_i \cast (\stab A_i) \to \set=A_i$, also ist $A_i$ eine disjunkte
    Vereinigung von Linksnebenklassen von $\stab A_i$ und mithin $\size {\stab A_i}\divides p^e$. Betrachten wir also obige
    gleichung modulo $pn$, so folgt
    $$
    {{p^e n} \choose {p^e}} = n N_{p^e} \mod pn, 
    $$
    denn alle Terme, in denen $\stab A_i<p^e$ ist in obiger Gleichung entfallen und die übrigen Terme zählen genau für jede
    $p^e$-elementige Untergruppe von $G$ ihre Linksnebenklassen (derer gibt es $n$).
    Daraus folgt
    $$
    \frac 1 n {{p^e n} \choose {p^e}} = {{p^e n-1} \choose {p^e-1}} = N_{p^e} \mod p,
    $$
    wobei der Ausdruck auf der Linken Seite gleich 1 ist modulo $p$. Dies sieht man einerseits daran, dass dies für die zyklische
    Gruppe mit $p^e n$ Elementen gilt, andererseits lässt sich auch das Theorem von \person{Lucas} (TODO : REF) auf den letzten
    Binomialkoeffizienten anwenden. Wir erhalten dann
    $$
    N_{p^e} = {{p^e n - 1} \choose {p^e -1}} = {{p-1} \choose {p-1}}^e = 1 \mod p.
    $$
\end{proof}

{\bfseries\scshape Text}

\begin{theorem}
    Jede endliche Gruppe $G$ hat $p$-\person{Sylow}-Gruppen. Für jede $p$-Untergruppe $U$ und eine $p$-\person{Sylow}-Gruppe von
    $G$ gibt es ein Element $g\in G$, sodass $U\monoto P^g$. Insbesondere sind alle $p$-\person{Sylow}-Gruppen konjugiert zueinander
    und ihre Anzahl ist $\size {G/N_G P}$. 
\end{theorem}

\section{Die Sätze von \person{Hall}}%

Die Sätze von \person{Hall} stellen eine Verallgemeinerung der \person{Sylow}'schen Sätze für auflösbare Gruppen dar. Entsprechende
Untergruppen nennt man auch \person{Hall}-Untergruppen.

\begin{theorem}[\person{Hall}'sches Theorem]
    Sei $G$ auflösbar und $\size G = mn$ mit teilerfremden $m$ und $n$. Dann gilt
    \begin{statements}
        \item Sei $U$ eine Untergruppe mit $\size U \divides m$ und $M$ eine Untergruppe der Ordnung $m$, dann gibt es ein $g\in G$
    sodass $U\leq M^g$.
        \item Für die Anzahl der Untergruppen der Ordnung $m$ von $G$ gilt:
    $$
    N_m = 1 \mod \rad m
    $$
    \end{statements}
\end{theorem}

\begin{proof}
    Der Beweis erfolgt per Induktion nach der Anzahl $k$ der Primfaktoren von $m$. Für $k=1$ gilt die Aussage schlicht aufgrund der
    \person{Sylow}'schen Sätze auch ohne Auflösbarkeit von $G$.
    
\end{proof}

\begin{theorem}[\person{Frattini}-Argument]
    Sie $G$ eine Gruppe und $N\leq_{\normalsubobjects G} G$. Weiter sei $P$ eine Untergruppe von $N$ derart, dass alle zu $P$ isomorphen
    Untergruppen in $H$ konjugiert sind (also z.B.~$P$ eine $p$-\person{Sylow}-Gruppe). Dann gilt
    $G = \normalizer_G P N$.
\end{theorem}

\begin{proof}
    Für $g\in G$ ist $P^g$ isomorph zu $P$ und gleichermaßen Untergruppe von $N$, da $N$ normal in $G$ liegt. Also sind $P$ und $P^g$ in $N$
    konjugiert und es folgt $P^{gn}=P$ für geignetes $n\in N$. Damit ist aber $gn\in \normalizer_G P$ und somit auch $g\in N
    \normalizer_G P$.
\end{proof}

\section{Auflösbarkeit von Gruppen}

\begin{definition}[Subnormalenverband, Subnormalenreihe]
    Ein Unterverband $\bfU$ von $\subobject G\in\group$ Subnormalenverband, falls für alle $U\in \bfU$ gilt
    $U\in\normalsubobjects \Meet_{V>U}V$. Ist ein solcher Verband isomorph zu einem Unterverband von $\cast \naturalnumber \to \lattice$, so nennen wir Ihn
    eine Subnormalenreihe. 
\end{definition}

\begin{definition}[Auflösbare Gruppe]
    Eine Gruppe $G$ heißt \keyword{auflösbar}\as{{auflösbare Gruppe}}, falls es einen Subnormalenverband von $G$ gibt, derart, dass
    $$
    \Meet_{V>U} V / U\ \textrm{kommutativ}
    $$
    für alle $U\in\bf U$. \keyword{endlich auflösbar}\as{{endliche auflösbare Gruppe}}\footnote{Dies meint auflösbar im herkömmlichen Sinne.}
\end{definition}

\begin{definition}[Kommutatoruntergruppe]
    Seien $\bfU\leq\cast \subobject G\in\group\to \set$. Dann bezeichnen wir mit $\commutatorsubgroup \bfA$ die
    \keyword{Kommutatoruntergruppe}\as{{Kommutatorgruppe}} von der Gruppen in $\bfA$. Sie wird erzeugt durch alle Kommutatoren $\commutator a b$ für $a\in
    A$, $b\in B$, $A,B\in\bfA$. 
\end{definition}

\begin{lemma}
    Die Kommutatoruntergruppe $\commutatorsubgroup \bfA$ ist charakterisiert durch folgende univerelle Eigenschaft.
    Sei $N$ ein Normalteiler von $\hull\bfA\in G$ derart, dass unter der kanonischen Projektion
    $\pi:\hull\bfA\in G\to\hull\bfA/N\in G$ die
    Bilder der Elemente von $\bfA$ paarweise elementweise kommutieren, d.h. $\commutatorsubgroup
    {\image{\iota_A\pi},\image{\iota_B\pi}}=1$, wobei $\iota_A$ bzw. $\iota_B$ die Inklusionen von Untergruppen $A,B\in\bfA$ sind,
    dann ist $\commutatorsubgroup \bfA\leq N$ und in äquivalenterweise gibt es einen eindeutigen Morphismus $\pi'$, sodass
    $\pi=p\pi'$ mit $p:\hull\bfA\in G\to\hull\bfA/N\in G$ die kanonische Abbildung.
\end{lemma}

\begin{proof}
    Der Beweis folgt aus dem allgemeineren Kontext (TODO).
\end{proof}

\begin{definition}[Kommutatorreihe]
    Wir definieren die \keyword{Kommutatorreihe}\as{{Kommutatorreihe}} $\sequence{\commutatorseries G i}_{i\in \naturalnumber}$ einer Gruppe $G$ als
    $$\commutatorseries G 0 \defeq G,\ \commutatorseries G {i+1} \defeq \commutatorsubgroup {\commutatorseries G i,\commutatorseries G i} \ (i\in\naturalnumber).$$
    Weiterhin setzen wir $\commutatorseries G \omega \defeq \Meet_{i\in\naturalnumber} \commutatorseries G i$ und bezeichnen es als
    \keyword{perfekten Kern}\as{{perfekter Kern}} von $G$.
\end{definition}

\begin{lemma}
    Eine Gruppe $G$ ist genau dann endlich auflösbar, falls ihre Kommutatorreihe nach endlich vielen Schritten in $1$ endet.
    Sie ist genau dann $\omega$-auflösbar, falls ihr perfekter Kern gleich $1$ ist.
\end{lemma}

\begin{proof}
    Sei $\bfG$
\end{proof}

\section{Der Satz von \person{Jordan}-\person{Hölder}}

\subsection{Das Modularitätsgesetz}

In der Verbandstheorie sind modulare Verbände gewissermaßen eine Abschwächung von distributiven
Verbänden.

\begin{lemma}
    Seien $A,B,C$ Untergruppen von $G$ und $C\leq A$, dann gilt
    $$
    \cast(A\meet B)\to\set \cast C\to\set=\cast A\to\set\meet\cast B\to\set\cast C\to\set.
    $$
\end{lemma}

\begin{proof}
    Sei $bc:A$, dann ist auch $b:A$, da $c:A$.
    Andererseits ...
\end{proof}

\subsection{Das \person{Zassenhaus}-Theorem (Schmetterlings-Theorem)}

\begin{theorem}[Schmetterlingslemma]
    Sei $A\interpret\leq\in{\normalsubobjects G} B$ und $C\interpret\leq\in{\normalsubobjects G} D$, dann gilt
    $$
    \cosubobject(B\meet D)\join A\over{(B\meet C)\join A}
    \isoto
    \cosubobject(B\meet D)\over{(B\meet C)\join (A\meet D)} 
    \isoto
    \cosubobject(B\meet D)\join C\over{(A\meet D)\join C}.
    $$
\end{theorem}

\begin{proof}
    
\end{proof}

\begin{theorem}[Verfeinerungssatz von \person{Schreier}]
    Seien $A$ und $B$ Normalketten. Dann existiert eine Verfeinerung.
\end{theorem}

\begin{theorem}[\person{Jordan}-\person{Hölder}'sches Theorem] 
    
\end{theorem}

\section{Nilpotenz von Gruppen}

Im Folgenden betrachten wir eine Eigenschaft von Gruppen die mit
\keyword{Nilpotenz}\as{{Nilpotenz}} bezeichnet wird. Sie stellt in gewissem Sinne
eine Verallgemeinerung der Eigenschaft einer Gruppe dar, kommutativ zu sein. Dies gilt in dem Sinne, dass die nilpotenten Gruppen
von Klasse 1 genau die abelschen Gruppen sind.

\begin{definition}[Normalkette, Normalreihe]
    Eine \keyword{Normalkette}\as{{Normalkette}} von $G$ ist eine Kette im Normalenverband
    $\normalsubobjects G$, ebenso ist eine \keyword{Normalreihe}\as{{Normalreihe}} eine Reihe im Normalenverband
    $\normalsubobjects G$.
\end{definition}

\begin{definition}[Zentralkette]
    Ein Normalreihe $\C$ heißt \keyword{Zentralreihe}\as{{Zentralreihe}}, wenn für alle
    $a,b\instanceof C$ mit $a\lessdot b$ gilt, dass $b/a$ im Zentrum
    von $G/a$ liegt oder in äquivalenterweise $\commutatorsubgroup {G, b} \leq a$.
\end{definition}

\begin{definition}[Untere Zentralreihe $\gamma_i$]
    Die \keyword{untere Zentralkette}\as{{untere Zentralkette}} $\hull{\lowercentralseries G i:i\in\naturalnumber}\in\set$ ist definiert als
    $$
    \lowercentralseries G 0 = G,\ \lowercentralseries G {i+1} = \commutatorsubgroup {G,\lowercentralseries G i} \for i\in\naturalnumber   
    $$
    und
    $$
    \lowercentralseries G \omega \defeq \Meet \lowercentralseries G i.
    $$
\end{definition}

\begin{definition}[obere Zentralreihe]
    Die \keyword{obere Zentralreihe}\as{{obere Zentralreihe}} $\hull{\uppercentralseries G i:i\in\naturalnumber}\in\set$ ist definiert durch
    $$
    \uppercentralseries G 0 = 1,\ \uppercentralseries G {i+1}/\uppercentralseries G i = \centre (G/\uppercentralseries G i) \for
    i\in\naturalnumber
    $$
    und
    $$
    \uppercentralseries G \omega \defeq \Join \uppercentralseries G i.
    $$
\end{definition}

\begin{lemma}[charakterisierende Eigenschaft der unteren Zentralreihe]
    Sei $\sequence{C_i}_{i\in\naturalnumber}$ eine aufsteigende Zentralreihe einer Gruppe $G$, dann gilt $\gamma_i G\leq C_i$ ($i\geq 0$). 
\end{lemma}

\begin{proof}
    Mit Induktion nach $i$. Für $i=0$ haben wir $\gamma_0 G= G= C_0$. 
    Für $i\in\naturalnumber$ gilt weiterhin $\gamma_{i+1} G = [G,\gamma_i G]\leq [G,C_i]\leq C_{i+1}$ (da $C_i/C_{i+1}\leq \centre
    (G/C_{i+1})$). Damit ist die Induktion abgeschlossen und es verbleibt die Zentralreiheneigenschaft von $\sequence{\gamma_i G}_{i\in
        I}$ nachzuweisen.
    Diese ist jedoch leicht zu überprüfen durch $\gamma_{i+1}=[G, \gamma_i G]\leq \gamma_{i+1}$, also liegt $\gamma_i
    G/\gamma_{i+1} G$ im Zentrum von $G/\gamma_{i+1} G$.    
\end{proof}


\begin{lemma}[charakterisierende Eigenschaft der oberen Zentralreihe]
    Sei $\sequence{C_i}_{i\in\naturalnumber}$ eine monotone Zentralreihe, dann gilt
    $$
    G_i\leq Z_i G
    $$
    und insbesondere ist $\sequence{Z_i G}_{i\in\naturalnumber}$ wirklich eine Zentralreihe.
\end{lemma}

\begin{proof}
    Mit Induktion nach $i$. Für $i=0$ gilt $Z_0 G = G = C_0$. Für $i\in\naturalnumber$ gilt weiterhin $C_{i+1}/C_i\leq\centre (G/C_i)$, was
    gleichbedeutend ist mit $[C_{i+1},G]\leq C_i$. Damit gilt aber $[C_{i+1},G]\leq Z_i G$ womit andererseits folgt, dass
    $C_{i+1}\leq Z_{i+1} G$. Damit ist die Induktion abgeschlossen und es verbleibt die Zentralreiheneigenschaft von $\sequence{Z_i
        G}_{i\in I}$ nachzuweisen. Diese folgt aber nach Definition trivial, denn $Z_{i+1} G/Z_i G=\centre (G/Z_i)\leq\centre(G/Z_i)$.
\end{proof}

\begin{definition}[Nilpotenz]
    Eine Gruppe $G$ heißt \keyword{nilpotent}\as{{nilotente Gruppe}}, falls es eine monotone Zentralreihe gibt, die gegen $G$ konvergiert und
    ..., falls es eine antitone Zentralreihe gibt, die gegen $1$ konvergiert. 
\end{definition}

\begin{theorem}[Charakterisierung von Nilpotenz]
    Die folgenden Aussagen sind für eine endliche Gruppe $G$ äquivalent.
    \begin{statements}
            \item $G$ ist nilpotent.
            \item Die untere Zentralreihe endet mit der trivialen Gruppe: $\exists n\in\naturalnumber:\gamma_n G = 1$.
            \item Die obere Zentralreihe von $G$ endet mit $G$: $\exists n\in\naturalnumber : Z_n G = G$.
            \item Für $U\in\subobject G\in\group$, $U\neq G$ gilt $U<N_G U$.
            \item Für $U\in\subobject G\in\group$, $U\neq 1$ gilt $[G,U]<U$.
            \item Jede maximale Untergruppe von $G$ ist normal in $G$.
            \item $G$ ist das direkte Produkt seine $p$-\person{Sylow}-Gruppe.
            \item Je zwei Elemente von koprimer Ordnung kommutieren.
    \end{statements}
\end{theorem}

\begin{proof}
    \begin{statements}
        \item Sei $P$ eine maximale $p$-Untergruppe, dann folgt aus $\normalizer_G P < G$, dass $\normalizer^2_G P = \normalizer_G
    P$. Also ist $G$ nicht nilpotent.
    \end{statements}
\end{proof}

\section{Die \person{Frattini}-Gruppe}

\begin{definition}
    Sei $G$ eine Gruppe mit der Eigenschaft, dass jede echte Untergruppe in einer maximalen Untergruppe liegt. Dann wird die \person{Frattini}-Gruppe $\Phi G$ definiert als der Durchschnitt aller maximalen Untegruppen von $G$.
\end{definition}

\begin{lemma}
    Die \person{Frattini}-Gruppe einer Gruppe $G$ ist charakteristische Untergruppe.
\end{lemma}

\begin{proof}
    Jeder Gruppenautomorphismus von $G$ induziert einen Verbandsautomorphismus von $\subobject G\in\group$. Unter diesem werden die maximalen
    Untergruppen auf sich abgebildet. Also $\Phi G^\alpha=\Phi G$.
\end{proof}

\begin{lemma}
    Genau dann gilt Element $g\in\Phi G$, falls für jede Menge $X\setleq \cast G\to \set$ gilt $\hull{g,X}\in G=G \implies \hull X\in G = G$. 
\end{lemma}

\begin{proof}
    Sei $\hull{g,X}\in G = G$ aber $\hull X \in G < G$, dann existiert eine maximale Untergruppe $M$ von $G$, sodass $\hull X \in
    G\leq M$, also $g\not\in M\geq \Phi G$.
    Andererseits sei $M$ eine maximale Untergruppe, dann gilt $\hull{g,M}\in G= M<G$, also $g\in M$.
\end{proof}

\section{Minimale Normalteiler und charakteristisch einfache Gruppen}

\begin{lemma}[Zusammenspiel  zwischen normal und charakteristisch]\label{nor-char-implies-nor}
    Sei $C$ charakteristisch in $N$ und $N$ normal in $G$. Dann ist $N$ normal in $G$.
\end{lemma}

\begin{proof}
Jede Konjugation (innere Automorphismus) in $G$ lässt $C$ fix, beschränkt sich also zu einem Automorphismus von $N$. Damit lässt
sie auch $C$ fest, nach Definition von charakteristisch.    
\end{proof}

\begin{lemma}
    Sei $N$ ein minimaler Normalteiler einer Gruppe $G$. Dann ist $N$ charakteristisch einfach.
\end{lemma}

\begin{proof}
    Sei $1<C$ eine charakteristische Untergruppe von $N$. Dann ist nach \ref{nor-char-implies-nor} auch $C$ normal in $G$. Nach
    Minimalität von $N$ folgt $N\leq C$ somit also $N=C$. Also ist $N$ charakteristisch einfach.
\end{proof}

\begin{theorem}[Charakterisierung charakteristisch einfacher Gruppen]
    Eine Gruppe $G$ mit minimalem Normalteiler $N$ ist genau dann charakteristisch einfach, falls $N$ einfach ist und sie eine Potenz von $N$ ist.
\end{theorem}

\begin{proof}
    Sei $G$ charakteristisch einfach und $N$ minimaler Normalteiler. Die Bilder von $N$ unter $\auto G$ erzeugen eine nicht-triviale
    charakteristische Untergruppe von $G$, also ganz $G$. Sind weiter $N$ und $M$ zwei solcher Bilder, dann gilt $N = M$ oder
    $N\meet M = 1$, da $N\meet M\leq N$ ein Normalteiler von $G$ ist. Also folgt für verschiedene $N$ und $M$, dass
    $\commutatorsubgroup M N \leq M\meet N = 1$, somit $M$ und $N$ elementweise kommutieren. Damit ist $G$ das Produkt all
    jener Bilder, da diese --- wie schon erwähnt --- $G$ erzeugen. Weiter muss dann $N$ einfach sein, da jeder Normalteiler $L$ von
    $N$ aufgrund der Produktdarstellung von $G$, dann auch normal in $G$ liegt.

    Für die Rückrichtung nehmen wir an, dass $G = \product S_i$.
\end{proof}



%%% Local Variables:
%%% coding: utf-8
%%% TeX-engine: xetex
%%% TeX-master: "Script-Main"
%%% End: 
