% My personal maths package

%\bibliography{Bibliography.bib}
\usepackage{amssymb,amsthm,amsmath,amsfonts}

%\usepackage{mathabx} % more and nicer symbols symbols

%\usepackage{mathabx} % more and nicer symbols symbols

% standard packages
\usepackage{color} % for color
%\usepackage[usenames,dvipsnames]{pstricks} % maybe there is a better one
%\usepackage{epsfig}
%\usepackage{pst-grad} % For gradients
%\usepackage{pst-plot} % For axes

% importing various types of grapics

%\usepackage{epsfig}  		% For postscript
%\usepackage{epic,eepic}       % For epic and eepic output from xfig


% The following is very useful in keeping track of labels while writing
%\usepackage[notcite]{showkeys}

%textwidth ...
%\setlength{\textwidth}{15cm}

%%%%%%%%%%%%%%%%%%%%%%%%%%%%%%%%%%%%%%%%%%%%%%%%%%%%%%%%%%%%%%%%%
%						Math environments						%
%%%%%%%%%%%%%%%%%%%%%%%%%%%%%%%%%%%%%%%%%%%%%%%%%%%%%%%%%%%%%%%%%

\newtheorem{theorem}{Theorem}[section]
\newtheorem{proposition}[theorem]{Proposition}
\newtheorem{lemma}[theorem]{Lemma}
\newtheorem{corollary}[theorem]{Corollary}
\newtheorem{conjecture}[theorem]{Conjecture} 

\theoremstyle{definition}
\newtheorem{definition}[theorem]{Definition}
\newtheorem{example}[theorem]{Example}

\newtheorem{note}[theorem]{Note}

\theoremstyle{remark}
\newtheorem{remark}[theorem]{Remark}

% common algebraic objects
\newcommand{\reals}{\mathbb{R}} 			% The real numbers
\newcommand{\nats}{\mathbb{N}} 				% The natural numbers
\newcommand{\ints}{\mathbb{Z}} 				% The integers
\newcommand{\rats}{\mathbb{Q}}				% The rationals
\newcommand{\complex}{\mathbb{C}}			% The complex numbers
\newcommand{\field}{\mathbb{F}}				% The standard arbitrary field
\newcommand{\card}{\boldsymbol{Cn}}	% The cardinal numbers
\newcommand{\ord}{\boldsymbol{On}}	% The ordinal numbers

% new operators and relations

%%%%%%%%%%%%%%%%%%%
% complex numbers %
%%%%%%%%%%%%%%%%%%%

\renewcommand{\Re}{\operatorname{Re}}		% real part
\renewcommand{\Im}{\operatorname{Im}}		% imaginary part
\DeclareMathOperator{\sgn}{sgn}				% the sign operator (0 for 0)

%%%%%%%%%%%%%%%%
% reel numbers %
%%%%%%%%%%%%%%%%

\newcommand{\floor}[1]{\lfloor #1 \rfloor}
\newcommand{\ceil}[1]{\lceil #1 \rceil}

%%%%%%%%%%%%%%%%%%
% set operations %
%%%%%%%%%%%%%%%%%%

\newcommand{\intersect}{\cap}			% intersect to sets
\newcommand{\setjoin}{\cup}				% join two sets
\newcommand{\setmeet}{\cap}                     % intersect to sets
\newcommand{\union}{\bigcup}			% the union of sets ... subscripts to be added
\newcommand{\distunion}{\dot{\bigcup}}	% disjoint union of sets ... subscripts to be added
\newcommand{\intersection}{\bigcap}		% intersection of sets
\newcommand{\powerset}{\mathcal{P}}		% powerset ... to be customized

%%%%%%%%%%%%%%%%%%%%%%%%%%%%%%%%%%%%%%%%
% composition operations of structures %
%%%%%%%%%%%%%%%%%%%%%%%%%%%%%%%%%%%%%%%%

%\newcommand{\setprod}{\bigtimes}			% setproduct - needed
\newcommand{\dirprod}{\bigotimes} 			% direct product for groups and spaces
\newcommand{\dirtimes}{\otimes}				% direct multiply for groups and spaces
\newcommand{\dirsum}{\bigoplus} 			% direct sum for groups and spaces
\newcommand{\dirplus}{\oplus}				% direct add for groups and spaces
\newcommand{\inprod}[2]{\left\langle #1,#2 \right\rangle}
%%%%%%%%%%%%%%
% categories %
%%%%%%%%%%%%%%

\renewcommand{\Vec}{\boldsymbol{Vec}} 		% vectorspaces (field must be added as subscript)
											% etc. ... things must be added

%%%%%%%%%%%%%%%%%%%%%%%%%%%%%%%%%%%%%%%%%%%%%%%%%%%%%
% metric spaces and normed spaces and vector spaces %
%%%%%%%%%%%%%%%%%%%%%%%%%%%%%%%%%%%%%%%%%%%%%%%%%%%%%

\newcommand{\dist}{\operatorname{dist}}				% distance operator ... dist(A,b), where A is a set and b a point
\DeclareMathOperator{\diam}{\operatorname{diam}}	% diameter operator for sets
\newcommand{\norm}[1]{\left\Vert #1 \right\Vert}	% norm in a normed space ... subscript to be added
\newcommand{\conv}{\operatorname{conv}} 			% convex hull - vectorspaces
\newcommand{\lin}{\operatorname{lin}} 				% linear hull - vectorspaces
\newcommand{\aff}{\operatorname{aff}}				% affine hull - vectorspaces

%%%%%%%%%%%%%%%%%%%%%%
% operators in rings %
%%%%%%%%%%%%%%%%%%%%%%

\newcommand{\lcm}{\operatorname{lcm}}				% least common multiple - in euclidean rings
\renewcommand{\gcd}{\operatorname{gcd}}				% greatest command devisor - in euclidean rings
\newcommand{\res}{\operatorname{res}}				% residue of p mod q is res(p,q)
\renewcommand{\mod}{\text{ mod }}

%%%%%%%%%%%%%%%%%%%
% logical symbols %
%%%%%%%%%%%%%%%%%%%

\renewcommand{\impliedby}{\Leftarrow}				% reverse implicatoin arrow
\renewcommand{\implies}{\Rightarrow}				% implication
\newcommand{\equival}{\Leftrightarrow}				% equivalence

%%%%%%%%%%%%%%%%%%%%%%%%%%%%%%%%%%
% functions - elementary symbols %
%%%%%%%%%%%%%%%%%%%%%%%%%%%%%%%%%%

\newcommand{\rest}[1]{\left. #1\right\vert}		% restriction of a function to a set / also used as restriction in other terms like differential expressions / evaluation of a function
\renewcommand{\to}{\rightarrow}							% arrow between domain and image
\newcommand{\dom}{\operatorname{dom}}					% domain of a function
\newcommand{\im}{\operatorname{im}}						% image of a function
\newcommand{\compose}{\circ}							% compose two functions
\newcommand{\cont}{\operatorname{C}}					% continuous functions from a domain into the reels or complex numbers 

%%%%%%%%%%%%%%%%%%%%
% groups - symbols %
%%%%%%%%%%%%%%%%%%%%

\newcommand{\stab}{\operatorname{stab}}					% the stabilizer ... subscripts to be added
\newcommand{\orb}{\operatorname{orb}} 					% orbit ... subscrit to be added (group)
\newcommand{\gen}[1]{\langle #1\rangle}					% generate ... kind of hull operator ---- to be thought of !!!!!!!

%%%%%%%%%%%%%%%%%%%%%%%%%%%%%%%%%
% matrices and linear operators %
%%%%%%%%%%%%%%%%%%%%%%%%%%%%%%%%%

\newcommand{\diag}{\operatorname{diag}}				% diagonal matrix or operator
\newcommand{\Eig}[1]{\operatorname{Eig}_{#1}}		% eigenspace for a certain eigenvalie		
\newcommand{\trace}{\operatorname{tr}}				% trace of a matrix
\newcommand{\trans}{\top} 							% transponse matrix

%%%%%%%%%%%%%
% constants %
%%%%%%%%%%%%%

\renewcommand{\i}{\boldsymbol{i}}			% imaginary unit
\newcommand{\e}{\boldsymbol{e}}				% the Eulerian constant

%%%%%%%%%%%%%%%%%%%%%%%%%%%%%%
% limit operators and arrows %
%%%%%%%%%%%%%%%%%%%%%%%%%%%%%%

\newcommand{\upto}{\uparrow}				% convergence from above
\newcommand{\downto}{\downarrow}			% convergence from below

%%
% other
%%
\newcommand{\cind}{\operatorname{Ind}}		% Cauchy index
\newcommand{\sgnc}{\sigma}					% sign changes
\newcommand{\wnumb}{\omega}					% winding number
\newcommand{\cfunc}{\operatorname{Cf}}		% Cauchy function of a compact curve in complex\setminus\{0\}


% evaluation of a function as a difference or single value

\newcommand{\abs}[1]{\left|#1\right|}
\newcommand{\conj}[1]{\overline{#1}}
\renewcommand{\d}{\operatorname{d}}

%%%%% test
\newcommand{\distjoin}{\mathaccent\cdot\cup}	% to be modified (name)
\newcommand{\cl}{\operatorname{cl}}				% topological closure
\newcommand{\sphere}{\mathbb{S}} % n-sphere
\newcommand{\ball}{\mathbb{B}} % n-ball
\newcommand{\bound}{\partial}
\newcommand{\bigmeet}{\operatorname{\bigwedge}}
\newcommand{\bigjoin}{\operatorname{\bigvee}}
\newcommand{\rchar}{\operatorname{char}}              % characteristic of a ring
\newcommand{\lgor}{\vee}                               % logical
\newcommand{\lgand}{\wedge}
\newcommand{\iso}{\cong}
\newcommand{\codim}{\operatorname{codim}}
\newcommand{\row}{\operatorname{row}}
\newcommand{\cone}{\operatorname{cone}}
\newcommand{\comp}{\operatorname{comp}}
\newcommand{\proj}{\operatorname{\bold{P}}}
\newcommand{\meet}{\wedge}
\newcommand{\join}{\vee}
\newcommand{\col}{\operatorname{col}}
\newcommand{\vol}{\operatorname{vol}}
%arrangements
\newcommand{\tpert}{\operatorname{tpert}}

%% set hulloperators -> define

%% test
\makeatletter
\newenvironment{proofof}[1]{\par
  \pushQED{\qed}%
  \normalfont \topsep6\p@\@plus6\p@\relax
  \trivlist
  \item[\hskip\labelsep
        \bfseries
    Proof of #1\@addpunct{.}]\ignorespaces
}{%
  \popQED\endtrivlist\@endpefalse
}
\makeatother



%%% environment test with enumerates

%\newcommand{\mathenv}[#1]{
%\newcounter{#1}								%define lemma counter
%\newcommand\the#1{\arabic{section}.\arabic{lemmax}} 

%\newenvironment{test}{\noindent {\bfseries Lemma %\thelemmanr.}\renewcommand{\theenumi}{\Roman{enumi}}
%\renewcommand{\labelenumi}{\theenumi}}{BS} }