\documentclass[12pt, twoside]{book}
\usepackage[inner=0.75in, outer=0.75in, top=0.75in, bottom=0.75in, paperwidth=6in, paperheight=9in]{geometry}

\usepackage{fancyhdr}

% ===== Set german language and enable typing umlauts
% for use of german language in a document
\usepackage[utf8]{inputenc} % this is needed for umlauts
\usepackage[ngerman]{babel} % this is needed for umlauts
\usepackage[T1]{fontenc}    % this is needed for correct output of umlauts in pdf


\begin{document}
% data for maketitle
\title{Die Erlebnisse meiner Kindheit\\ \emph{Bericht einer Zeitzeugin}}
\author{Käte Halbach}
\date{11. März 2014}

% no headers for empty pages
\makeatletter
\def\cleardoublepage{\clearpage\if@twoside \ifodd\c@page\else
    %\hbox{}
    \vspace*{\fill}
    %\begin{center}
    %    This page intentionally contains only this sentence.
    %\end{center}
    %\vspace{\fill}
    \thispagestyle{empty}
    \newpage
    \if@twocolumn\hbox{}\newpage\fi\fi\fi}
\makeatother

\maketitle

\chapter{Familie und Herkunft}
\pagestyle{fancy}

So da wohnten wir. Und meine Eltern hatten einen Bauernhof.
Rauschnicken hieß der Ort. Rauschnicken, Kreis Wehlau, Ostpreußen.
Mein Vater war dort Bürgermeister von der Gemeinde Bartenhof, Rauschnicken und Fichtenhof.
Und unser Dorf war ein reines Bauerndorf und Bartenhof war ein reines Gutsdorf mit einem Gutsherren --- Meyer hieß er.
Das war bei uns früher also so, dass manche Dörfer reine Gutsdörfer und andere Bauerndörfer waren.
Und auf dem Gut gab es Knechte und Mägde, einen Schmied, Gartenmeister und viele andere Bedienstete.
Einen Stellmacher gab es auch, also jemanden, der so Holzarbeiten erledigte --- alles eigentlich, alle Berufe hatten sie.

Rauschnicken war dreißig Kilometer von Königsberg entfernt und ebenso dreizig Kilometer von der Kreisstadt Wehlau.

\paragraph{Mutter Martha Klara Therese Neumeier geborene Neumann}

Meine Mutter Martha wurde geboren am 14.8.1900 in Rauschnicken.
Sie stammte aus einem Bauernhaus mit Gaststätte. Dort haben sie jedes Jahr Ferienkinder aufgenommen.
Die sind dann aus den Städten gekommen von Königsberg oder Berlin. Und da war auch meine Tante Käte dabei, die ist von Kindheit an dann jedes Jahr nach Ostpreußen gekommen, um Ferien zu machen, Sommerferien.
Sie ist natürlich nicht meine richtige Tante, aber meine Patentante, der ich auch meinen Namen zu verdanken habe.
Mit Tante Käte hatten wir ein Leben lang Verbindung bis zu ihrem Tod. Sie ist später auch von Berlin aus hier öfter [Wendhausen] zu Besuch gekommen und hat sogar unsere Kinder Karl-Friedrich und Doris je eine Woche nach Berlin eingeladen, damals war Doris fünfzehn Jahre alt und Karl-Friedrich sechzehn.

\paragraph{Jüngste Schwester Frieda Beutler, geborene Neumann}

Von meiner Tante Frieda weiß ich noch, dass sie und ihre Familie einen sehr großen Bauernhof mit Schweizer [Melker, zuständig für die Viehversorgung] in Grünlauken hatten.
In Erinnerung sind mir auch noch die Weihnachtstage, die wir alle zwei Jahre dort verlebt haben, geblieben. Da sind wir dort mit dem Schlitten [mit Pferden bespannt] vorbeigefahren.
In den anderen Jahren sind sie dann bei uns vorbeigekommen. Außerdem kann ich mich noch sehr gut an die Konfirmation meiner Cousine Erika erinnern --- das war immer etwas Besonderes für uns Kinder.
Sie [Tante Frieda] und ihre Tochter Erika sind später unter den Russen gekommen. Und dann mussten sie dort auf ihrem eigenen Hof als Bedienstete arbeiten --- es war damals so, dass alle Flüchtlinge zurückgetrieben wurden nach ihrem Zuhause und dort schuften mussten.
Erika wurde auch vergewaltigt und bekam dann ein Kind vom Russen. Aber, als sie hochschwanger war, durfte sie mit ihrer Mutter raus aus Ostpreußen. 

Der Mann meiner Tante Frieda, Fritz Beutler, ist damals auch von den Russen verschleppt worden --- dadurch ist er verschollen.

\paragraph{Cousin Gerhard Beutler} Das hier ist mein Cousin Gerhard Beutler [ein Bild zeigend]. Zu dem fällt mir noch eine Geschichte ein.
Damals sind die Nazis mit einem Zettel in die Schule gegangen und haben gesagt:
"<Unterschreibt mal eben hier, das ist für einen guten Zweck."> Und die ganze Klasse hat dann unterschrieben, das war eine Jungenklasse.
Und damit hatten sie für die Waffen-SS unterschrieben. Dann sind die Eltern in die Schule gegangen und haben gesagt: "<Das sind doch Kinder. Ihr könnt doch mit unseren
Kindern das nicht machen.">
Doch dann mussten sie weg zur Ausbildung. Danach war Schule zu Ende für sie --- war Kriegspielen [Ausbildung]. 
Und nachher wurde seine Einheit nach Italien versetzt, und in Italien ist er gefallen als Soldat. Er war damals sechzehn Jahre alt.

\paragraph{Älteste Schwester Gertrud Corinth, geborene Neumann} Die älteste Schwester meiner Mutter hieß Gertrud Corinth, der Mann Walter und die Tochter Irmgard.
Er ist früh verstorben, war Lokomotivführer. Auch diese Schwester ist in Königsberg unter den Russen gekommen und auch vermisst. 
Königsberg wurde doch damals zur Festung erklärt und zweimal bombardiert. Die Stadt wurde total zerstört.
August 1944 fand die erste Bombardierung statt --- zwei Drittel der Stadt waren kaputt. Ja, die Menschen mussten viel Leid ertragen.
Und ihre Tochter Irmgard wurde verschleppt nach Sibirien.
Sie hat uns sogar von dort eine Karte nach Dänemark geschrieben --- das war 1947. Es gab ja damals nur solche Karten, da durfte man immer nur so eine bestimmte Wortzahl schreiben, die lasen das ja und zensierten das.
Und da hat sie unter anderem geschrieben: "<Wisst ihr nicht, wo meine Mutti ist?"> Aber wir wussten es nicht. Wir waren in Dänemark im Lager und wussten noch nicht einmal, woher sie unsere Anschrift hatte.
Das kann nur über das Rote Kreuz gegangen sein.
Das Rote Kreuz hat ja damals sehr viel geleistet, auch für die Zusammenführung der Familien.
Da konnte man dann auch Suchanzeigen angeben. Das hat meine Mutter dann auch getan --- Suchanzeigen von allen ihren Verwandten, auch von meinem Vater.
Na jedenfalls ist Irmgard dann tödlich verunglückt bei Waldarbeiten. Sie mussten im Bergwerk und im Wald schuften, da in Sibirien. Das haben wir von einer ihrer Mitleidenden erfahren, die mit verschleppt wurde. Sie hat uns von dem Unglück unterrichtet [geschrieben].

\paragraph{Jüngster Bruder Karl Neumann} Der jüngste Bruder hieß Karl Neumann, verheiratet mit Erna Neumann und sie hatten einen Sohn Horst. Der Bruder Karl meiner Mutter war auch Lokführer. Auch er ist vermisst, war bis zum Schluss im Dienst.
Meine Tante Erna war gelernte Uhrmacherin und ist mit ihrem Sohn unter den Russen gekommen --- genau wie Tante Gertrud. Nach ihren Erzählungen --- da wir bis zum Schluss ja Kontakt hatten, hat sie dann aus der Zeit auch Etliches erzählt ---
mussten sie 20-Kilometer-Fußmärsche von Königsberg aus antreten, Winter 1945 bei über 20 Grad Frost. Sie hatte dort noch ihre betagten Eltern bei sich.
Während dieser Strapazen sind die Eltern gestorben, wahrscheinlich als sie gerastet haben.
Die haben sie dann in einem Sack im Schnee vergraben und sind dann einfach weitergezogen.
Und diese Märsche waren nur Schikane, die haben die Leute nur hin und zurückgetrieben.
Mit ihrem Handwerk konnte sie aber einigermaßen überleben. Damals sind ja auch viele verhungert, es gab ja nichts. Auch Tante Käte hat in ihren Briefen von Verwandten geschrieben, die dort verhungert sind.
Das kann man gar keinem erzählen, so schlimm ist das.

Eine andere Geschichte muss ich noch erzählen:
Ein 17-jähriges Mädchen sprach meine Tante an und sagte zu ihr: "<Die Züge fahren ja von Moskau nach Berlin --- leer hin und voll zurück. Weil sie ja die Wohnungen ausgeräubert haben. Wir wissen, dass die leeren Züge dann über Königsberg nach Berlin fahren. Wir schleichen uns auf einen Wagen und verstecken uns und fahren damit nach Berlin. Ich habe vom Herrgott die Zusage, dass wir gesund
in Berlin ankommen."> Es war damals Sommer. Meine Tante sagte mir: "<Wir sind drei Tage von Königsberg nach Berlin gefahren. Das waren offene Züge gewesen, wir wurden nass und sind wieder getrocknet und gesund in 
Berlin angekommen."> Sie sagte auch: "<Die Russen haben uns gesehen, aber sie haben uns gelassen. Es war wirklich Bewahrung.">

Ja dann hat sie ihre Lehrstelle in der Nähe von Berlin aufgesucht, wo sie Uhrmacherin gelernt hatte, und dort auch Zuflucht gefunden.
Über die Geschichte könnte man noch ein ganzes Buch schreiben.
Dann hat sie dort im Ausbesserungswerk der Bahn arbeiten müssen, in Stendal, DDR.
Ihr Sohn ist als 18-jähriger über die Grenze gekommen, zu meinen Eltern ins Weser Bergland, ging aber nicht wieder zurück, weil sie ihn an der Grenze so schikaniert hatten.
Dadurch hat Tante Erna viel Ärger durch gehabt.
Sie wurde dann aufgefordert sich zu melden, ihrem Sohn wurde dann Republikflucht vorgeworfen. Als der Sohn später heiraten wollte, die
Mutter zur Hochzeit einlud, durfte sie nicht kommen, weil er Republikflucht begangen hatte. "<Und selbst wenn ihr Sohn auf dem Sterbebett läge, dürften Sie auch nicht rüberfahren.">
Das haben wurde ihr gesagt.
Aber als sie Rentnerin wurde, durfte sie mit Sack und Pack rüber nach Wolfsburg zur ihrem Sohn.
Sie ist dann mit neunundneunzig und einem dreiviertel Jahr verstorben in Wolfsburg --- bei ihrem Sohn im Hause.
Rüstig bis zu einem halben Jahr davor.

\paragraph{Mittlerer Bruder Richard Neumann} Onkel Richard, das war der mittlere Bruder meiner Mutter. Ja, da weiß ich nicht viel. Der hat das elterliche Anwesen und die Gaststädte mit Konialwaren [Lebensmittelgeschäft] übernommen.
Er ist früh verstorben, hatte vier Kinder --- Georg, Karl-Heinz, Richard und Walter --- und das jüngste starb auch früh. Seine Frau hieß Antonie Neumann --- von uns auch Tante Toni genannt. Sie war eine geborene Fischer.
Da wir direkt gegenüber, auf der anderen Straßenseite, wohnten, haben wir als Kinder oft miteinander gespielt --- wir waren ungefähr im gleichen Alter.
Tante Toni hat das Anwesen nach Onkel Richards Tod weitergeführt bis wir zusammen auf die Flucht gegangen sind. Bis nach Dänemark. Damals nahm nur die französische Zone Flüchtlinge auf und so hat sie sich dorthin gemeldet und kam dann bei Teilfingen, am Bodensee, unter mit ihren vier Kindern.

\paragraph{Ältester Bruder Ernst Neumann} Onkel Ernst hatte auch einen Bauernhof und seine Frau hieß Martha Neumann, geborene Fischer [Schwester von Tante Toni]. Er hatte drei Töchter: Hanna, Irene und Annemarie.
Die sind nachher auch alle in Dänemark gelandet, aber getrennt. Und später sind sie dann 47 am Bodensee gelandet --- die ganze Familie. Der Onkel Ernst musste ja zum Volkssturm, ist aber dann trotzdem irgendwie nach Dänemark gelangt.
Die amerikanische Zone nahm damals auf, die englische nicht, weil sie überfüllt war.

\paragraph{Die Eltern meiner Mutter} Der Vater hieß Karl Neumann, geboren 28.01.1865 in Warginen. Und die Mutter der Mutter ist Marie Neumann, geborene Gottschall am geboren am 03.03.1875 in Rauschnicken.
Die Eheschließung war am 04.11.1891 in Cremitten.

\paragraph{Kinder meiner Mutter} Die Kinder meiner Mutter sind Fritz Karl August Neumeier (26.07.1930, Rauschnicken), Käte Martha Marie (ich, 04.10.1931, Rauschnicken), Helmut Ernst Willi (06.06.1933, Rauschnicken), Dora Gertrud Minna (21.04.1935, Rauschnicken), Eva Frieda Antonie (28.10.1937, Rauschnicken).
Die Namen wurden damals auch teilweise von Tanten und Patentanten übernommen.

\paragraph{Vater Fritz Otto Gustav Neumeier} geboren in Dubeningken am 20.10.1903, Kreis Goldap. 

\paragraph{Eltern und Schwestern des Vaters} Vater August Neumeier, geboren 22.04.1870 in Jörkischken. Standesamt Goldap.
Die Mutter hieß Marie --- daher mein Zweitname ---, geborene Pridat am 16.10.1861 in Dubeningken.
Und zwar war das der zweite Mann der Frau. Der erste war gestorben und da hatte sie zwei Mädchen: Minna und Bertha.
Später hat sie dann den Junggesellen, den Vater meines Vaters, geheiratet.
Eheschließung dieser beiden war am 18.09.1902.

\chapter{Die Flucht}

Wie ich schon gesagt habe, hatten meine Eltern ja einen Bauernhof und mein Vater war Bürgermeister die ganzen Jahre lang, weswegen er auch nicht eingezogen wurde, da ja einer die Verwaltungsangelegenheiten regeln musste.
Später wurde er dann doch --- wie alle anderen --- für den Volkssturm eingezogen, ist dann aber, weil die Front ganz nahe kam, nach Hause gelaufen, um der Familie beim Flüchten zu helfen. 

Der 20. Januar 1945 war unser letzter Schultag, das war ein Samstag. Am 23. Januar jedenfalls sind wir geflüchtet. Wir durften nämlich vorher nicht, weil das laut Kriegsgesetz verboten war.
Wir hatten aber Flüchtlinge schon seit Oktober aufgenommen, die jedoch selbst schon Anfang August 1944 geflüchtet waren. Es war also deren zweite Station. Sie kamen direkt von der Grenze, Kreis Schlossberg.
Dieselben wurden aufgefordert, nach Westdeutschland zu flüchten. Sie sind dann im Weser Bergland gelandet. Zum Beispiel Familie Schönberger, die wir auch später besucht haben. Wir sind also am 23. Januar 1945
geflüchtet, da hat sich kein Mensch mehr darum gekümmert, ob man flüchten würde oder nicht, so nah war die Front schon. Da wurden, das weiß ich noch, die Wagen mit einer Plane bespannt und dann wurden 
auch noch ein Schwein geschlachtet, Federbetten und Essen eingepackt. Es waren 25 Grad Frost. Wir nahmen dann auf dem Wagen noch zwei Familien mit Säuglingen mit, die nachher erfroren sind.
Alles in allem bestand unser Treck dann aus sieben Wagen. Das Vieh hatten wir im Stall zurückgelassen und die Wohnungen blieben offen. Wir nahmen mit, was auf den Wagen Platz hatte. Zu diesem Zeitpunkt dachten
wir noch, wir würden irgendwann zurückkommen. Die Soldaten waren im Rückzug, die Straßen verstopft mit Flüchtlingen und Soldaten. Den ersten Tag sind wir nur fünf Kilometer gekommen, weil alles so voll
war. Alle wollten nach Königsberg, weil sie dachten, das wäre jetzt die rettende Stadt.
Wir konnten laufen im Schnee, aber für die meisten Säuglinge gab es keine Rettung, viele sind erfroren. Unsere Kolonne fuhr nicht nach Königsberg in die Stadt, sondern zur Ostküste (Samland-Küste).
Das war unser Glück, denn diejenigen die nach Königsberg hereinfuhren, kamen dort unter den Russen. Erstmal wurde das (Königsberg) mit der Stalinorgel (Artillerie) beschossen. Dann wurde es zur Festung erklärt und sie gingen 
von Haus zu Haus und holten sich, was sie brauchten und was da war.

Wir fuhren an der Küste entlang. Auf dem Weg zur Küste übernachteten wir in Häusern, Kuh-, Pferde- und Hühnerställen. Ich habe sogar einmal in einer Krippe gelegen. Die Leute waren ja alle weg.
Sie waren ja selbst auch schon geflüchtet. Und die Häuser waren leer.
Als wir nach Rauschen wollten, erzählten uns andere Flüchtlinge, dass dort der Russe schon Fuß gefasst hätte. 

Ich kann mich noch an ein anderes Erlebnis erinnern. In einem Ort, da standen die letzten deutschen Soldaten schon in den Häuserecken, zum Nahkampf bereit. Und da kamen die Panzer, die Tiger (Deutsche). Und der Russe
kam ihnen ebenfalls mit Panzern entgegen. Und wir ließen den Wagen stehen und liefen in die Häuser. Aber wir haben es irgendwie geschafft.
Jedenfalls haben sie dann irgendwie den Russen zurückgedrängt und wir haben wieder versucht weiterzuziehen. Uns ist jedenfalls nichts geschehen.
	Wir sind immer an der Front gewesen --- dreieinhalb Wochen --- und nie unter den Russen gekommen. Das war ein Segen.

An eine andere Sache erinnere ich mich auch noch. Das war an einer anderen Stelle. Wir waren ja sieben Wagen und wurden dann getrennt (an dieser Stelle). Und da war am Waldrand im Hintergrund eine ganze Front Soldaten zu sehen.
Die schossen und alles brannte. Und dann war immer wieder die Stalinorgel zu hören. Dann sah man sie mit Schlitten die Verwundeten ziehen.
Und der Pole, der bei uns gearbeitet hatte, wurde verwundet. Er wurde von einer Kugel am Bein getroffen. Wir brachten ihn ins Feldlazarett. Er hieß Vladislav Mikula. 
Der Pole war aber mal Soldat gewesen und daher wusste er, dass die Lebensmittelmagazine gesprengt werden sollten, damit sie nicht den Russen in die Hände fallen würden. So besorgte er uns noch Schokolade. Er blieb dann dort zurück.
Wir freuten uns, dass wir was etwas im Bauch hatten, mussten aber trotzdem davon brechen.

Ich kann mich auch noch an eine Frau erinnern, die als einzige in einem Dorf geblieben war, um für die Flüchtigen Brot zu backen. Das Brot, was wir eingepackt hatten, war ja auf dem Wagen alles gefroren.
Mein Großvater war wider allem guten Zureden nicht mit auf die Flucht gekommen. Er sagte wörtlich: "<Kinder, was wollt ihr denn. Einer muss doch hier auf die Höfe aufpassen."> --- Wir haben nie wieder etwas von ihm gehört.

Am Ende fanden wir uns dann aber wieder.
Nach dreieinhalb Wochen Treck sind wir in einem Fischerdorf Rothenen an der Ostseeküste gelandet. Keine 200 Meter von der See entfernt. Da waren nur noch drei Jungs in dem Dorf. Eines Morgens lag der ganze Strand
voller Fische, ich meine, es waren Heringe. Und uns wurde von den Jungen erzählt, dass dies nur alle 25 Jahre vorkommen würde. Das würde an irgendwelchen Strömungen liegen. So hatten sie es von ihren Eltern erzählt bekommen.
Ja, für die Flüchtlinge war das natürlich ein Segen. Wir gingen dann also mit zum Strand und haben die Fische eingesammelt. Es waren so viele, dass wir sie gar nicht alle verwerten konnten.
Tage später sind dann am gleichen Strand lauter Leichen angespült wurden. Wir sind natürlich als Kinder hin und haben sie uns angesehen, nachdem einer davon erzählt hatte. Deswegen habe ich nie schwimmen gelernt.
Da wurde dann gerätselt woher die kamen. Man konnte sich ja vorstellen, dass da ein Schiff untergegangen war. Aber heute weiß ich, dass es nicht so war. Dort in der Nähe, so 3-4 Kilometer entfernt in
Palmnicken war ein großes Bernsteinwerk (das größte was es gab). Ich nehme auch an, dass die Fischer dort teilweise Steine abgegeben haben. Dort haben die Nazis wohl viele Juden ins Wasser getrieben. So wurde es mir später erzählt.
Ich sehe nur noch vor mir, wie die Leichen aussahen, so aufgedunsen, und wie sie um sich schlugen.

Jedenfalls, mein Bruder wurde noch in einer Kirche konfirmiert. Mein Bruder, ein Cousin und noch ein Cousin. Dort war ein Soldatenpfarrer und hinten lagen noch verwundete Soldaten im Stroh.
Und es hieß dann: "<Wir sind jetzt noch zusammen. Jetzt können wir sie noch konfirmieren lassen."> Da waren also nur wir mit ein paar Angehörigen in der Kirche, die dem beiwohnten.

Die Nahrung bestand zu der Zeit aus den Sachen, die man noch in den Häusern fand, Pferde wurde teilweise geschlachtet und das Fleisch verteilt. Dann wurde mein Vater wieder, wie vor der Flucht, eingezogen
zum Volkssturm und ist nach Wochen in Gefangenschaft geraten und dann in ein Lager in Georgenburg verschleppt worden, wo 8000 Gefangene sich aufhielten.
Während der ganzen Zeit haben wir uns nicht waschen können und die Sachen nicht gewechselt.
Ich selbst hab auf der Flucht Läuse gekriegt. Wir schliefen in übervollen Betten, wie die Heringe nebeneinander. Und da waren ja auch meist noch die Soldaten. 
Eines Tages ging eine Rot-Kreuz-Schwester durch den Ort. Unter anderem kam sie dann zu uns, zu meiner Mutter, und sagte: "<Es geht ein Schiff nach Dänemark, Familien mit Kindern werden bevorzugt. Es ist das letzte Schiff.">
Meine Mutter meinte dazu: "<Mein Mann findet mich dort nie. Ich will nicht nach Dänemark."> Dort war aber auch meine Tante Toni (Antonie) dabei und eine Soldatenkommandantur. Und ein Offizier aus in diesem Raum 
sagte zu meiner Mutter: "<Frau Neumeier, wenn Sie auch nicht dorthin wollen. Um ihrer Kinder willen müssen Sie schon."> Und das war dann auch der Grund, dass wir dann doch fuhren. 

Handgepäck und Rucksack --- das konnte man mitnehmen aufs Schiff, so hieß es. Ja und die Papiere, das Stammbuch und alles waren meiner Mutter unterwegs gestohlen wurden. Jedenfalls unser Weißrusse Leo
brachte uns mit dem Wagen nach Germau zum Bahnhof am 13. April 45. Ich weiß noch wie meine Mutter zu ihm sagte: "<Leo. Möchtest du nicht mitkommen? Du möchtest doch auch mit, oder?">, "<Ja"> sagt er, "<ich habe Angst."> Männer durften ja nicht mit und er
erst recht nicht (weil er Weißrusse war). Er war Anfang 20 und hatte Lehramt studiert.
Es ging außerdem das Gerücht um, dass, wenn der Russe Polen oder Weißrussen unter deutschen Flüchtlingen finden würde, so würde er sie als Kriegsverbrecher deklarieren und erschießen. Es würde als "<deutschfreundliches"> Verhalten
bestraft werden.
Dann waren wir also auf dem Bahnhof mit meiner Tante und ihren drei Söhnen. Und noch das Ehepaar Schablowski mit ihrem Enkelkind. Deren Tochter wollte wohl noch Sachen holen und kam nun nicht zurück.
Da blieben sie bei uns und waren auch mit uns mitgetreckt. Die stammten eigentlich aus dem Ort, wo Schönbergers auch herkamen.
Naja, die Schablowskis waren jetzt in einen Viehwagen gekommen, dort wollten wir auch hinein, sind dann aber in einen Personenwagen, wo weniger Platz war. Der Zug wurde dann beschossen und einer mit einer
braunen Uniform ging von Wagen zu Wagen (Nazi). Und deshalb kamen sie im Tiefflug und beschossen den Wagen, weil sie den Nazi natürlich sehen konnten.
Und wir lagen dann immer flach auf der Erde. Oben war durchgeschossen worden, da splitterten die Scheiben. Und in dem Wagon, wo wir hinein wollten waren drei Tote. Das blieb uns erspart.
Dann wurde die Lok abgehängt. Es war Stille. Dann hieß es, wir sollen alle aussteigen, aber keiner stieg aus. Alles wartete. Es schien uns ja wie Stunden, wir wissen es nicht, dann kam die Lok wieder zurück, wurde wieder angehängt und es 
ging nach Pillau, Hafenstadt Pillau.
Wir stiegen alle aus. Die Stadt brannte. Leute wickelten ihre Toten in irgendetwas ein und gingen zum Schiff, das schon da lag. Männliche Personen ab 15 durften ja nicht aufs Schiff und wenn mein Bruder den Ausweis nicht dabeigehabt
hätte, denn hätte er nicht mitgedurft, er wurde nämlich im Juli fünfzehn Jahre alt. Ich fasste meine jüngste Schwester bei der Hand und dann stiegen wir aufs Schiff. 3000 Menschen waren auf dem Schiff. Es war ein Frachtschiff.
Und wir landeten ganz unten und da saßen wir dann Rücken an Rücken. Schließlich sind wir alle seekrank geworden, haben viel gebrochen. Das war 13. April. Am 15. sind wir dann in Dänemark, Kopenhagen angekommen.
Die See war ein bisschen stürmisch. Es wurde wohl auch mal geschossen auf uns und es rummste einmal heftig, aber sie sagten, wir sollen ruhig bleiben. Dabei hatte ich meine Tasche fallen lassen und traute mich
erst nicht es meiner Mutter zu sagen. Später fanden wir sie dann aber wieder.

Dänemark.
Und am 15. April --- das war ein Sonntagmorgen und die Sonntagsglocken läuteten --- kamen wir in Kopenhagen an.
Ostpreußen --- unsere Heimat --- hatten wir nun hinter uns gelassen, aber die Erinnerung war natürlich immer noch groß. Ich persönlich dachte
daran, als der Krieg begann, als die Wehrmacht durch Ostpreußen 1939 in den Krieg gegen Polen und Russland zog, und auch der Bauernhof meiner Eltern Station war.
Die jungen Soldaten hatten fröhlich und vergnügt ausgesehen und für sie sah es aus, als wenn sie ins Manöver zogen. Den Rückzug hatten wir ja
hautnah erlebt --- wie schon geschildert --- mit der Wehrmacht und den Flüchtlingen. Dort war nur noch Elend gewesen.
Abends, bei einer Nacht-und-Nebel-Aktion wurden wir mit LKWs in das Internierungslager Kasemose bei Melbourne gebracht. Dieses war ehemals ein Erholungsheim für die deutschen
Soldaten gewesen, die ja Dänemark noch besetzt hatten. Zuerst wurden wir dort entlaust, denn wir hatten auf dem Schiff alle Kleiderläuse
bekommen. Denn das Schiff hatte mehrmals verwundete Soldaten von Ostrpeußen nach Kopenhagen gebracht, die an der Ostfront gekämpft hatten.
Dann brach Typhus aus und einige starben, an uns wurden daraufhin mehrere Impfungen durchgeführt. 
Wir hatten noch gut einen Monat unsere Freiheit, bis im Mai die Kapitulation Deutschlands kam.

 Danach wurden wir dann verschifft von Helsingör nach Aalburg. Wir waren dort in der Stadt auf einem Fabrikgelände untergebracht zusammen mit
1000 anderen Menschen. Die Unterkunft bestand aus 3 Blöcken mit einem Stahldraht ringsherum.
Wir schliefen in dreistöckigen Betten mit Strohmatraze und übriggebliebenen Felddecken, die ehemals für die Soldaten gewesen waren.
Von Juli bis Oktober 1945 waren wir in diesem Lager.
 
Anschließend wurden verlegt in ein größeres Lager in Aalburg West, das ehemals ein Flughafen gewesen war --- die Flugzeughallen standen zum Teil noch. Dort fanden 3000 Flüchtlinge eine Bleibe. 
Wir wohnten dort in Holzbaracken, wobei in jeder Baracke ungefähr 200 Personen wohnten. Da gab es die Eckzimmer, in die so um die sechs Personen hineinpassten, und größere Zimmer, in denen ungefähr zehn Personen
Platz fanden. Meine Mutter und wir fünf Kinder bekamen ein Eckzimmer, wo wir in 2 dreistöckigen Betten mit Strohmatratzen und Pferdedecken schliefen. Vorne im Zimmer war ein Schrank, ein kleiner Tisch und eine Bank, sowie rechts ein Kanonenofen, der mit Torf beheizt wurde.
Als Verköstigung gab es Kaltverpflegung, die zweimal die Woche ausgeteilt wurde. Es gab nur eine Gemeinschaftsküche, die für 3000 Personen kochte. In jeder Baracke wurde einzeln Bescheid gegeben, wenn morgens Tee, mittags Essen, und abends nochmals Tee 
geholt werden sollte. 
Wir holten dann das Essen in Kannen.

Also, da dort nun viele Kinder waren, versuchte man ein wenig Ordnung zu schaffen, in dem man eine Art Flüchtlingsschule einrichtete.
Im Lager war ein alter pensionierter Lehrer, der dann die Kinder unterrichtete. Wir 14-Jährigen wurden direkt entlassen, wovon ich hier noch das 
Abschlusszeugnis habe. Genauso hatten wir auch einen Lagerältesten; Barackenälteste wurden gewält. Genauso wurde auch was für die Seele getan --- 
ein Pfarrer Russack wurde uns als Seelsorger zugeteilt. Er war ein Soldatenpfarrer gewesen, der in Stahlingrad einen Lungensteckschuss bekommen hatte, sodass er zum Ausheilen nach Dänemark gekommen war.
Gottesdienst wurde dann immer in einem ehemaligen Kinosaal gefeiert. Dort wurde nachher auch Theater gespielt --- es wurde halt so ein bisschen zum kulturellen Höhepunkt umfunktioniert.
Da wurden auch Gedichte und Lieder gedichtet:
"<In unserem Lager ist es schön, sagt er, ist zusammen, sagt er, groß und klein, sagt er, ist zusammen, sagt er, arm und reich, denn im Flüchtlingslager ist alles gleich.">

Nach einem vierteljährigen Konfirmandenunterricht, dreimal pro Woche, wurden wir --- eine Gruppe von 14 Jugendlichen --- am 2. Juni 1946 konfirmiert.
Tage davor hatten wir eine große Freude, denn wir bekamen die Nachricht, dass unser Vater lebte. Als zuvor die deutschen Soldaten ausgewiesen worden waren, hatten sie uns angeboten, Post in Deutschland aufzugeben.
Meine Mutter hatte sodann einen Brief an meine Tante Käte in Berlin geschrieben, denn wir hatten vereinbart, uns in Berlin wiederzutreffen, falls wir getrennt werden würden.
Diesen Brief hat der Soldat mitgenommen; er ist dann angekommen in Berlin, morgens; und nachmittags kam mein Vater zum ersten Mal dort hin --- zu meiner Tante (Patentante).
Ja und da sagte die Tante noch: "<Das hat doch der Herrgott so gefügt.">
Mein Vater war ja zuvor aus russischer Gefangenschaft entlassen worden, weil er so krank war, ist erst mit dem Transport nach Mecklenburg-Vorpommern gekommen und dann nach der Ausheilung nach Berlin gefahren.
Er hatte damals gesagt: "<Den Russen kenne ich. Jetzt gehe ich nach Westdeutschland.">
Einige Zeilen aus dem Brief meiner Mutter, die sich auf das Lagerleben bezogen: "<Wir säen nicht, wir ernten nicht und unser himmlischer Vater ernährt uns doch.">
Dieses Gottvertrauen hat meine Eltern und uns immer sehr getragen. Mein Vater ging also dann nach Westdeutschland, landete sodann im Weser Bergland, Nähe Golmbach, Kreis Holzminden. Dort suchte er sich Arbeit und
Wohnung für seine Familie, was er bei einem Bauern auch fand --- beides. Es war ja ein 800-Seelen-Dorf  mit 800 Flüchtlingen aus Schlesien. Deshalb gab es keine Wohnungen mehr.
Mein Vater holte sich also eine Zuzugsgenehmigung vom Bürgermeister, die dann nach Dänemark übersandt wurde. Zwei Zimmer standen also für uns bereit. Das war 1946.

Aber die englische Zone nahm zur Zeit keine Flüchtlinge mehr auf, da sie überfüllt war, sodass wir bis Februar 1948 in Dänemark bleiben mussten.
Inzwischen wurde dort im Lager auch mein Bruder Helmut konfirmiert im Februar 1948.
Der Pfarrer war damals der Meinung, dass die Konfirmation lieber noch dort im Lager stattfinden, als dann vergessen werden sollte.
Deshalb zog er sie vor.
Na jedenfalls sind wir dann am 28. Februar über Hamburg in Stadt Oldendorf, Nähe Holzminden, angekommen, wo uns unser Vater mit Pferd und Wagen im Schnee abholte.
Wir sind damals 7 Kilometer zu Fuß im Schnee nach Hause gegangen.

Zuletzt im Lager:
Die dagelassene Kleidung der deutschen Soldaten (nach 1946) wurde von den Frauen umgearbeitet zu Kinderkleidung, mit der Hand.
Außerdem bekamen wir ab und zu Schwedenspenden. Ich bekam zum Beispiel auch eine Bibel vom Schweden gespendet für meinen Unterricht.
Für die Kinder unter 15 Jahren gab es täglich einen halben Liter Milch, aber kein Obst und Gemüse. Es gab einmal Fisch pro Woche, dann Sauerkraut, sowie dunkle Grützsuppe und weiße Grützsuppe.
Sonst aber hatten wir unseren Frieden und wir hatten uns. Das Wichtigste, so sagte meine Mutter, war, dass wir unser Leben gerettet hatten und zusammen geblieben waren.

Eine Sache, an die ich mich auch noch erinnere: Man machte aus Zaunholz Schuhsolen, dann den Schuh aus Soldatenkleidung.


\paragraph{Flucht des Vaters beim ersten Weltkrieg 1914} Da das Heimatdorf meines Vaters Dubeningken, Kreis Goldap, war --- was Grenzgebiet war --- ist der Russe schon im Sommer 1914 dort eingefallen. Nach Erzählungen meines Vaters sind sie damals in die Wälder geflüchtet und als die Soldaten ins Dorf kamen, haben sie schon die Frauen in den Kellern versteckt.
Dann kam die Parole heraus, "<Es geht nach dem Westen">, das Dorf wurde evakuiert und die Bewohner sollten also Richtung Westen fahren --- bis Westpreußen sind sie gekommen. Dort haben sie Pferd und Wagen stehen lassen und sind mit der Eisenbahn nach Kamen in Westfahlen gefahren, wo sie ein ganzes Jahr verbrachten.
Sie sind da auch zur Schule gegangen und mein Vater war immer so stolz darauf, dass sie dem Stoff, den sie hier behandelten, schon ein Jahr voraus waren.
Danach ging es wieder heimwärts --- aus Westpreußen holten sie ihr Gepferd ab und fuhren dann ganz nach Hause. Mein Vater war damals 13 oder 14 Jahre alt.
Dann war Ruhe im Grenzbegiet bis zum Rückzug des zweiten Weltkrieges.

\paragraph{Meine Geschwister und Familien, heute 13.03.2014} Fritz Neumeier verheiratet mit Malies Neumeier (geborene Schewe), Tochter Friederieke mit Kindern Jiad und Amin.
Helmut Neumeier inzwischen verstorben, verheiratet mit Lieselotte Neumeier, geborene Kipnick, Söhne Siegfried und Hartmut Neumeier, verheiratet mit Iris, Kinder Benjamin und Julia.
Dora Albrecht, geborene Neumeier, verheiratet mit Wilfried Albrecht, Sohn Heino Albrecht, verheiratet mit Siegried.
Eva Schaper, geborene Neumeier, verheiratet mit Herbert Schaper (inzwischen auch verstorben), Sohn Bernd Schaper verheiratet mit Barbara, Kinder Anna und Julia.

Mein lieber Mann Karl Halbach, geboren am 24.08.1935 in Zelle, verheiratet seit 07.11.1958.

\paragraph{Unsere Kinder} Karl-Friedrich Halbach, geboren am 13.10.1959, verheiratet mit Eveline Halbach geborene Klett, Kinder Sarah (05.03.1991) und Maik (24.08.1994).
Doris Schneider, geboren am 06.02.1964, verheiratet mit Torsten Schneider geboren am 28.05.1961, Kinder Paul (26.09.1991), Jakob (26.01.1994), Hannah (23.11.1996), Jonathan (13.08.2001).  
\end{document}