\part{Grundlagen}

\section{Verbände}

\begin{definition}
    Die Kategorie der Verbände
\end{definition}

\section{Kategorien}

In diesem Abschnitt werden die nötigen kategorientheoretischen Kenntnisse (bzw.~Terminologie) bereitgestellt, welche für die
Gruppentheorie (von einem modernen Standpunkt aus) unentbehrlich sind.

\subsection{Terminale und finale Objekte}

\begin{definition}[initiale und terminale Objekte]
    Sei $\bfA$ eine Kategorie. Dann heißt ein Objekt $T:\bfA$ \keyword{terminal}\as{{terminales Objekt}{koinitiales Objekt}}, falls es für jedes andere Objekt $O:\bfA$ genau
    einen Morphismus $\alpha:O\to T$ gibt. Mit anderen Worten: $T$ ist maximal bezüglich der transitiven Relation $\to$.
    In analoger Weise heißt $I$ ein \keyword{initiales}\as{{initiales Objekt}{koterminales Objekt}}, falls es für jedes $O:\bfA$
    genau einen Morphismus $\beta:I\to O$ gibt. Mit anderen Worten: $I$ ist minimal bezüglich der transitiven Relation $\to$. 
\end{definition}

\begin{definition}[Nullobjekt]
    Ein Objekt $0:\bfA$ heißt \keyword{Nullobjekt}\as{{Nullobjekt}}, falls $0$ sowohl initial als auch terminal ist.
\end{definition}

\subsection{Initiale und terminale Pfeile und Nullpfeile}

\begin{definition}[Initiale und terminale Pfeile]
    Ein Pfeil $\gamma:A\to B$ heißt \keyword{terminaler Morphismus}\as{{terminaler Morphismus}{konstanter Morphismus}}, falls $\alpha\gamma=\beta\gamma$
    für alle $\alpha,\beta:\to A$. Analog heißt $\gamma$ \keyword{initialer Morphismus}\as{{initialer Morphismus}{kokonstanter Morphismus}}, falls
    $\gamma\alpha=\gamma\beta$ für $\alpha,\beta:B\to$ (d.h. $\dual\gamma$ ist konstant in $\dual\bfA$).
\end{definition}

\begin{remark}
    Initiale und terminale Pfeile sind genau die initialen und terminalen Objekte in der Morphismenkategorie von $\bfA$. 
\end{remark}

\begin{definition}[Nullpfeil]
    Ein Pfeil $0:A\to B$ heißt \keyword{Nullmorphismus}\as{{Nullmorphismus}}, falls er konstant und kokonstant zugleich ist.
\end{definition}

\begin{remark}
    Nullpfeile sind genau die Nullobjekte in der Morphismenkategorie von $\bfA$.
\end{remark}

\begin{remark}
    Gibt es ein Nullobjekt $0:\bfA$, so auch ein kanonischen Nullmorphismus zwischen Objekten $A,B:\bfA$ via $A\to 0\to B$, wobei
    die beiden Morphismen aufgrund der Nullobjekteigenschaft schon eindeutig sind. 
\end{remark}


\subsection{Kerne und Kokerne}

\begin{definition}[Kern und Kokern]
Sei $\alpha:A\to_\bfA B$ ein Morphismus. Dann wird ist der \keyword{Kern}\as{{Kern eines Morphismus}} $\ker\phi$ das Unterobjekt mit der universellen Eigenschaft, dass jedes
Unterobjekt $U$ von $A$ gilt, dass $U\alpha=\interpret 0\in{\subobjects A}$, dann gilt $\ker\phi\leq U$.
In analoger Weise definieren wir den \keyword{Kokern}\as{{Kokern}} $\cokernel \alpha$ von $\alpha$ als das Quotientenbojekt $Q$ von $B$, welches die
Eigenschaft hat, dass $\alpha Q=\interpret 0\in{\cosubobjects A}$ gilt $\cokernel \alpha\leq Q$.
\end{definition}

\subsection{Unterobjekte und Quotientenobjekte}

\begin{definition}
    Sei $\bfA$ eine Kategorie. Der Verband der \keyword{Unterobjekte}\as{{Unterobjekt}} $\subobjects A$ für jedes Objekt $A:\bfA$ als die
    Isomorphieklassen der Kategorie $\monoto_\bfA A$ (also $\monoto_\bfA A/\isoto$).
    Analog definieren wir den Verband der \keyword{Quotientenobjekte}\as{{Quotientenobjekt}{Kounterobjekt}} von $A$ als $\cosubobjects A$ durch die Isomorphieklassen von $A\epito_\bfA/\isoto$.
\end{definition}

\begin{remark}
    Die beiden Konzepte sind also genau dual zueinander.
\end{remark}

\subsection{Normale Unterobjekte und konormale Quotientenobjekte}

\begin{definition}[Normales Unterobjekt und konormales Quotientenobjekt]
        Ein Unterobjekt $N$ von $A$ heißt \keyword{normal}\as{{normales Unterobjekt}}, falls es einen Morphismus $\alpha:A\to$ gibt, sodass $N=\ker\alpha$.
        Ein Quotientenobjekt $Q$ heißt \keyword{konormal}\as{{konormales Quotientenobjekt}}, falls es einen Morphismus $\beta:\to\alpha$ gibt mit $\cokernel \beta=Q$.
\end{definition}

\subsection{Bilder}

\begin{definition}
    Sei $\alpha:A\to B$ ein Morphismus, dann bezeichnet $image \alpha$ das induzierte Unterobjekt von $\alpha$.
\end{definition}

\subsection{Normale Morphismen}

\begin{definition}
    Ein Morphismus $\alpha:A\interpret\to\in\bfA B$ heißt \keyword{normal}\as{{normaler Morphismus}}, falls $\image\alpha$ ein
    normales Unterobjekt von $B$ ist.
\end{definition}

\subsection{Morphiesätze}

\begin{theorem}
    Sei $\bfA$ eine Kategorie mit Bildern und Kernen. Sei $\phi:A\interpret\to\in\bfA B$ ein Morphismus. Dann gibt es ein Objekt
    $C$ mit $\pi:A\epito C$, $\iota:C\monoto B$, sodass $\phi = \pi\iota$. 
\end{theorem}

\begin{proof}
    
\end{proof}

\section{Produkte und Koprodukte}

Sei $\bfB\monoto\bfA$ eine Unterkategorie. 


\section{Gruppenaxiome}

Unter einer Gruppe verstehen wir eine Struktur vom Typ $\Grp=\struct{\compose,^{-1},1}$, derart dass folgende Identitäten gelten
\begin{itemize}
        \item $(a\compose b)\compose c = a\compose (b\compose c)$ (Assoziativität)
        \item $a^{-1}\compose a = a\compose a^{-1} = 1$ (Inversenabblildung)
    \item $a\compose 1 = 1\compose a = a$ (neutrales Element)
\end{itemize}

\begin{theorem}
    Hallo
\end{theorem}

\section{Aufsteigende und absteigende Kettenbedingung}

\begin{definition}
    Sei $P:\poset$. Dann genügt $P$ der \keyword{aufsteigenden Kettenbedingung}\as{{aufsteigende
            Kettenbedingung}{\person{Noether}'sche Eigenschaft}}, wenn jede aufsteigende Kette nach endlich vielen Gliedern
    abbricht. Analog genügt $P$ der \keyword{absteigenden Kettenbedingung}\as{{absteigende Kettenbedingung}{\person{Artin}'sche Eigenschaft}} 
\end{definition}

\begin{theorem}[Charakterisierung von endlicher Erzeugbarkeit]
    Sei $A$ eine Algebra. Die folgenden Aussagen sind äquivalent
    \begin{statements}
            \item $A$ ist endlich erzeugt.
            \item $\subobjects A$ genügt der aufsteigenden Kettenbedingung.
            \item Jedes Unterobjekt $U:\subobjects A$, $U\neq A$ liegt in einem maximalen Unterobjekt.
    \end{statements}
\end{theorem}

\begin{proof}
    \begin{implications}
            \item Ist $A$ endlich erzeugt und $C$ eine aufsteigende Kette in $\subobjects A$, dann wird $C$ stationär.
            \item
        \item blub
    \end{implications}
\end{proof}

\section{Die \person{Sylow}'schnen Sätze}

Eine natürliche Frage, welche sich aus dem Theorem von \person{Lagrange} ergibt, welche Aussagen über die Anzahl und Art der
Untergruppen von Ordnung $n$ einer endlichen Gruppe $G$ getroffen werden können.
Für $n\not\divides \size G$ ist selbige Anzahl nach dem Theorem von \person{Lagrange} (TODO: REF) gleich null. Ist $G$ zyklisch, so
ist jene Anzahl im Falle $n\divides \size G$ genau eins. Tatsächlich muss es aber für $n\divides \size G$ keine Untergruppen dieser
Ordnung geben, was man am leichtesten an der symmetrischen Gruppe $\Aut m$ sieht, denn wählt man nun $n$ als eine Zyklizität
erzwingende Zahl, sodass $m<n\divides m!$, dann gibt es offensichtlich keine Untergruppen von $\Aut m$ dieser Ordnung.
ist es ob bei einer endlichen Gruppe $G$ der

Tatsächlich lassen sich aber befriedigende Aussagen treffen, falls $n=p^e$ die Potenz einer Primzahl $p$ ist. Diese werden
gemeinhin als \person{Sylow}'sche Sätze bezeichnet.

\begin{definition}[$p$-Gruppe]
    Sei $G$ eine Gruppe derart, dass jedes Element $g\in G$ eine Primzahlpotenz $p^{e_g}$ als Ordnung hat (wobei $p$ eine feste
    Primzahl sei.
\end{definition}

\begin{remark}
    Eine triviale Konsequenz der \person{Sylow}'schen Theoreme wird es sein, dass jede endliche $p$-Gruppe selbst von
    Primzahlpotenzordnung $p^e$ ist.
\end{remark}
%
\begin{theorem}[Existenz von $p$-Untergruppen jeder Ordnung]
    Sei $G$ eine endliche Gruppe mit $\size G = p^e n$. Für die Anzahl $N_{p^e}\defeq \size{\hull{U\leq G:\size U=p^e\in\set}}$ gilt dann
    $$
    N_{p^e} = 1 \mod p
    $$
\end{theorem}
%
\begin{proof}
    Wir betrachten die Aktion von $G$ auf den $p^e$-elementigen Untermengen von $\cast G \to \set$ welche gegeben wird durch
    elementweise Rechtsmultiplikation. Die Bahnengleichung für diese Aktion wird dann zu
    $$
    \size {\binom {\cast G \to \set} {p^e}} = \sum_i {\size {G/\stab A_i}},
    $$
    wobei $A_i$ Repräsentanten der $G$-Bahnen sind.
    Für $A_i\monoto \cast G \to \set$, $\size {A_i}=p^e$ gilt allerdings dann $A_i \cast (\stab A_i) \to \set=A_i$, also ist $A_i$ eine disjunkte
    Vereinigung von Linksnebenklassen von $\stab A_i$ und mithin $\size {\stab A_i}\divides p^e$. Betrachten wir also obige
    gleichung modulo $pn$, so folgt
    $$
    \binom {p^e n}{p^e} = n N_{p^e} \mod pn, 
    $$
    denn alle Terme, in denen $\stab A_i<p^e$ ist in obiger Gleichung entfallen und die übrigen Terme zählen genau für jede
    $p^e$-elementige Untergruppe von $G$ ihre Linksnebenklassen (derer gibt es $n$).
    Daraus folgt
    $$
    \frac 1 n \binom {p^e n}{p^e} = \binom {p^e n-1} {p^e-1} = N_{p^e} \mod p,
    $$
    wobei der Ausdruck auf der Linken Seite gleich 1 ist modulo $p$. Dies sieht man einerseits daran, dass dies für die zyklische
    Gruppe mit $p^e n$ Elementen gilt, andererseits lässt sich auch das Theorem von \person{Lucas} (TODO : REF) auf den letzten
    Binomialkoeffizienten anwenden. Wir erhalten dann
    $$
    N_{p^e} = \binom {p^e n - 1} {p^e -1} = {\binom {p-1} {p-1}}^e = 1 \mod p.
    $$
\end{proof}

{\bfseries\scshape Text}

\begin{theorem}
    Jede endliche Gruppe $G$ hat $p$-\person{Sylow}-Gruppen. Für jede $p$-Untergruppe $U$ und eine $p$-\person{Sylow}-Gruppe von
    $G$ gibt es ein Element $g\in G$, sodass $U\monoto P^g$. Insbesondere sind alle $p$-\person{Sylow}-Gruppen konjugiert zueinander
    und ihre Anzahl ist $\size {G/N_G P}$. 
\end{theorem}

\section{Die Sätze von \person{Hall}}%

Die Sätze von \person{Hall} stellen eine Verallgemeinerung der \person{Sylow}'schen Sätze für auflösbare Gruppen dar. Entsprechende
Untergruppen nennt man auch \person{Hall}-Untergruppen.

\begin{theorem}[\person{Hall}'sches Theorem]
    Sei $G$ auflösbar und $\size G = mn$ mit teilerfremden $m$ und $n$. Dann gilt
    \begin{statements}
        \item Sei $U$ eine Untergruppe mit $\size U \divides m$ und $M$ eine Untergruppe der Ordnung $m$, dann gibt es ein $g\in G$
    sodass $U\leq M^g$.
        \item Für die Anzahl der Untergruppen der Ordnung $m$ von $G$ gilt:
    $$
    N_m = 1 \mod \rad m
    $$
    \end{statements}
\end{theorem}

\begin{proof}
    Der Beweis erfolgt per Induktion nach der Anzahl $k$ der Primfaktoren von $m$. Für $k=1$ gilt die Aussage schlicht aufgrund der
    \person{Sylow}'schen Sätze auch ohne Auflösbarkeit von $G$.
    
\end{proof}

\begin{theorem}[\person{Frattini}-Argument]
    Sie $G$ eine Gruppe und $N\leq_{\normalsubobjects G} G$. Weiter sei $P$ eine Untergruppe von $N$ derart, dass alle zu $P$ isomorphen
    Untergruppen in $H$ konjugiert sind (also z.B.~$P$ eine $p$-\person{Sylow}-Gruppe). Dann gilt
    $G = \normalizer_G P N$.
\end{theorem}

\begin{proof}
    Für $g\in G$ ist $P^g$ isomorph zu $P$ und gleichermaßen Untergruppe von $N$, da $N$ normal in $G$ liegt. Also sind $P$ und $P^g$ in $N$
    konjugiert und es folgt $P^{gn}=P$ für geignetes $n\in N$. Damit ist aber $gn\in \normalizer_G P$ und somit auch $g\in N
    \normalizer_G P$.
\end{proof}

\section{Auflösbarkeit von Gruppen}

\begin{definition}[Subnormalenverband, Subnormalenreihe]
    Ein Unterverband $\bfU$ von $\subobjects G$ Subnormalenverband, falls für alle $U\in \bfU$ gilt
    $U\in\normalsubobjects \Meet_{V>U}V$. Ist ein solcher Verband isomorph zu einem Unterverband von $\cast \nats \to \lattice$, so nennen wir Ihn
    eine Subnormalenreihe. 
\end{definition}

\begin{definition}[Auflösbare Gruppe]
    Eine Gruppe $G$ heißt \keyword{auflösbar}\as{{auflösbare Gruppe}}, falls es einen Subnormalenverband von $G$ gibt, derart, dass
    $$
    \Meet_{V>U} V / U\ \textrm{kommutativ}
    $$
    für alle $U\in\bf U$. \keyword{endlich auflösbar}\as{{endliche auflösbare Gruppe}}\footnote{Dies meint auflösbar im herkömmlichen Sinne.}
\end{definition}

\begin{definition}[Kommutatoruntergruppe]
    Seien $\bfU\leq\cast \subobjects G\to \set$. Dann bezeichnen wir mit $\commutatorsubgroup \bfA$ die
    \keyword{Kommutatoruntergruppe}\as{{Kommutatorgruppe}} von der Gruppen in $\bfA$. Sie wird erzeugt durch alle Kommutatoren $\commutator a b$ für $a\in
    A$, $b\in B$, $A,B\in\bfA$. 
\end{definition}

\begin{lemma}
    Die Kommutatoruntergruppe $\commutatorsubgroup \bfA$ ist charakterisiert durch folgende univerelle Eigenschaft.
    Sei $N$ ein Normalteiler von $\hull\bfA$ derart, dass unter der kanonischen Projektion $\pi:\hull\bfA\to\hull\bfA/N$ die
    Bilder der Elemente von $\bfA$ paarweise elementweise kommutieren, d.h. $\commutatorsubgroup
    {\image{\iota_A\pi},\image{\iota_B\pi}}=1$, wobei $\iota_A$ bzw. $\iota_B$ die Inklusionen von Untergruppen $A,B\in\bfA$ sind,
    dann ist $\commutatorsubgroup \bfA\leq N$ und in äquivalenterweise gibt es einen eindeutigen Morphismus $\pi'$, sodass
    $\pi=p\pi'$ mit $p:\hull\bfA\to\hull\bfA/N$ die kanonische Abbildung.
\end{lemma}

\begin{proof}
    Der Beweis folgt aus dem allgemeineren Kontext (TODO).
\end{proof}

\begin{definition}[Kommutatorreihe]
    Wir definieren die \keyword{Kommutatorreihe}\as{{Kommutatorreihe}} $\sequence{\commutatorseries G i}_{i\in \nats}$ einer Gruppe $G$ als
    $$\commutatorseries G 0 \defeq G,\ \commutatorseries G {i+1} \defeq \commutatorsubgroup {\commutatorseries G i,\commutatorseries G i} \ (i\in\nats).$$
    Weiterhin setzen wir $\commutatorseries G \omega \defeq \Meet_{i\in\nats} \commutatorseries G i$ und bezeichnen es als
    \keyword{perfekten Kern}\as{{perfekter Kern}} von $G$.
\end{definition}

\begin{lemma}
    Eine Gruppe $G$ ist genau dann endlich auflösbar, falls ihre Kommutatorreihe nach endlich vielen Schritten in $1$ endet.
    Sie ist genau dann $\omega$-auflösbar, falls ihr perfekter Kern gleich $1$ ist.
\end{lemma}

\begin{proof}
    Sei $\bfG$
\end{proof}

\section{Der Satz von \person{Jordan}-\person{Hölder}}

\subsection{Das \person{Zassenhaus}-Theorem (Schmetterlings-Theorem)}

\begin{theorem}[Schmetterlingslemma]
    Sei $A\interpret\leq\in{\normalsubobjects G} B$ und $C\interpret\leq\in{\normalsubobjects G} D$, dann gilt
    $$
    \quotientobject(B\meet D)\join A\over{(B\meet C)\join A}
    \isoto
    \quotientobject(B\meet D)\over{(B\meet C)\join (A\meet D)} 
    \isoto
    \quotientobject(B\meet D)\join C\over{(A\meet D)\join C}.
    $$
\end{theorem}

\begin{proof}
    
\end{proof}


%%% Local Variables:
%%% coding: utf-8
%%% TeX-engine: xetex
%%% TeX-master: "Script-Main"
%%% End: 
