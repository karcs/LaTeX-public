\section{Transportoptimierung}
%% bug : footnote cannot be made in section header
    \subsection{Aufgabenstellung --- das Standard-TOP}%\footnote{steht hier for Transportoptimierungsproblem}

    \begin{definition}
        Gegeben sind $r$ Betriebe (Produzenten) $A_1,\ldots,A_r$, die ein gleiches Gut produzieren.
        (Kapazitäten $\alpha_1,\ldots,\alpha_r\geq 0$), $s$ Verbraucher $B_1,\ldots,B_s$ mit Bedarf $\beta_1\ldots,\beta_s$.
        Als mathematisches Modell ergibt sich mithin:
        Seien $c_{ik}$ die Transportkosten (je Einheit) und $x_{ik}$ die Mengen (die von $A_i$ nach $B_k$ gehen), dann ist 
        $$
        c=\sum_{i=1}^r\sum_{k=1}^s{c_{ik}x_{ik}}\to \min\ \textrm{bei}\ x_{ik}\geq 0\label{stdtop}
        $$
        zu optimieren mit den Bedingungen
        $$
        \sum_{k=1}^s{x_{ik}}=\alpha_i
        $$
        für $i=1,\ldots,r$, sowie
        $$
        \sum_{i=1}^r{x_{ik}}=\beta_k
        $$
        für $k=1,\ldots,s$.
    \end{definition}

    \begin{remark}
        In praktischen Aufgaben gilt $\alpha_i,\beta_k>0$ bzgl. der Zielfunktionskoeffizienten gilt
        $c_{ik}\geq 0$ wird nicht gefordert. Auch möglich: $\sum_{c_{ik}x_{ik}}\to \max$ (aus Sicht der Transportunternehmens.
    \end{remark}

    Dies ist klarerweise eine LOA\footnote{lineare Optimierungsaufgabe} spezieller Struktur (!).
    Dabei sei $A$ die Matrix $r s x(r+s)$
    $$
    x = (x_{ik}),\ c=(c_{ik}),\ b=(\alpha_1,\ldots,\alpha_r,\beta_1,\ldots,\beta_r)
    $$
    $a^{ij}$ Spaltenvektor $
    \begin{pmatrix}
        e_i\\
        e_j
    \end{pmatrix}
    \in\reals^r\settimes\reals^s$, 
    $$
    c^\top x\to\min\ \textrm{bei}\ Ax=b,\ x\geq 0,\label{stdtopgnf}
    $$
    was die LOA in Gleichungsnormalform darstellt.

    \begin{theorem}\label{top-lbkt}
        Das Transportproblem \autoref{stdtop} ist genau dann lösbar, wenn gilt
        $$
        \sum_{i=1}^r{\alpha_i}=\sum_{k=1}^s{\beta_k} \label{sbed},
        $$
        was man auch als \keyword{Sättigungsbedingung} bezeichnet.
    \end{theorem}

    \begin{proof}
        Wir zeigen zunächst $G\neq \emptyset$ genau dann, wenn die Sättigungsbedingung \autoref{sbed} gilt
        \begin{implications}
                \item[$\implies$:] Für $x\in G$ folgt:
            $$
            \sum_{i=1}^r{\alpha_i}=\sum_{i=1}^r\sum_{k=1}^s{x_{ik}}=\sum_{k=1}^s\sum_{i=1}^r{x_{ik}}=\sum_{k=1}^s\beta_k
            $$
                    \item[$\impliedby$:] Definieren wir $\tilde x$ mit $\tilde x_{ik}:=\alpha_i\beta_k/\sigma$ mit $\sigma:=\sum_{i=1}^r{\alpha_i}=\sum_{i=1}^s{\beta_i}$. Offenbar gilt dann $\tilde x\in G$.
        \end{implications}
    
        Weiterhin ist die zulässige Menge polyedrisch, insbesondere abgeschlossen. $G$ ist aber auch beschränkt, denn $0\leq x_{ik}\leq \min\set{\alpha_i,\beta_i}$ --- also kompakt.

        Die Zielfunktion $x\mapsto c^\top x$ ist nun stetig, also nimmt sie auf der zulässigen Menge ihr Minimum an (Satz von \person{Weierstraß}).
    \end{proof}

    Die duale LOA zu \autoref{stdtopgnf} bzw. \autoref{stdtop} lautet nun $b^\top \to \max$ bei $u\in\reals^{r\cotimes s}$ (`frei').
    Dabei setzen wir $u^\top = (u_1,\ldots,u_r,v_1,\ldots,v_s)$, $u_i+v_j\leq c_{ij}$,
    $A^{\top}_{ij}=(e^\top_i e^\top_j)$.

    \begin{theorem}[Optimalitätskriterium]
        Sei $x\in G$ ($Ax=b$, $x\geq 0$). Dann sind äquivalent
        \begin{statements}
                \item $x$ ist optimal.
            \item $\exists u_1,\ldots,u_r,v_1,\ldots,v_s$ mit $u_i+v_k-c_{ik}\leq 0$, $x_{ik}(u_i+v_k-c_{ik})=0$ für alle $i,j$.
        \end{statements}
    \end{theorem}

    \begin{proof}
        Sei $x$ optimal fpr \autoref{stdtop} und $u$ optimal für das duale Problem. Mit \autoref{top-lbkt} ist dies äquivalent zu $c^\top x=b^\top u$, was äquivalent zu $x^\top(c-A^\top u)=0$ und $A^\top u\leq c$, das ist die Behauptung.
    \end{proof}

    \begin{theorem}
        Jedes Simplexschema zu \autoref{stdtopgnf} hat genau $r+s-1$ Basisvariable.
    \end{theorem}

    \begin{proof}
        Zunächst $\rg A\leq r+s-1
        $. Dies ist unmittelbar klar, denn $\sum_{i,k}{x_{ik}}=\sum_{k=1}^s\sum_{i=1}^r{x_{ik}}=\sum_{i=1}^r\sum_{k=1}^s{x_{ik}}$ (damit muss der Defekt mindestens 1 sein). Nun noch $\rg A\geq r+s-1$, denn es lässt sich eine reguläre Untermatrix dieser Größe finden: Wähle einfach die Spalten $a_{11},\ldots,a_{1s}, a_{21},\ldots,a_{r1}$ stellen sich als linear unabhängig heraus.

        \hfill{16.06.2014}
        
    \end{proof}

    \begin{corollary}
        Jede Ecke hat maximal $r+s-1$ positive Komponenten.
    \end{corollary}

    \begin{definition}[Zyklus]
        Eine Folge von $2l$ Zellen $(i_1,k_1),(i_1,k_2),(i_2,k_2),\ldots,(i_l,k_1)$ mit $i_\nu\neq i_\mu$, $k_\nu\neq k_\mu$ heißt Zyklus.
    \end{definition}

    \begin{example}
        Es ist
        $$
        (3,1), (3,5), (2,5), (2,2), (4,2), (4,3), (1,3), (1,3), (1,1)
        $$
        ein Zyklus.
    \end{example}

    \begin{lemma}[Existenz von Zyklen]\label{top-zyklen}
        \begin{statements}
                \item Sei $\cJ$ eine Menge von Zellen. Falls $\card\cJ\geq r+s$, so enthält $\cJ$ einen Zyklus.
                \item Sei $x=(x_{ik})$ ein zulässiger Transportplan. $x$ ist genau dann eine Ecke von $G$, wenn $\cJ_+:=\set{(i,k):x_{ik}>0}$ keinen Zyklus enthält.
        \end{statements}
    \end{lemma}

    \begin{proof}
        Übung.
    \end{proof}

    \begin{theorem}
        Es seien $u_i$ ($i=1,\ldots,r$) $v_j$ ($j=1,\ldots,s$) beliebige Zahlen. Es sei eine weitere Zielfunktion gegeben:
        $$
        \bar z = \bar z (x) = \sum\sum{\bar c_{ik}x_{ik}}
        $$
        mit $\bar c_{ik}=c_{ik}-(u_i+v_k)$. Dann gilt für alle $x\in G$.
        $\bar z (x)=z(x) - w$ mit $w=\sum_{i=1}^r{\alpha_i u_i}+\sum_{k=1}^s{\beta_k v_k}$.
    \end{theorem}

    \subsection{Erzeugen eines ersten Simplexschemas}

    Es gibt drei `Grundregeln':
    \begin{enumerate}
            \item Markieren eines (Tableau-)Elementes und maximale Belegung $\bar x_{ik}=\min\set{\alpha_i,\beta_k}$.
            \item Korrigieren von $\alpha_i$ und $\beta_k$:
        $$ \tilde \alpha_i = \alpha_i-\bar x_{ik}, $$
        $$ \tilde \beta_k = \beta_k - \bar x_{ik}. $$
            \item Streichung von `abgesättigten' Zeilen bzw.\ Spalten ($\tilde \alpha_i=0$ bzw. $\tilde \beta_k=0$). 
    \end{enumerate}
    Methoden unterscheiden sich nur in der Auswahl der ersten der obigen Regel.

    \begin{description}%% buggy : fix this
            \item[Nord-West-Ecken-Regel:] Markierung des linken oberen Eintrags.
            \item[Minimale-Kosten-Regel:] Auswahl $(\bar i,\bar k)$ mit $c_{\bar i,\bar k}=\min_{i,k}{c_{i,k}}$.
            \item[Methode von Vogel:] Bestimme in jeder Zeile und Spalte die Differenz zwischen den beiden kleinsten Elementen. Belege diejenige Zeile bzw.\ Spalte, in denen diese Differenz maximal wird.
    \end{description}

    \begin{remark}
        Falls weniger als $r+s-1$ Zellen mit $v_{ik}>0$ belegt werden, so `Auffüllen' mit `0', dabei ist auf Zyklenfreiheit zu achten.
    \end{remark}

    \paragraph{Abarbeiten im `Rechteckschema':}
    Daten sind in einer rechteckigen Anordnung gegeben.
    $$
    \begin{matrix}
        R & \beta_1 & \cdots & \beta_s\\
        \midrule
        \alpha_1 & c_{11} & \cdots & c_{1s}\\
        \vdots & \vdots & \ddots & \vdots\\
        \alpha_r & c_{r1} & \cdots & c_{rs}
    \end{matrix}
    $$

    \begin{example}[Nord-West-Ecken-Regel, Minimale Kosten]
        Gegeben das Schema
        $$
        \begin{matrix}
            R & 12 & 5 & 6 & 7 & 7\\
            \midrule
            4 & 12 & 6 & 10 & 9 & 5\\
            19 & 10 & 16 & 17 & 3 & 7\\
            14 & 4 & 11 & 5 & 8 & 10
        \end{matrix}
        $$
        NW-Ecken-Regel liefert:
        $$
        \begin{matrix}
            NW & 12 & 5 & 6 & 7 & 7\\
            4 & 4 & \ast & \ast & \ast & \ast\\
            19 & 8 & 5 & 6 & 0 & \ast\\
            14 & \ast & \ast & \ast & 7 & 7
        \end{matrix}
        $$
        Die $0$ entsteht hier gemäß der Bemerkung.
        Minimale Kosten:
        $$
        \begin{matrix}
            MK & 12 & 5 & 6 & 7 & 7\\
            4 & \ast & \ast & \ast & \ast & 4\\
            19 & \ast & 5 & 4 & 7 & 3 \\
            14 & 12 & \ast & 2 & \ast & \ast
        \end{matrix}
        $$
    \end{example}

    Komplementäres Schema ($d_{ik}=c_{ik}-(u_i+v_k)$):
    $$
    \begin{matrix}
        T & \bar v_1 & \cdots & \bar v_s\\
        \bar u_1 & x_{11} & d_{ik} & \cdots & \\
        \vdots & x_{12} & \\
        \bar u_r & 
    \end{matrix}
    $$
    \subsection{Das Simplexverfahren für TOP}

    Verfahren zur Bestimmung eines optimalen Transportplans: aktuelle Basislösung (Startlösung).

    \begin{enumerate}
            \item Sei $\cJ_B$ die Indexmenge der Basisvariablen (`Zellen'). Bestimmen von $u_i$, $v_k$ (Optimalitätsindikatoren) aus $u_i+v_k-c_{ik}=0$ ($(i,k)\in \cJ_B$).
        Wegen $\card\cJ=r+s-1$ (und $\rg X=r+s-1$). Das System ist also unterbestimmt.
        Setzen von $u_1=0$ (z.B. eine Variable beliebig wählbar). Die Lösung kann also sukzessive explizit bestimmt werden.
            \item Setze (für Nicht-Basis-Variablen) $d_{ik}=c_{ik}-u_i-v_k$. Falls gilt $d_{ik}\geq 0$ für alle $i,k$, dann ist die aktuelle Basislösung optimal.
        \item Falls die Basislösung nicht optimal ist (letzte Bedingungen nicht erfüllt), so ermittle Indexpaar $(p,q)$ mit $d_{pq}<0$ (mit $d_{pq}=\min_{i,j}{d_{ij}}$). Bestimme einen Zyklus in $\cJ_B\setjoin\set{(p,q)}$ (immer möglich wegen \autoref{top-zyklen}). Markiere die Indexpaare des Zyklus ausgehend von $(p,q)$ mit `$+$' und `$-$' ($(i,k)\in\cJ_-,\cJ_+$). Bestimme $\hat t:= x_{\bar p \bar q}=\min\set{x_{ik}:(i,k)\in\cJ_-}$. Setze $\tilde x_{pq}=x_{pq}$, $\tilde x_{ik}=x_{ik}+\hat t$, ($(i,k)\in\cJ_+$), $\tilde x_{ik} = x_{ik}-\hat t$ ($(i,k)\in \cJ_-$) und $\cJ_{\tilde B}=\cJ_B\setminus\set{(\bar p,\bar q)}\setjoin{p,q}$
    \end{enumerate}
    \begin{example}
            Daten seien:
            $$
            \begin{matrix}
                R & 3 & 4 & 2 & 6 \\
                \midrule
                5 & 2 & 5 & 1 & 3 \\
                7 & 2 & 6 & 4 & 1 \\
                3 & 1 & 3 & 1 & 5
            \end{matrix}
            $$
            Man erhält mit der Nord-West-Ecken-Regel (Basiselemente in dieser Reihenfolge $(i)$ selektiert).
            $$
            \begin{matrix}
                T0 & 2 & 5 & 3 & 0 \\
                \midrule
                0 & 3 (1) & 2 (2) & -2 & 3 & \\
                1 & -1 & 2 (3) & 2 (4) & 3 (5) & \\
                5 & -6 & -7 & \rightarrow -7 & 3 (6)
            \end{matrix}
            $$
            ($z=54$), ($\hat t=2$)
            Der Zyklus ist das $2\settimes 2$-Quadrat unten rechts. Also $(p,q)=(3,3)$.
            Damit ist der Zyklus $(3,3)+,(3,4)-,(2,4)+,(2,3)-$.
            $$
            \begin{matrix}
                T1 & 2 & 5 & -4 & 0\\
                \midrule
                0 & 3(1) & 2(2) & 5 & 3\\
                1 & -1  & 2(3) & 7 & \rightarrow 5 \\
                5 & -6 & -7(4) & 2(5) & 1(2) 
                
            \end{matrix}
            $$
            ($z=40$)
        \end{example}

        \begin{remark}[Optimalitätstest]
            Die reduzierten Kosten $d_{ik}=c_{ik}-u_i-v_k$ sind unabhängig von der speziellen Wahl der Lösung in (*) (insbesondere unabhängig von der Wahl $u_1=0$)
            \begin{proof}
                $u_i,v_k$ Lösung von (*) ist gleichbedeutend mit $\tilde u_i= u_i+s$, $\tilde v_k = v_k -s$ ($s\in \reals$). $d_{ik}=\tilde d_{ik}$
            \end{proof}
        \end{remark}

        \begin{remark}[Modifikationen]
            \begin{enumerate}
                    \item "Verbotene Wege" $A_i$ soll $B_j$ nicht beliefern, dann $c_{ij}=\infty$ ($\geq K$) bzw. $c_{ij}=K\geq \max\set{c_{ik}}\sum{a_i}>>1$.
                    \item $A_i$ soll mindestens $r$ Einheiten nach $B_k$ liefern, dann $\tilde \alpha_i = \alpha_i-r$, $\tilde \beta_k = \beta_k-r$ ($c_{ik}^0=r$)
                    \item "Überkapazität" $\sum\alpha_i>\sum\beta_i$, "Strohmann" $\beta_{k+1}=\sum\alpha_i\sum\beta_k$ ("Lagerkosten"+Transportkosten)
                    \item `Unterversorgung' $\alpha_{r+1}=\sum\beta_k-\sum\alpha_i$ (`Strafkosten')
                    \item Kapazitätsbeschränkungen $x_{ik}\leq s_{ik}$ (`upper bounding')
                \item Offene TOP: `$=$' (weitere Ungleichungsbeschränkungen).
            \end{enumerate}
        \end{remark}

        