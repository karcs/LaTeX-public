\section{The proof of the MDS main conjecture for $n\leq 2p-2$}%
\makeatletter%
\def\@currentlabel{Section \thesection}%
\makeatother% TODO better section command
\label{sec-msd-nleq2p-2}

In the following we aim to present the results of the two papers~\cite{ball2012setsI} and~\cite{ball2012setsII} simplifying some aspects of the proofs.

\subsection{\person{Segre}'s lemma of tangents}%

Any attempt to answer the MDS main conjecture uses a lemma which was first stated by \person{B. Segre} and can be considered as a key result. To state it we need the following

\begin{definition}[tangent polynomials]
    Let $\cA$ be an arc in $\field{q}^n$. Then for any $(n-2)$-set $X\setleq \cA$ define the tangent polynomial $T_X$ as the product
    $$
    T_X\defdoteq\prod_{\substack{H\in\Sub \field{q}^n\\ \codim{H}=1\\ H\setmeet\cA=X}}{l_H}
    $$
    where $l_H$ are linear forms such that $\ker{l_H}=H$. This defines (up to scalar factor from $\units{\field{q}}$) a homogeneous polynomial in $n$ variables which is of degree $q+1-(\cA-(n-2))=q-1+n-\card{\cA}\eqdef t$ (by counting the corresponding hyperplanes).
\end{definition}

    The main idea of this section is to present a way to choose the tangent polynomials canonically with respect to a given order on the elements of the arc. This will simplify the calculation in the proofs of various lemmas a lot.

\begin{remark}
    From $t\geq 0$ one gets the weak bound $\card\cA\leq q+n-1$ which we already derived in \autoref{sec-gen-hyperplane-arr}.
\end{remark}

Now we are able to state the lemma.

\begin{lemma}[\person{Segre}'s lemma of tangents, original version]\label{segre-tangent-lemma-orig}
  Let $n\geq3$ and $\cA$ be a representation of an arc in $\P \field{q}^n$. Then for pairwise distinct $x_i\in \cA$ ($i=0,1,2$) and $X\setleq \cA\setminus\set{x_0,x_1,x_2}$, $\card X=n-3$ we have
  $$    \frac{T_{X\setjoin\set{x_0}}(x_1)T_{X\setjoin\set{x_1}}(x_2)T_{X\setjoin\set{x_2}}(x_0)}{T_{X\setjoin\set{x_1}}(x_0)T_{X\setjoin\set{x_2}}(x_1)T_{X\setjoin\set{x_0}}(x_2)}={(-1)}^{t+1}.
  $$
\end{lemma}

\begin{remark}
    The expression in the lemma is well-defined since the scalar factors in $T_Y$ vanish in the fraction.
\end{remark}

\begin{proof}
    Denote by $\dual x_i$ the linear form corresponding to $x_i$ in the dual basis of $X'\defeq X\setjoin\set{x_0,x_1,x_2}$.
    For $i\in\set{0,1,2}$ let $\cH_i$ be the set of hyperplanes $H$ containing $\gen{X,x_i}$ but not $x_j$ for $j\neq i$.
    For each hyperplane $H\in \cH_i$ there are two possibilities:
    \begin{casebycase}
        \item There is an $a\in\cA\setminus X'$ with $a\in H$. In this case define the linear form
    $$
    l_H\defeq \det(\arg,x_i,X,a)
    $$
        \item Otherwise, $l_H$ is already defined via the tangent polynomials.
    Now, one may define the polynomial $P_i\defeq \prod_{H\in \cH_i}{l_H}$ (of degree $q-1$) and one computes that
    $$
    P_i(x_j)=T_{X\setjoin\set{x_i}}(x_j)\prod_{a\in\cA\setminus X'}{\det(x_j,x_i,X,a)}.
    $$%%
    \end{casebycase}
    When $\set{0,1,2}\setminus\set{i}=\set{j,k}$ it is clear that $P_i(x_j)/P_i(x_k)=-1$ which can be deduced from the fact that the product of all units of a finite field is $-1$ and the observation that all hyperplanes in $\cH_i$ are given by $\ker(\dual x_j+\lambda \dual x_k)$ where $\lambda\in\units{\field{q}}$.
    Thus we have
    \begin{align*}
        \frac{P_1(x_1)P_2(x_2)P_3(x_0)}{P_2(x_0)P_3(x_1)P_1(x_2)} &= \frac{T_{X\setjoin\set{x_0}}(x_1)T_{X\setjoin\set{x_1}}(x_2)T_{X\setjoin\set{x_2}}(x_0)}{T_{X\setjoin\set{x_1}}(x_0)T_{X\setjoin\set{x_2}}(x_1)T_{X\setjoin\set{x_0}}(x_2)}\\
        & \quad\setprod\prod_{\substack{i,j\in\set{0,1,2}\\ i<j}}\prod_{a\in\cA\setminus X'}{\frac{\det(x_i,x_j,X,a)}{\det(x_j,x_i,X,a)}}\\
        & = {(-1)}^3=-1.
    \end{align*}
    Since $\cA\setminus X'$ has $\card{\cA}-n=q-1-t$ elements, the last two products evaluate to ${(-1)}^{3(q-1-t)}={(-1)}^t$.
    This finishes the proof.
\end{proof}

For convenience we now state a `simplified version' of \person{Segre}'s lemma which is the only version we shall use in the following.

At first it becomes necessary to recall the meaning of the following notation from the preliminaries.

\begin{convention}[sequence notation]
    By $\seq{\range{A_0}{A_{m-1}}}$ we mean the sequence as explained in \autoref{not-conv} where $A_i$ is a subset of some ordered set $A$.%% TODO REF
\end{convention}


\begin{corollary}[Simplification of \person{Segre}'s lemma]\label{mds-segre-simplified}
    Let $\cA$ be a representation of an arc in $\P \field{q}^n$ ($n\geq 3$) and $\leq$ some linear
    order on $\cA$. Then we can define the tangent polynomials $T_Y$
    ($Y\setleq \cA$, $\card Y=n-2$) such that for all $x_0,x_1,X$ we have
    $$    \frac{T_{X\setjoin\set{x_1}}(x_0)}{T_{X\setjoin\set{x_0}}(x_1)}={\left(\frac{\sgn(X\setjoin\set{x_1},x_0)}{\sgn(X\setjoin\set{x_0},x_1)}\right)}^{t+1}
    $$
    where $X\setleq \cA$ has $n-3$ elements and $x_0,x_1\in \cA\setminus X$ are distinct.
\end{corollary}

\begin{proof}
    We define the directed graph $G=(V,E)$ by $V\defeq \binom{\cA}{n-2}$ and $E\defeq \set{(u,v)\in V^2:\card{u\setminus v}=1}$.
    Moreover, we define a labeling $\lambda:E\to \units{\field{q}}$. For $(u,v)\in E$ let $u_v\in u\setminus v$ and $v_u\in v\setminus u$ be the elements of the singletons. Then define
    $$
    \lambda(u,v)\defeq \frac{T_u(v_u)}{T_v(u_v)}\text{.} 
    $$
    For the two types of triangles in $G$ we have two different relations holding.
    A triangle of the first type consists of vertices $u,v,w\in V$ such that $\card{u\setjoin v\setjoin w}=n-1$.
    Here it clearly holds that
    $$
    \lambda(u,v)\lambda(v,w)\lambda(w,u)=\frac{T_u(v_u)T_v(w_v)T_w(u_w)}{T_v(u_v)T_w(v_w)T_u(w_u)}=1
    $$
    which is a trivial equation since $v_u=w_u$, $u_v=w_v$, $u_w=v_w$.
    For a triangle of the second type consisting of vertices $u,v,w\in V$ such that $\card{u\setjoin v\setjoin w}=n$ we obtain
    $$
    \lambda(u,v)\lambda(v,w)\lambda(w,u)={(-1)}^{t+1}
    $$
    by \autoref{segre-tangent-lemma-orig} where $X\defeq u\setmeet v\setmeet w$ and $x_0\defeq u_v=u_w$, $x_1\defeq v_u=v_w$, $x_2\defeq w_u=w_v$.
    It is thus clear, that $\lambda$ is uniquely defined by any restriction $\rest{\lambda}_{E_T}$ where $T=(V,E_T)$ is a (directed) rooted spanning tree of $G$ with root $r\in V$ (by the above relations in triangles of $G$ and as $G$ is obviously strongly connected).
    Moreover, when replacing $T_u$ by $T'_u\defeq \mu T_u$ one just modifies $\lambda$ to $\lambda'$ where 
    $$
    \lambda'(v,w)=
    \begin{cases} 
        \lambda(v,w) & : u\notin\set{v,w}\\
        \mu\lambda(v,w) &: v=u\\
        \mu^{-1}\lambda(v,w) &: w=u
    \end{cases}\text{.}
    $$
    This idea can be used to modify the tangent polynomials step by step to achieve any values of $\lambda$ among $E_T$.
    Define the sets $V_l\defeq \set{v\in V}[d_T(r,v)=l]$ ($l\in\nats$, here $d_T(\arg_1,\arg_2)$ means the metric of the shortest path on $T$). Since $G$ is finite there is some $L\in\nats$ such that $\set{V_l}[l=\range 0 L]$ is a partition of $V$. Moreover, one notes that the sets $E_l\defeq \set{(u,v)\in E_T}[u\in V_{l-1},v\in V_l]$ for $l=\range 1 L$ form a partition of $E_T$. Thus one can modify the labeling $\lambda$ at first on $E_1$ then on $E_2$ etc. As there is no edge in $T$ between the sets $V_m$ and $V_n$ where $\card{n-m}\geq 2$ this procedure works and one does not destroy former changes on some $E_i$. 
    This shows that $\lambda$ can be changed to any labeling satisfying the two triangle conditions.
    Lastly, we check that these are satisfied for the labeling $\bar{\lambda}$ given in the lemma, where for $(u,v)\in E$

    $$
    \bar{\lambda}(u,v)\defeq {\left(\frac{\sgn(u,v_u)}{\sgn(v,u_v)}\right)}^{t+1}\text{.}
    $$
    
    For a triangle $uvw$ of the first type ($\card{u\setjoin v\setjoin w}=n-1$) we obtain $v_u=w_u$, $u_v=w_v$, $u_w=v_w$ and thus one gets
    $$
    \bar{\lambda}(u,v)\bar{\lambda}(v,w)\bar{\lambda}(w,u) ={\left(\frac{\sgn(u,v_u)}{\sgn(v,u_v)}\frac{\sgn(v,w_v)}{\sgn(w,v_w)}\frac{\sgn(w,u_w)}{\sgn(u,w_u)}\right)}^{t+1}\\
    =1
    $$
    Similarly, for a triangle $uvw$ of the second type ($\card{u\setjoin v\setjoin w}=n$) one gets the desired identity by the following reasoning. W.l.o.g.\ we may write $u=X\setjoin\set{x_0}$, $v=X\setjoin\set{x_1}$, $w=X\setjoin\set{x_2}$ for an $(n-3)$-element set $X\defeq u\setmeet v\setmeet w$ and elements $x_i\in \cA$ ($i=0,1,2$) such that $\set{x_0,x_1,x_2}\setjoin X=u\setjoin v\setjoin w$. Furthermore, we may assume that $X_0<x_0<X_1<x_1<X_2<x_2<X_3$, where $X_0,X_1,X_2,X_3$ partitions $X$ (otherwise interchange the labeling of $u$, $v$ and $w$; some sets $X_i$ may of course be empty for $i=\range 0 3$). Then compute
    \begin{align*}
        \bar{\lambda}(u,v)\bar{\lambda}(v,w)\bar{\lambda}(w,u) 
 & = {\left(\frac{\sgn(X_0,x_0,X_1,X_2,X_3,x_1)}{\sgn(X_0,X_1,x_1,X_2,X_3,x_0)}\right)}^{t+1}              \\
 & \quad\setprod {\left(\frac{\sgn(X_0,X_1,x_1,X_2,X_3,x_2)}{\sgn(X_0,X_1,X_2,x_2,X_3,x_1)}\right)}^{t+1} \\
 & \quad\setprod {\left(\frac{\sgn(X_0,X_1,X_2,x_2,X_3,x_0)}{\sgn(X_0,x_0,X_1,X_2,X_3,x_2)}\right)}^{t+1} \\
 & = {(-1)}^{((|X_2|+|X_3|)+(|X_1|+1+|X_2|+|X_3|))(t+1)}                                                 \\
 & \quad\setprod {(-1)}^{(|X_3|+(|X_2|+1+|X_3|))(t+1)}                                                   \\
 & \quad\setprod {(-1)}^{((|X_1|+|X_2|+1+|X_3|)+|X_3|)(t+1)}                                            \\
 & = {(-1)}^{t+1}
\end{align*}

  to end the proof.
\end{proof}

This lemma enables us to make the following definition.

\begin{definition}
    Let $\cA\setleq\field{q}^n$ be a representation of an arc. We say that its tangent polynomials are defined canonically with respect to the linear order $\leq$ on $\cA$ if they satisfy for all $x_0,x_1,X$ the identity
    $$
    \frac{T_{X\setjoin\set{x_1}}(x_0)}{T_{X\setjoin\set{x_0}}(x_1)} = {\left(\frac{\sgn(X\setjoin\set{x_1},x_0)}{\sgn(X\setjoin\set{x_0},x_1)}\right)}^{t+1}
    $$
    where $X\setleq \cA$ has $n-3$ elements and $x_0,x_1\in \cA\setminus X$ are distinct.
\end{definition}

    This simple but effective trick enables us to prove some results of \person{Ball} and \person{de Beule} avoiding the occurrence of some inconvenient terms in the calculation.
    Note that in the above definition the tangent polynomials are still only defined up to scalar factor, but their quotients are fixed.
    In the following we do only work with the tangent polynomials chosen in that manner.

    Actually, we shall use new symbols $P(X)$ for the evaluation of tangent polynomials in subsequent calculations.

\begin{definition}[abbreviation of tangent polynomial evaluations]\label{tang-pol-eval}
    Let $\cA$ be a representation of an arc in $\P \field{q}^n$ such that the tangent polynomials $T_Y$ are defined canonically with respect to some linear order $\leq$ on $\cA$. Then we set $P(X)\defeq T_{X\setminus\set{x}}(x)$ where $x$ is the biggest element in  $X$.
\end{definition}

\section{Interpolation formulae}

An elementary but particularly nice idea is to use interpolation to capture information about the arc.

There are two basic possibilities to apply interpolation. 

\begin{lemma}[interpolation of the tangent polynomial]
    Let $\cA\setleq\field{q}^n$ be a representation of an arc and $A,B\setleq \cA$ be disjoint subsets with
    $\card{A}=t+2$ and $\card{B}=n-2$. Then
    $$
    \sum_{a\in A}{T_{B}(a)\prod_{z\in A\setminus \set a}{\det(z,B,a)}^{-1}} =0
    $$
    holds or equivalently, when the tangent polynomials are defined canonically
    with respect to some linear order $\leq$,
    $$
    \sum_{a\in A}{P(\set a\setjoin B)\prod_{z\in A\setminus\set a}\det(z,\set a\setjoin B)^{-1}}=0.
    $$
\end{lemma}
%
\begin{proof}
    As $T_B(x+y)=T_B(x)$ for all $x\in\field{q}^n$ and $y\in\gen B$ we may
    interpolate the polynomial $\bar{T}_B(x+\gen B)\defeq T_{B}(x)$ as a homogeneous polynomial of
    degree $t$ over $\field{q}^n/\gen B$. To do this, pick $a\in A$ to get
    $$
    T_B(x)=\sum_{a'\in A\setminus\set a}{T_B(a')\prod_{z\in A\setminus\set{a,a'}}{\frac{\det(z,B,x)}{\det(z,B,a')}}},
    $$
    since both sides are polynomials in $x$ of degree $t$
    and both are constant on cosets of $\gen B$ and agree on $t+1$
    points of $\field{q}^n/\gen B$ (namely $a'+\gen B$
    for $a'\in A\setminus\set a$), for which the right hand side is a \person{Lagrange} interpolation formula. Replacing $x$ by $a$ and dividing by
  $\prod_{z\in A\setminus\set a}{\det(z,B,a)}$ one gets 
  $$
  T_B(a)\prod_{z\in  A\setminus\set a}{{\det(z,B,a)}^{-1}} = {\det(a',B,a)}^{-1}\sum_{a'\in A\setminus\set a}{T_B(a')\prod_{z\in A\setminus\set{a,a'}}{{\det(z,B,a')}^{-1}}}
  $$
  which is what we wanted to prove.
  The second formulation in the lemma follows from the fact that the tangent polynomials are defined canonically with respect to an underlying linear order and the definition of $P(X)$ together with \autoref{mds-segre-simplified}. 
\end{proof}

Another idea is to interpolate the determinants themselves.
%% todo
\begin{lemma}[interpolation of determinants]\label{mds-tan-poly-interpol-det}
    Let $A,B,C\setleq \field{q}^n$ such that $\gen{A\setjoin
    C}=\field{q}^n$, $\card A+\card C=n+1$ and $\card B+\card C=n-1$, and let $\leq$
    be some linear order on $A\setjoin B\setjoin C$. Then we have
    $$
    \sum_{a\in A}{\sgn(a,A\setminus\set a\setjoin C)\det(a,B\setjoin C)\det(A\setminus\set a\setjoin C)}=0\text{.}
    $$
    Here, $\sgn$ is taken with respect to $\leq$.
\end{lemma}

\begin{proof}
  Picking $a\in A$ and interpolating $\det(\arg,B\setjoin C)$ as a linear form in
  $\field{q}^n/\gen C$ gives
  $$
  \det(x,B\setjoin C) 
  = \sum_{a'\in A\setminus\set a}{\det(a',B\setjoin C)\frac{\det(x,A\setminus\set{a,a'}\setjoin C)}{\det(a',A\setminus\set{a,a'}\setjoin C)}}
  $$
  which holds as it holds for $x\in A\setminus\set a\setjoin C$ which is a basis of $\field{q}^n$. Replacing $x$ by $a$ and
  rearranging the terms yields
  \begin{align*}
    \det(a,B\setjoin C)\det(A\setminus\set a\setjoin C)
    &= 
    \sum_{a'\in A\setminus\set a}{\det(a',B\setjoin C)\frac{\det(a,A\setminus\set{a,a'}\setjoin C)}{\sgn(a',A\setminus\set{a,a'}\setjoin C)}}\\
    &= 
    \sum_{a'\in A\setminus\set a}{\det(a',B\setjoin C)\det(A\setminus\set {a'}\setjoin C)\frac{\sgn(a,A\setminus\set{a,a'}\setjoin C)}{\sgn(a',A\setminus\set{a,a'}\setjoin C)}}\\
    &= 
    -\sum_{a'\in A\setminus\set a}{\det(a',B\setjoin C)\det(A\setminus\set{a'}\setjoin C)\frac{\sgn(a',A\setminus\set{a'}\setjoin C)}{\sgn(a,A\setminus\set a\setjoin C)}},
  \end{align*}
  which is the desired result.
\end{proof}

\subsection{Manipulation of interpolation identities}

\begin{genassumptions}
For the rest of this section we fix $\cA\setleq\field{q}^n$ as a representation of an arc in $\P \field{q}^n$ with a linear order $\leq$ explained on it, $P(X)$ as the evaluations of tangent polynomials of $\cA$ as defined in \autoref{tang-pol-eval}, $p=\rchar \field{q}$ as the characteristic of the finite field $\field{q}$ and $t\defeq q+n-1-\card \cA$ as the degree of the tangent polynomials of $\cA$.  
\end{genassumptions}

Now, the idea is to play with the interpolation identities to reach a
contradiction in the case where $t\leq n-3$ (i.e. $\card{\cA}\geq q+2$).

\paragraph{First attempt.} The proof of the main conjecture for MDS codes of \person{Ball} and \person{de Beule} for the case in which $n\leq p$ and the classification of $(q+1)$-arcs in that case is based on the following key result which can be derived by elementary means from the interpolation of tangent polynomials.

\begin{lemma}[\person{Ball} \& \person{de Beule}'s $ABC$ lemma]\label{mds-abc-lemma}
    Let $0\leq r\leq\min\set{n,p}-1$ and $A,B,C\setleq \cA$ disjoint sets such
    that $\card{A}+\card{B}=r+t+1$, $\card{C}=n-1-r$. We then have
    \begin{align*}
        &{(-1)}^r\sum_{\substack{A'\setleq A\\ \card{A'}=r}}{P(A'\setjoin C)\prod_{z\in (A\setminus A')\setjoin B}{{\det(z,A'\setjoin C)}^{-1}}}\\
        &= \sum_{\substack{B'\setleq B\\ \card{B'}=r}}{P(B'\setjoin C)\prod_{z\in (B\setminus B')\setjoin A}{{\det(z,B'\setjoin C)}^{-1}}}\text{.}
    \end{align*}
\end{lemma}

\begin{proof}
    The proof happens by induction on $r$. For $r=0$ the statement is a
    trivial.
    Now, suppose the lemma is proven for $r-1\geq 0$ and let
    $r\leq\min\set{n,p}-1$.
    Moreover, let $A,B,C\setleq \cA$ be disjoint sets such
    that $\card{A}+\card{B}=r+t+1$, $\card{C}=n-1-r$. Then pick $a\in A$ and
    apply the lemma for $r-1$ and sets $A\setminus\set a$, $B$,
    $\set{a}\setjoin C$ (if $A$ and $B$ are empty the lemma is obvious --- moreover, the roles of $A$ and $B$ are symmetric).
    This yields
    \begin{align*}
        &{(-1)}^{r-1}\sum_{\substack{A'\setleq A\setminus\set a\\ \card{A'}=r-1}}{P(A'\setjoin\set a\setjoin C)\prod_{z\in (A\setminus A')\setjoin B}{{\det(z,A'\setjoin\set a\setjoin C)}^{-1}}}\\
        &= \sum_{\substack{B'\setleq B\\ \card{B'}=r-1}}{P(B'\setjoin\set a\setjoin C)\prod_{z\in (B\setminus B')\setjoin A\setminus\set a}{{\det(z,B'\setjoin\set a\setjoin C)}^{-1}}}\text{.}
    \end{align*}
    Summing over $a\in A$ gives
    \begin{align*}
        &{(-1)}^{r-1}r\sum_{\substack{A'\setleq A\\ \card{A'}=r}}{P(A'\setjoin C)\prod_{z\in (A\setminus A')\setjoin B}{{\det(z,A'\setjoin C)}^{-1}}}\\
        &= \sum_{\substack{B'\setleq B\\ \card{B'}=r-1}}{\sum_{a\in A}{P(B'\setjoin\set a\setjoin C)\prod_{z\in (B\setminus B')\setjoin A\setminus\set a}{{\det(z,B'\setjoin\set a\setjoin C)}^{-1}}}}\\
        &= \sum_{\substack{B'\setleq B\\ \card{B'}=r-1}}{\sum_{a\in A}{P(B'\setjoin\set a\setjoin C)\prod_{z\in (B\setminus B'\setjoin A)\setminus\set a}{{\det(z,B'\setjoin\set a\setjoin C)}^{-1}}}}\\
        &= -\sum_{\substack{B'\setleq B\\ \card{B'}=r-1}}{\sum_{b\in B\setminus B'}{P(B'\setjoin\set b\setjoin C)\prod_{z\in(B\setminus B'\setjoin A)\setminus\set b}{{\det(z,B'\setjoin\set b\setjoin C)}^{-1}}}}\\
        &= -r\sum_{\substack{B'\setleq B\\ \card{B'}=r}}{P(B'\setjoin C)\prod_{z\in (B\setminus B')\setjoin A}{{\det(z,B'\setjoin C)}^{-1}}}\text{.}
    \end{align*}
    Here we used the interpolation of tangent polynomials in the fourth
    line for the sets $B'\setjoin C$ and $B\setminus B'\setjoin A$ when
    $r-1\leq\card{B}$. In the case $\card{B}< r-1$ the left hand side is zero (as it
    had been zero before).
    If $r\leq p-1$ it is a unit and we can divide the above by $-r$ to complete
    the induction.
\end{proof}

We thus immediately obtain

\begin{corollary}[the case $n\leq p$]\label{mds-bound-n-leq-p}
  If $n\leq p$ then $\card\cA\leq q+1$.
\end{corollary}

\begin{proof}
  Assume that $n\leq p$ and $\card{\cA}\geq q+2$ or equivalently $t=q+n-1-\card{\cA}\leq
  n-3$. Then apply \autoref{mds-abc-lemma} with
  $r=\card{A}=n-1\leq p-1$ and appropriate subsets $B$, $C$ (as $\min\set{n,\card{\cA}-n}\leq\card{\cA}/2$, using the dual arc \see{def-dual-arc} if necessary we may assume w.l.o.g.\ that $n+t\leq 2n-3\leq\card{\cA}$).
  We have $\card{B}=t+1\leq n-2$ and thus the lemma gives
  $$
    {(-1)}^{n-1}P(A)\prod_{z\in B}{{\det(z,A)}^{-1}}=0
  $$
  which is a contradiction.
\end{proof}

\begin{remark}
    This is as we will see the optimal result using \emph{only} the interpolation
    of the tangent polynomial in the sense that the corresponding system of equations for $t=n-3$ is regular if and only if $n\leq p$ \see{sec-mds-knes}.  % TODO sectioning
\end{remark}

Moreover, in that case the $(q+1)$-arcs can be identified as normal rational
curves.

\begin{corollary}[classification of $(q+1)$-arcs for $n\leq p$]\label{mds-class-n-leq-p}
    Let $n\leq p$. Then $\cA$ is a normal rational curve.
\end{corollary}

\begin{proof}
    In the case $\card{\cA}=q+1$ one has $t=n-2$.
    Again, we apply \autoref{mds-abc-lemma} for $r=n-1\leq p-1$ and $A\setleq \cA$ with
    $\card A=n$ and appropriate sets $B,C\setleq \cA$ (here $\card{B}=t=n-2<r$ and $C=\emptyset$). This gives
    $$ {(-1)}^{n-1}\sum_{\substack{A'\setleq A \\
            \card{A'}=n-1}}{P(A')\prod_{z\in (A\setminus A')\setjoin
            B}{{\det(z,A')}^{-1}}} = 0.
    $$
    Applying the above for $A$ and $B_b\defeq B\setminus\set{b}\setjoin\set{x}$ for some fixed point $x\in \cA\setminus A$ we obtain the $n-2$ equations
    $$
    \sum_{a\in A}{P(A\setminus\set{a})\prod_{z\in B}{{{\det(z,A\setminus\set{a})}^{-1}}}\frac{\det(b,A\setminus\set{a})}{\det(a,A\setminus\set{a})}{\det(x,A\setminus\set{a})}^{-1}} = 0\quad (b\in B).\label{mds-class-n-leq-p-keysys}
    $$
    We could also have written $\dual a(b)$ (where $\dual a$ means an element of the dual basis of $A$) for the fraction of determinants showing that the matrix $M\in\field{q}^{B\setprod A}$ defined by
    $$
    M\defeq {(m_{ba})}_{(b,a)\in B\setprod A},\quad
    m_{ba} \defeq \frac{\det(b,A\setminus\set{a})}{\det(a,A\setminus\set{a})}
    $$
    has full rank as its rows are just the coordinate vectors of each $b\in B$ with respect to the basis $A$.
    Thus it follows that the matrix $N\defeq MD$, where
    $$
    D\defeq \diag\seq{P(A\setminus\set{a})\prod_{z\in\set a\setjoin B}{{{\det(z,A\setminus\set{a})}^{-1}}}}[a\in A],
    $$ has full rank which is the matrix of the linear system~\theref{mds-class-n-leq-p-keysys} in
    $$
    \seq{\frac{\det(a,A\setminus\set a)}{\det(x,A\setminus\set a)}}_{a\in A}.
    $$
    Hence the kernel of this system has dimension $\card{A}-\card{B}=n-(n-2)=2$ showing that the image of $x$ for all $x\in \cA\setminus A$ under the map
    $$
    \gamma:{(\units{\field{q}})}^n\to{(\units{\field{q}})}^n
    $$
    where ${(\units{\field{q}})}^n\defeq \field{q}^n\setminus\Setjoin_{a\in A}{\gen{A\setminus\set{a}}}$ and
    $$
    \sum_{a\in A}{\lambda_a a}\mapsto \sum_{a\in A}{\lambda_a^{-1} a}
    $$
    must lie on a (projective) line i.e.~in a two dimensional subspace of $\field{q}^n$.
    Using an appropriate element of $M\in\PGL(n,\field{q})$ which maps $\gen a \mapsto \gen a$ for $a\in A$ and $\gen{\hat x}\mapsto \gen{\sum_{a\in A}{a}}$ for some $\hat x\in \cA\setminus A$ we may assume w.l.o.g.~that $\hat a\defeq \sum_{a\in A}{a}\in \cA$. Set $\hat{A}\defeq A\setjoin\set{\hat a}$. 

    Rescaling $\gamma(x)$ ($x\in\cA\setminus \hat A$) appropriately we obtain scalars $\alpha_x$ such that
    $$
    \alpha_x\gamma(x)=\check a-\hat a\lambda_x
    $$
    lie on an affine line parallel to $\gen{\hat a}$ in $\field{q}^n$ (for some $\check a\in\field{q}^n$ and $\lambda_x\in\field{q}$). This is possible since the point $\gen{\hat a}$ lies on the same projective line as all $\gen x\in\cA\setminus \hat A$ so the latter lie in an affine line parallel to $\gen{\hat a}$.
    Changing the parameters if necessary by a translation $x\mapsto x+\mu$ ($\mu\in\field{q}$), we may assume that $0=\lambda_x$ for some $x\in \cA\setminus \hat{A}$ whence $\check a\in {(\units{\field{q}})}^n$.

    The line $\lambda\mapsto \check a-\hat a\lambda$ intersects the $n$ hyperplanes $\gen{A\setminus\set a}$ (for $a\in A$) in $n$ distinct points (i.e.~in the coordinate representation $\check a =\sum_{a\in A}{\nu_a a}$ all $\nu_a$ are distinct for $a\in A$). This holds as the assumption $\nu_{a'}=\nu_{a''}$ for $a',a''\in A$, $a'\neq a''$ leads to the contradiction $\set{\hat a,\check a}\setjoin A\setminus\set{a',a''}\setleq \cA$ forming a linearly dependent $n$-set.

    Therefore, we may deduce that
    $$
    \set{\nu_a:a\in A}\setjoin\set{\lambda_x:x\in \cA\setminus\hat A}=\field{q}
    $$ is a disjoint union as all $\check a-\hat a\lambda_x$ have no coordinates equal to zero in the basis $A$. However, considering the set $\hat\cA\defeq \hat A\setjoin\set{\gamma^{-1}\compose\alpha_x\compose\gamma(x):x\in \cA\setminus\hat A}$ we have a representation of the same arc which is a \person{Cauchy}-representation shown in \autoref{cauchy-rep}. So the arc represented by $\cA$ is a normal rational curve.
\end{proof}

\begin{remark}
    The argument can also be used to prove that the cardinality of an arc in $\P \field{q}^n$ ($n\leq p$) can at most become $q+1$ (similarly to \autoref{mds-bound-n-leq-p}).
\end{remark}

\paragraph{Second attempt.} In this paragraph we prove the same result for $n\leq 2p-2$ and will bring in the interpolation of determinants. A classification of $(q+1)$-arcs is not given.

\begin{lemma}[\person{Ball}'s \& \person{de Beule}'s $ABCDE$ lemma]
    Let $n>p$ and $0< r\leq p-1$ and $0\leq m\leq \min\set{n-1-r,t+2}$. Moreover, let $A$, $B$, $C$, $D$ and $E$ be disjoint subsets of $\cA$ with $\card A=\card B=m$, $\card C=t+2-m$, $\card D=n-1-r-m$, $\card E=r-1$ and let there be given bijections $m\to A$ and $m\to B$ such that $A_\tau, B_\tau$ denote the images of $\tau\setleq m=\set{\range 0 {m-1}}$. Then we have
    \begin{align*}
        0 & = \sum_{\substack{C'\setleq C                                                                       \\
                \card{C'}=r}}
        {\sum_{\tau\setleq m}{{(-1)}^{\card{\tau}}P(A_{\tau}\setjoin B_{m\setminus\tau}\setjoin C'\setjoin D)}} \\
          & \quad\setprod\prod_{\substack{z\in A_{m\setminus\tau}\setjoin B_{\tau}                                  \\
                  \setjoin(C\setminus C')\setjoin E}}
          {{\det(z,A_{\tau}\setjoin B_{m\setminus\tau}\setjoin C'\setjoin D)}^{-1}}.
\end{align*}
\end{lemma}
%%
\begin{proof}
    The proof happens by induction on $m$.
    We have $n>p$ so we may apply the \autoref{mds-abc-lemma} for $r\leq p-1$ and sets $A,B,C$ with $\card{A}=t+2$, $\card{B}=r-1$ and $\card{C}=n-1-r>0$.
    This gives
    $$ 0 = \sum_{\substack{A'\setleq A\\\card{A'}=r}}{P(A'\setjoin C)\prod_{z\in (A\setminus A')\setjoin B}{{\det(z,A'\setjoin C)}^{-1}}}. $$
    proving the lemma for $m=0$ when replacing $A$ by $C$, $B$ by $E$ and $C$ by $D$.

    For the induction step, assume the lemma holds for $m-1$ and for given $A$, $B$, $C$, $D$ and $E$ with $\card{A}=\card{B}=m$, $\card{C}=t+2-m$, $\card{D}=n-1-r-m$, $\card{E}=r-1$ pick $a\defeq a_m\in A$ and $b\defeq b_m\in B$ and apply the induction hypothesis for $\bar{A}\defeq A\setminus\set{a}$, $\bar{B}\defeq B\setminus\set{b}$, $C\setjoin\set{a}$, $D\setjoin\set{b}$, $E$ and $\bar{A}$, $\bar{B}$, $C\setjoin\set{b}$, $D\setjoin\set{a}$, $E$ (and $m-1$), respectively. 
    This yields (the terms where $a\in C'$ on the left hand side and $b\in C'$ on the right hand side cancel out) 
    \begin{align*}
 & \sum_{\substack{C'\setleq C                                                                   \\ \card{C'}=r}}\sum_{\tau\setleq m-1}{{(-1)}^{\card{\tau}}P(\bar{A}_{\tau}\setjoin \bar{B}_{(m-1)\setminus\tau}\setjoin\set{b}\setjoin C'\setjoin D)}\\
 & \quad\setprod\prod_{\substack{z\in \bar{A}_{(m-1)\setminus\tau}\setjoin\set{a}\setjoin \bar{B}_{\tau} \\ \setjoin(C\setminus C')\setjoin E}}{{\det(z,\bar{A}_{\tau}\setjoin \bar{B}_{(m-1)\setminus\tau}\setjoin\set{b}\setjoin C'\setjoin D)}^{-1}}\\
 & = \sum_{\substack{C'\setleq C                                                               \\
         \card{C'}=r}}\sum_{\tau\setleq m-1}{{(-1)}^{\card{\tau}}P(\bar{A}_{\tau}\setjoin\set{a}\setjoin \bar{B}_{(m-1)\setminus\tau}\setjoin C'\setjoin D)}\\
 & \quad\setprod\prod_{\substack{z\in \bar{A}_{(m-1)\setminus\tau}\setjoin \bar{B}_{\tau}\setjoin\set{b} \\\setjoin(C\setminus C')\setjoin D}}{{\det(z,\bar{A}_{\tau}\setjoin\set{a}\setjoin \bar{B}_{(m-1)\setminus\tau}\setjoin C'\setjoin D)}^{-1}}.
    \end{align*}
    Rearranging this to one side proves the induction.
\end{proof}

Applying the above corollary to the condition $\card\cA=q+2$, i.e. $t=n-3$ leads to

\begin{corollary}
    When $\card{\cA}=q+2$ and $m=n-1-r\geq n-p$ we have
    $$
  0 = \sum_{\tau\setleq m}{{(-1)}^{\card{\tau}}P(A_{\tau}\setjoin B_{m\setminus\tau}\setjoin C)}\prod_{\substack{z\in A_{m\setminus\tau}\setjoin B_{\tau}\setjoin E}}{{\det(z,A_{\tau}\setjoin B_{m\setminus\tau}\setjoin C)}^{-1}}.
    $$
\end{corollary}

\begin{proof}
    Since $\card {C\setminus C'}=t+2-m-r=(n-1)-(n-1-r)-r=0$ the corresponding factors vanish in the product of the last corollary. For the same reason $D=\emptyset$.
\end{proof}

\begin{corollary}[the case $n\leq 2p-2$]
    Any arc $\cA$ in $\P \field{q}^n$ with $n\leq 2p-2$ satisfies the bound $\card\cA\leq q+1$.
\end{corollary}

\begin{proof}
    We may assume that $n>p$ by the previous work.
    Apply the previous corollary for $r=p-1$, then $\card E=p-2$, $\card C=p-1$ and $\card A=\card B=n-p$.
    Write $E$ as $E=F\setjoin G$ where $\card F = 2p-2-n$ (here we use the assumption) and $\card G = n-p>0$.
    Rewriting the equation of in the last corollary delivers
    $$
  0 = \sum_{\tau\setleq m}{{(-1)}^{\card{\tau}}P(A_{\tau}\setjoin B_{m\setminus\tau}\setjoin C)}\prod_{\substack{z\in A_{m\setminus\tau}\setjoin B_{\tau}\setjoin F\setjoin G}}{{\det(z,A_{\tau}\setjoin B_{m\setminus\tau}\setjoin C)}^{-1}}.
    $$
    We now aim to prove the following equation for $0\leq s\leq n-p$ for which the above is the base of induction (inducting on $s\defeq \card D = \card F-(2p-2-n)$)
    $$
  0 = \sum_{\tau\setleq m}{{(-1)}^{\card{\tau}}P(A_{\tau}\setjoin B_{m\setminus\tau}\setjoin C)}\prod_{w\in D}{\det(w,A_{\tau}\setjoin B_{m\setminus\tau}\setjoin C)}\prod_{\substack{z\in A_{m\setminus\tau}\setjoin B_{\tau}\setjoin F\setjoin G}}{{\det(z,A_{\tau}\setjoin B_{m\setminus\tau}\setjoin C)}^{-1}}.
    $$
    for $D\setleq A$ an $s$-element set (which is possible since $\card A=n-p\geq s$).
    
    For the induction step we pick $d\in D$ and $g\in G$, $f\in F$, assume the hypothesis to be proven for $s-1$ and apply this to our sets $A$, $B$, $C$, $D\setminus\set d$, $F\setminus\set f$, $G\setminus\set g\setjoin\set f$
    $$
    \eqalign{%
  0 &= \sum_{\tau\setleq m}{{(-1)}^{\card{\tau}}P(A_{\tau}\setjoin B_{m\setminus\tau}\setjoin C)}\prod_{w\in D\setminus\set d}{\det(w,A_{\tau}\setjoin B_{m\setminus\tau}\setjoin C)}\cr &\quad \setprod\det(g,A_{\tau}\setjoin B_{m\setminus\tau}\setjoin C)\prod_{\substack{z\in A_{m\setminus\tau}\setjoin B_{\tau}\setjoin F\setjoin G}}{{\det(z,A_{\tau}\setjoin B_{m\setminus\tau}\setjoin C)}^{-1}}.
}%
$$
Now, we use in the interpolation formula for the determinants as given in \autoref{mds-tan-poly-interpol-det}. Multiplying the above by
$$
\sgn(g,\set d\setjoin (G\setjoin\set f)\setminus\set g\setjoin C)\det(\set d\setjoin (G\setjoin\set f)\setminus\set g\setjoin C)
$$
and summing over $g\in G\setjoin\set f$ gives (by interpolation of determinants)
$$
    \eqalign{%
  0 &= -\sgn(d,G\setjoin\set f\setjoin C)\det(G\setjoin\set f\setjoin C)\sum_{\tau\setleq m}{{(-1)}^{\card{\tau}}P(A_{\tau}\setjoin B_{m\setminus\tau}\setjoin C)}\prod_{w\in D\setminus\set d}{\det(w,A_{\tau}\setjoin B_{m\setminus\tau}\setjoin C)}\cr &\quad \setprod\det(d,A_{\tau}\setjoin B_{m\setminus\tau}\setjoin C)\prod_{\substack{z\in A_{m\setminus\tau}\setjoin B_{\tau}\setjoin F\setjoin G}}{{\det(z,A_{\tau}\setjoin B_{m\setminus\tau}\setjoin C)}^{-1}},
}%
$$
where we can omit the $\sgn$ and $\det$ at the beginning to see that we are done with the induction step. Of course, the above argument does only work for $s=\card D\leq\card A=n-p$.
Applying the formula which we have just proven for $s=n-p$ (i.e. $D=A$) we get
$$
P(B\setjoin C)\prod_{w\in A}{\det(w,B\setjoin C)}\prod_{z\in A\setjoin F\setjoin G}{{\det(z,B\setjoin C)}^{-1}}=0
$$%%%
since all terms where $\tau\neq\emptyset$ vanish. This clearly is a contradiction.  
Lastly, we have to verify that $A\setjoin B\setjoin C\setjoin F\setjoin G$ is not bigger than $q+2$. Adding the cardinalities leads to
$$
\card{A\setjoin B\setjoin C\setjoin F\setjoin G}=2(n-p)+(p-1)+(p-2)+(n-p)=3n-3-p.
$$
But this is no restriction since $p\leq\sqrt q$ and thus $n\leq 2\sqrt q - 2$ so $3n-3-p<3n-3\leq 6\sqrt q -9\leq q$.
\end{proof}

\begin{remark}
    The author's proposal is to call the above the $A$--$G$ lemma.
\end{remark}
%%%