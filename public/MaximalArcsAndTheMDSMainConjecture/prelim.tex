\section{Preliminaries}
%\addcontentsline{toc}{section}{Preliminaries}

In this thesis we aim to consider several subsets of finite projective spaces admitting certain `independence relations'.

\subsection{Notation conventions}
%\addcontentsline{toc}{subsection}{Notation conventions}

We want to introduce several notation conventions used throughout the thesis.
These are 
%%%%%%%%%%%%%%
\begin{definition}[generalized binomial coefficient]
    Define the generalized binomial coefficient as
    $$
    \binom[q]{n}{k}:= \prod_{i=1}^{k}{\frac{q^n-q^{n-i}}{q^k-q^{k-i}}}.
    $$
    This counts the number of $(k-1)$-dimensional projective subspaces of an $(n-1)$-dimensional projective space.
\end{definition}

When defining a polynomial only up to scalar factor of the underlying field, we write `$\defdoteq$' instead of `$\defeq$'.

For most operations which can be interpreted as meet and join of an underlying lattice, we write `$\meet$' and `$\join$' where sometimes the corresponding expression is only defined up to some natural equivalence relation (this is the case for e.g.~the greatest common divisor or the least common multiple).


