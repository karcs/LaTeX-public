\section{The MDS main conjecture}\label{sec-mds-connections}

In this section being the core part of this thesis, we come to a very interesting open problem which is formulated in coding theory, finite projective geometry, the theory of hyperplane arrangements and matroid theory.

\subsection{Projective arcs}

We now introduce another concept which coincides with the one of a $(k,m)$-arc (see \autoref{kmarc-def}) in dimension two for $m=2$.

\begin{definition}[projective arc]
    A set $\cA\setleq \P(K^n)$ is called a \keyword{projective arc} if no $n$ points of $\cA$ lie in a common hyperplane (if we aim to emphasize the number of points in $\cA$ we call it a $k$-arc when $\card\cA=k$).
    A set $\cA'\setleq K^n$ such that $\cA\defeq\set{\gen a}[a\in\cA']$ is a projective arc and $\gen a\neq\gen b$ for $a,b\in\cA'$ is called an \keyword{representation} of the arc $\cA$.
    Analogously, a set $\cA\setleq K^n$ is called an \keyword{affine arc} if any $n+1$ points are affinely independent.
\end{definition}

\begin{remark}
    As we consider only Desarguian spaces, the above definition is not given for non-Desarguian planes.
\end{remark}

Apart from the `duality principle' for maximal $(k,m)$-arcs there is another duality principle for projective arcs.

\begin{definition}[dual projective arc]\label{def-dual-arc}
    Let $\cA'\setleq K^n$ a representation of a projective arc $\cA$ in $\P(K^n)$ and $n+1\leq \card\cA$. Consider the canonical map
    $$\pi:\dirsum_{a\in\cA'}{Ka}\to K^n$$
    given by
    $$\dirsum_{a\in\cA'}{\lambda_a a}\mapsto \sum_{a\in\cA'}{\lambda_a a}$$
    and define
    $$\pi_a:\dirsum_{a\in \cA'}{Ka}\to K$$
    as the projection to the summand $a$.
    Choose a basis $\cB$ of $\ker \pi$. Then
    $$
    \cA''\defeq\set{\sum_{b\in\cB}{(\pi_a b)b}}[a\in\cA']
    $$
    is a representation of a projective arc $\dual\cA$ in $\ker \pi \iso K^{\card\cA -n}$.
    This is called the \keyword{dual arc} of $\cA$.
\end{definition}
\begin{remark}
Of course this definition depends on the choice of the basis $\cB$, but it defines $\dual \cA$ up to $\PGL$-equivalence.
\end{remark}

We are left to prove that $\dual\cA$ is indeed a projective arc in $\ker\pi\iso K^{\card\cA-n}$.

\begin{proof}    
To see that $\dual\cA$ has the desired property, we may consider a non-trivial linear combination of elements of $\cA''$ which evaluates to zero.
$$
\sum_{a\in\cA'}{\lambda_a\sum_{b\in\cB}{(\pi_a b)b}}=\sum_{b\in\cB}{\left(\sum_{a\in\cA'}{\lambda_a(\pi_a b)}\right)b}=0.
$$
As $\cB$ is linearly independent, we have that
$$
\left(\sum_{a\in\cA}{\lambda_a\pi_a}\right)b=0
$$
for all $b\in\cB$ --- that is $\ker\pi\setleq \ker{\sum_{a\in\cA}{\lambda_a\pi_a}}$. But any element of $c\in\ker \pi$ must either be zero or at least $n+1$ summands $\pi_a c$ of it do not vanish ($a\in \cA'$). As $\card\cA>n$ we have $\ker\pi\neq 0$ and it follows that $\lambda_a\neq 0$ for at least $\card\cA-n+1$ elements $a\in\cA'$. Assume the contrary, that $\lambda_a\neq 0$ for $a\in A\setleq \cA'$ with $\card A\leq \card\cA-n$. Then choose an $(n+1)$-subset $B$ of $\cA'$ such that $B\setmeet A=\set{c}$ and a non-trivial linear combination $\sum_{b\in B}{\mu_b b}=0$ with $\lambda_b\neq 0$ for $b\in B$.
This implies that $\dirsum_{b\in B}{\mu_b b}\in\ker \pi$ and
$$
\left(\sum_{a\in A}{\lambda_a\pi_a}\right)\left(\dirsum_{b\in B}{\mu_b b}\right)=\lambda_c\mu_c\neq 0, 
$$
a contradiction. This shows that any $\card\cA-n$ elements of $\cA''$ are linearly independent.
\end{proof}

This duality principle is completely analogous to the duality principle for MDS codes (see \autoref{lincode-dual}).

A simple fact which is implied by the above is the following
%%
\begin{corollary}\label{mds-ngeqq}
    Let $\cA$ be a projective arc in $\P(\field{q}^n)$ where $q\leq n$. Then $\card\cA\leq n+1$ and all maximal examples with respect to cardinality are $\PGL$-equivalent to $\set{\gen{e_1},\ldots,\gen{e_n},\gen{e_1+\cdots+e_n}}$.
\end{corollary}

\begin{proof}
    Assume that $\cA$ is a projective arc in $\P(\field{q}^n)$ of cardinality $n+2$. Then $\dual\cA$ has the same cardinality in is a projective arc in $\P(\field{q}^2)$ --- thus $\card{\dual\cA}\leq q+1$ (as there are only $q+1$ one-dimensional linear subspaces of $\field{q}^2$). This implies $\card\cA=n+2\leq q+1$ so $n<q$.
    The second claim follows from the fact that $\PGL(K^n)$ acts sharply transitive on the $(n+1)$-tuples in general linear position.
    We breefly demonstrate this. Let ${(a_i)}_{i=0}^n$ and ${(b_i)}_{i=0}^n$ be in general linear position. We may assume then that $a_n=\sum_{i=0}^{n-1}{\lambda_i a_i}$ and $b_n=\sum_{i=1}^{n-1}{\mu_i b_i}$ ($\lambda_i,\nu_i\in \units K$, $i=1,\ldots,n-1$).
    A preimage $A\in\GL(K^n)$ of an element $B\in\PGL(K^n)$ which sends $\gen{a_i}$ to $\gen{b_i}$ must map $a_i$ to $\nu_i b_i$ ($\nu_i\in \units K$, $i=0,\ldots,n$). By linearity of $A$ it follows from
    $$
    Aa_n=A\left(\sum_{i=0}^{n-1}{\lambda_i a_i}\right)=\nu_n b_n=\nu_n\left(\sum_{i=0}^{n-1}{\mu_i b_i}\right)=\sum_{i=0}^{n-1}{\lambda_i\nu_i b_i}
    $$
    that $\nu_i=\nu_n\frac{\mu_i}{\lambda_i}$ for $i=0,\ldots,n-1$. Thus $A$ is defined up to scalar factor $\nu_n\in \units K$ so $B$ exists uniquely.
\end{proof}

\subsection{Generic hyperplane arrangements}

We start by introducing a new concept.

\begin{definition}[generic and central generic hyperplane arrangement]
    An arrangement $\cA$ of (affine) hyperplanes in $V=K^n$ is called \keyword[generic arrangement of hyperplanes]{generic} if for any set of hyperplanes $\cH\setleq\cA$ it holds that $\codim_{\aff}{\Setmeet\cH}=\card\cH$ if $\card{\cH}\leq n$ and $\Setmeet\cH=\emptyset$ otherwise.
    Similarly, $\cA$ is called a \keyword[central generic arrangement of hyperplanes]{central generic arrangement} if for any set of hyperplanes $\cH\setleq\cA$ it holds that $\codim{\Setmeet\cH}=n\meet\card{\cH}$.
\end{definition}

\begin{remark}
    It is then clear that the concept of a central generic arrangement is the same as the concept of an arc in $\proj(\dual V)$.
\end{remark}

\subsection{Linear codes and the \person{Singleton} bound}

Recall the following definitions

\begin{definition}[$(n,k)$-linear code]
    A linear code of length $n$ and rank $k$ (also abbreviated by $(n,k)$-linear code) over a field $K$ is a $k$-dimensional subspace of $K^n$.  
\end{definition}

\begin{definition}
    Let $\cC\leq K^n$ be a linear code. A \keyword{generator matrix} $G$ of $\cC$ is a matrix having as its columns the coordinate vectors of a basis of $\cC$ (with respect to the canonical basis of $K^n$). A \keyword{check matrix} $C$ of $\cC$ is a fully ranked matrix having as its kernel the coordinate vectors of $\cC$.
\end{definition}

\begin{definition}[minimum weight and \person{Hamming} distance]
    The \keyword{\person{Hamming} distance} $d_H$ is a metric on $K^n$ given by
    $$
    d_H(a,b)\defeq\card{\set{i\in n}[a_i\neq b_i]}
    $$
    for $a,b\in K^n$ where the $a_i$ and $b_i$ denote the coordinates of $a$ and $b$ in the canonical basis.
    Similarly, we define
    $$
    w_H(a)\defeq d_H(a,0)
    $$
    as the \person{Hamming} distance to zero and call it the \keyword{weight} of $a$.
    Moreover, the \keyword{\person{Hamming} weight} or \keyword{minimum weight} of an $(n,k)$-linear code is the minimum \person{Hamming} distance between two of its points.
    This is
    $$
    d(\cC)\defeq \min\set{w_H(c)}[c\in\cC\setminus\set{0}].
    $$
    An $(n,k)$-linear code with \person{Hamming} weight $d$ is also denoted as an $(n,k,d)$-linear code.
\end{definition}

We are interested in special linear codes called MDS codes. To see from where they arise, consider the following lemma.

\begin{lemma}[Singleton bound]
    Let $\cS\setleq \field{q}^n$ such that any two distinct points of $\cS$ have at least hamming distance $d$. Then
    $$
    \card{\cS}\leq q^{n-d+1}.
    $$
    Similarly, a $(n,k)$-linear code $\cC$ over an arbitrary field $K$ with $d(\cC)\leq d$ has dimension $k\leq n-d+1$.
\end{lemma}

\begin{proof}
    Consider the projection map $\pi:\field{q}^n\to\field{q}^{n-d+1}$ which forgets $d-1$ coordinates.
    Then $\pi$ is injective on $\cS$ since otherwise $\cS$ would have two distinct points differing in less than $d$ coordinates contradicting the assumptions, proving the first claim.
    The second claim follows analogously from the fact that $\pi:K^n\to K^{n-d+1}$ maps $\cC$ injectively into a $(n-d+1)$-dimensional space showing that $\dim{\cC}\leq n-d+1$.
    This finishes the proof.
\end{proof}

This motivates the definition of MDS codes

\begin{definition}[linear MDS code]
    A linear $(n,k,d)$-code $\mathcal{C}$ is called \keyword[maximum distance separable code]{maximum distance separable} or \keyword{MDS code} if the singleton bound is fulfilled with equality, i.e. $k=n-d+1$. 
\end{definition}

A very important concept for linear codes and especially for MDS codes is \keyword{duality}.

\begin{definition}[dual code]\label{lincode-dual}
    %% K^{n\ast} typo
    Let $\cC\leq K^n$ be a linear code of dimension $k$. The \keyword{dual code} $\dual\cC$ of $\cC$ is defined as $\dual\cC\defeq\set{c^{\ast}\in K^{n\ast}}[c^{\ast}c=0]$. Here $K^{n\ast}$ is equipped with the dual basis of $x_1,\ldots,x_n$ (where ${(x_i)}_{i=1}^n$ is the standard basis of $K^n$).
\end{definition}

It is now easy to characterize linear MDS codes via their checkmatrix

\begin{lemma}[characterization of MDS codes]\label{mdschar}
    Let $\cC$ be an $(n,k,d)$-linear code. Then the following are equivalent
    \begin{statements}
            \item\label{mds} $\cC$ is MDS.
            \item\label{mds-chkmtrx} A check matrix $C \in K^{(n-k)\settimes n}$ has the property that any maximal square submatrix of it is regular.
            \item\label{mds-gmtrx} A generator matrix $G \in K^{k\settimes n}$ has the property that any maximal square submatrix is regular.
            \item\label{mds-dual} The dual code $\dual \cC$ is MDS.
    \end{statements}
\end{lemma}

\begin{proof}
    \begin{implications}
            \item[$\autoref{mds}\equival\autoref{mds-chkmtrx}$:]
        The property of $\cC$ that $d(\cC)=d=n-k+1$ is equivalent translates to the property of a check matrix $C$ of $\cC$ that any $d-1=n-k$ of its columns are linearly independent (as $c\in\cC\equival Cc=0$).
            \item[$\autoref{mds}\equival\autoref{mds-gmtrx}$:]
        The property of $\cC$ that $d(\cC)=n-k+1$ is equivalent to the property of a generatormatrix $G$ of $\cC$ that the coordinate representation of any non-trivial linear combination of the rows of $G$ contains at least $d=n-k+1$ non-zero entries --- that is any $k\settimes k$ submatrix of $G$ is regular.
            \item[$\autoref{mds}\equival\autoref{mds-dual}$:]
        A generator matrix of $\cC$ is a check matrix of $\dual\cC$ and vice versa. By the last two equivalences $\cC$ and $\dual\cC$ can only be MDS codes at the same time. 
    \end{implications}
This completes the proof.
\end{proof}

To fix this interesting type of matrices which are the check (or generator) matrix of an MDS code we make the following

\begin{definition}[principally regular matrix]
    A matrix $A$ over some commutative ring $R$ such that any maximal square submatrix of $A$ is regular is called \keyword{principally regular matrix}.
\end{definition}

There is a very simple connection between this type of matrix and another type of matrix.

\begin{definition}[totally regular matrix]
    Let $B$ be a matrix such that any square submatrix of $B$ is regular. Then $B$ is called \keyword[totally regular matrix]{totally regular}\footnote{Regrettably, the denotation of these matrices in the literature and different papers is not very consistent}. 
\end{definition}

We briefly demonstrate the connection between these two types of matrices.

\begin{lemma}[connection between principally and totally regular matrices]
    Let $C\defeq(A|B)\in R^{m\settimes n}$ be some matrix over a commutative ring $R$ and $m\leq n$ where $A\in R^{m\settimes m}$.
    Then the following two are equivalent.
    \begin{statements}
            \item $C$ is principally regular.
            \item $A$ is invertible and $A^{-1}B$ is totally regular.
    \end{statements}
\end{lemma}

\begin{proof}
    When $C$ is principally regular then $A$ is invertible so that $A^{-1}C$ is also principally regular. Now let $k\leq \min\set{m,n-m}$. Choosing $k$-sets $I\setleq m$ and $J\setleq n\setminus m$ we can check the matrix $\rest{A^{-1}C}_{m\settimes (m\setminus I\setjoin J)}$ for regularity which is easily seen using \person{Laplace}'s formula for determinants to be equivalent to the fact that $\rest{A^{-1}B}_{I\settimes J}$ is regular.
\end{proof}

\subsection{Connections}

We start by drawing some connections between some objects related to this last question.
It turns out, that there is a (up to projective equivalence) `one-to-one' correspondance between the following concepts.

We already saw how MDS codes and principally or totally regular matrices are related.
To draw the connection to projective arcs we interprete the one-dimensional subspaces of the columns of a check matrix $C$ of an MDS code $\cC\leq K^n$ ($\dim\cC=k$) as elements of the projective space $\proj(\col C)\iso \P(K^{d-1})$ (as $d=n-k+1$). This gives us an arc of size $n$ (we can also construct $H(\mathcal{C})$ from that arc).
Starting from a $k$-arc $\cA$ in $\P(K^n)$ we can get a central generic arrangement $\cB$ of hyperplanes by interpreting the elements of $\cA$ as (non-zero) linear functionals $\cF$ in $K^{n\ast}$ (which can be chosen up to scalar factor). Then $\cB\defeq\set{\ker f}[f\in \cF]$ is the associated central generic arrangement.
Finally, we can get an (affine) generic arrangement from a non-empty central generic one by a deconing construction. If $P_{\cA}(X_0,\ldots,X_{n-1})=\prod{\cF}\in\field{q}[X_0,\ldots,X_{n-1}]$ is the defining polynomial of $\cA$ with respect to the basis $x_0,\ldots,x_{n-1}$ (i.e. $P_\cA(\dual x_0,\ldots,\dual x_{n-1})=\prod_{H\in\cA}{l_H}$ where $\ker l_H =H$) and $\ker \dual x_0\in\cA$ then the affine generic arrangement $\cA^{\aff}$ arising from that by deconing $\mathcal{A}$ has defining polynomial $P_{\cA^{\aff}}(X_1,\ldots,X_{n-1})=P_{\cA}(1,X_1,\ldots,X_{n-1})$ with respect to the basis $x_1,\ldots,x_{n-1}$ of $\pi_{(1,\ldots,n-1)}\field{q}^n$ (that is the intersection of the hyperplanes in $\mathcal{A}$ with the one given by $x_0=1$).

\begin{conjecture}[main conjecture on MDS codes, projective arcs, generic arrangements in $\field{q}^n$]
  Let $m(n,q)$ be the minimum of all integers $m$ such that $\card\cA\leq m$ for any arc $\cA$ in a projective space of dimension $n$ and order $q$ then
  $$
    m(n,q)=\begin{cases}
      n+2 :& n \geq q\\
      q+2 :& n<q \lgand n\in\{2,q-1\} \lgand 2\divides q\\
      q+1 :& \otherwise
    \end{cases}.
    $$
\end{conjecture}

\begin{remark}%%%
    We fully proved the conjecture for $n\geq q$ (see \autoref{mds-ngeqq}) characterizing the sets $\cA$ of maximal cardinality. For $n=2$ it is (as we have shown) a consequence of \autoref{kmarc-size} and \autoref{kmarc-dual} und the classification of sets $\cA$ of maximal cardinality in the case of even $q$ or non-Desarguian planes seems to be very difficult. In the case where $q$ is odd and the plane is Desarguian, we will see that these sets $\cA$ are precisely non-singular conics (this will be a consequence of \autoref{mds-class-n-leq-p}).
\end{remark}

\begin{remark}\label{mds-mainconjotherobj}
    Equivalently, $m(n,q)-1$ is the minimum of integers $m'$ such that for all generic hyperplane arrangements $\cB$ in $\field{q}^n$ it holds that $\card{\cB}\leq m'$ or $m(k-1,q)$ is minimal among all $m''$ such that all linear MDS $(n,k)$-codes $\cC$ over $\field{q}$ satisfy $n\leq m''$.
\end{remark}

%\begin{proof}[\autoref{mds-mainconjotherobj}]% BUG
    %% %todo
%\end{proof}




