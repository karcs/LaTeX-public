\documentclass{article}
\usepackage[left=3cm,right=3cm,top=0cm,bottom=2cm]{geometry} % page settings
\usepackage{amsmath} % provides many mathematical environments & tools

\setlength{\parindent}{0mm}

\begin{document}
\title{MTH 251: Week 2 lab write up}
\author{C. M. Hughes}
\date{\today}
\maketitle

We begin by recalling the ancient but famous key result by Beniamino Segre in a more modern version.





\subsection*{Lab activity 1.2.4}
Find the difference quotient of $f(x)$ when $f(x)=x^3$.

We proceed as demonstrated in the lab manual; assuming that $h\ne 0$ 
we have
\begin{align*}
    \frac{f(x+h)-f(x)}{h} & =  \frac{(x+h)^3-x^3}{h}   \\
                          & =  \frac{x^3+3x^2h+3xh^2+h^3 - x^3}{h}\\
                          & =  \frac{3x^2h+2xh^2+h^3}{h}\\
                          & =  \frac{h(3x^2+2xh+h^2)}{h}\\
                          & =  3x^2+2xh+h^2
\end{align*} 

\subsection*{Lab activity 2.3.4}
Use the definition of the derivative to find $f'(x)$ when $f(x)=x^{\frac{1}{4}}$.

Using the definition of the derivative, we have
\begin{align*}
            f'(x)           &= \lim_{h\rightarrow 0}\frac{(x+h)^{1/4}-x^{1/4}}{h}   \\
                            &=  \lim_{h\rightarrow 0}\frac{(x+h)^{1/4}-x^{1/4}}{h}\cdot \frac{((x+h)^{1/4}+x^{1/4})((x+h)^{1/2}+x^{1/2})}{((x+h)^{1/4}+x^{1/4})((x+h)^{1/2}+x^{1/2})}\\
                            &=  \lim_{h\rightarrow 0}\frac{(x+h)-x}{h((x+h)^{1/4}+x^{1/4})((x+h)^{1/2}+x^{1/2})}    \\  
                            &=  \lim_{h\rightarrow 0}\frac{1}{((x+h)^{1/4}+x^{1/4})((x+h)^{1/2}+x^{1/2})}   \\
                            &= \frac{1}{(x^{1/4}+x^{1/4})(x^{1/2}+x^{1/2})} \\
                            &=  \frac{1}{(2x^{1/4})(2x^{1/2})}  \\
                            &=  \frac{1}{4x^{3/4}}  \\
                            &=  \frac{1}{4}x^{-3/4}
\end{align*}
Note: the key observation here is that
\begin{align*}
    a^4-b^4 &= (a^2-b^2)(a^2+b^2)   \\
        &= (a-b)(a+b)(a^2+b^2), 
\end{align*}
with 
\[
    a = (x+h)^{1/4}, \qquad b = x^{1/4},
\]
which allowed us to rationalize the denominator.

\end{document}

%%% Local Variables:
%%% mode: latex
%%% TeX-master: t
%%% End:
