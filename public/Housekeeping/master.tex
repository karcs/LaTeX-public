\documentclass{article}
\usepackage[top=0.5in, bottom=0.5in, left=0.5in, right=0.5in]{geometry}

% for use of german language in a document
\usepackage[utf8]{inputenc} % this is needed for umlauts
\usepackage[ngerman]{babel} % this is needed for umlauts
\usepackage[T1]{fontenc}    % this is needed for correct output of umlauts in pdf



\begin{document}
\pagestyle{empty}
\section*{Haushaltsplan}

\subsection*{Grundlegende Regeln}

\begin{itemize}
        \item Jeder Mitbewohner ist für den gemeinsam genutzten Teil der Wohngemeinschaft gleichermaßen verantwortlich (d.h. Küche, Flur und Bad).
    Dies umfasst regelmäßige Reinigung und nach Möglichkeit sauberes Hinterlassen entsprechender Örtlichkeiten.
    \item Bezüglich der \textbf{Küche} umfassen die entsprechenden Pflichten
    \begin{itemize}
            \item Abwaschen benutzten Geschirrs,
            \item Reinigung des Ofens bei hoher Verschmutzung nach Benutzung,
            \item Reinigung des Herds nach Benutzung,
            \item Einhaltung der Mülltrennung,
            \item regelmäßiges Leeren der Abfallbehälter und Reinigung selbiger bei hoher Verschmutzung,
            \item Kehren und anschließendes Wischen bzw.\ (eventuell) Saugen des Fußbodens je nach Verschmutzung,
            \item Einhaltung des eigenen Bereiches zur Aufbewahrung von Nahrungsmitteln (dies betrifft die Kühlschrank-, Gefrier- und sonstigen Fächer),
            \item wenn nötig Reinigung der Fenster.
    \end{itemize}
        \item Bezüglich des jeweils genutzten \textbf{Bades} umfassen die entsprechenden Aufgaben
    \begin{itemize}
            \item regelmäßige Reinigung von Dusche und Toilette, sowie des Waschbeckens,
            \item Wischen des Bodens,
            \item Beseitigung anfallender Abfälle (wie z.B. Toilettenpapier),
            \item sauberes Hinterlassen der Dusche,
            \item regelmäßiges Waschen des Duschvorhangs.
    \end{itemize}
        \item Absolut indiskutable \textbf{Tabus} sind
    \begin{itemize}
            \item das wochenlange Stehenlassen von (verderbenden) Essensresten,
            \item das Benutzen von Küchenutensilien ohne diese anschließend zu reinigen,
            \item das Verteilen von erworbenen Lebensmitteln in der Küche, sodass diese später niemandem mehr zugeordnet werden können.
            \item das übermäßige Horten von nicht eigenem Geschirr im eigenen Zimmer.
            \item offensichtliche Nichtachtung der Mülltrennung.
    \end{itemize}
        \item Bis auf Absprache darf Geschirr maximal ein bis zwei Tage herumstehen, alle für die Allgemeinheit zur Verfügung gestellten Arbeitsflächen sind frei zu halten (auch Fußboden und Herd, sowie Arbeits- und Abstellflächen).
        \item Kühlschrank- und Gefriertruhe sind, falls nötig, nach Absprache abzutauen bzw.\ zu reinigen.
        \item Geschirr oder sonstiges Eigentum, das von anderen Mitbewohnern nicht ohne zu fragen benutzt werden soll, ist als solches zu kennzeichnen oder es sind entsprechende Personen zu informieren.
        \item Für von allen Mitbewohnern benutzte Güter wie Toilettenpapier, Schwämme, Müllbeutel, eventuell Staubsaugerbeutel etc.\ sollte jeder im gleichen Maße aufkommen.
\end{itemize}

\subsection*{Zeitplan}

\paragraph{Küche.} Die Küche ist wöchentlich zu reinigen, wenn notwendig. Falls die Mitglieder der Wohngemeinschaft langfristig nicht in der Lage sind, selbige dauerhaft in einem einigermaßen benutzbaren Zustand zu halten, wird eine Einteilung für die Verantwortlichkeit zur Reinigung vorgenommen.
Bei wiederholter Nichtachtung obiger Regeln, wird die Wohngemeinschaft einer kostenpflichtigen Reinigung unterzogen.

\paragraph{Bad.} Das Bad ist zwei- bis dreiwöchentlich zu reinigen. 

\end{document}
