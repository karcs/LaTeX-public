
\subsection{The polynomials $Q(X)$ and $P(X)$}

The previous section we discuss some results by \person{Blokhuis} und draw some relations to \person{Ball}'s and \person{de Beule}'s interpolation idea.

\begin{lemma}[Blokhuis]
  Let $\cA$ be an arc of $\PG(n,\field{q})$ and let $\cA^*$ be the set of lines dual hyperplanes
  to the points of $\cA$. Let
  \begin{align*}
    t:=q+n-\mathcal{\card{\cA}}
  \end{align*}
  Then there is a polynomial $Q(X)\in\symalg(\field{q}^{{n+1}*})$ homogeneous in $n+1$ variables of degree 
  \begin{equation}
    k:=\begin{cases}
      t &: 2|q\\
     2t &: \otherwise 
       \end{cases}
  \end{equation}
  whose zeros include the set $\mathcal{Z}$ of points that lie on exactly $n-1$ hyperplanes of $\cS^*$.
  Moreover, when $q$ is odd, for each point $z\in \mathcal{Z}$, if $l_z$ is line of $\cS^*$ incident with $z$ then $Q$ mod $l_z$ has a zero of degree $2$ at $z$.
  When 
  \begin{equation}
    \card{\cA}\geq \begin{cases}
      n+\frac{q}{2} &: 2|q \\
      n+\frac{2}{3}q &: \otherwise
  \end{cases}
  \end{equation}
  then $Q$ is unique.
\end{lemma}

In this latter case we call this $Q$ the \person{Segre}-polynomial associated with the arc.

--- TO BE CONTINUED ---
