% Vorlage fuer Stundenplaene (von Christina, Sept. 2007)
% --------------------------------------------------------------
% 1. Die Veranstaltungen werden mit dem Kommando 
% \va{Veranstaltungsbezeichnung}{Veranstaltungsart}{Raum}
% in der unteren Tabelle eingefuegt. 
%
% 2. Geht eine Veranstaltung ueber
% zwei Stunden, so benutze man den Befehl
% \dzv{...}
% wobei in den Klammern die Veranstaltung steht (s.o.). Die Zelle der
% Tabelle unter dieser muss dann frei bleiben. (Zur Erinnerung kann man
% den Merker |so benutzen.) Ausserdem muss dann eine Zeilenhoehenkorrektur 
% \hoehe in zwei Zellen aus diesen beiden Zeilen eingefuegt werden. Auf 
% diesen duerfen die Befehle \dzv{} und \dzvv{} nicht angewendet werden.
%
% 3. Um zwei sich ueberschneidenden Veranstaltungen einzutragen, kann man
% \vaa{Veranstaltung1/ Veranstaltung2}{Art1}{Raum1}{Art2}{Raum2}
% benutzen. 
%
% 4. Gehen diese ueber zwei Stunden, so kann man sie mithilfe von
% \dzvv{...} 
% zusammenfassen, Benutzung wie oben. 
%
% 5. Die Stundentrennlinien koennen von Spalte1 bis Spalte2 mit 
% \cline{Spalte1-Spalte2} gezogen werden.




\documentclass[a4paper, landscape]{article}
\usepackage[ngerman]{babel}
\usepackage[utf8]{inputenc}
\usepackage{color}
\usepackage[right=1cm, left=1cm, top=1.5cm, bottom=1.5cm,noheadfoot]{geometry}
\usepackage{multirow}
\usepackage{booktabs}
%\usepackage{lscape}

\newlength{\spv}
\setlength{\spv}{0.15\textwidth} % Die Breite der Spalten fuer die Veranstaltungen
\newlength{\spz}
\setlength{\spz}{0.05\textwidth} % Die Breite der Spalte fuer die Zeit

\newcommand{\va}[3]{\begin{minipage}{\spv} % Eine Veranstaltung
		{\vspace{5pt}\begin{center} \hspace{0pt} #1 \\  % Veranstaltung
					\tiny \hspace{0pt} #2  % Art
					\hspace{\fill}	\footnotesize\itshape{\hspace{0pt} #3}  % Raum
		\end{center}}
	\end{minipage}}
\newcommand{\vaa}[5]{\begin{minipage}{\spv} % Zwei sich ueberschneidende Veranstaltungen
		{\vspace{5pt}\begin{center} \hspace{0pt} #1 \\  % Veranstaltung1/ Veranstaltung2
					\tiny \hspace{0pt} #2  % Art1
					\hspace{\fill}	\footnotesize\itshape{\hspace{0pt} #3}\\  % Raum1
					\tiny \upshape\hspace{0pt} #4  % Art2
					\hspace{\fill}	\footnotesize\itshape{\hspace{0pt} #5}  % Raum2
		\end{center}}
	\end{minipage}}

\renewcommand{\k}[1]{} % Kommentar
\newcommand{\so}{} % heisst: siehe oben, als Kommentar

\newcommand{\hoehe}{\begin{minipage}{\spv}{\vspace{23pt}}\end{minipage}}% Eine Zeilenhoehenkorrektur, 
					% die in zwei freie Zellen in zwei auf einander folgendnen Zeilen 
					% eingefuegt werden muss, wenn sich zwei Veranstaltungen ueberschneiden:
\newcommand{\dzv}[1]{\multirow{2}{*}{#1}} % Blockveranstaltung ueber zwei Stunden
\newcommand{\dzvv}[1]{\multirow{2}{*}[7pt]{\centering #1}} % Das gleiche, nur fuer Veranstaltungen, 
					% die sich ueberschneiden
\newcommand{\dzz}[1]{\multirow{2}{*}{\centering{#1}}} % Zeitblock

\definecolor{grau}{gray}{0.9}


\begin{document}
\pagestyle{empty}
 %%%% Ueberschrift %%%%
\begin{center}
\fcolorbox{black}{grau}{\parbox{0.7\textwidth}{\vspace{-8pt}
	\begin{center}
		\textbf{Stundenplan f"ur das SoSe 2014} 
	\end{center}\vspace{-8pt}}}
\end{center}
%%%%%%%
% Hier beginnt der Stundenplan.
\begin{tabular}{c|*{5}{|c}}
   \toprule
 & \textbf{Montag} & \textbf{Dienstag} & \textbf{Mittwoch} & \textbf{Donnerstag} & \textbf{Freitag} \\
\midrule \midrule
	7.30\,-\,9.00	
 & \va{FSG}{Ü}{Wu5 115}
 & \va{KAT}{VL}{WIL A120}	
 & \va{SCHTECH}{VL}{BAR SCHÖ}
 & \va{}{}{}
 & \va{NTECH}{Ü}{BAR SCHÖ}                                                                          \\\midrule
	9.20\,-\,10.50
 & \va{FSG}{Ü}{Wu5 115}
 & \va{HASK}{Ü/VL}{INF E042}	
 & \va{HANA}{VL/Ü}{WIL C129}	
 & \va{INFTH}{Ü}{GÖR 226H}	
 & \va{KONGEO}{VL/Ü}{A120}                                                                          \\\midrule
	11.10\,-\,12.40
 & \va{FSG}{Ü}{Wu5 115}	
 & \va{HANA}{VL/Ü}{WIL C129}	
 & \va{}{}{}
 & \va{SCHTECH}{Ü}{BAR SCHÖ}	
 & \va{}{}{}                                                                                        \\\midrule
	13.00\,-\,15.30	
 & \va{OPT}{VL}{WIL C129}	
 & \va{KONGEO}{VL}{WIL A120}	
 & \va{}{}{}	
 & \va{}{}{}	
 & \va{DGEO}{VL}{WIL C129}                                                                          \\\midrule
	15.50\,-\,16.20
 & \va{ALGSTR}{VL/Ü}{WIL C133}	
 & \va{}{}{}	
 & \va{ALGSTR}{VL/Ü}{WIL C133}	
 & \va{DGEO}{VL/Ü}{WIL C132}	
 & \va{RNETZ}{Ü}{WIL C103}                                                                          \\\midrule
        16.40\,-\,18.10
 & \va{OL}{}{GG/Nöth. Str.}	
 & \va{SCPROP}{VL}{WIL A317}	
 & \va{}{}{}	
 & \va{DB}{Ü}{WIL C106}	
 & \va{}{}{}                                                                                        \\
\bottomrule
\end{tabular}                                                                                       \\
% Wenn noch etwas Platz ist, kann man hier einige Hinweise hinschreiben. Sonst koennen sie auch auf der Rueckseite stehen.

\end{document}