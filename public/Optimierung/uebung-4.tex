
\section{Übung (Teil 4)}\hfill{02.06.2014}

\begin{exercise}
    Man gebe die Gleichungsnormalform der folgenden Optimierungsaufgaben an und löse sie mit dem Simplexverfahren:
    \begin{tasks}
        \item Maximierungsproblem
        $$
        z = 2x_1 -x_2 \to \max
        $$
        bei
        $$
        \begin{pmatrix}
            -1 & 2\\
            2 & 2\\
            1 & -4\\
        \end{pmatrix}
        \begin{pmatrix}
            x_1\\
            x_2
        \end{pmatrix}
        \leq
        \begin{pmatrix}
            4\\
            1\\
            4
        \end{pmatrix}
        $$
        und $x_1,x_2\geq 0$.
        \item
    \end{tasks}
\end{exercise}

\begin{solution}
    \begin{tasks}
            \item Es ergibt sich das Minimierungsproblem:
        $$ \tilde z = -2x_1+x_2 \leftarrow \min $$
        bei
        $$
        \begin{pmatrix}
            -1 & 2 & 1 & 0 & 0\\
            2 & 2 & 0 & 1 & 0\\
            1 & -4 & 0 & 0 & 1 
        \end{pmatrix}
        \begin{pmatrix}
            x_1\\
            \vdots\\
            x_5
        \end{pmatrix}
        = \begin{pmatrix}
            4\\
            1\\
            4
            \end{pmatrix}
            $$
            als erstes Tableau ergibt sich
            $$
            \begin{matrix}
                & x_1 & x_2 & 1\\
                \midrule
                x_3 & 1 & -2 & 4\\
                x_4 & -2 & -2 & 1\\
                x_5 & -1 & 4 & 4\\
                \midrule
                \tilde z & 2 & 1 & 0 
            \end{matrix}
            $$
            was ein zulässiges Tableau darstellt, da der Vektor $x=(0,0,4,1,4)\geq 0$ im zulässigen Bereich liegt. Dies Basisvariablen sind hier $x_3,x_4,x_5$ und Nicht-Basisvariablen $x_1,x_2$.
            Da weder der Vektor $q^\trans$ vollständig nicht-negativ ist, noch eine Spalte $i$ existiert, sodass $q_i<0$ und alle $P_{ji}\geq 0$ (Nicht-Lösbarkeit mit unbeschränktem Bereich), muss ein Pivotelement gewählt werden.
            Wähle also $-2$ als Pivotelement (negatives $q_i$) und bilde die Quotienten $P_{j1}/q_1$ für alle. Diese sind $1/2$ und $4$. Ihr Minimum ist also $1/2$.

            Wir führen nun den Austauschschritt durch. Dazu muss die Basisvariable $x_1$ durch $x_4$ ersetzt werden.
            Am besten rechnet man mit einer sogenannten Kellerzeile:
            $$
            \begin{matrix}
                & x_1 & x_2 & 1\\
                \midrule
                x_3 & 1 & -2 & 4\\
                x_4 & -2 & -2 & 1\\
                x_5 & -1 & 4 & 4\\
                \midrule
                \tilde z & 2 & 1 & 0\\
                \midrule
                 & * & -1 & \frac 1 2
            \end{matrix}.
            $$
            Diese ergibt sich einfach durch
            $$
            -\frac{\textrm{``Pivotzeile''}}{\textrm{``Pivotelement''}}.
            $$
            Das neue Tableau hat dann die Gestalt:
            $$
            \begin{matrix}
                & x_4 & x_2 & 1\\
                \midrule
                x_3 & -\frac 1 2 & -3 & \frac 9 2\\
                x_1 & -\frac 1 2 & -1 & \frac 1 2\\
                x_5 & \frac 1 2 & 5 & \frac 7 2\\
                \midrule
                \tilde z & 1 & 3 & -1
            \end{matrix}
            $$
            Man erhält hier beispielsweise die $-3$ in der Spalte von $x_2$ durch $-3=-2+1\cdot(-1)$, wobei $-2$ der alte Matrixeintrag ist, $1$ der assoziierte eintrag in der Pivotspalte und $-1$ der assoziierte Eintrag der Kellerzeile.
            Die Pivotspalte berechnet sich über die Vorschrift
            $$
            \textrm{``neue Einträge''}=\frac{\textrm{``alte Pivotspalte''}}{\textrm{``altes Pivotelement''}}
            $$
            bis auf das Pivotelement, was sich über
            $$
            \textrm{``neuer Eintrag''}=\frac{1}{\textrm{``altes Pivotelement''}}.
            $$
            Alle anderen Einträge ergeben sich aus der Vorschrift
            $$
            \eqalign{
                &\textrm{``neuer Eintrag''}\cr
                &=\cr
                &\textrm{``alter Eintrag''}+\textrm{``Eintrag der alten Pivotspalte''}\cdot\textrm{``Eintrag der Kellerzeile''}.
            }
            $$
                \item Es ergibt sich das Tableau
            $$
            \begin{matrix}
                & x_1 & x_2 & 1\\
                \midrule
                x_3 & -2 & 1 & 10\\
                x_4 & -1 & 3 & 15\\
                x_5 & -1 & 0 & 4\\
                \midrule
                z & 1 & -3 & 0
            \end{matrix}
            $$
            was nicht lösbar ist, da die zweite Spalte $q_2=-3<0$ und $P_{j2}\geq 0$ erfüllt.
        \end{tasks}
    \end{solution}

    \begin{exercise}
        Gegeben
        $$ z = x_2 + x_4 - x_5 + 3 \to \min $$
        bei
        $$
        \begin{pmatrix}
            2 & 3 & 1 & 1 & -2\\
            1 & -1 & 3 & -2 & 4
        \end{pmatrix}
        \begin{pmatrix}
            x_1 & x_2 & x_3 & x_4 & x_5
        \end{pmatrix}^{\trans}=
        \begin{pmatrix}
            10\\
            0
        \end{pmatrix}
        $$
        bei $x_1,\ldots,x_5\geq 0$
    \end{exercise}

    \begin{solution}
        Wir führen Schlupfvariablen ein
        $$
        \begin{pmatrix}
            2 & 3 & 1 & 1 & -2 & 1 & 0\\
            1 & -1 & 3 & -2 & 4 & 0 & 1
        \end{pmatrix}
        \begin{pmatrix}
            x_1 & x_2 & x_3 & x_4 & x_5 & y_1 & y_2
        \end{pmatrix}^{\trans}=
        \begin{pmatrix}
            10\\
            0
        \end{pmatrix}.
        $$
        Zur Bestimmung eines ersten zulässigen Tableaus lösen wir das Hilfsproblem $h=y_1+y_2\to\min$.
        Es ergibt sich das Tableau (mit Kellerzeile und dem Koeffizientenvektor von $z$)
        $$
        \begin{matrix}
            & x_1 & x_2 & x_3 & x_4 & x_5 & 1\\
            \midrule
            y_1 & -2 & -3 & -1 & -1 & 2 & 10\\
            y_2 & -1 & 1 & -3 & 2 & -4 & 0\\
            \midrule
            h & -3 & -2 & -4 & 1 & -1 & 3\\
            \midrule
            z & 0 & 1 & 0 & 1 & -1 & 3\\
            \midrule
            & * & 1 & -3 & 2 & -4 & 0
        \end{matrix}
        $$
        Ein Austauschschritt von $x_1$ und $y_2$ (wähle die Spalte, in der $h$ den Koeffizienten $-3$ hat und noch dem Quotientenvergleich die $-1=P_{21}$ als Pivotelement). Wir bekommen das neue Tableau
        $$
        \begin{matrix}
            & y_2 & x_2 & x_3 & x_4 & x_5 & 1\\
            \midrule
            y_1 & 2 & -5 & 5 & -5 & 10 & 10\\
            x_1 & -1 & 1 & -3 & 2 & -4 & 0\\
            \midrule
            h & 3 & -5 & 5 & -5 & 10 & 10\\
            \midrule
            z & 0 & 1 & 0 & 1 & -1 & 3
        \end{matrix},
        $$
        wobei man die Spalte, welche zu $y_2$ gehört, schon weglassen kann.
        Wir müssen nun noch $y_2$ eliminieren, um nur noch $x$-Variablen im Tableau zu haben.
        Ein weiterer Austauschschritt mit $x_2$ und $y_1$ liefert:
        $$
        \begin{matrix}
            & x_3 & x_4 & x_5 & 1\\
            \midrule
            x_2 & 1 & -1 & 2 & 2\\
            x_1 & -2 & 1 & -2 & 2\\
            \midrule
            z & 1 & 0 & 1 & 5
        \end{matrix}.
        $$
        Dieses Tableau ist offensichtlich schon optimal.
        Um nun alle Lösungen zu bestimmen, gilt es zu beachten, dass $q$ noch eine $0$ enthält.
        Wir sehen also, dass wir $x_4\geq 0$ beliebig wählen können, sodass noch $x_1$ und $x_2$ nicht-negativ bleiben. Daraus ergibt sich aus $x_2=2-x_4$ und $x_1=2+x_4$, dass $x_4\in[0,2]$, also eine ganze Kante die Lösungsmenge darstellt.
    \end{solution}

    \begin{exercise}
        Lösen Sie das Optimierungsproblem
        $$
        z = 3x_1 + 2x_2 + 3 \to \min
        $$
        bei
        $$
        \begin{pmatrix}
            1 & 1 & -2\\
            -1 & 1 & -1\\
            -1 & 0 & 2
        \end{pmatrix}
        \begin{pmatrix}
            x_1\\
            x_2\\
            x_3
        \end{pmatrix}
        \geq
        \begin{pmatrix}
            2\\
            2\\
            1
        \end{pmatrix}
        $$
        und $x_1,x_2,x_2\geq 0$ mit dem dualen Simplexverfahren.
    \end{exercise}

    \begin{solution}
        Wieder führen wir Schlupfvariablen $x_4,x_5\geq 0$ ein.
        $$
        \begin{pmatrix}
            1 & 1 & -2 & -1 & 0 & 0\\
            -1 & 1 & -1 & 0 & -1 & 0\\
            -1 & 0 & 2 & 0 & 0 & -1
        \end{pmatrix}
        \begin{pmatrix}
            x_1\\
            x_2\\
            x_3\\
            x_4\\
            x_5
        \end{pmatrix}
        =
        \begin{pmatrix}
            2\\
            2\\
            1
        \end{pmatrix}
        $$
    
    Das erste Simplextableau des dualen Simplexverfahrens ist also
    $$
    \begin{matrix}
        & x_1 & x_2 & x_3 & 1\\
        \midrule
        x_4 & 1 & 1 & -2 & -2\\
        x_5 & -1 & 1 & -1 & -2\\
        x_6 & -1 & 0 & 2 & -1\\
        \midrule
        z & 3 & 2 & 0 & 3
    \end{matrix}.
    $$
    Dieses ist nicht entscheidbar. Wir wählen die erste Zeile als Pivotzeile.
    Nun bilden wir die Quotienten $3/1$ und $2/1$, wobei das Minimum bei $2$ angenommen wird (hier sind nur die Quotienten mit positiven Elementen ausschlaggebend). Unser Pivotelement ist also $P_{12}=1$.

    Ein Austauschschritt liefert
    $$
    \begin{matrix}
        & x_1 & x_4 & x_3 & 1\\
        \midrule
        x_2 & -1 & 1 & 2 & 2 \\
        x_5 & -2 & 1 & 1 & 0 \\
        x_6 & -1 & 0 & 2 & -1 \\
        \midrule
        z & 1 & 2 & 4 & 7
    \end{matrix}
    $$
    Lösung ist also: $x_1=0$, $x_2=3$, $x_3=1/2$, $x_4=0$, $x_5=1/2$, $x_6=0$.
\end{solution}

21
\begin{exercise}
    $$
    \begin{matrix}
        & x_1 & x_2 & x_3 & x_4 & 1 \\
        x_5 & 1 & 3 & 0 & -2 & 2-s \\
        x_6 & -1 & 0 & -3 & -1 & 3-s \\
        x_7 & s & 2 & -1 & 3 & 5 \\
        z & 4 & s+2 & s+5 & s & 1 
    \end{matrix}
    $$
\end{exercise}

\begin{solution}
    \begin{tasks}
            \item \emph{Optimalität} liegt bei $p,q\geq 0$ vor, also $s+2,s+5,s\geq 0$ und $2-s,3-s\geq 0$, also $0\leq s\leq 2$.
        \emph{Nicht lösbar durch unbeschränktes Fallen der Zielfunktion} ist das Tableau bei $s< -2$, denn das Schema ist primal zulässig und in der zweiten Spalte ist der Zielfunktionskoeffizient negativ, aber alle $P_{ij}$ ($j=2$) darüber nicht-negativ.
        \emph{Nicht-Lösbarkeit wegen Leerheit des zulässigen Bereiches} gilt für $s>3$, denn dann ist die $z$-Zeile nicht-negativ in der zweiten Zeile der Wert rechts kleiner als $0$, die Werte $P_{ij}$ danaben aber alle nicht-positiv (duale Unbeschränktheit).
            \item Bei $s=0$ ist die Lösung nicht eindeutig. Setze $x_1=x_2=x_3=0$, $x_4=t\geq 0$. Die Bedingungen an $t$ sind dann
        \begin{enumerate}
                \item $x_5 = -2t+2\geq 0$, also $t\leq 1$,
                \item $x_6 = -t+3\geq 0$, also $t\leq 3$,
            \item $3t+5\geq 0$, also $t\geq -3/5$,
        \end{enumerate}
        womit sich die Kante $0\leq t\leq 1$ als Lösungsmenge im Falle $s=0$ herausstellt.
            \item Für $2<s\leq 3$ lässt sich das duale Simplexverfahren anwenden, da der erste Eintrag der rechten Seite kleiner als $0$ ist. Nun sind beim Ermitteln des Pivotelements die Quotienten $4/1$ und $(s+2)/3$ zu vergleichen. Da $s\leq 3$, ist das Minimum dieser immer $(s+2)/3$. Das Pivotelement ist also $P_{12}=3$.
            \item Nach einem Simplexschritt ergibt sich
        $$
        \begin{matrix}
            & x_1 & x_2 & x_3 & x_5 & 1 \\
            \midrule
            x_4 & 1/2 & 3/2 & 0 & -1/2 & 3/8 \\
            x_6 & -3/2 & -3/2 & -3 & 1/2 & 5/2 \\
            x_7 & 1/2 & 13/2 & -1 & 3/2 & 19/2 \\
            z & 7/2 & -1/2 & 4 & 1/2 & -1/2 
        \end{matrix}.
        $$
    \end{tasks}
    
\end{solution}