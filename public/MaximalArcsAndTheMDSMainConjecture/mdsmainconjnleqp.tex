In the following we aim to present the results of the two papers~\cite{ball2011mdsmainconjI} and~\cite{ball2012mdsmainconjII} simplifying some aspects of the proofs.

\subsection{\person{Segre}'s lemma of tangents}%

Any attempt to answer the MDS main conjecture uses a lemma which was first stated by \person{B. Segre} and can be considered as a key result. To state it we need the following

\begin{definition}[tangent polynomials]
    Let $\cA$ be an arc in $\field{q}^n$. Then for any $(n-2)$-set $X\setleq \cA$ define the tangent polynomial $T_X$ as the product
    $$
    T_X:\doteq\prod_{\substack{H\in\Sub(\field{q})\\ \codim{H}=1\\ H\setmeet\cA=X}}{l_H}
    $$
    where $l_H$ are linear forms such that $\ker{l_H}=H$. This defines (up to scalar factor from $\units{\field{q}}$) a homogenous polynomial in $n$ variables which is of degree $q+1-(\cA-(n-2))=q-1+n-\card{\cA}=:t$ (by counting the corresponding hyperplanes).
\end{definition}

\begin{remark}
    From $t\geq 0$ one gets the weak bound $\card\cA\leq q+n-1$.
\end{remark}

Now we are able to state the lemma.

\begin{lemma}[\person{Segre}'s lemma of tangents, original]
  Let $n\geq3$ and $\cA$ be the representation of an arc in $\P(\field{q}^n)$. Then for pairwise distinct $x_i\in \cA$ ($i=1,2,3$) and $X\setleq \cA\setminus\set{x_1,x_2,x_3}$, $\abs{X}=n-3$ we have
  $$    \frac{T_{X\setjoin\set{x_1}}(x_2)T_{X\setjoin\set{x_2}}(x_3)T_{X\setjoin\set{x_3}}(x_1)}{T_{X\setjoin\set{x_2}}(x_1)T_{X\setjoin\set{x_3}}(x_2)T_{X\setjoin\set{x_1}}(x_3)}={(-1)}^{t+1}.
  $$
\end{lemma}

\begin{remark}
    The expression in the lemma is well-defined since scalar factors in $T_Y$ vanish in the fraction.
\end{remark}

\begin{proof}
    Denote by $x_i^{\ast}$ the linear form corresponding to $x_i$ in the dual basis of $X':=X\setjoin\set{x_1,x_2,x_3}$.
    For $i\in\set{1,2,3}$ let $\cH_i$ be the set of hyperplanes $H$ containing $\gen{X,x_i}$ but not $x_j$ for $j\neq i$.
    For each hyperplane $H\in \cH_i$ there are two possibilities:
    \begin{casebycase}
        \item There is an $a\in\cA\setminus X'$ with $a\in H$. In this case define the linear form
    $$
    l_H(x):=\det(x,x_i,X,a)
    $$
        \item Otherwise, $l_H$ is already defined in via the tangent polynomials.
    Now, one may define the polynomial $P_i:=\prod_{H\in \cH_i}{l_H}$ (of degree $q-1$) and one computes that
    $$
    P_i(x_j)=T_{X\setjoin\set{x_i}}(x_j)\prod_{a\in\cA\setminus X'}{\det(x_j,x_i,X,a)}.
    $$%%
    \end{casebycase}
    When $\set{1,2,3}\setminus\set{x_i}=\set{j,k}$ it is clear that $P_i(x_j)/P_i(x_k)=-1$ which can be deduced from the fact that the product of all units of a finite field is $-1$ and the observation that all hyperplanes in $\cH_i$ are given by $\ker(x_j^{\ast}+\lambda x_k^{\ast})$ where $\lambda\in\units{\field{q}}$.
    Thus we have that
    \begin{align*}
        \frac{P_1(x_2)P_2(x_3)P_3(x_1)}{P_2(x_1)P_3(x_2)P_1(x_3)} &= \frac{T_{X\setjoin\set{x_1}}(x_2)T_{X\setjoin\set{x_2}}(x_3)T_{X\setjoin\set{x_3}}(x_1)}{T_{X\setjoin\set{x_2}}(x_1)T_{X\setjoin\set{x_3}}(x_2)T_{X\setjoin\set{x_1}}(x_3)}\\
        & \quad\settimes\prod_{\substack{i,j\in\set{1,2,3}\\ i<j}}\prod_{a\in\cA\setminus X'}{\frac{\det(x_i,x_j,X,a)}{\det(x_j,x_i,X,a)}}\\
        & = {(-1)}^3=-1.
    \end{align*}
    Since $\cA\setminus X'$ has $\card{\cA}-n=q-1-t$ elements, the last two products evaluate to ${(-1)}^{3(q-1-t)}={(-1)}^t$.
    This finishes the proof.
\end{proof}

Later we will do some calculations with the polynomials $T_Y$ for $Y\in\binom{\cA}{d-2}$. Thus it is desirable to choose the scalar factors in front of each of them in an appropriate way to simplify the calculation.
%%%%%
\begin{corollary}[Simplification of \person{Segre}'s lemma]\label{mds-segre-simplified}
    Let $\cA$ be a representation of an arc in $\P(\field{q}^n)$ ($n\geq 3$) and $\leq$ some linear
    order on $\cA$. Then we can define the tangent polynials $T_Y$
    ($Y\setleq \cA$, $\abs{Y}=n-2$) such that for all $x_1,x_2,X$ we have
    $$    \frac{T_{X\setjoin\set{x_2}}(x_1)}{T_{X\setjoin\set{x_1}}(x_2)}={\left(\frac{\sgn(X\setjoin\set{x_2},x_1)}{\sgn(X\setjoin\set{x_1},x_2)}\right)}^{t+1}
    $$
    where $X\setleq \cA$ has $d-3$ elements and $x_1,x_2\in \cA\setminus X$ are distinct.
\end{corollary}

\begin{remark}
    Here, by $(A_1,\ldots,A_n)$ we mean the tuple
    which is given by writing down the elements of $A_1,\ldots,A_n$
    where each $A_i$ is written in the order $\leq$
    ($i=1,\ldots,n$). Moreover, by $\sgn(A_1,\ldots,A_n)$ we mean
    the sign of the permutation which corresponds to the tuple $(A_1,\ldots,A_n)$
    where the identity is given by $(\bigsetjoin_{i=1}^n{A_i})$. To abbreviate notation, for $A_i$ a singleton we just write its
    element without curly brackets.  
    This convention shall be used in the following.
\end{remark}

\begin{proof}
    We define the directed graph $G=(V,E)$ by $V:=\binom{\cA}{n-2}$ and $E:=\set{(u,v)\in V^2:\abs{u\setminus v}=1}$.
    Moreover, we define a labeling $\lambda:E\to \units{\field{q}}$. For $(u,v)\in E$ let $u_v\in u\setminus v$ and $v_u\in v\setminus u$ be the elements of the singletons. Then 
    $$
    \lambda(u,v):=\frac{T_u(v_u)}{T_v(u_v)}\text{.} 
    $$
    Then in the two types of triangles in $G$ we have two different relations that hold.
    The first type of triangles consists of vertices $u,v,w\in V$ such that $\abs{u\setjoin v\setjoin w}=d-1$.
    Here it clearly holds that
    $$
    \lambda(u,v)\lambda(v,w)\lambda(w,u)=1
    $$
    which is simply a tautology when replacing $\lambda$ by its definition.
    In the second type of triangles consisting of vertices $u,v,w\in V$ such that $\abs{u\setjoin v\setjoin w}=n$ we have
    $$
    \lambda(u,v)\lambda(v,w)\lambda(w,u)={(-1)}^{t+1}
    $$
    which turns out to be just a reformulation of \person{Segre}'s lemma of tangents.
    Now it is clear, that $\lambda$ is uniquely defined by any restriction $\rest{\lambda}_{E_T}$ where $T=(V,E_T)$ is a (directed) rooted spanning tree of $G$ with root $r\in V$ (by the above relations in triangles of $G$ and as $G$ is obviously strongly connected).
    Moreover, when replacing $T_u$ by $\tilde{T}_u:=\mu T_u$ one just modifies $\lambda$ to $\tilde{\lambda}$ where 
    $$
    \tilde{\lambda}(v,w)=
    \begin{cases} 
        \lambda(v,w) & : u\notin\set{v,w}\\
        \mu\lambda(v,w) &: v=u\\
        \mu^{-1}\lambda(v,w) &: w=u
    \end{cases}\text{.}
    $$
    This idea can be used to modify the tangent polynomials step by step to achieve any values of $\lambda$ among $E_T$.
    Define the sets $V_l:=\set{v\in V: d_T(r,v)=l}$ ($l\in\nats$, here $d_T(\arg_1,\arg_2)$ means the metric of the shortest path on $T$). Since $G$ is finite there is some $L\in\nats$ such that $\set{V_l:l\in\set{0,\ldots,L}}$ is a partition of $V$. Moreover, one notes that the sets $E_l:=\set{(u,v)\in E_T:u\in V_{l-1},v\in V_l}$ for $l=1,\ldots,L$ form a partition of $E_T$. Thus one can modify the labeling $\lambda$ at first on $E_1$ then on $E_2$ etc. As there is no edge in $T$ between the sets $V_m$ and $V_n$ where $\abs{n-m}\geq 2$ this procedure works and one does not destroy former changes on some $E_i$. 
    This shows that $\lambda$ can be changed to any labeling satisfying the two triangle conditions.
    At last we check that these are satisfied for the labeling $\bar{\lambda}$ given in the lemma, where for $(u,v)\in E$

    $$
    \bar{\lambda}(u,v):={\left(\frac{\sgn(u,v_u)}{\sgn(v,u_v)}\right)}^{t+1}\text{.}
    $$
    
    For a triangle $uvw$ of the first type ($\abs{u\setjoin v\setjoin w}=n-1$) we obtain $v_u=w_u$, $u_v=w_v$, $u_w=v_w$ (with the same notation as before) and thus one gets
    $$
    \bar{\lambda}(u,v)\bar{\lambda}(v,w)\bar{\lambda}(w,u) ={\left(\frac{\sgn(u,v_u)}{\sgn(v,u_v)}\frac{\sgn(v,w_v)}{\sgn(w,v_w)}\frac{\sgn(w,u_w)}{\sgn(u,w_u)}\right)}^{t+1}\\
    =1
    $$
    Similarly, for a triangle $uvw$ of the second type ($\abs{u\setjoin v\setjoin w}=n$) one gets the desired identity by the following reasoning. W.l.o.g.\ we may write $u=X\setjoin\set{x_1}$, $v=X\setjoin\set{x_2}$, $w=X\setjoin\set{x_3}$ for a $(n-3)$-element set $X:=u\setmeet v\setmeet w$ and elements $x_i\in \cA$ ($i=1,2,3$) such that $\set{x_1,x_2,x_3}\setjoin X=u\setjoin v\setjoin w$. Furthermore we may assume that $X^1<x_1<X^2<x_2<X^3<x_3<X^4$, where $X^1,X^2,X^3,X^4$ partitions $X$ (otherwise we interchange the labeling of $u$, $v$ and $w$). Then we may compute

    \begin{align*}
        \bar{\lambda}(u,v)\bar{\lambda}(v,w)\bar{\lambda}(w,u) 
 & = \left(\frac{\sgn(X^1,x_1,X^2,X^3,X^4,x_2)}{\sgn(X^1,X^2,x_2,X^3,X^4,x_1)}\right)^{t+1}              \\
 & \quad\settimes \left(\frac{\sgn(X^1,X^2,x_2,X^3,X^4,x_3)}{\sgn(X^1,X^2,X^3,x_3,X^4,x_2)}\right)^{t+1} \\
 & \quad\settimes \left(\frac{\sgn(X^1,X^2,X^3,x_3,X^4,x_1)}{\sgn(X^1,x_1,X^2,X^3,X^4,x_3)}\right)^{t+1} \\
 & = {(-1)}^{((|X^3|+|X^4|)+(|X^2|+1+|X^3|+|X^4|))(t+1)}                                                 \\
 & \quad\settimes {(-1)}^{(|X^4|+(|X^3|+1+|X^4|)(t+1)}                                                   \\
 & \quad\settimes {(-1)}^{((|X^2|+|X^3|+1+|X^4|)+|X^4|)(t+1)}                                            \\
 & = {(-1)}^{t+1}
\end{align*}

  which finishes the proof.
\end{proof}

This lemma enables us now to make the following definition

\begin{definition}
    Let $\cA\setleq\field{q}^n$ be the representation of an arc. We say that its tangent polynomials are defined canonically with respect to the linear order $\leq$ on $\cA$ if they satisfy for all $x_1,x_2,X$ the identity
    $$
    \frac{T_{X\setjoin\set{x_2}}(x_1)}{T_{X\setjoin\set{x_1}}(x_2)} = {\left(\frac{\sgn(X\setjoin\set{x_2},x_1)}{\sgn(X\setjoin\set{x_1},x_2)}\right)}^{t+1}
    $$
    where $X\setleq \cA$ has $n-3$ elements and $x_1,x_2\in \cA\setminus X$ are distinct.
\end{definition}

\begin{remark}
    This simple but effective trick enables us to prove some results of \person{Ball} and \person{de Beule} avoiding some inconvenient terms to occur.
    Note that in the above definition the tangent polynomials are still only defined up to scalar factor, but their quotients are fixed.
    In the following we do only work with the tangent polynomials defined in that manner.
\end{remark}

\begin{definition}[abbreviation of tangent polynomial evaluations]\label{tang-pol-eval}
    Let $\cA$ be the representation of an arc in $\P(\field{q}^n)$ such that the tangent polynomials $T_Y$ are defined canonically with respect to some linear order $\leq$ on $\cA$. Then we set $P(X):=T_{X\setminus\set{x}}(x)$ where $x$ is the biggest element in  $X$.
\end{definition}

\section{Interpolation formulae}

An elementary but particularly nice idea is to use interpolation to capture information about the arc.

There are two basic possibilities to apply interpolation. 

\begin{lemma}[interpolation of the tangent polynomial]
    Let $\cA$ be an arc in $\field{q}^n$ and $A,B\setleq \cA$ be disjoint subsets with
    $\abs{A}=t+2$ and $\abs{B}=n-2$. It then holds that
    $$
    \sum_{a\in A}{T_{B}(a)\prod_{z\in A\setminus \set a}{\det(z,B,a)}^{-1}} =0
    $$
    or equivalently when the tangent polynomials are defined canonically
    with respect to the linear order $\leq$
    $$
    \sum_{a\in A}{P(\set a\setjoin B)\prod_{z\in A\setminus\set a}\det(z,\set a\setjoin B)^{-1}}=0.
    $$
\end{lemma}
%
\begin{proof}
    As $T_B(x+y)=T_B(x)$ for all $x\in\field{q}^n$ and $y\in\gen B$ we may
    interpolate the polynomial $\bar{T}_B(x+\gen B):=T_{B}(x)$ as a homogenous polynomial of
    degree $t$ over $\field{q}^n/\gen B$. To do this, pick $a\in A$ to get
    $$
    T_B(x)=\sum_{a'\in A\setminus\set a}{T_B(a')\prod_{z\in A\setminus\set{a,a'}}{\frac{\det(z,B,x)}{\det(z,B,a')}}}
    $$
    since both sides are polynomials in $x$ of degree $t$
    and both are constant on cosets of $\gen B$ and agree on $t+1$
    points of $\field{q}^n/\gen B$ (namely $\overline{a}'$
    for $a'\in A\setminus\set a$) for which the RHS is a \person{Lagrange} interpolation formula. Replacing $x$ by $a$ and dividing by
  $\prod_{z\in A\setminus\set a}{\det(z,B,a)}$ one gets 
  $$
  T_B(a)\prod_{z\in  A\setminus\set a}{{\det(z,B,a)}^{-1}} = {\det(a',B,a)}^{-1}\sum_{a'\in A\setminus\set a}{T_B(a')\prod_{z\in A\setminus\set{a,a'}}{{\det(z,B,a')}^{-1}}}
  $$
  which is what we wanted to prove.
  The second formulation only uses the definition of $P$.
\end{proof}

The second idea is to interpolate the determinants themselves
%% todo
\begin{lemma}[interpolation of determinants]\label{mds-tan-poly-interpol-det}
    Let $A,B,C\setleq \field{q}^n$ such that $\gen{A\setjoin
    C}=\field{q}^n$, $\card A+\card C=n+1$ and $\card B+\card C=n-1$ and $\leq$
    be some linear order on $A$. Then it holds that
    $$
    \sum_{a\in A}{\sgn(a,A\setminus\set a\setjoin C)\det(a,B\setjoin C)\det(A\setminus\set a\setjoin C)}=0\text{.}
    $$
    Here, $\sgn$ is taken with respect to $\leq$.
\end{lemma}

\begin{proof}
  Picking $a\in A$ and interpolating $\det(\arg,B\setjoin C)$ as a linear form in
  $\field{q}^n/\lin{C}$ gives
  $$
  \det(x,B\setjoin C) 
  = \sum_{a'\in A\setminus\set a}{\det(a',B\setjoin C)\frac{\det(x,A\setminus\set{a,a'}\setjoin C)}{\det(a',A\setminus\set{a,a'}\setjoin C)}}
  $$
  which holds as it holds for $x$ an element of the basis $\bar a'$ for $a'\in
  A\setminus\set a$ of $\field{q}^n/\gen{C}_{\lin}$. Replacing $x$ by $a$ and
  rearranging the terms yields
  \begin{align*}
    \det(a,B\setjoin C)\det(A\setminus\set a\setjoin C)
    &= 
    \sum_{a'\in A\setminus\set a}{\det(a',B\setjoin C)\frac{\det(a,A\setminus\set{a,a'}\setjoin C)}{\sgn(a',A\setminus\set{a,a'}\setjoin C)}}\\
    &= 
    \sum_{a'\in A\setminus\set a}{\det(a',B\setjoin C)\det(A\setminus\set {a'}\setjoin C)\frac{\sgn(a,A\setminus\set{a,a'}\setjoin C)}{\sgn(a',A\setminus\set{a,a'}\setjoin C)}}\\
    &= 
    -\sum_{a'\in A\setminus\set a}{\det(a',B\setjoin C)\det(A\setminus\set{a'}\setjoin C)\frac{\sgn(a',A\setminus\set{a'}\setjoin C)}{\sgn(a,A\setminus\set a\setjoin C)}}
  \end{align*}
  which is the desired result.
\end{proof}

\begin{remark}
    In the statement one could also require that $A<C$ such that the argument of $\sgn$ simplifies a bit --- here we wrote down the formula in its most general version (forming sets instead of `blocks' whenever possible).
\end{remark}

\subsection{Manipulation of interpolation identities}

Now, the idea is to play with the interpolation identities to reach a
contradiction in the case where $t\leq n-3$ (i.e. $\card{\cA}\geq q+2$).

\paragraph{First attempt.}

The proof of the main conjecture for MDS codes of \person{Ball} and \person{de Beule} for the case where
$n\leq p=\rchar[\field{q}]$ that any arc $\cA$ in $\field{q}^n$ satisfies $\abs{\cA}\leq q+1$ and is equivalent to a normal rational
curve, mainly bases on the following key result which can be derived elementarily from the interpolation of tangent polynomials.

\begin{lemma}[\person{Ball} \& \person{de Beule}'s $ABC$ lemma]\label{mds-abc-lemma}
    Let $\cA$ be the representation of an arc in $\P(\field{q}^n)$ ($p=\rchar[\field{q}]$) and let
    $P(X)$ be the abbreviation as met in \autoref{tang-pol-eval} for the order $\leq$. Let $0\leq
    r\leq\min\set{n,p}-1$ and $A,B,C\setleq \cA$ disjoint sets such
    that $\card{A}+\card{B}=r+t+1$, $\card{C}=n-1-r$. Then it holds that
    \begin{align*}
        &(-1)^r\sum_{\substack{A'\setleq A\\ \card{A'}=r}}{P(A'\setjoin C)\prod_{z\in (A\setminus A')\setjoin B}{\det(z,A'\setjoin C)^{-1}}}\\
        &= \sum_{\substack{B'\setleq B\\ \card{B'}=r}}{P(B'\setjoin C)\prod_{z\in (B\setminus B')\setjoin A}{\det(z,B'\setjoin C)^{-1}}}\text{.}
    \end{align*}
\end{lemma}

\begin{proof}
    The proof happens by induction on $r$. For $r=0$ the statement is a
    tautology.
    Now, suppose the lemma is proven for $r-1\geq 0$ and let
    $r\leq\min\set{n,p}-1$.
    Moreover, let $A,B,C\setleq \cA$ be disjoint sets such
    that $\card{A}+\card{B}=r+t+1$, $\card{C}=n-1-r$. Then pick $a\in A$ and
    apply the lemma for $r-1$ and sets $A\setminus\set a$, $B$,
    $\set{a}\setjoin C$ (if $A$ and $B$ ar empty the lemma is obvious --- moreover, the roles of $A$ and $B$ are symmetric).
    This yields
    \begin{align*}
        &{(-1)}^{r-1}\sum_{\substack{A'\setleq A\setminus\set a\\ \card{A'}=r-1}}{P(A'\setjoin\set a\setjoin C)\prod_{z\in (A\setminus A')\setjoin B}{{\det(z,A'\setjoin\set a\setjoin C)}^{-1}}}\\
        &= \sum_{\substack{B'\setleq B\\ \card{B'}=r-1}}{P(B'\setjoin\set a\setjoin C)\prod_{z\in (B\setminus B')\setjoin A\setminus\set a}{{\det(z,B'\setjoin\set a\setjoin C)}^{-1}}}\text{.}
    \end{align*}
    Now, we sum this over $a\in A$ to get
    \begin{align*}
        &{(-1)}^{r-1}r\sum_{\substack{A'\setleq A\\ \card{A'}=r}}{P(A'\setjoin C)\prod_{z\in (A\setminus A')\setjoin B}{{\det(z,A'\setjoin C)}^{-1}}}\\
        &= \sum_{\substack{B'\setleq B\\ \card{B'}=r-1}}{\sum_{a\in A}{P(B'\setjoin\set a\setjoin C)\prod_{z\in (B\setminus B')\setjoin A\setminus\set a}{{\det(z,B'\setjoin\set a\setjoin C)}^{-1}}}}\\
        &= \sum_{\substack{B'\setleq B\\ \card{B'}=r-1}}{\sum_{a\in A}{P(B'\setjoin\set a\setjoin C)\prod_{z\in (B\setminus B'\setjoin A)\setminus\set a}{{\det(z,B'\setjoin\set a\setjoin C)}^{-1}}}}\\
        &= -\sum_{\substack{B'\setleq B\\ \card{B'}=r-1}}{\sum_{b\in B\setminus B'}{P(B'\setjoin\set b\setjoin C)\prod_{z\in(B\setminus B'\setjoin A)\setminus\set b}{{\det(z,B'\setjoin\set b\setjoin C)}^{-1}}}}\\
        &= -r\sum_{\substack{B'\setleq B\\ \card{B'}=r}}{P(B'\setjoin C)\prod_{z\in (B\setminus B')\setjoin A}{{\det(z,B'\setjoin C)}^{-1}}}\text{.}
    \end{align*}
    Here we used the interpolation of tangent polynomials in the fourth
    line for the sets $B'\setjoin C$ and $B\setminus B'\setjoin A$ when
    $r-1\leq\card{B}$. In the case $\card{B}< r-1$ the LHS is zero (as it
    had been zero before).
    If $r\leq p-1$ it is a unit and we can divide the above by $-r$ to complete
    the induction.
\end{proof}

We thus immediately obtain

\begin{corollary}[the case $n\leq p$]\label{mds-bound-n-leq-p}
  If $n\leq p$ then $\card\cA\leq q+1$.
\end{corollary}

\begin{proof}
  Assume that $n\leq p$ and $\card{\cA}\geq q+2$ or equivalently $t=q+n-1-\card{\cA}\leq
  n-3$. Then apply \autoref{mds-abc-lemma} with
  $r=\card{A}=n-1\leq p-1$ and appropriate subsets $B$, $C$ (as $\min\set{n,\card{\cA}-n}\leq\card{\cA}/2$, using the dual arc \see{def-dual-arc} if necessary we may assume w.l.o.g.\ that $n+t\leq 2n-3\leq\card{\cA}$). Then
  we have that $\card{B}=t+1\leq n-2$ and thus the lemma gives
  $$
    {(-1)}^{n-1}P(A)\prod_{z\in B}{{\det(z,A)}^{-1}}=0
  $$
  which is a contradiction.
\end{proof}

\begin{remark}
    This is as we will see the optimal result using \emph{only} the interpolation
    of the tangent polynomial.
\end{remark}

Moreover, in that case the $(q+1)$-arcs can be identified as normal rational
curves.

\begin{corollary}[classification of $(q+1)$-arcs for $n\leq p$]\label{mds-class-n-leq-p}
    Let $\cA$ be the representation of a $(q+1)$-arc in $\P(\field{q}^n)$ where $n\leq
    p=\rchar[\field{q}]$. Then $\cA$ is a normal rational curve.
\end{corollary}

\begin{proof}
    In the case $\card{\cA}=q+1$ we have that $t=n-2$.
    Again, we apply \autoref{mds-abc-lemma} for $r=n-1\leq p-1$ and $A\setleq \cA$ with
    $\card A=n$ and appropriate sets $B,C\setleq \cA$ (then
    $\card{B}=t=n-2<r$ and $C=\emptyset$). This gives us
    $$ {(-1)}^{n-1}\sum_{\substack{A'\setleq A \\
            \card{A'}=n-1}}{P(A')\prod_{z\in (A\setminus A')\setjoin
            B}{{\det(z,A')}^{-1}}} = 0.
    $$
    Applying the above for $A$ and $B_b:=B\setminus\set{b}\setjoin\set{x}$ for some fixed point $x\in \cA$ we get the $n-2$ equations
    $$
    \sum_{a\in A}{P(A\setminus\set{a})\frac{\det(b,A\setminus\set{a})}{\det(a,A\setminus\set{a})}{\det(x,A\setminus\set{a})}^{-1}} = 0\quad (b\in B).\label{mds-class-n-leq-p-keysys}
    $$
    We could also have written $\dual a(b)$ (where $\dual a$ means an element of the dual basis of $A$) for the fraction showing that the matrix $M\in \field{q}^{B\settimes A}$ defined by
    $$
    M:={(m_{ab})}_{(a,b)\in A\settimes B},\quad
    m_{ab} := \frac{\det(b,A\setminus\set{a})}{\det(a,A\setminus\set{a})}
    $$
    has full rank as its rows are just the coordinate vectors of the $b\in B$ with respect to the basis $A$.%
    From this it follows that also the matrix $N:=MD$, where $N:=\diag(P(A\setminus\set{a}):a\in A)$, has full rank which is the matrix of the linear system~\theref{mds-class-n-leq-p-keysys}.
    Hence the kernel of this system has dimension $\card{A}-\card{B}=n-(n-2)=2$ showing that the image of $x$ for all $x\in \cA\setminus A$ under the map
    $$
    \gamma:{(\units{\field{q}})}^n\to{(\units{\field{q}})}^n
    $$
    where ${(\units{\field{q}})}^n:=\field{q}^n\setminus\bigsetjoin_{a\in A}{\gen{A\setminus\set{a}}_{\lin}}$ and
    $$
    \sum_{a\in A}{\lambda_a a}\mapsto \sum_{a\in A}{\lambda_a^{-1} a}
    $$
    must lie on a (projective) line i.e.~in a two dimensional subspace of $\field{q}^n$.
    Using an appropriate element of $M\in\PGL(\field{q}^n)$ which does $\gen a \mapsto \gen a$ for $a\in A$ and $\gen{\hat x}\mapsto \gen{\sum_{a\in A}{a}}$ for some $\hat x\in \cA\setminus A$ we may assume w.l.o.g.~that $\hat a:=\sum_{a\in A}{a}\in \cA$. Set $\hat{A}:=A\setjoin\set{\hat a}$. 

    Now, rescaling the $\gamma(x)$ ($x\in\cA\setminus \hat A$) appropriately we get scalars $\alpha_x$ such that
    $$
    \alpha_x\gamma(x)=\check a-\hat a\lambda_x
    $$
    lie on an affine line parallel to $\gen{\hat a}_{\lin}$ in $\field{q}^n$ (for some $\check a\in\field{q}^n$ and $\lambda_x\in\field{q}$). This is possible since the point $\gen{\hat a}_{\lin}$ lies on the same projective line as all $\gen{x}_{\lin}\in\cA\setminus \hat A$ so the latter lie in an affine line parallel to $\gen{\hat a}$.
    Changing the parameters if necessary by a tranlation $x\mapsto x+\mu$ ($\mu\in\field{q}$), we may assume that $0=\lambda_x$ for some $x\in \cA\setminus \hat{A}$ whence $\check a\in {(\units{\field{q}})}^n$.

    The line $\lambda\mapsto \check a-\hat a\lambda$ intersects the $n$ hyperplanes $\gen{A\setminus\set a}_{\lin}$ (for $a\in A$) in $n$ distinct points (i.e.~in the coordinate representation $\check a =\sum_{a\in A}{\nu_a a}$ all $\nu_a$ are distinct for $a\in A$). This holds as the assumption that $\nu_{a'}=\nu_{a''}$ for $a',a''\in A$, $a'\neq a''$ leads to the contradiction that $\set{\hat a,\check a}\setjoin A\setminus\set{a',a''}\setleq \cA$ forms a linearly dependend $n$-set.

    Hence we may deduce that
    $$
    \set{\nu_a:a\in A}\setjoin\set{\lambda_x:x\in \cA\setminus\hat A}=\field{q}
    $$ is a disjoint union as all $\check a+\hat a\lambda_x$ have no zero coordinates in the basis $A$. However, considering the set $\hat\cA:=\hat A\setjoin\set{\gamma^{-1}\compose\alpha_x\compose\gamma(x):x\in \cA\setminus\hat A}$ we have a representation equivalent to $\cA$ which is a \person{Cauchy}-representation shown in \autoref{cauchy-rep}. So the arc represented by $\cA$ is a normal rational curve.
\end{proof}

\begin{remark}
    The argument can also be used to prove that the cardinality of an arc in $\P(\field{q}^n)$ ($n\leq p$) can at most become $q+1$ (similarly to \autoref{mds-bound-n-leq-p}).
\end{remark}

\paragraph{Second attempt.} In this paragraph we try to prove the same result for $n\leq 2p-2$ and will bring in the interpolation of determinants.

\begin{lemma}[\person{Ball}'s \& \person{de Beule}'s $ABCDE$ lemma]
    Let $\cA$ be an arc in $\field{q}^n$ ($p=\rchar{\field{q}}$) with $n>p$ and $0< r\leq p-1$ and $0\leq m\leq \min\set{n-1-r,t+2}$. Moreover, let $A$, $B$, $C$, $D$ and $E$ be disjoint subsets of $\cA$ with $\card A=\card B=m$, $\card C=t+2-m$, $\card D=n-1-r-m$, $\card E=r-1$ and let there be given bijections $m\to A$ and $m\to B$ such that $A_\tau, B_\tau$ denote the images of $\tau\setleq m$. Then it holds that
    \begin{align*}
        0 & = \sum_{\substack{C'\setleq C                                                                       \\
                \card{C'}=r}}
        {\sum_{\tau\setleq m}{{(-1)}^{\card{\tau}}P(A_{\tau}\setjoin B_{m\setminus\tau}\setjoin C'\setjoin D)}} \\
          & \quad\settimes\prod_{\substack{z\in A_{m\setminus\tau}\setjoin B_{\tau}                                  \\
                  \setjoin(C\setminus C')\setjoin E}}
          {{\det(z,A_{\tau}\setjoin B_{m\setminus\tau}\setjoin C'\setjoin D)}^{-1}}
\end{align*}
\end{lemma}
%%
\begin{proof}
    The proof happens by induction on $m$.
    We have that $n>p$ so we may apply the \autoref{mds-abc-lemma} for $r\leq p-1$ and sets $A,B,C$ with $\card{A}=t+2$, $\card{B}=r-1$ and $\card{C}=n-1-r>0$.
    This gives us
    $$ 0 = \sum_{\substack{A'\setleq A\\\card{A'}=r}}{P(A'\setjoin C)\prod_{z\in (A\setminus A')\setjoin B}{{\det(z,A'\setjoin C)}^{-1}}}. $$
    This proves the lemma for $m=0$ when replacing $A$ by $C$, $B$ by $E$ and $C$ by $D$.

    For the induction step, assume the lemma holds for $m-1$ and let there be given $A$, $B$, $C$, $D$ and $E$ with $\card{A}=\card{B}=m$, $\card{C}=t+2-m$, $\card{D}=n-1-r$, $\card{E}=r-1-m$. Then pick $a:=a_m\in A$ and $b:=b_m\in B$ and apply the induction hypothesis for $\bar{A}:=A\setminus\set{a}$, $\bar{B}:=B\setminus\set{b}$, $C\setjoin\set{a}$, $D\setjoin\set{b}$, $E$ and $\bar{A}$, $\bar{B}$, $C\setjoin\set{b}$, $D\setjoin\set{a}$, $E$ (and $m-1$), respectively. 
    This yields (distinguishing whether $a\in C'$ or not and analogously for $b$, the terms where $a\in C'$ on the LHS and $b\in C'$ on the RHS whipe out) 
    \begin{align*}
 & \sum_{\substack{C'\setleq C                                                                   \\ \card{C'}=r}}\sum_{\tau\setleq m-1}{{(-1)}^{\card{\tau}}P(\bar{A}_{\tau}\setjoin \bar{B}_{m\setminus\tau}\setjoin\set{b}\setjoin C'\setjoin D)}\\
 & \quad\settimes\prod_{\substack{z\in \bar{A}_{m\setminus\tau}\setjoin\set{a}\setjoin \bar{B}_{\tau} \\ \setjoin(C\setminus C')\setjoin E}}{{\det(z,\bar{A}_{\tau}\setjoin \bar{B}_{m\setminus\tau}\setjoin\set{b}\setjoin C'\setjoin D)}^{-1}}\\
 & = \sum_{\substack{C'\setleq C                                                               \\
         \card{C'}=r}}\sum_{\tau\setleq m-1}{{(-1)}^{\card{\tau}}P(\bar{A}_{\tau}\setjoin\set{a}\setjoin \bar{B}_{m\setminus\tau}\setjoin C'\setjoin D)}\\
 & \quad\settimes\prod_{\substack{z\in \bar{A}_{m\setminus\tau}\setjoin \bar{B}_{\tau}\setjoin\set{b} \\\setjoin(C\setminus C')\setjoin D}}{{\det(z,\bar{A}_{\tau}\setjoin\set{a}\setjoin \bar{B}_{m\setminus\tau}\setjoin C'\setjoin D)}^{-1}}.
    \end{align*}
    Rearranging this to one side proves the induction.
\end{proof}

Applying the above corollary to the condition $\card\cA=q+2$, i.e. $t=n-3$ leads to

\begin{corollary}
    When $\card{\cA}=q+2$ and $m=n-1-r\geq n-p$ we have that
    $$
  0 = \sum_{\tau\setleq m}{{(-1)}^{\card{\tau}}P(A_{\tau}\setjoin B_{m\setminus\tau}\setjoin C)}\prod_{\substack{z\in A_{m\setminus\tau}\setjoin B_{\tau}\setjoin E}}{{\det(z,A_{\tau}\setjoin B_{m\setminus\tau}\setjoin C)}^{-1}}.
    $$
\end{corollary}

\begin{proof}
    The as $\card {C\setminus C'}=t+2-m-r=(n-1)-(n-1-r)-r=0$ the corresponding factors vanish in the product of the last corollary. For the same reason $D=\emptyset$.
\end{proof}

\begin{corollary}[the case $n\leq 2p-2$]
    Any arc $\cA$ in $\P(\field{q}^n)$ with $n\leq 2p-2$ ($p=\rchar\field{q}$) satisfies the bound $\card\cA\leq q+1$.
\end{corollary}

\begin{proof}
    We may assume that $n>p$ by the previous work.
    Apply the previous corollary for $r=p-1$, then $\card E=p-2$, $\card C=p-1$ and $\card A=\card B=n-p$.
    Write $E$ as $E=F\setjoin G$ where $\card F = 2p-2-n$ (here we use the assumption) and $\card G = n-p>0$.
    Rewriting the previous corollary we obtain
    $$
  0 = \sum_{\tau\setleq m}{{(-1)}^{\card{\tau}}P(A_{\tau}\setjoin B_{m\setminus\tau}\setjoin C)}\prod_{\substack{z\in A_{m\setminus\tau}\setjoin B_{\tau}\setjoin F\setjoin G}}{{\det(z,A_{\tau}\setjoin B_{m\setminus\tau}\setjoin C)}^{-1}}.
    $$
    We now aim to prove the following equation for $\card F\leq p-2$ for which the above is the base of induction ($\card G=n-p$, inducting on $s:=\card D = \card F-(2p-2-n)$).
    $$
  0 = \sum_{\tau\setleq m}{{(-1)}^{\card{\tau}}P(A_{\tau}\setjoin B_{m\setminus\tau}\setjoin C)}\prod_{w\in D}{\det(w,A_{\tau}\setjoin B_{m\setminus\tau}\setjoin C)}\prod_{\substack{z\in A_{m\setminus\tau}\setjoin B_{\tau}\setjoin F\setjoin G}}{{\det(z,A_{\tau}\setjoin B_{m\setminus\tau}\setjoin C)}^{-1}}.
    $$
    for $D\setleq A$ an $s$-element set (which is possible since $\card A=n-p\geq s$).
    
    For the induction step we pick $d\in D$ and $g\in G$, $f\in F$, assume the hypothesis to be proven for $s-1$ and apply this to our sets $A$, $B$, $C$, $D\setminus\set d$, $E$, $F\setminus\set f$, $G\setminus\set g\setjoin\set f$
    $$
    \eqalign{%
  0 &= \sum_{\tau\setleq m}{{(-1)}^{\card{\tau}}P(A_{\tau}\setjoin B_{m\setminus\tau}\setjoin C)}\prod_{w\in D\setminus\set d}{\det(w,A_{\tau}\setjoin B_{m\setminus\tau}\setjoin C)}\cr &\quad \settimes\det(g,A_{\tau}\setjoin B_{m\setminus\tau}\setjoin C)\prod_{\substack{z\in A_{m\setminus\tau}\setjoin B_{\tau}\setjoin F\setjoin G}}{{\det(z,A_{\tau}\setjoin B_{m\setminus\tau}\setjoin C)}^{-1}}.
}%
$$
Now, we bring in the interpolation formula for the determinants as given in \autoref{mds-tan-poly-interpol-det}. Multiplying the above by
$$
\sgn(g,\set d\setjoin (G\setjoin\set f)\setminus\set g\setjoin C)\det(\set d\setjoin (G\setjoin\set f)\setminus\set g\setjoin C)
$$
and summing over $g\in G\setjoin\set f$ gives (by interpolation of determinants)
$$
    \eqalign{%
  0 &= -\sgn(d,G\setjoin\set f\setjoin C)\det(G\setjoin\set f\setjoin C)\sum_{\tau\setleq m}{{(-1)}^{\card{\tau}}P(A_{\tau}\setjoin B_{m\setminus\tau}\setjoin C)}\prod_{w\in D\setminus\set d}{\det(w,A_{\tau}\setjoin B_{m\setminus\tau}\setjoin C)}\cr &\quad \settimes\det(d,A_{\tau}\setjoin B_{m\setminus\tau}\setjoin C)\prod_{\substack{z\in A_{m\setminus\tau}\setjoin B_{\tau}\setjoin F\setjoin G}}{{\det(z,A_{\tau}\setjoin B_{m\setminus\tau}\setjoin C)}^{-1}},
}%
$$
where we can forget the $\sgn$ and $\det$ at the beginning to see that we are done with the induction step. Of course, the above argument does only work for $s=\card D\leq\card A=n-p$, i.e. $\card G\leq 2(n-p)$.
Applying the formula which we have just proven for $s=n-p$ (i.e. $D=A$) we get
$$
P(B\setjoin C)\prod_{w\in A}{\det(w,B\setjoin C)}\prod_{z\in A\setjoin F\setjoin G}{{\det(z,B\setjoin C)}^{-1}}=0
$$%%%
since all terms where $\tau\neq\emptyset$ vanish. This clearly is a contradiction.  
Lastly, we have to check that $A\setjoin B\setjoin C\setjoin E\setjoin F\setjoin G$ is not bigger that $q+2$. Adding the cardinalities we get
$$
\card{A\setjoin B\setjoin C\setjoin F\setjoin G}=2(n-p)+(p-1)+(p-2)+(n-p)=3n-3-p.
$$
But this is no restriction since $p\leq\sqrt q$ and thus $n\leq 2\sqrt q - 2$ so $3n-3\leq 6\sqrt q -9\leq q$. This does hold for all $q>0$. 
\end{proof}

\begin{remark}
    The author's proposal is to call the above the $A$--$G$ lemma.
\end{remark}
%%%