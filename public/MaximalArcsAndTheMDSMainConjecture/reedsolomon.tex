\section{Extended \person{Reed-Solomon}-Codes and normal rational curves}%
\makeatletter%
\def\@currentlabel{Section \thesection}%
\makeatother% TODO better section command
\label{sec-reedsolomon}

In this section we discuss the most popular (probably) maximal examples of the MDS main conjecture in the case where $\cA\setleq \P K^n$ is an arc for $n\geq 3$ and $K$ a field with $n\leq\card K$.

\subsection{Normal rational curves}

\begin{definition}[normal rational curve]\label{norratcurve}
    A \keyword{normal rational curve} is the image of the projective map $\proj{f}:\proj{K^2}\to\proj{K^n}$ where $f$ is given by
    $$
    (X_0,X_1) \mapsto (X_0^{n-1},X_0^{n-2}X_1,\ldots,X_0 X_1^{n-2},X_1^{n-1})
    $$
    modulo $\PGL$.
\end{definition}

\begin{convention}
For the rest of this section the symbol $f$ denotes the above mapping.    
\end{convention}

A natural question is whether a normal rational curve is a complete arc.
An obvious fact is that it is indeed a projective arc.

\begin{definition}[projective zeros]
    Let $p\in K[X_0,X_1]$ be a homogeneous polynomial. We say that $z\in\P K^2$ is a projective zero of $p$ if $z=\gen{(z_0,z_1)}$ such that $p(z_0,z_1)=0$. 
\end{definition}

\begin{lemma}\label{norratcurve-arc}
    A normal rational curve is a projective arc.
\end{lemma}

\begin{proof}
    Let $l=\sum_{i=0}^{n-1}{\lambda_i \dual e_i}$ be an arbitrary linear form. Then the equation
    $$
    lf(X_0,X_1)=\sum_{i=0}^{n-1}{\lambda_i X_0^{i}X_1^{n-1-i}}
    $$
    is a homogeneous polynomial of degree $n-1$ in two variables and thus has at most $n-1$ projective zeros. Thus $\ker l\setmeet \im\P f$ consists of at most $n-1$ points.
\end{proof}

\begin{notation}[polynomial coefficients]
    For a polynomial $p\in R[X_0,\ldots,X_{m-1}]$ we write $[X_0^{e_0}\cdots X_{m-1}^{e_{m-1}}]p$ for the coefficient which is in front of the monomial $X_0^{e_0}\cdots X_{m-1}^{e_{m-1}}$ in $p$.
\end{notation}

\begin{lemma}[description of $(n-2)$-secant hyperplanes of a normal rational curve]\label{norrattandesc}
    The hyperplanes $\cH$ intersecting a normal rational curve $\cA=\im\proj{f}\setleq \proj{K^n}$ in the representation of \autoref{norratcurve} in exactly the points of an $(n-2)$-set $\proj{f}(A)$ ($A$ is the corresponding $(n-2)$-set of preimages in $\P K^2$) are given as the kernels of the linear forms 
    $$
    l_{A,\alpha}:\doteq\sum_{i=0}^{n-1}{[X_0^i X_1^{n-1-i}]\left(\det(X,\alpha)\prod_{a\in A}{\det(X,a)}\right)\dual e_i}
    $$
    where $\alpha\in A$ and $\seq{e_i}_{i=0}^{n-1}$ denotes the standard basis of $K^n$ and $\seq{\dual e_i}_{i=0}^{n-1}$ its dual basis and $X=(X_0,X_1)$, $a=(a_0,a_1)$ ($a\in A$).  
\end{lemma}

\begin{proof}
    It is clear that the $l_{A,\alpha}(f(a_0,a_1))=0$ if and only if $\gen{(a_0,a_1)}\in A$. On the other hand, any linear form $l$, which has as a kernel a hyperplane $H$ with $H\setmeet\cA=A$, must be of the form
    $$
    l=\sum_{j=0}^{n-1}{[X_0^i X_1^{n-1-i}]\left(\prod_{a\in A}{\det(X,a)}\det(X,\alpha')\right)\dual e_i}
    $$
    since its coefficients in $\dual e_i$ can be interpreted as the corresponding coefficients of a homogeneous polynomial in two variables of degree $n-1$. But this polynomial must have all $(n-2)$ elements of $A$ as its zeros (as the assumption on the form $l$ delivers). Hence, there is another zero $\alpha'=\gen{(\alpha'_0,\alpha'_1)}\in\P K^2$ since the polynomial factors into linear polynomials. Since then $\gen{f(\alpha'_0,\alpha'_1)}\in H\setmeet\cA=\proj f(A)$ we must have $\alpha'\in A$ (clearly $\proj f$ is injective).    
\end{proof}
%% here is work TODO
In the sense of the proof of this lemma, we make the following definition.
Henceforth, we assume that $\cA$ is an incomplete normal rational curve.

\begin{definition}[associated polynomials and linear forms for incomplete normal rational curves]
    Assume the normal rational curve $\cA\setleq \proj{K^n}=\proj{V}$ can be extended to a $(q+2)$-arc $\hat{\cA}:=\cA\setjoin\set{\hat{a}}$ by a point $\hat a$. Then for each $(n-2)$-set $A\setleq\proj{K^2}$ there exists a hyperplane $H_A:=\gen{\P f(A),\hat{a}}$ and $H_A\setmeet \hat{\cA}=\proj f(A)\setjoin\set{\hat{a}}$. Choose $l_A\in\dual V$ such that $\ker{l_A}=H_A$ and call it an \keyword[associated linear form]{associated linear form of $A$}.
    Define $p_A:=\sum_{i=0}^{n-1}{l_A(e_i)X_0^{n-1-i}X_1^i}$ and call it an \keyword[associated polynomial]{associated polynomial of $A$}.
\end{definition}

\begin{remark}
    Clearly, both $p_A$ and $l_A$ are only determined up to scalar factor from $\units K$.
\end{remark}

\begin{remark}
    The polynomials $p_A$ are precisely the polynomials which are used to construct the forms $l_{A,\alpha}$ in the proof of \autoref{norrattandesc}. Thus $p_A$ factors completely into $n-1$ linear factors and has the elements of $A$ as a projective zeros (with one zero of order two).
\end{remark}

\begin{definition}
    Define $P_m$ as the subspace of $K[X_0,X_1]$ of homogeneous polynomials of degree $m$.
\end{definition}

\begin{lemma}
    The space $P:=\gen{p_A\in P_{n-1}:A\setleq \proj{K^2},\card A=n-2}_{\Sub P_{n-1}}$ is of dimension $n-1$.
\end{lemma}

\begin{proof}
    The map $\phi:V^{\ast}\to P_{n-1}$ mapping associated linear forms to associated polynomials given by $l\mapsto \sum_{i=0}^{n-1}{l(e_i)X_0^{n-1-i}X_1^i}$ is clearly linear and an isomorphism (the inverse is $p\mapsto\sum_{i=0}^{n-1}{([X_0^i X_1^{n-1-i}]p)\dual e_i}$ as used in the proof of \autoref{norrattandesc}). Thus,
    $$
\dim{P}=\dim\gen{l_A\in V^{\ast}:A\setleq \proj{K^2},\card A=n-2}_{\Sub\dual V}.
$$
But the intersection of the kernels of the $l_A$'s contains the extending point $\hat{a}\neq 0$. Hence, 
$$
\dim\gen{l_A\in V^{\ast}:A\setleq \proj{K^2},\card A=n-2}_{\Sub \dual V}=n-\dim\Meet_{\substack{A\setleq\proj{K^2}\\ \card A=n-2}}{\ker{l_A}}\leq n-1.
$$

On the other hand, we can choose an $(n-1)$-set $B\setleq \proj{K^2}$ and directly verify that the system of polynomials $p_A$, where $A\setleq B$, $\card{A}=n-2$, is linearly independent by the interpolation formula
$$
\left(\sum_{\substack{A\setleq B\\ \card{A}=n-2}}{\lambda_A p_A}\right)(a)=\lambda_{B\setminus\set{a}}p_{B\setminus\set{a}}(a).
$$
This proves the linear combination to be trivial if it evaluates to zero (since $p_{B\setminus\set{a}}(a)\neq 0$).
Thus,
$$
\dim{P}=\dim\gen{l_{A}\in V^{\ast}:A\setleq \proj{K^2},\card{A'}=n-2}_{\Sub \dual V}=n-1.
$$
\end{proof}

\begin{lemma}
    The space $P$ does not contain a non-zero separable polynomial which splits in $\P K^2$.
\end{lemma}

\begin{proof}
    Assume there is such polynomial $p$ with a set of $n-1$ zeros $B$.
    As we have seen in the proof of the previous lemma, the set $\set{p_A:A\setleq B,\card A=n-2}$ forms a basis of $P$. Thus we can interpolate $p$ by
    $$
    p=\sum_{a\in A}{p(a)\frac{p_{A\setminus\set{a}}}{p_{A\setminus\set{a}}(a)}}=0.
    $$
    A contradiction.
\end{proof}

We thus arrive at a much more `algebraic' version of the assumption. 

\begin{lemma}\label{char-incomp-nor-rat-curve}
    Let $n\leq \card K -1$. The following two are equivalent.
    \begin{statements}
            \item A normal rational curve in $\P K^n$ is incomplete.\label{norrat-incomp}
            \item There is an $(n-1)$-dimensional subspace $P$ of the space $P_{n-1}$ of homogeneous polynomials of degree $n-1$ containing no splitting separable polynomials.\label{subspace-without-sep-spl-poly}
    \end{statements}
\end{lemma}

\begin{proof}
    We have just shown that \autoref{norrat-incomp} implies \autoref{subspace-without-sep-spl-poly}. Conversely, if there is a subspace $P$ as described in \autoref{subspace-without-sep-spl-poly}, then it intersects in a one-dimensional subspace with each subspace $P_A:=P_1\prod_{a\in A}{\det(X,a)}$ (for $A\setleq \P K^2$ an $(n-2)$-set) --- the intersection must be one-dimensional as $\dim P +\dim P_A=n+1$ and $P\join P_A=P_{n-1}$ since $P_A$ contains a splitting separable polynomial as $n-2\leq \card K$. Hence we may define $p_A$ via $\gen{p_A}=P_A\setmeet P$ and $l_A\defdoteq\sum_{i=0}^{n-1}{([X_0^i X_1^{n-1-i}]p_A)\dual e_i}$. One derives that the hyperplanes $H_A\defeq\ker l_A$ contain only the points $\P f(A)$ of $\cA$ and intersect in a unique point $\hat a$ which extends $\cA$.
\end{proof}

Furthermore, we want to prove that a normal rational curve is complete in more than half of the expected cases.
For this we need the following elementary facts.

\begin{lemma}
    Let $Q\leq P_2$ be of codimension one. Then if $Q$ is not of the form $Q=P_1\det(X,\alpha)$ for some $\alpha\in\P K^2$ the following statements hold.
    \begin{statements}
            \item If $\rchar K = 2$, then either $Q=P_1^2$ (i.e. $Q$ consists entirely of squares) or $Q$ contains one square, $\frac{\card K} 2$ splitting separable polynomials and the same number of irreducible polynomials (up to scalar factor).
        \item If $\rchar K \neq 2$, then $Q$ contains $1\pm 1$ squares, $\frac{\card K \mp 1} 2$ splitting separable polynomials and the same number of irreducible polynomials.
    \end{statements}
\end{lemma}

\begin{proof}
    \begin{statements}
            \item If $\P Q$ contains two distinct squares, then it is clearly $P_1^2$ (since they already span a two-dimensional space). Otherwise, the kernel of the map $[X_0X_1]:P_2\to K$ intersects with $Q$ in a one-dimensional space (showing that $Q$ contains a square). The intersection of $Q$ with $P_1\det(X,\alpha)$ ($\alpha\in \P K^2$) is always of dimension one, since $Q\meet P_1\det(X,\alpha)<Q$ and $\dim (Q\meet P_1\det(X,\alpha))= \dim Q+\dim (P_1\det(X,\alpha))-\dim (Q\join P_1\det(X,\alpha))\geq 2+2-3$. Thus there are $\frac{\card K} 2$ splitting separable polynomials in $\P Q$, one square and $\frac{\card K} 2$ irreducible ones.
            \item Assume $Q$ contains two squares. Choosing appropriate coordinates, we may assume that these are $X_0^2$ and $X_1^2$. But then any linear combination $\mu X_0^2+\lambda X_1^2$ ($\lambda,\mu\in\units K$) cannot be a square since then $\mu X_0^2+\lambda X_1^2={(\alpha X_0 +\beta X_1)}^2=\alpha^2 X_0^2+2\alpha\beta X_0X_1 + \beta^2X_1$ implying that $\alpha\beta=0$ which leads to $\mu=0$ or $\lambda=0$.
        Moreover, if $Q$ contains one square, say $X_0^2$, and is not of the form $P_1\det(X,\alpha)$, then it contains a polynomial $aX_0^2+bX_0X_1+X_1^2$ which can be completed to a second square by adding $\left(\frac{b^2} 4 -a\right)X_0^2$.
    \end{statements}
    Hence we are done.
\end{proof}

\begin{remark}
    It is easy to see that for $K=\field{q}$ and $q$ even there exist exactly $q + 1$ subspaces of codimension one in $P_2$ of type $P_1\det(X,\alpha)$, one subspace of type $P_1^2$ and $q^2-1$ subspaces containing only one square.
    When $q$ is odd, there are $q+1$ such subspaces of type $P_1\det(X,\alpha)$ as well as $\binom {q+1} 2$ subspaces with two squares and $\binom q 2$ subspaces without square.  
\end{remark}

\begin{lemma}
    If a normal rational curve is incomplete in $P K^n$ then $\rchar K =2$ and $n=3$ or
    $$
    \ceil{\frac{\card K + 5} 2}\leq n.
    $$
\end{lemma}

\begin{proof}
    If $n\geq \card K + 1$ we have already seen that no $\card K +1$ arc is complete and embedded in an arc which is projectively equivalent to the arc $\set{\gen{e_0},\ldots,\gen{e_{n-1}},\gen{e_0+\cdots+e_{n-1}}}$ \see{mds-ngeqq}. Thus assume $3\leq n\leq \card K$, since for $n=2$ the statement holds obviously ($f$ is the identity map).

    Let $\rchar K\neq 2$. Pick two $(n-2)$-sets $A,B\setleq \P K^2$ such that $A\setmeet B =C$ is an $(n-3)$-set and the zeros $z_A$ and $z_B$ of order two of $p_A$ and $p_B$ are distinct (for this we need $2 \leq n-1\leq \card K +1$, e.g.~we can then choose $A$ first and then $B\setleq \P K^2\setminus\set{z_A}$).
    The space $\P\gen{p_A,p_B}$ contains exactly one polynomial with a set of zeros $C\setjoin\set c$ for all $c\in\P K^2\setminus C$ --- that is $\card K + 1 - (n-3)$ polynomials of this kind (the uniqueness follows from \autoref{char-incomp-nor-rat-curve}). Thus it contains exactly $\card K + 1 - (\card K + 1 - (n-3))=n-3$ polynomials which have exactly $C$ as their zeros (the rest must be of that latter kind by \autoref{char-incomp-nor-rat-curve}). In the case where $\rchar K\neq 2$ we then immediately deduce that the space
    $$
    \P\gen{p_A/(p_A\meet p_B),p_B/(p_A\meet p_B)}
    $$
    contains $1\pm 1$ squares, $\frac{\card K \mp 1} 2$ splitting separable polynomials, and thus $\frac{\card K \mp 1} 2$ irreducible polynomials.
    From this we get the inequality
    $$
    \frac{\card K - 1} 2\leq n-3
    $$
    since all irreducible polynomials $p$ of the above space lead to a polynomial of the second kind by multiplying it by $\prod_{c\in C}{\det(X,c)}$. Thus
    $$
    \frac{\card K + 5} 2\leq n
    $$
    if $\rchar K\neq 2$.

    If $\rchar K = 2$ we need to pick $A$ and $B$ such that the cases 
    $z_A=z_B$ and $\set{z_A}=A\setminus B$, $\set{z_B}=B\setminus A$ do not occur. So assume the second case occurs, then choose an $(n-2)$-set $D\setleq\P K^2$ such that $D\setmeet B$ is an $(n-3)$-set which contains $z_B$ (i.e.~$D\setmeet B\not\eq C$; here we need that $4\leq n$, since $1\leq\card{\set{z_B}}\leq\card{D\setmeet B}=n-3$). Define $A':=B$ and $B':=D$, then $A'$ and $B'$ cannot be of the second case by the choice of $D$. If these sets are of the first case, then $p_B$ and $p_D$ have the same zero of order two $z_B$ and so $\gen{p_B,p_D}=P_1(p_B\meet p_D)$ contains a polynomial $p_{B\setmeet D\setjoin\set{z_A}}$ (with the same zero of order two). $A'':=A$ and $B'':=B\setmeet D\setjoin\set{z_A}$ are sets avoiding the two bad cases since there double zeros are $z_A$ and $z_B$ (which are distinct by assumption) and $z_A\in B''$.
    Assuming that $A$ and $B$ are appropriately chosen it is simple to verify that the space $\P\gen{p_A/(p_A\meet p_B),p_B/(p_A\meet p_B)}$ (of homogeneous quadratic polynomials) contains one square, $\frac{\card K} 2$ splitting separable polynomials and $\frac{\card K} 2$ irreducible polynomials.
    This yields the inequality
    $$
    \frac{\card K} 2 \leq n-3, \text{ i.e. } \frac{\card K+6} 2\leq n,
    $$
    completing the proof.
\end{proof}

The proof of this last fact seems to be simple, but although there are much better bounds for $n$ --- which I also rediscovered --- it is not known to me, that the completeness of normal rational curves is established.
We omit to present a proof of these other bounds since they are also not very satisfactory and the proof would be rather sophisticated.

%TODO: continue
%\begin{align*}
%    &\binom{q+1} 2\binom{\frac{q+1} 2}{q+1-(n-3)} 2^{q+1-(n-3)}\\
%    &??+\binom q 2\left(\binom{\frac{q-1} 2}{q+1-(n-3)}2^{q+1-(n-3)}+2\binom{\frac{q-1} %2}{q-(n-3)}2^{q-(n-3)}+\binom{\frac{q-1} 2}{q-1-(n-3)}2^{q-1-(n-3)}\right)\\
%    &??\geq \frac 1 2\binom{q+1}{n-2}
%\end{align*}

\subsection{Related matrices}

Next, we discuss some interesting generator matrices of the MDS codes corresponding to a normal rational curve which seem natural to be mentioned at this point.

\paragraph{Extended \person{Vandermonde} matrices.}
Denote the elements of $\field{q}$ by $\nu_i$ ($i=0,\ldots,q-1$) and define a representation $V$ of a normal rational curve as given in \autoref{norratcurve} by
$$
V\defeq\set{\begin{pmatrix} 1 \\ \nu_i\\ \vdots\\ \nu_i^{n-1}\end{pmatrix}}[i=\range 0 {n-1}]\setjoin\set{
        \begin{pmatrix}
            0 \\
            \vdots \\
            0 \\
            1
        \end{pmatrix}
}
$$
and a check matrix (by writing down the vectors of the above representation in a `natural order')
$$
G_V\defeq
\begin{pmatrix}
    1      & \cdots & 1      & 0      \\
    \nu_0    & \cdots & \nu_{q-1}    & 0      \\
    \vdots & \ddots & \vdots & \vdots \\
    \nu_0^n  & \cdots & \nu_{q-1}^n  & 1
\end{pmatrix}\textrm{.}
$$
One then checks easily that $G_V$ has the property that any $n$ distinct column vectors are linearly independent since any submatrix

$$
U \defeq\begin{pmatrix}
    1             & \cdots & 1       \\
    \nu_{k_1}       & \cdots & \nu_{k_n} \\
    \vdots        & \ddots & \vdots  \\
    \nu_{k_1}^{n-1} & \cdots & \nu_{k_n}^{n-1} 
\end{pmatrix}
$$
is a \keyword{\person{Vandermonde} matrix} in the case it does not contain the last vector having determinant
$$
    \det U = \prod_{i<j}{(\nu_{k_i}-\nu_{k_j})}
$$%
which is not zero as all $\nu_k$ are distinct. In the other case, the determinant is also non-zero, as one notes by applying \person{Laplace}'s formula to the last column. Actually, it is immediately clear that any $n$ column vectors of the above matrix are linearly independent by \autoref{norratcurve-arc} but we just wanted to point out another way to see this.

The representation $V$ of a classical arc introduced in this section is called \keyword{\person{Vandermonde} representation} and $G_V$ is called extended \person{Vandermonde} matrix in the sequence of elements $\seq{\range{\nu_0}{\nu_{q-1}}}$.
It is a generator matrix of the \keyword{extended \person{Reed-Solomon} code}\footnote{Basically, there are many possibilities to introduce \person{Reed-Solomon} codes and mostly the definitions do coincide only up to $\units{\field{q}}\wr S_m$-equivalence where $m$ is the length of the code --- here $m=\card K+1$.}

\paragraph{Extended \person{Cauchy} matrices.}% good
%
In the last paragraph we saw that the arcs corresponding to extended \person{Reed-Solomon} codes are precisely normal rational curves. We want to construct another type of generator matrix from a normal rational curve (so that the corresponding code will be $\units{\field{q}}\wr S_n$-equivalent to the extended \person{Reed-Solomon} code).
The first step is to realize that 
$$ C' \defeq \set{
    \begin{pmatrix} P_0(z)             \\
        \vdots                         \\
        P_{n-1}(z)\end{pmatrix}\in\field{q}^n}[z\in \field{q}]
\setjoin\set{\begin{pmatrix} p^{n-1}_0 \\
        \vdots                         \\
        p^{n-1}_{n-1}\end{pmatrix}
} $$
is a representation of a normal rational curve (since it is obviously $\PGL$-equivalent to the `one' given in \autoref{norratcurve} by the mapping $f$) where $P_i=\sum_{j=0}^{n-1}{p_i^j X^j}$ ($i=0,\ldots,n-1$) form a basis of the polynomials of degree $k\leq n-1$. Now we pick the first $n$ distinct elements $\nu_0,\ldots,\nu_{n-1}\in\field{q}$ and set
$$ P_i := \prod_{\substack{j=0\\ j\neq i}}^{n-1}{(X-\nu_j)}\in\field{q}[X]\text{.} $$
%% TODO buggy \set with [ in first argument
It is then immediately clear that the $P_i$ are linearly independent, since $\frac 1 {\prod_{j\not\eq i}{(\nu_i-\nu_j)}}P_i$ form a \person{Lagrange} basis with respect to the points $\nu_i$ ($i=0,\ldots,n-1$). The point $\gen{e_{\infty}}$ corresponding to the preimage $\infty$ now evaluates to
$$ e_{\infty} = \begin{pmatrix} 1\\ \vdots\\ 1\end{pmatrix} = \sum_{i=0}^{n-1}{e_i}.$$
To abbreviate notation we also set
$$
P(z)\defeq\begin{pmatrix} P_0(z)\\ \vdots\\ P_{n-1}(z)\end{pmatrix}\textrm{.}
$$

Starting from the representation $C'$ and scaling all vectors $P(\nu_i)$ of the representation ($i=\range 0 {n-1}$) by
$$ \prod_{\substack{j=0\\ j\neq i}}^{n-1}{\frac{1}{\nu_i-\nu_j}}\textrm{,} $$
the vectors $P(\nu_i)$ ($i=\range n {q-1}$) by 
$$ \prod_{j=0}^{n-1}{\frac{1}{\nu_i-\nu_j}} $$
and leaving $e_{\infty}=\sum_{i=0}^{n-1}{e_i}$ fixed we get a representation
$$
C = \set{e_1,\ldots,e_n,\sum_{i=1}^n{e_i}}\setjoin\set{\begin{pmatrix} \frac{1}{\nu_i-\nu_0} \\ \vdots\\ \frac{1}{\nu_i-\nu_{n-1}}\end{pmatrix}}[i=\range n {q-1}]
$$

of a curve being $\PGL$-equivalent to the initial curve.
However, writing down the generator matrix of the corresponding MDS code (in a `natural order') we obtain
$$ G_C =
    \begin{pmatrix}
        1      & 0      & \cdots & 0      & 1      & \frac{1}{\nu_n-\nu_0} & \cdots & \frac{1}{\nu_{q-1}-\nu_0} \\
        0      & \ddots & \ddots & \vdots & \vdots & \vdots                & \ddots & \vdots            \\
        \vdots & \ddots & \ddots & 0      & \vdots & \vdots                & \ddots & \vdots            \\
        0      & \cdots & 0      & 1      & 1      & \frac{1}{\nu_n-\nu_{n-1}} & \cdots & \frac{1}{\nu_{q-1}-\nu_{n-1}}
    \end{pmatrix} $$

which contains a \keyword{\person{Cauchy} matrix}. Hence, we call the representation $C$ of a classical arc the \keyword{\person{Cauchy} representation}. The matrix consisting of the columns of index $n$ till $q$ (starting from zero) is called \keyword{extended \person{Cauchy} matrix} in the sequences $\seq{\range{\nu_0}{\nu_{n-1}}}$ and $\seq{\range{\nu_n}{\nu_{q-1}}}$ (which is a totally regular matrix).\label{cauchy-rep}

