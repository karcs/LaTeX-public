% My personal maths package

%\bibliography{Bibliography.bib}
\usepackage{amssymb,amsthm,amsmath,amsfonts}

%\usepackage{mathabx} % more and nicer symbols symbols

%\usepackage{mathabx} % more and nicer symbols symbols
% MEINE PACKAGES!!!!!!
\usepackage{color}
\usepackage[usenames,dvipsnames]{pstricks}
\usepackage{epsfig}
\usepackage{pst-grad} % For gradients
\usepackage{pst-plot} % For axes

% importing various types of grapics

\usepackage{epsfig}  		% For postscript
\usepackage{epic,eepic}       % For epic and eepic output from xfig


% The following is very useful in keeping track of labels while writing
%\usepackage[notcite]{showkeys}

%textwidth ...
\setlength{\textwidth}{14cm}


\newtheorem{theorem}{Theorem}[section]
\newtheorem{proposition}[theorem]{Proposition}
\newtheorem{lemma}[theorem]{Lemma}
\newtheorem{corollary}[theorem]{Corollary}
\newtheorem{conjecture}[theorem]{Conjecture} 

\theoremstyle{definition}
\newtheorem{definition}[theorem]{Definition}
\newtheorem{example}[theorem]{Example}

\newtheorem{note}[theorem]{Note}

\theoremstyle{remark}
\newtheorem{remark}[theorem]{Remark}

% common algebraic objects
\newcommand{\reals}{\mathbb{R}}  % The real numbers
\newcommand{\nats}{\mathbb{N}} % The natural numbers
\newcommand{\ints}{\mathbb{Z}} % The integers
\newcommand{\rats}{\mathbb{Q}}
\newcommand{\complex}{\mathbb{C}}
\newcommand{\field}{\mathbb{F}}
\DeclareMathOperator{\cardinals}{Cn}


% new operators and relations

% setproduct
\newcommand{\setprod}{\bigtimes}
\newcommand{\dirprod}{\bigotimes} % direct product for vector spaces
\newcommand{\dirtimes}{\otimes}
\newcommand{\dirsum}{\bigoplus} % direct sum for vector spaces
\newcommand{\dirplus}{\oplus}
\newcommand{\rest}[1]{\left. #1\right\vert}
\DeclareMathOperator{\vecspace}{vec} % forgetful functor to vector spaces
\DeclareMathOperator{\ord}{ord} % forgetful functor to partial orders
\DeclareMathOperator{\lex}{lex}
\renewcommand{\mod}{\text{mod }}
\newcommand{\lcm}{\text{lcm}}
\DeclareMathOperator{\dist}{dist}
\DeclareMathOperator{\diam}{diam}

%\renewcommand{\to}{\rightarrow}
\newcommand{\norm}[1]{\left\Vert #1 \right\Vert}
\renewcommand{\impliedby}{\Leftarrow}
\renewcommand{\implies}{\Rightarrow}
\newcommand{\equival}{\Leftrightarrow}
\newcommand{\intersect}{\cap}
\newcommand{\unify}{\cup}
\newcommand{\union}{\bigcup}
\newcommand{\distunion}{\dot{\bigcup}}
\newcommand{\intersection}{\bigcap}
%\newcommand{\setminus}{\char`\\}
\newcommand{\level}{\boldsymbol{l}}
\newcommand{\restrict}{\vert}
\newcommand{\compose}{\circ}
\DeclareMathOperator{\conv}{conv} % convex hull
\DeclareMathOperator{\lin}{lin} % linear hull
\DeclareMathOperator{\aff}{aff}
\DeclareMathOperator{\img}{im}
\renewcommand{\int}{\text{int}} % interior
\newcommand{\module}[1]{\left| #1\right|}
\DeclareMathOperator{\diag}{diag}
\DeclareMathOperator{\Eig}{Eig}
\newcommand{\mi}{\boldsymbol{i}}
\newcommand{\e}{\boldsymbol{e}}
\DeclareMathOperator{\trace}{tr}
\DeclareMathOperator{\trans}{T} % transponse matrix
\DeclareMathOperator{\stab}{stab} % stabilizer
\DeclareMathOperator{\orb}{orb} % stabilizer
\newcommand{\generate}[1]{\langle #1\rangle}
%\DeclareMathOperator{\Re}{Re}
%\DeclareMathOperator{\Im}{Im}
%\DeclareMathOperator{\ord}{ord} % order
