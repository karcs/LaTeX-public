\documentclass[10pt,a4paper]{article}
%My personal maths package
%\bibliography{Bibliography.bib}

%%%%%%%%%%%%%%%%%%%%%%%%%%%%%%%%%%%%%%%%%%%%%%%%%%%%%%%%%%%%%%%%%%%%%%%%%%%%%%
%%%%% MATH PACKAGES %%%%%%%%%%%%%%%%%%%%%%%%%%%%%%%%%%%%%%%%%%%%%%%%%%%%%%%%%%
%%%%%%%%%%%%%%%%%%%%%%%%%%%%%%%%%%%%%%%%%%%%%%%%%%%%%%%%%%%%%%%%%%%%%%%%%%%%%%

% very good package
%\usepackage{mathtools}

%%% font stuff
\usepackage[T1]{fontenc}        % for capitals in section /paragraph etc.
\usepackage[utf8]{inputenc}     % use utf8 symbols in code

\usepackage{amssymb,amsmath,amsfonts} % amsthm not needed -- use my own envs
%\usepackage{mathtools}
%\mathtoolsset{showonlyrefs}
% further alternative math packages: unicode-math, abx-math
\usepackage{bm}
\usepackage{mathrsfs} % used for: fraktal math letters
\usepackage[bigsqcap]{stmaryrd} % used for: big square cap symbol
\usepackage{xargs}

% standard packages
%\usepackage{color} % for color

%%
%%index

\newcommand*{\keyword}[2][\empty]{\emph{#2}\ifx#1\empty\index{#2}\else\index{#1}\fi}
\newcommand*{\person}[1]{\textsc{#1}}

%%%%%%%%%%%%%%%%%%%%%%%%%%%%%%%%%%%%%%%%%%%%%%%%%%%%%%%%%%%%%%%%%%%%%%%%%%%%%%%%%%%%%%%%%%%%%%%%%%%%%%%%%%%%
%%%%% MATH ALPHABETS & SYMBOLS %%%%%%%%%%%%%%%%%%%%%%%%%%%%%%%%%%%%%%%%%%%%%%%%%%%%%%%%%%%%%%%%%%%%%%%%%%%%%
%%%%%%%%%%%%%%%%%%%%%%%%%%%%%%%%%%%%%%%%%%%%%%%%%%%%%%%%%%%%%%%%%%%%%%%%%%%%%%%%%%%%%%%%%%%%%%%%%%%%%%%%%%%%

% w: http://milde.users.sourceforge.net/LUCR/Math/math-font-selection.xhtml

% ===== Set quick commands for math letters ================================================================
% calagraphic letters (only upper case available; standard)
\newcommand{\cA}{\mathcal{A}}
\newcommand{\cB}{\mathcal{B}}
\newcommand{\cC}{\mathcal{C}}
\newcommand{\cD}{\mathcal{D}}
\newcommand{\cE}{\mathcal{E}}
\newcommand{\cF}{\mathcal{F}}
\newcommand{\cG}{\mathcal{G}}
\newcommand{\cH}{\mathcal{H}}
\newcommand{\cI}{\mathcal{I}}
\newcommand{\cJ}{\mathcal{J}}
\newcommand{\cK}{\mathcal{K}}
\newcommand{\cL}{\mathcal{L}}
\newcommand{\cM}{\mathcal{M}}
\newcommand{\cN}{\mathcal{N}}
\newcommand{\cO}{\mathcal{O}}
\newcommand{\cP}{\mathcal{P}}
\newcommand{\cQ}{\mathcal{Q}}
\newcommand{\cR}{\mathcal{R}}
\newcommand{\cS}{\mathcal{S}}
\newcommand{\cT}{\mathcal{T}}
\newcommand{\cU}{\mathcal{U}}
\newcommand{\cV}{\mathcal{V}}
\newcommand{\cW}{\mathcal{W}}
\newcommand{\cX}{\mathcal{X}}
\newcommand{\cY}{\mathcal{Y}}
\newcommand{\cZ}{\mathcal{Z}}

% bold math letters (standard)
\newcommand{\bfA}{\mathbf{A}}
\newcommand{\bfB}{\mathbf{B}}
\newcommand{\bfC}{\mathbf{C}}
\newcommand{\bfD}{\mathbf{D}}
\newcommand{\bfE}{\mathbf{E}}
\newcommand{\bfF}{\mathbf{F}}
\newcommand{\bfG}{\mathbf{G}}
\newcommand{\bfH}{\mathbf{H}}
\newcommand{\bfI}{\mathbf{I}}
\newcommand{\bfJ}{\mathbf{J}}
\newcommand{\bfK}{\mathbf{K}}
\newcommand{\bfL}{\mathbf{L}}
\newcommand{\bfM}{\mathbf{M}}
\newcommand{\bfN}{\mathbf{N}}
\newcommand{\bfO}{\mathbf{O}}
\newcommand{\bfP}{\mathbf{P}}
\newcommand{\bfQ}{\mathbf{Q}}
\newcommand{\bfR}{\mathbf{R}}
\newcommand{\bfS}{\mathbf{S}}
\newcommand{\bfT}{\mathbf{T}}
\newcommand{\bfU}{\mathbf{U}}
\newcommand{\bfV}{\mathbf{V}}
\newcommand{\bfW}{\mathbf{W}}
\newcommand{\bfX}{\mathbf{X}}
\newcommand{\bfY}{\mathbf{Y}}
\newcommand{\bfZ}{\mathbf{Z}}
\newcommand{\bfa}{\mathbf{a}}
\newcommand{\bfb}{\mathbf{b}}
\newcommand{\bfc}{\mathbf{c}}
\newcommand{\bfd}{\mathbf{d}}
\newcommand{\bfe}{\mathbf{e}}
\newcommand{\bff}{\mathbf{f}}
\newcommand{\bfg}{\mathbf{g}}
\newcommand{\bfh}{\mathbf{h}}
\newcommand{\bfi}{\mathbf{i}}
\newcommand{\bfj}{\mathbf{j}}
\newcommand{\bfk}{\mathbf{k}}
\newcommand{\bfl}{\mathbf{l}}
\newcommand{\bfm}{\mathbf{m}}
\newcommand{\bfn}{\mathbf{n}}
\newcommand{\bfo}{\mathbf{o}}
\newcommand{\bfp}{\mathbf{p}}
\newcommand{\bfq}{\mathbf{q}}
\newcommand{\bfr}{\mathbf{r}}
\newcommand{\bfs}{\mathbf{s}}
\newcommand{\bft}{\mathbf{t}}
\newcommand{\bfu}{\mathbf{u}}
\newcommand{\bfv}{\mathbf{v}}
\newcommand{\bfw}{\mathbf{w}}
\newcommand{\bfx}{\mathbf{x}}
\newcommand{\bfy}{\mathbf{y}}
\newcommand{\bfz}{\mathbf{z}}

% fractal math letters (standard)
\newcommand{\fkA}{\mathfrak{A}}
\newcommand{\fkB}{\mathfrak{B}}
\newcommand{\fkC}{\mathfrak{C}}
\newcommand{\fkD}{\mathfrak{D}}
\newcommand{\fkE}{\mathfrak{E}}
\newcommand{\fkF}{\mathfrak{F}}
\newcommand{\fkG}{\mathfrak{G}}
\newcommand{\fkH}{\mathfrak{H}}
\newcommand{\fkI}{\mathfrak{I}}
\newcommand{\fkJ}{\mathfrak{J}}
\newcommand{\fkK}{\mathfrak{K}}
\newcommand{\fkL}{\mathfrak{L}}
\newcommand{\fkM}{\mathfrak{M}}
\newcommand{\fkN}{\mathfrak{N}}
\newcommand{\fkO}{\mathfrak{O}}
\newcommand{\fkP}{\mathfrak{P}}
\newcommand{\fkQ}{\mathfrak{Q}}
\newcommand{\fkR}{\mathfrak{R}}
\newcommand{\fkS}{\mathfrak{S}}
\newcommand{\fkT}{\mathfrak{T}}
\newcommand{\fkU}{\mathfrak{U}}
\newcommand{\fkV}{\mathfrak{V}}
\newcommand{\fkW}{\mathfrak{W}}
\newcommand{\fkX}{\mathfrak{X}}
\newcommand{\fkY}{\mathfrak{Y}}
\newcommand{\fkZ}{\mathfrak{Z}}
\newcommand{\fka}{\mathfrak{a}}
\newcommand{\fkb}{\mathfrak{b}}
\newcommand{\fkc}{\mathfrak{c}}
\newcommand{\fkd}{\mathfrak{d}}
\newcommand{\fke}{\mathfrak{e}}
\newcommand{\fkf}{\mathfrak{f}}
\newcommand{\fkg}{\mathfrak{g}}
\newcommand{\fkh}{\mathfrak{h}}
\newcommand{\fki}{\mathfrak{i}}
\newcommand{\fkj}{\mathfrak{j}}
\newcommand{\fkk}{\mathfrak{k}}
\newcommand{\fkl}{\mathfrak{l}}
\newcommand{\fkm}{\mathfrak{m}}
\newcommand{\fkn}{\mathfrak{n}}
\newcommand{\fko}{\mathfrak{o}}
\newcommand{\fkp}{\mathfrak{p}}
\newcommand{\fkq}{\mathfrak{q}}
\newcommand{\fkr}{\mathfrak{r}}
\newcommand{\fks}{\mathfrak{s}}
\newcommand{\fkt}{\mathfrak{t}}
\newcommand{\fku}{\mathfrak{u}}
\newcommand{\fkv}{\mathfrak{v}}
\newcommand{\fkw}{\mathfrak{w}}
\newcommand{\fkx}{\mathfrak{x}}
\newcommand{\fky}{\mathfrak{y}}
\newcommand{\fkz}{\mathfrak{z}}

% script math symbols (only uppercase; package: mathrsfs)
\newcommand{\sA}{\mathscr{A}}
\newcommand{\sB}{\mathscr{B}}
\newcommand{\sC}{\mathscr{C}}
\newcommand{\sD}{\mathscr{D}}
\newcommand{\sE}{\mathscr{E}}
\newcommand{\sF}{\mathscr{F}}
\newcommand{\sG}{\mathscr{G}}
\newcommand{\sH}{\mathscr{H}}
\newcommand{\sI}{\mathscr{I}}
\newcommand{\sJ}{\mathscr{J}}
\newcommand{\sK}{\mathscr{K}}
\newcommand{\sL}{\mathscr{L}}
\newcommand{\sM}{\mathscr{M}}
\newcommand{\sN}{\mathscr{N}}
\newcommand{\sO}{\mathscr{O}}
\newcommand{\sP}{\mathscr{P}}
\newcommand{\sQ}{\mathscr{Q}}
\newcommand{\sR}{\mathscr{R}}
\newcommand{\sS}{\mathscr{S}}
\newcommand{\sT}{\mathscr{T}}
\newcommand{\sU}{\mathscr{U}}
\newcommand{\sV}{\mathscr{V}}
\newcommand{\sW}{\mathscr{W}}
\newcommand{\sX}{\mathscr{X}}
\newcommand{\sY}{\mathscr{Y}}
\newcommand{\sZ}{\mathscr{Z}}

%%%%%%%%%%%%%%%%%%%%%%%%%%%%%%%%%%%%%%%%%%%%%%%%%%%%%%%%%%%%%%%%%%%%%%%%%%%%%%%%%%%%%%
%%%% BOLD MATH IN BOLD TEXT ENVIRONMENT %%%%%%%%%%%%%%%%%%%%%%%%%%%%%%%%%%%%%%%%%%%%%%
%%%%%%%%%%%%%%%%%%%%%%%%%%%%%%%%%%%%%%%%%%%%%%%%%%%%%%%%%%%%%%%%%%%%%%%%%%%%%%%%%%%%%%

% for bold math in bold text (e.g. sections)
\makeatletter
\g@addto@macro\bfseries{\boldmath}
\makeatother

\def\brackets#1{\ifx#1\empty\else\left(#1\right)\fi}

%%%%%%%%%%%%%%%%%%%%%%%%%%%%%%%%%%%%%%%%%%%%%%%%%%%%%%%%%%%%%%%%%%%%%%%%%%%%%%%%%%%%%%
%%%%% CATEGORY THEORY %%%%%%%%%%%%%%%%%%%%%%%%%%%%%%%%%%%%%%%%%%%%%%%%%%%%%%%%%%%%%%%%
%%%%%%%%%%%%%%%%%%%%%%%%%%%%%%%%%%%%%%%%%%%%%%%%%%%%%%%%%%%%%%%%%%%%%%%%%%%%%%%%%%%%%%

% ===== Category theory concepts =====================================================

\newcommand{\Ob}{\mathop\mathrm{Ob}}
\newcommand{\Mor}{\mathop\mathrm{Mor}}

\newcommand{\ccoprod}{\bigsqcup}
\newcommand{\cprod}{\bigsqcap}
\newcommand{\cincl}{\mathop\mathrm{incl}}
\newcommand{\cproj}{\mathop\mathrm{pr}}


% ===== Define standard categories ===================================================

% Define sets
\newcommand{\Set}{\mathbf{Set}}
% Define set-builder operator (equivalent to gen for algebras)
\newcommand{\set}[1]{\left\{#1\right\}}
% define interval operator: o - open, c - closed
\newcommand{\intervalcc}[2]{\left[#1,#2\right]}
\newcommand{\intervalco}[2]{\left[#1,#2\right)}
\newcommand{\intervaloc}[2]{\left(#1,#2\right]}
\newcommand{\intervaloo}[2]{\left(#1,#2\right)}

\newcommand{\inter}{\mathop\mathrm{int}}
\newcommand{\face}{\mathop\mathrm{F}}
\newcommand{\Pol}{\mathop\mathrm{Pol}}
\newcommand{\Inv}{\mathop\mathrm{Inv}}
\def\struct#1{\gen{#1}}

\let\originaltimes\times%
\renewcommand{\times}{\mathbin{\sqcap}}
\newcommand\settimes{\originaltimes}
\newcommand{\setleq}{\subseteq}
\newcommand{\setgeq}{\supseteq}

\newcommand{\pderive}[2]{\frac{\partial{#1}}{\partial{#2}}}
\renewcommand{\div}{\mathop\mathrm{div}}

%% Diffgeo
\def\Ric{\mathop\mathrm{Ric}}
\def\ric{\mathop\mathrm{ric}}
\def\tr{\mathop\mathrm{tr}}

\def\cotimes{\mathbin{\sqcup}}

\def\@rightopen#1{\ifx#1]{\right]}\else{\interval@errmessage}\fi}
\def\@leftclosed[#1){\left[#1\right)}
\makeatother

% finite
\newcommand{\fin}{\mathrm{fin}}

% Define groups (optarg: properties such as -> abelian, noetherian (acc), artinian (dcc) etc.)
\newcommand{\Grp}[1][\empty]{\if\empty{#1}{\mathbf{Grp}}\else{\mathbf{Grp}_{#1}}}
\def\PGL{\mathrm{PGL}}
\def\PGammaL{\mathrm{P\Gamma L}}
\def\GL{\mathrm{GL}}
% Define rings
\newcommand{\rg}{\mathrm{rg}} %rank of a matrix
\newcommand{\Rg}[1][\empty]{\if\empty{#1}{\mathbf{Rg}}\else{\mathbf{Rg}_{#1}}}
\edef\units#1{#1^{\settimes}}
\def\dual#1{#1^{\ast}}

%% redefine the command \P to produce the projective functor in math mode
\let\parsymb\P%
\def\P{\ifmmode\mathrm{P}\else\parsymb\fi}
\renewcommand{\iff}{\ifmmode\equival\else{if and only if}\fi}
\newcommand{\quotring}{\mathop{\mathrm{Q}}}
\newcommand{\rad}{\mathrm{rad}}
% Standard rings
% integral domains
\newcommand{\ID}{\mathbf{ID}}
% unique factorization domains
\newcommand{\UFD}{\mathbf{UFD}}
% principal ideal domains
\newcommand{\PID}{\mathbf{PID}}

% Define modules over a group or ring
\newcommand{\Mod}[1]{\mathbf{Mod}_{#1}}
% Define vector space over a field
\renewcommand{\Vec}[1]{\mathbf{Vec}_{#1}}%

% when cases
\def\otherwise{\textrm{otherwise}}

% new concepts
\newcommand{\new}[1]{\emph{#1}}

\usepackage{xifthen,xstring}

% replace the bar command by overline when argument just one character (shorter and better)
%$\let\oldbar\bar
%\renewcommand{\bar}[1]{\StrLen{#1}[\length]\ifthenelse{\length > 1}{\overline{#1}}{\oldbar{#1}}}
\def\bar{\overline}

%% argument in equatoin
\def\arg{\bullet}

% groups and algebras
\newcommand{\Con}{\mathop\mathrm{Con}}
\newcommand{\Sub}{\mathop\mathrm{Sub}}
\newcommand{\Hom}{\mathop\mathrm{Hom}}
\newcommand{\Aut}{\mathop\mathrm{Aut}}
\newcommand{\Out}{\mathop\mathrm{Out}}
\newcommand{\End}{\mathop\mathrm{End}}
\newcommand{\id}{\mathop\mathrm{id}}
\newcommand{\rk}{\mathop\mathrm{rk}} % rank of a group module/ lattice
\newcommandx{\con}[1][1=\empty]{\ifx#1\empty{\mathop{\mathrm{con}}}\else{\mathop{\mathrm{con}}\left(#1\right)}\fi}
\newcommand{\leftsemidirprod}[1][]{\mathbin{\ifx&#1&\ltimes\else{\ltimes_{#1}}\fi}}
\newcommand{\rightsemidirprod}[1][]{\ifx#1\empty\rtimes\else{\rtimes_{#1}}\fi}
\newcommand{\normalisor}[2][]{\ifx#1\empty{\mathrm{N}\left(#2\right)}\else{\mathrm{N}_{#1} \left(#2\right)}\fi}
% support
\newcommand{\spt}{\mathop{\mathrm{spt}}}
% commutator
\newcommand{\gcom}[2]{\left[#1,#2\right]}

% physics stuff
\newcommand{\float}[3][\empty]{\ifx#1\empty{{#2}\cdot{10^{#3}}}\else{{#2}\cdot{{#1}^{#3}}}\fi}
\makeatletter
\def\newunit#1{\@namedef{#1}{\mathrm{#1}}}
\def\mum{\mathrm{\mu m}}
\def\ohm{\Omega}
\newunit{V}
\newunit{mV}
\newunit{kV}
\newunit{s}
\newunit{ms}
\def\mus{\mathrm{\mu s}}
\newunit{m}
\newunit{nm}
\newunit{cm}
\newunit{mm}
\newunit{fF}
\newunit{A}

\newunit{fA}
\newunit{C}

% elements
\def\newelement#1{\@namedef{#1}{\mathrm{#1}}}
\newelement{Si}
\makeatother


% groups
\newcommand{\ord}{\mathop\mathrm{ord}}
\newcommand{\divides}{|}

\newcommand{\conleq}{\trianglelefteq}
\newcommand{\congeq}{\trianglerighteq}

% common algebraic objects
\newcommand{\reals}{\mathbb{R}} 			% real numbers
\newcommand{\nats}{\mathbb{N}} 				% natural numbers
\newcommand{\ints}{\mathbb{Z}} 				% integers
\newcommand{\rats}{\mathbb{Q}}				% rationals
\newcommand{\complex}{\mathbb{C}}			% complex numbers
\newcommand{\field}[1]{\mathbb{F}_{#1}}  		% finite field
\newcommand{\cards}{\boldsymbol{Cn}}                     % The cardinal numbers
\newcommand{\ords}{\boldsymbol{On}}                      % The ordinal numbers

% graphs
\def\KG{\mathop\mathrm{KG}}                     % Knesergraph

\newcommand{\uvect}{\boldsymbol{e}}

% new operators and relations

%%%%%%%%%%%%%%%%%%%
% complex numbers %
%%%%%%%%%%%%%%%%%%%

\renewcommand{\Re}{\mathop\mathrm{Re}}		% real part
\renewcommand{\Im}{\mathop\mathrm{Im}}		% imaginary part
\newcommand{\sgn}{\mathop\mathrm{sgn}}				% the sign operator (0 for 0)

%%%%%%%%%%%%%%%%
% reel numbers %
%%%%%%%%%%%%%%%%

\newcommand{\floor}[1]{\left\lfloor#1\right\rfloor}
\newcommand{\ceil}[1]{\left\lceil#1\right\rceil}

%%%%%%%%%%%%%%%%%%
% set operations %
%%%%%%%%%%%%%%%%%%

\newcommand{\intersect}{\cap}			% intersect to sets
\newcommand{\setjoin}{\cup}				% join two sets
\newcommand{\setmeet}{\cap}                     % intersect to sets
\newcommand{\bigsetjoin}{\bigcup}			% the union of sets ... subscripts to be added
\newcommand{\distunion}{\dot{\bigcup}}	% disjoint union of sets ... subscripts to be added
\newcommand{\bigsetmeet}{\bigcap}		% intersection of sets
\newcommand{\powerset}[1][]{\ifx&#1&\mathcal{P}\else\mathcal{P}_{#1}\fi}		% powerset ... to be customized
\newcommand{\card}[1]{\left|#1\right|}

%%%%%%%%%%%%%%%%%%%%%%%%%%%%%%%%%%%%%%%%
% composition operations of structures %
%%%%%%%%%%%%%%%%%%%%%%%%%%%%%%%%%%%%%%%%

%\newcommand{\setprod}{\bigtimes}			% setproduct - needed
\newcommand{\dirprod}{\bigotimes} 			% direct product for groups and spaces
\newcommand{\dirtimes}{\otimes}				% direct multiply for groups and spaces
\newcommand{\dirsum}{\bigoplus} 			% direct sum for groups and spaces
\newcommand{\dirplus}{\oplus}				% direct add for groups and spaces
\newcommand{\inprod}[2]{\left\langle #1,#2 \right\rangle}

\newcommand{\tuple}{\meet}
\newcommand{\cotuple}{\join}

%%%%%%%%%%%%%%
% categories %
%%%%%%%%%%%%%%

% combinatorics
%%

\renewcommand{\binom}[3][\empty]{\if\empty{#1}{{#2 \choose #3}}\else{{#2 \choose #3}_{#1}}}

%%%%%%%%%%%%%%%%%%%%%%%%%%%%%%%%%%%%%%%%%%%%%%%%%%%%%
% metric spaces and normed spaces and vector spaces %
%%%%%%%%%%%%%%%%%%%%%%%%%%%%%%%%%%%%%%%%%%%%%%%%%%%%%

\newcommand{\dist}{\mathop\mathrm{dist}}				% distance operator ... dist(A,b), where A is a set and b a point
\newcommand{\diam}{\mathop\mathrm{diam}}	% diameter operator for sets
\newcommand{\norm}[1]{\left\Vert #1 \right\Vert}	% norm in a normed space ... subscript to be added
\newcommand{\conv}{\mathop\mathrm{conv}} 			% convex hull - vectorspaces
\newcommand{\lin}{\mathop\mathrm{lin}} 				% linear hull - vectorspaces
\newcommand{\aff}{\mathop\mathrm{aff}}				% affine hull - vectorspaces

%%%%%%%%%%%%%%%%%%%%%%
% operators in rings %
%%%%%%%%%%%%%%%%%%%%%%

\newcommand{\lcm}{\mathop\mathrm{lcm}}				% least common multiple - in euclidean rings
\renewcommand{\gcd}{\mathop\mathrm{gcd}}				% greatest command devisor - in euclidean rings
\newcommand{\res}{\mathop\mathrm{res}}				% residue of p mod q is res(p,q)
\renewcommand{\mod}{\textrm{ mod }}

%%%%%%%%%%%%%%%%%%%
% logical symbols %
%%%%%%%%%%%%%%%%%%%

%\newcommand{\impliedby}{\Leftarrow}				% reverse implicatoin arrow
%\newcommand{\implies}{\Rightarrow}				% implication
\newcommand{\equival}{\Leftrightarrow}				% equivalence

%%%%%%%%%%%%%%%%%%%%%%%%%%%%%%%%%%
% functions - elementary symbols %
%%%%%%%%%%%%%%%%%%%%%%%%%%%%%%%%%%

\newcommand{\rest}[1]{\left. #1\right\vert}		% restriction of a function to a set / also used as restriction in other terms like differential expressions / evaluation of a function
\newcommand{\rto}[3][]{#2\ifx&#1&\rightarrow\else\stackrel{#1}{\rightarrow}\fi#3}
\renewcommand{\to}{\rightarrow}							% arrow between domain and image
\newcommand{\dom}{\mathop\mathrm{dom}}					% domain of a function
\newcommand{\im}{\mathop\mathrm{im}}						% image of a function
\newcommand{\compose}{\circ}							% compose two functions
\newcommand{\cont}{\mathop\mathrm{C}}					% continuous functions from a domain into the reels or complex numbers 

%%%%%%%%%%%%%%%%%%%%
% groups - symbols %
%%%%%%%%%%%%%%%%%%%%

\newcommand{\stab}[1][]{\if&#1{\mathop\mathrm{stab}}&\else{\mathop\mathrm{stab}_{#1}}\fi}					% the stabilizer ... subscripts to be added
\newcommand{\orb}[1][]{\ifx&#1&\mathrm{orb}\else\mathrm{orb}_{#1}\fi} 					% orbit ... subscrit to be added (group)
\newcommand{\gen}[2][\empty]{\ifx#1\empty{\left\langle#2\right\rangle}\else{\left\langle#2\right\rangle_{#1}}\fi}					% generate ... kind of hull operator ---- to be thought of !!!!!!!

\def\Clo{\mathrm{Clo}}
\def\Loc{\mathrm{Loc}}
%% nets
\newcommand{\net}[2][\empty]{\ifx#1\empty{\left(#2\right)}\else{{\left(#2\right)}_{#1}}\fi}

%% open half ray
\newcommand{\ray}[2]{R_{#1}(#2)}

%% new
\let\oldcong\cong%
\newcommand{\iso}{\oldcong}

\def\cong{\equiv}
\newcommand{\base}[2]{\left[#2\right]_{#1}}                                   % base n expansion of some number
%%%%%%%%%%%%%%%%%%%%%%%%%%%%%%%%%
% matrices and linear operators %
%%%%%%%%%%%%%%%%%%%%%%%%%%%%%%%%%

\newcommand{\diag}{\mathop\mathrm{diag}}				% diagonal matrix or operator
\newcommand{\Eig}[1]{\mathop\mathrm{Eig}_{#1}}		% eigenspace for a certain eigenvalie		
\newcommand{\trace}{\mathop\mathrm{tr}}				% trace of a matrix
\newcommand{\trans}{\top} 							% transponse matrix

%%%%%%%%%%%%%
% constants %
%%%%%%%%%%%%%

\renewcommand{\i}{\boldsymbol{i}}			% imaginary unit
\newcommand{\e}{\boldsymbol{e}}				% the Eulerian constant

%%%%%%%%%%%%%%%%%%%%%%%%%%%%%%
% limit operators and arrows %
%%%%%%%%%%%%%%%%%%%%%%%%%%%%%%

\newcommand{\upto}{\uparrow}				% convergence from above
\newcommand{\downto}{\downarrow}			% convergence from below

%%
% other
%%
\newcommand{\cind}{\mathop\mathrm{Ind}}		% Cauchy index
\newcommand{\sgnc}{\sigma}					% sign changes
\newcommand{\wnumb}{\omega}					% winding number
\newcommand{\cfunc}{\mathop\mathrm{Cf}}		% Cauchy function of a compact curve in complex\setminus\{0\}


% evaluation of a function as a difference or single value

\newcommand{\abs}[1]{\left|#1\right|}
\newcommand{\conj}[1]{\overline{#1}}
\newcommand{\diff}{\mathop\mathrm{d}}

%%%%% test
\newcommand{\distjoin}{\mathaccent\cdot\cup}	% to be modified (name)
\newcommand{\cl}{\mathop\mathrm{cl}}				% topological closure
\newcommand{\sphere}{\mathbb{S}} % n-sphere
\newcommand{\ball}{\mathbb{B}} % n-ball
\newcommand{\bound}{\partial}
\newcommand{\bigmeet}{\mathop\mathrm{\bigwedge}}
\newcommand{\bigjoin}{\mathop\mathrm{\bigvee}}
\newcommandx{\rchar}[1][1=\empty]{\mathop\mathrm{char}\brackets{#1}}              % characteristic of a ring
\newcommand{\lgor}{\vee}                               % logical
\newcommand{\lgand}{\wedge}
\newcommand{\codim}{\mathop\mathrm{codim}}
\newcommand{\row}{\mathop\mathrm{row}}
\newcommand{\cone}{\mathop\mathrm{cone}}
\newcommand{\comp}{\mathop\mathrm{comp}}
\newcommand{\proj}{\mathrm{P}}
\def\PG{\mathrm{PG}}           % projective space
\newcommand{\meet}{\wedge}
\newcommand{\join}{\vee}
\newcommand{\col}{\mathop\mathrm{col}}
\newcommand{\vol}{\mathop\mathrm{vol}\nolimits}
%arrangements
\newcommand{\tpert}{\mathop\mathrm{tpert}}
\renewcommand{\epsilon}{\varepsilon}
% groups
\newcommand{\symgr}{\mathop\mathrm{Sym}}
\newcommand{\symalg}{\mathop\mathrm{S}}
\newcommand{\extalg}{\mathop\mathrm{\Lambda}}
\newcommand{\extpow}[1]{\mathop\mathrm{\Lambda}^{#1}}

%% set hulloperators -> define



\newcommandx{\homl}[3][1=1,2=2,3=3]{\ifx#1\empty{\mathrm{Hom}}\else{\mathrm{Hom}_{#1}}\fi(#2,#3)}
\makeatletter
\newenvironment{myproofof}[1]{\par
  \pushQED{\qed}%
  \normalfont \topsep6\p@\@plus6\p@\relax
  \trivlist
  \item[\hskip\labelsep
        \bfseries
    Proof of #1\@addpunct{.}]\ignorespaces
}{%
  \popQED\endtrivlist\@endpefalse
}
\makeatother



%%% environment test with enumerates

    

% Local variables:
% mode: tex
% End:


% for use of german language in a document
\usepackage[utf8]{inputenc} % this is needed for umlauts
\usepackage[ngerman]{babel} % this is needed for umlauts
\usepackage[T1]{fontenc}    % this is needed for correct output of umlauts in pdf

% math environments in german language

\newcounter{theoremcounter}
%\renewcommand{\thefpcounter}{\thechapter.\arabic{fpcounter}}
%\renewcommand{\thefpcounter}{\thesection.\arabic{fpcounter}}
\renewcommand{\thetheoremcounter}{\arabic{theoremcounter}}

% theorem environment
% parameters
\newcommand{\theoremheaderstyle}{\bf}
\newcommand{\theoremname}{Satz}
\newcommand{\theorembodystyle}{
    % appearance in references
    %\renewcommand\theenumi{\@roman\c@enumi}
    \renewcommand{\theenumi}{(\Roman{enumi})}
    %\renewcommand\theenumi{\@roman\c@enumii}
    \renewcommand{\theenumii}{(\roman{enumii})}
    % appearance in enumerate
    \renewcommand{\labelenumi}{\theenumi}
    \renewcommand{\labelenumii}{\theenumii}
    \it}

%\renewcommand*\descriptionlabel[1]{\hspace\labelsep\normalfont #1}

\newenvironment{theorem}[1][]{
\refstepcounter{theoremcounter}
\begin{trivlist}\item[\hskip\labelsep{\theoremheaderstyle\theoremname\space\thetheoremcounter\ifx&#1&\else{\space(#1)}\fi.}]\theorembodystyle}{\end{trivlist}}

% proof environment
% parameters 
\newcommand{\proofname}{Beweis}             % 'proof' word
\newcommand{\proofof}{von}                  % proof 'of' word
\newcommand{\proofheaderstyle}{\it}         % style of the header
\newcommand{\proofheaderskip}{\space}         % how to skip proofheader (par or space)
\newcommand{\proofbodystyle}{\normalfont}   % style of the body
\newcommand{\proofmark}{$\square$}          % qed mark at the end

% if no argument is passed, just write 'Proof.' otherwise 'Proof of ... .'
\newenvironment{proof}[1][]{\par\noindent{\proofheaderstyle\proofname\ifx&#1&\else{\space\proofof\space#1}\fi.\proofheaderskip}\proofbodystyle}{\hfill\proofmark}

%\newtheorem{theorem}{Satz}[section]
%\newtheorem{proposition}[theorem]{Proposition}
%\newtheorem{lemma}[theorem]{Lemma}
%\newtheorem{corollary}[theorem]{Korallar}
%\newtheorem{conjecture}[theorem]{Vermutung}
 
%\theoremstyle{definition}
%\newtheorem{definition}[theorem]{Definition}
%\newtheorem{example}[theorem]{Beispiel}
 
%\newtheorem{note}[theorem]{Anmerkung}

%\theoremstyle{remark}
%\newtheorem{remark}[theorem]{Bemerkung}

\newenvironment{remark}{Bemerkung.}{}

\newcounter{definitioncounter}[section]
\renewcommand{\thedefinitioncounter}{\arabic{section}.\arabic{definitioncounter}}

\newcommand{\definitionname}{Definition}
\newcommand{\definitionheaderfont}{\bfseries}
\newcommand{\definitionheaderskip}{\space}
\newcommand{\definitionvspacebefore}{.5em}
\newcommand{\definitionvspaceafter}{.5em}

\newenvironment{definition}[1][]{\refstepcounter{definitioncounter}\vspace{\definitionvspacebefore}\par\noindent       {\definitionheaderfont\definitionname\space\thedefinitioncounter\ifx&#1\else\space(#1)\fi.}\definitionheaderskip}{\par\vspace{\definitionvspaceafter}}

\newenvironment{lemma}{Lemma.}{}


\usepackage{tikz}
\usetikzlibrary{arrows}

\begin{document}
\title{Der Satz von \textsc{Nielsen-Schreier}}
\author{Jakob Schneider\\ Seminarleitender: Friedrich Martin Schneider}
\maketitle
\paragraph{Danksagung.} Mein Dank gebührt Shoji Sayaka für ihre ausgezeichnete Interpretation von Beethovens Violinenkonzert in D-dur (Op. 61),
welche mir die Stunden ein wenig versüßt hat.

\section{Der Satz von \textsc{Nielsen-Schreier}}

\paragraph{Einleitung.} In diesem Abschnitt wollen wir die Untergruppen einer freien Gruppe selbst als frei identifizieren. 

\subsection{Wiederholung einiger Begriffe der Graphentheorie}

Zunächst wiederholen wir einige Definitionen, um Konfusionen zu vermeiden.

\begin{definition}[Graph]
  Unter einem \emph{Graphen} verstehen wir in diesem Abschnitt ein Paar von Mengen $(V,E)$ (Ecken und Kanten) zusammen mit zwei Abbildungen $\alpha,\omega:E\to V$ ($\alpha(e)$ ist dabei als Anfangsknoten und $\omega(e)$ als Endknoten einer Kante $e\in E$ zu verstehen).
\end{definition}

\begin{remark}
    Die Eckenmenge eines Graphen $G$ bezeichnen wir im Weiteren auch mit $V_{G}$ und analog dazu meint $E_{G}$ seine Kanten. Ähnliche Kennzeichnungen werden für $\alpha$ und $\omega$ verwandt.
\end{remark}

\begin{remark}
    Die obige Definition entspricht jenem Objekt, welches man oft auch als \emph{gerichteten Multigraphen} bezeichnet.
\end{remark}

\begin{definition}[Homomorphismus von Graphen]
    Ein \emph{Homomorphismus} $\phi:G\to F$ von Graphen ist ein Paar von von Abbildungen $(\phi_V,\phi_E)$ mit
    $\phi_V:V_G\to V_F$ und $\phi_E:E_G\to E_F$ mit der Eigenschaft, dass
    \begin{align*}
        \alpha_{F}\compose\phi_E & = \phi_V\compose\alpha_G\\
        \omega_{F}\compose\phi_E & = \phi_V\compose\omega_G\text{.}
    \end{align*}
    Die Identität $\id_G$ auf einem Graphen $G$ wird dabei durch $(\id_{V_G},\id_{E_G})$ gebildet und die Komposition zweier Homomorphismen durch die komponentenweise Komposition der Ecken- und Kantenabbildungen.
    Wir nennen einen Homomorphismus von Graphen \emph{injektiv}, \emph{surjektiv}, \emph{bijektiv}, wenn seine Ecken- und Kantenabbildungen beide die jeweilige Eigenschaft haben.
\end{definition}

\begin{remark}
    In der üblichen Konvention notieren wir statt $\phi_V(v)$ kürzer $\phi(v)$ und ebenso für $\phi_E(e)$ kürzer $\phi(e)$ ($v\in V$, $e\in E$).
\end{remark}

\begin{remark}
    Die Begriffe \emph{Isomorphismus}, \emph{Epimorphismus}, \emph{Monomorphismus} werden im Weiteren im üblichen kategorientheoretischen Sinne verstanden. 
    Weiterhin ist zu beachten, dass der obige Homomorphiebegriff (i.b. surjektiv, injektiv, bijektiv) algebraisch ist, d.h. ein bijektiver Homomorphismus ist ein Isomorphismus (was unmittelbar ersichtlich ist, wenn man obige Gleichung von rechts mit $\phi_V^{-1}$ und von links mit $\phi_E^{-1}$ multipliziert).
    Dies weicht zwar von der üblichen Konvention eines eher topologisch motivierten Homomorphiebegriffes ab, ist jedoch für unsere Zwecke brauchbarer.
\end{remark}

\begin{definition}[Gruppenaktion auf Graph]
    Sei $F$ ein Graph, $G$ eine Gruppe. Unter einer Gruppenaktion von $G$ auf $F$ verstehen wir einen Homomorphismus $\gamma:G\to\Aut(F)$. Spricht man von einer Linksaktion, so notieren wir statt $\gamma(g)(v)$ bzw. $\gamma(g)(e)$ kürzer $gv$ bzw. $ge$ ($v\in V$, $e\in E$; in analogerweise notiert man Rechtsaktionen). 
\end{definition}

\begin{Remark}
    Die Aktion $\Gamma$ Selbst Wird meist gar nicht notiert, sondern als verstanden angenommen.
\end{remark}

\begin{definition}[Kongruenzrelationen für Graphen]
    Eine Kongruenzrelation $\sim$ auf einem Graphen $F$ bezüglich einer Gruppenaktion $\gamma:G\to\Aut(F)$
    ist eine Äquivalenzrelation $\sim\subseteq (V_F\setjoin E_F)^2$, sodass
    \begin{align*}
        a \sim b &\implies ga \sim gb\\
        e \sim f &\implies (\alpha_F(e) \sim \alpha_F(f) \lgand \omega_F(e)\sim\omega_F(f))
    \end{align*}
    für $a,b\in V_F\setjoin E_F$, $e,f\in E_F$.
    Der \emph{Quotientengraph} $\bar{F}:=F/\sim$ ist dann erklärt durch das Paar von Mengen $(E',V')$, wobei $E'$ die
    Menge der Äquivalenzklassen von $\sim$ bezeichnet, die keine Ecke enthalten, und $V'$ alle anderen.
    Weiterhin werden $\alpha',\omega':E'\to V'$ definiert durch $\alpha_{\bar{F}}[e]_{\sim}:=[\alpha_F(e)]_{\sim}$ und
    $\omega_{\bar{F}}[e]_{\sim}:=[\omega_F(e)]_{\sim}$ (für $[e]_{\sim}\in E'$, Wohldefiniertheit ergibt
    sich aus der letzten der beiden obigen Bedingungen). 
\end{definition}

\begin{remark}
    Anders als beim Faktorgraphen haben wir hier \emph{keinen} natürlichen Homomorphismus, was durch den
    gewählten Homomorphiebegriff bedingt ist. Man könnte diesen noch besser wählen, indem man einen
    Homorphismus durch eine einzige Abbildung zwischen der Vereinigung von Ecken und Kanten in sich
    auffasst (der gegebenenfalls Kanten zu Ecken kollabiert). Dies sei hier aus schreibtechnischen Gründen
    allerdings nicht getan.
\end{remark}

\begin{lemma}[Gruppenaktion auf Quotientengraph]
    Sei $F$ Graph mit einer $G$-Linksaktion und $\sim$ eine Kongruenzrelation bezüglich dieser, dann agiert
    $G$ auf $F/\sim$ durch die Zuordnungen $g[a]_{\sim}:=[ga]_{\sim}$ ($a\in V_F\setjoin E_F$).
\end{lemma}

\begin{proof}
    Folgt direkt aus der ersten Bedingung der vorhergehenden Definition.
\end{proof}

\subsection{Das Pfadgruppoid}

Wir wollen nun ein Gruppoid aus einem Graphen heraus definieren, in das die Kanten desselben eingebettet sind, welches wir auch als \emph{Pfadgruppoid} bezeichnen werden.

\begin{definition}[Pfadgruppoid]
    Sei $G$ ein Graph mit Kanten $E_G$ und Ecken $V_G$. Bezeichne $E_G^{-1}$ eine disjunkte Kopie der Kantenmenge
    $E_G$ bestehend aus Symbolen $e^{-1}$ für $e\in E_G$ (die formalen Inversen) und weiter sei 
    \begin{align*}
        ^{-1}:E_G\distjoin E_G^{-1}; e\mapsto e^{-1}, e^{-1}\mapsto e
    \end{align*}
    Ein \emph{Pfad} $p$ in $G$ ist dann ein Element der Menge $P_G:=V\times (E_G\distjoin E_G^{-1})^{*}\times V$
    mit der Eigenschaft, dass falls $p=(a,(e_i)_{i=1}^n,b)$, dann $\alpha(e_{i+1})=\omega(e_i)$ ($i=1,\ldots,n-1$)
    und $\alpha(e_1)=a$, $\omega(e_n)=b$, falls $n>0$, und sonst $a=b$ (hier gelte
    $M^{*}:=\union_{n\in\nats}{M^n}$(\textsc{Kleene}-Stern)), weiterhin sei für $e\in E$ der Wert $\alpha(e^{-1})$ als $\omega(e)$ verstanden und analog für $\omega(e^{-1})$). 
    Im letzteren Falle schreiben wir $p=\epsilon_a=\epsilon_b$ und nennen $p$ den 
    \emph{leeren} oder \emph{degenerierten Pfad} bei $a$ (bzw. $b$). Ein Pfad $p$ mit $\alpha(p)=a$ und $\omega(p)=b$ heiße \emph{Pfad von $a$ nach $b$} 
    und falls $a=b$ Schleife bei $a$ (bzw. $b$).
    Wir setzen nun $\alpha$ und $\omega$ auf $P_G$ fort, indem wir $\alpha=\pi_1$ und $\omega=\pi_3$ (Projektionen $\pi_i$ auf $i$-te Koordinate) setzen.
    Dabei identifizieren wir die Kante $e\in E_G$ mit $(\alpha(e),e,\omega(e))$ (und auf diese Weise sind die Kanten $E_G$ in $P_G$ eingebettet).
    Um die Notation kompakt zu halten führen wir keine neuen Symbole für die Fortsetzungen ein. Der Pfad
    $p$ heiße nun \emph{reduziert}, wenn er keine aufeinanderfolgenden Elemente $e$ und $e^{-1}$ in seiner Kantenfolge hat.
    Wir definieren nun die partielle Multiplikation $*:Z\subseteq P_G\times P_G\to P_G$ für Pfade $p_1=(a,f_1,b)$ und $p_2=(b,f_2,c)$ durch $(a,f_1f_2,c)$
    (wobei hier $f_1f_2$ das Konkatenat der Folgen $f_1$ und $f_2$ bezeichnet), dementsprechend sei auch $Z$ definiert. 
    Weiterhin definieren wir $^{-1}:P_G\to P_G$ durch $(a,f,b)^{-1}:=(b,f^{-1},a)$ 
    (hierbei bezeichne $f^{-1}$ die Folge $f$ in umgekehrter Reihenfolge angeordnet, wobei alle Symbole $e$
    und $e^{-1}$ vertauscht wurden). 
\end{definition}

Man sieht leicht ein, dass dann zu jedem Pfad $p$ ein äquivalenter reduzierter Pfad existiert, wobei man
unter äquivalent hier den reflexiven, transitiven, symmetrischen Abschluss der Relation von Pfaden verstehe, für welche genau die Pfade $p=(a,(e_1,\ldots,e_n),b)$ und $p'=(a,(e_1,\ldots,e_i,e,e^{-1},e_{i+1}\ldots,e_n),b)$ (für $e\in E$ und $i\in\{1,\ldots,n-1\}$) in Beziehung stehen (siehe dazu vorheriger Vortrag zu Fundamentalgruppen von Graphen).

\begin{definition}[Zusammenhang]
Ein Graph $G$ heißt zusammenhängend, wenn für alle $u,v\in V_G$ ein Pfad von $u$ nach $v$ existiert.
\end{definition}

\begin{definition}[Baum] Sei $G$ ein Graph, dann ist $G$ ein \emph{Baum}, wenn er zusammenhängend ist und $P_G$ keine nicht-entarteten reduzierten Schleifen enthält (d.h. keine reduzierten Schleifen, die Kanten enthalten).
\end{definition}

Eine kleine Übung, welche wir hier kurz demonstrieren, ist das folgende

\begin{corollary}[Äquivalente Charakterisierung von Bäumen]
Die folgenden Aussagen sind für einen Graphen $G$ äquivalent:
\begin{enumerate}
    \item $G$ ist Baum,
    \item für alle $u,v\in V_G$ gibt es einen eindeutigen reduzierten Pfad von $u$ nach $v$,
    \item jeder endliche Teilgraph mit $n>0$ Ecken hat höchstens $n-1$ Kanten und $G$ ist zusammenhängend.
\end{enumerate}
\end{corollary}

\begin{proof}
    \emph{1. $\Rightarrow$ 2.:} Seien $p=(a,(e_1,\ldots,e_n),b),q=(a,(f_1,\ldots,f_m),b)$ reduzierte Pfade von $u$ nach $v$, dann können wir den Pfad 
    \[r:=p*q^{-1}=(a,(e_1,\ldots,e_n,f_m^{-1},\ldots,f_1^{-1}),a)\] betrachten. 
    Da nun $G$ ein Baum ist, muss ein zu $r$ äquivalente reduzierte Pfad eine triviale reduzierte Schleife sein. 
    Damit sieht man allerdings sofort (mit Induktion), dass $e_{n-i}=f_{m-i}$ für ($i=1,\ldots,\min\{n,m\}-1$ gelten muss und ebenso $n=m$. 
    Daraus folgt $p=q$. Dies sichert die Eindeutigkeit. Die Existenz eines reduzierten Pfades $p$ von $u$ nach $v$ folgt aus dem Zusammenhang von $G$
    und sukzessiver Anwendung von Reduktionen. 
    
    \emph{2. $\Rightarrow$ 3.:} Ist die zweite Aussage erfüllt, dann ist $G$ natürlich zusammenhängend. Sei weiterhin $H\leq G$ ein induzierter endlicher
    Untergraph auf minimaler Eckenzahl $n>0$, der mehr als $n-1$ Kanten habe. Es ist unmittelbar ersichtlich, dass dann $n>1$, 
    denn ein von einer Ecke induzierter Untergraph hat 0 Kanten. Damit können wir aus der Eckenmenge $V_H$ eine Ecke $v$ entfernen sodass der induzierte
    Untergraph auf $V_G\setminus\{v\}$ die gewünschte Eigenschaft der dritten Aussage hat. Gäbe es nun aber innerhalb von $G$ zwei Kanten $e_1,e_2$, 
    die eine Ecken $u_1,u_2\in E_G\setminus\{v\}$ mit $v$ verbinden, so würde es auch zwei verschiedene reduzierte Pfade von einer Ecke 
    $u\in E_H\setminus\{v\}$ nach $v$ geben (nämlich indem man jene beiden Kanten (möglicherweise invertiert) an reduzierte Pfade von $u$ nach $u_1$ bzw.
    $u_2$ anhängt, erstere existieren aufgrund des Zusammenhangs). Damit kann aber auch $H$ höchstens $n-1$ Kanten haben.

    \emph{3. $\Rightarrow$ 1.:} Wenn $G$ der dritten Aussage genügt, so kann es insbesondere keine nicht-triviale reduzierte Schleife enthalten, denn aus
    einer solchen lässt sich immer ein in $G$ eingebetteter zyklischer Untergraph (mit irgendeiner
    Orientierung der Kanten) gewinnen, der der dritten Aussage 
    widerspräche. Dies geschieht in folgender Manier: Für die reduzierte Schleife $s=(a,(e_1,\ldots,e_n),a)$ wählt man das $m$ als maximales $i>0$,
    sodass für alle $1\leq j,k\leq i$ gilt $\omega(e_j)\neq \omega(e_k)$. Dann ist der Graph $H$ auf den Kanten $e_1,\ldots,e_m$ eine gewünschte Einbettung des
    $C_m$ in $G$. Zusammenhang war in beiden Punkten gefordert. 
\end{proof}

\begin{definition}[\textsc{Cayley}-Graph]
    Sei $G$ eine Gruppe und $S\subseteq G$. Dann bezeichnen wir mit $\Gamma(G,S)$ den Graphen $F$ der gegeben ist durch die Mengen 
    $V_F:=G$, $E_F:=\{(g,gs):(g,s)\in G\times S\}$ und die Abbildungen $\alpha,\omega:E_F\to V_F$ mit $\alpha:=\pi_1$, $\omega:=\pi_2$ 
    (Projektionen auf erste und zweite Koordinate), welche auf das Pfadgruppoid $P_F$ fortgesetzt natürlich der Projektion der ersten und dritten
    Koordinate entsprechen.
    Weiterhin definieren wir eine (Kanten-)Färbung des eben definierten \textsc{Cayley}-Graphen 
    $\lambda:E_F\to G$ durch $\lambda(e):=\alpha(e)^{-1}\omega(e)$, die wir ebenfalls gleich auf das Pfadgruppoid fortgesetzt verstehen (hier entspricht
    sie der Projektion auf die zweite Koordinate, d.h. der Kantenfolge des Pfades).
\end{definition}

Wir beginnen mit einem 

\begin{corollary}
    Sei $\Gamma(G,S)$ der \textsc{Cayley}-Graph einer Gruppe $G$ bezüglich einer Menge $S\subseteq G$. Dann ist $\Gamma(G,S)$ genau dann ein Baum, wenn $G$ frei erzeugt von $S$ ist.
\end{corollary}

Der Beweis dieser Tatsache basiert lediglich auf stupider Anwendung der Definitionen:

\begin{proof}
    \emph{$\Leftrightarrow$:} Falls $F:=\Gamma(G,S)$ ein Baum ist, dann finden wir zu zwei beliebigen Punkten $u,v\in V_F$ genau einen reduzierten Pfad
    $p=(u,(e_i)_{i=1}^n,v)$ mit $\alpha(p)=v$ und $\omega(p)=v$. Übersetzen wir dies zurück in die Gruppe, so erhalten wir für beliebige 
    Elemente $u,v\in G$ eine eindeutige reduzierte Darstellung von $u^{-1}v$ als Produkt von Elementen von $S$ oder Inversen davon (vgl. Vortrag zu
    freien Gruppen). Setzen wir $u=e_G$ (neutrales Element von $G$), so ergibt sich, dass $G$ frei erzeugt von $S$ sein muss (wieder vgl. Vortrag zu freien
    Gruppen). In analoger Weise lässt sich umgekehrt argumentieren (da die Reduktion von Pfaden in $F$ genau der Reduktion von Darstellung von 
    Elementen von $G$ in Elementen von $S$ entspricht). 
\end{proof}

\subsection{Das Theorem}

Zum Beweis des anfänglich erwähnten Theorems benötigen wir einige Hilfsaussagen und Definitionen

\begin{definition}[Faktorgraph]
    Sei $F$ ein Graph und $G$ eine Gruppe mit Linksaktion auf $F$. 
    Dann definieren wir den Faktorgraph $\bar{F}:=G\backslash F$ durch $V_{\bar{F}}:=G\backslash V_F$ (Menge der Eckenorbits), $E_{\bar{F}}:=G\backslash E_F$ (Kantenorbits). 
    Weiterhin werden die entsprechenden Abbildungen $\bar{\alpha},\bar{\omega}:E_{\bar{F}}\to V_{\bar{F}}$ durch $\bar{\alpha}(Ge):=G\alpha(e)$ und $\bar{\omega}(Ge):=G\omega(e)$ (Wohldefiniertheit is unmittelbar nachzuprüfen).
    Wir nennen den Homomorphismus $\pi:F\to G\backslash F$, der gegeben ist durch die Orbitbildung von
    Ecken und Kanten den \emph{natürlichen Homomorphismus} oder \emph{natürliche Projektion}. 
\end{definition}

\begin{example}
    \begin{figure}[htb]
\begin{tikzpicture}[->,>=stealth',shorten >=1pt,auto,node distance=3cm,thick,main
    node/.style={circle,fill=blue!20,draw}]
    \node[main node] (1) {123};
    \node[main node] (2) [below left of=1] {312};
    \node[main node] (3) [below right of=1] {231};
    \node[main node] (4) [below of=1] {213};
    \node[main node] (5) [below of=2] {132};
    \node[main node] (6) [below of=3] {321};

    \path[every node/.style={font=\sffamily\small}]
    (1) edge [bend right] node {$\tau$} (4)
        edge node {$c$} (2)
    (2) edge [bend right] node {$\tau$} (5)
        edge node {$c$} (3)
    (3) edge [bend right] node {$\tau$} (6)
        edge node {$c$} (1)
    (4) edge [bend right] node {$\tau^{-1}$} (1)
        edge node {$c$} (6)
    (5) edge [bend right] node {$\tau^{-1}$} (2)
        edge node {$c$} (4)
    (6) edge [bend right] node {$\tau^{-1}$} (3)
        edge node {$c$} (5);
\end{tikzpicture}
\caption{\textsc{Cayley}-Graph von $S_3$ mit $S=(c,\tau)$ (Zyklus und Transposition).}
\end{figure}

    \begin{figure}[htb]
\begin{tikzpicture}[->,>=stealth',shorten >=1pt,auto,node distance=2cm,thick,main
    node/.style={circle,fill=blue!20,draw}]
    \node[main node] (1) {$[123]$};
    \node[main node] (2) [below left of=1] {$[312]$};
    \node[main node] (3) [below right of=1] {$[231]$};
    
    \path[every node/.style={font=\sffamily\small}]
    (1) edge [loop above] node {} (1)
        edge node {} (2)
    (2) edge [loop left] node {} (2)
        edge node {} (3)
    (3) edge [loop right] node {} (3)
        edge node {} (1);
\end{tikzpicture}
\caption{Faktorgraph von $\Gamma(S_3,S)$ bzgl. $\gen{\tau}$.}
\end{figure}


\end{example}

\begin{definition}[freie Aktion] Sei $F$ Graph und $G$ eine Gruppe mit Linksaktion auf $F$. Dann heißt
    diese Aktion \emph{frei}, wenn aus der Existenz eines Elementes $v\in V_G$ mit $gv=v$ oder eines
    Elementes $e\in E_G$ mit $ge=e$ folgt, dass $g=e_G$ (neutrales Element von $G$).
\end{definition}

\begin{lemma}[Hochhebung von Bäumen]
    Sei $F$ ein Graph mit einer $G$-Linksaktion und $T$ ein Baum in $\bar{F}:=G\backslash F$ und $s\in
    V_F$, sodass $\pi(s)\in V_T$, wobei $\pi:F\to \bar{F}$ die natürliche Projektion sei. 
    Dann existiert ein Baum $S$ in $F$, sodass $\rest{\pi}^T_{S}:S\to T$ ein Isomorphismus von Graphen ist. 
    Ist die Aktion frei, so ist die \emph{Hochhebung} $S$ eindeutig.
\end{lemma}

\begin{proof}
    \emph{Beweis der Aussage für Kanten:}
    Zunächst beweisen wir die Aussage des Lemmas für eine einzelne Kante $\bar{e}\in E_{\bar{F}}$ und eine
    Ecke $s\in V_F$ mit $\pi(s)=\bar{\alpha}(\bar{e})$ (wir nehmen hier o.B.d.A. $\alpha$). Da $\pi$
    natürliche Abbildung ist, gibt es ein $e\in E_F$, sodass $\bar{e}=Ge=\pi(e)$ und damit
    $\bar{\alpha}(\bar{e})=G\alpha(e)=Gs=\pi(s)$. Dies impliziert die Existenz eines Elementes $g\in G$ mit
    $g\alpha(e)=s$. Somit gilt für $e'=ge$, dass $\pi(e')=Ge=\bar{e}$ und ebenso $\alpha(e')=g\alpha(e)=s$,
    was die Existenz der Hochhebung zeigt. 
    Falls die Aktion von $G$ auf $F$ frei ist, so ist diese Hochhebung der Kante auch eindeutig, denn falls
    $e,e'\in E_F$ Kanten sind mit $\pi(e)=Ge=Ge'=\pi(e')=\bar{e}$ und $\alpha(e)=\alpha(e')=s$, so folgt die
    Existenz eines Elementes $g\in G$ mit $ge=e'$ und damit 
    $\alpha(e)=\alpha(e')=g\alpha(e)$, also $g=e_G$ und damit $e=e'$.

    \emph{Beweis der Aussage für Bäume:}
    Für den allgemeinen Fall betrachte man alle Bäume $S'$, für welche $\rest{\pi}_S:S\to\bar{F}$ injektiv
    nach $T$ abbildet. Diese Menge ist offensichtlich nicht leer (denn sie enthält den leeren Graphen) und
    induktiv geordnet (die Vereinigung einer Kette injektive Abbildungen ist wieder injektive Abbildung und
    die zugehörigen vereinigten Bäume haben einen gemeinsamen Punkt $s$, ihre Vereinigung ist also auch ein Baum). Das Lemma von
    \textsc{Zorn} liefert also ein maximales Element $S$ (bzgl. Inklusion) dieser Menge. Wenn wir einmal
    annehmen, dass $\rest{\pi}_S^T$ nicht surjektiv ist, führt dies schnell zum Widerspruch, indem wir dann
    eine Kante $e\in T\setminus\im(\rest{\pi}_S^T)$ finden, welche mit einer Ecke in $\im(\rest{\pi}_S^T)$
    inzidiert (dies geht streng genommen nur, wenn $T$ tatsächlich Kanten enthält, der Fall eines Punktes
    ist jedoch trivial; ansonsten inzidiert jede Ecke von $T$ mit einer Kante). Jene Kante könnten wir nun
    wieder hochheben zu einer Kante, welche mit einer Ecke von $S$ inzidiert. Also könnte man jene Kante
    (mit zugehörigem Anfangs- bzw. Endpunkt) einfach zu $S$ hinzufügen, was der Maximalität von $S$
    widerspräche (eine Schleife kann dabei ebenfalls nicht entstehen, da diese sonst durch $\pi$ zu einer
    Schleife in $T$ würde).
    Damit muss $\rest{\pi}_S^T$ bijektiv sein (und im Falle, dass die Aktion frei ist, folgt ebenfalls
    analog zu obiger Argumentation die Eindeutigkeit von $S$, denn der Baum $S$ ist - wie man mit Induktion
    sieht - bis zu jeder endlichen Tiefe
    (ausgehend von $s$ als Wurzel - \emph{Tiefe} meint dabei den Abstand zu $s$) bereits eindeutig, da die
    Kanten immer eindeutig hochheben, also auch insgesamt).
    Mit unserem Homomorphiebegriff ist $\rest{\pi}_S^T$ damit ein Isomorphismus.
\end{proof}

\begin{theorem}[\textsc{Nielsen-Schreier}]
    Sei $G$ eine Gruppe mit freier Aktion auf einem Baum $T$. Dann ist $G$ frei und ihr Rank entspricht der
    Kardinalität der Kanten im Komplement eines maximalen Baumes von $\bar{T}:=G\backslash T$. Ist
    $\bar{T}$ endlich, so ist also $\rk(G)=\card{E_{\bar{T}}}-\card{V_{\bar{T}}}+1$.
\end{theorem}

\begin{proof}
    Sei wieder $\pi:T\to\bar{T}$ die natürliche Projektion. Sei nun $\bar{S}$ ein maximaler Baum in
    $\bar{T}$. Fixiere ein $s\in V_T$. Dann existiert eine eindeutige Hochhebung $S$ mit $s\in S$ von
    $\bar{S}$ nach dem vorhergehenden Lemma, denn $\pi(s)\in V_{\bar{S}}$, sonst wäre $\bar{S}$ nicht
    maximal.
    Betrachte nun die Kanten $\bar{E}:=E_{\bar{T}}\setminus E_{\bar{E}}$. Jede dieser Kanten hat ihren Anfangs
    und Endpunkt in $V_{\bar{T}}$ (sonst wäre $S$ nicht maximal). Damit können wir diese Kanten eindeutig
    zu Kanten $E$ hochheben, die ihre Anfangspunkte in $S$ haben, denn die Aktion ist frei und $V_S$ ist
    ein Repräsentantensystem der $G$-Eckenorbits (nach Definition von $\bar{S}$). Dabei muss für jedes
    $e\in E$ gelten $\omega(e)\not\in S$, denn sonst enthielte $S$ einen Kreis. Andererseits muss die durch die 
    Hochhebung der Kanten $\bar{E}$ nach $E$ gegebene Korrespondenz bijektiv sein, denn schon alle Anfangspunkte 
    der hochgehobenen Kanten sind verschieden (und natürlich kann jede Kante hochgehoben werden). Weiterhin ist
    für jedes $e\in E$ der Endpunkt $\omega(e)$ $G$-äquivalent zu genau einem Element $s_e$ von $V_S$ 
    (Definition von $S$). Da die $G$-Aktion frei ist, lässt sich also eine Abbildung $\nu:E\to G$
    definieren, sodass $s_v=\nu(e)\omega(e)$ ($e\in E$), denn gibt es zwei Elemente $g,g'\in G$ mit
    $s_v=g\omega(e)=g'\omega(e)$, dann folgt aus $g^{-1}g'\omega(e)=\omega(e)$, dass $g=g'$. 
    Die Bäume $gS$ (für $g\in G$) sind nun allesamt Hochhebungen von $\bar{S}$ und aufgrund der Definition
    von $S$ gilt $\union_{g\in G}{V_{gS}}=V_T$ (da $V_S$ Repräsentantensystem der $G$-Eckenorbits von $T$
    ist). Daraus folgt nun schon, dass jede Hochhebung von $\bar{S}$ der Form $gS$ sein muss
    (vorhergehendes Lemma). Weiterhin sind jene Bäume auch paarweise disjunkt (d.h. ihre Ecken- und Kantenmengen sind
    disjunkt), denn wären es nicht, so müssten sie wegen der Eindeutigkeit der Hochhebung schon gleich sein. 
    Außerdem muss jede Kante $e'\in E_T\setminus\union_{g\in G}{E_{gS}}$ der Form $ge$ für ein $e\in E$ und
    $g\in G$ sein, denn sie hat einen Anfangspunkt in einem Baum der Form $gS$. 
    Wir sehen daraus sofort, dass die durch $\{V_{gS}\setjoin E_{gS},\{ge\}:g\in G, e\in E\}$ gegebene Partition 
    eine Kongruenzrelation $\sim\subset (V_T\setjoin E_T)^2$ auf $T$ induziert, denn offensichtlich wird jede
    Klasse dieser Partition (vorher demonstriert) durch ein Element $g\in G$ auf eine andere abgebildet. 
    Andererseits ist auch
    die zweite für Kongruenzrelationen geforderte Bedingung erfüllt, denn für eine Klasse der Form
    $V_{gS}\setjoin E_{gS}$ bleiben Anfangs- und Endpunkte zugehöriger Kanten klar in derselben, und für
    ein Singleton $\{ge\}$ ist nichts zu prüfen. Wir zeigen nun, dass $T/\sim \cong \Gamma(G,\im(\nu))$.
    Dies geschieht, indem wir den Isomorphismus $\phi:T/\sim\to\Gamma(G,\im(\nu))$ durch
    \begin{align*}
        \phi_E([ge]_{\sim}) &:= (g,g\nu(e)^{-1})\\
        \phi_V(V_{gS}\setjoin E_{gS}) &:= g 
    \end{align*}
    definieren. Klarerweise ist sind $\phi_V,\phi_E$ hier bijektiv (wegen Definition des \textsc{Cayley}-Graphen.
    Bleibt noch zu zeigen, dass $\phi=(\phi_V,\phi_E)$ ein Homomorphismus ist. Dies zeigt die folgende
    Rechnung (in der wir $\Gamma(G,\im(\nu))$ mit $\Gamma$ abkürzen)
    \begin{align*}
        \alpha_{\Gamma}(\phi_E[ge]_{\sim})=\alpha_{\Gamma}(g,g\nu(e))=g & =
        \phi_V(V_{gS}\setjoin E_{gS})=\phi_V(g(V_S\setjoin E_S))\\
        & = \phi_V(g[\alpha_F(e)]_{\sim})=\phi_V(\alpha_{F/\sim}[ge])\\
        \omega_{\Gamma}(\phi_E[ge]_{\sim})=\omega_{\Gamma}(g,g\nu(e)^{-1})=g\nu(e)^{-1} & =
        \phi_V(V_{g\nu(e)^{-1}S}\setjoin E_{g\nu(e)^{-1}S})\\
        & =\phi_V(g\nu(e)^{-1}(V_S\setjoin E_S))\\
        & = \phi_V(g\nu(e)^{-1}[\nu(e)\omega_F(e)]_{\sim})\\
        & =\phi_V(\omega_{F/\sim}[ge])\text{.}
    \end{align*}
    Damit folgt aus dem vorhergehenden Korollar, dass $G\cong F(E)\cong F(\bar{E})$. Wir erhalten also im
    Falle, dass $\bar{T}$ endlich ist die gewünschte Formel, indem wir die dritten Punkt der
    Charakterisierung von Bäumen anwenden.
\end{proof}

\begin{corollary}[\textsc{Nielsen-Schreier}]
    Sei $G=F(S)$ eine freie Gruppe und $H\leq G$, dann ist $H$ frei.
\end{corollary}

\begin{proof}
    Die Gruppe $H$ agiert frei auf $\Gamma(G,S)$, also ist sie mit dem vorhergehenden Theorem frei.
\end{proof}

Eine weitere Folgerung ist

\begin{corollary}[Schreier's Formel]
    Ist $G=F(S)$ frei und von endlichem Rank und $H$ die freie Untergruppe vom Index $n$. Dann gilt
    $\rk(H)-1=n(\rk(G)-1)$.
\end{corollary}

\begin{proof}
    Man erhält für den Faktorgraph $\bar{F}:=H\backslash \Gamma(G,S)$, dass
    $\card{V_{\bar{F}}}=\card{H\backslash G}=n$ und $\card{E_{\bar{F}}}=\card{\{[(g,gs)]:g\in G\lgand s\in
    S\}}=\card{H\backslash G}\card{S}$ (wenn man $H$ von links herausteilt). Dies liefert nach
    vorangegangenem Theorem
    \begin{align*}
        \rk(H)=n\rk(G)-n+1
    \end{align*}
    
\end{proof}

\begin{remark}
    Man kann beispielsweise leicht Untergruppen vom Index $n$ in der freien Gruppe $F(S)$ mit $k>0$ Erzeugern
    finden, indem man einen Homomorphismus von $F(S)$ nach $\ints_n$ definiert, der alle bis auf einen
    Erzeuger trivial macht und den letzten auf $1$ schickt. Der Kern eines solchen Homomorphismus ist ein
    Normalteiler von endlichem Index $n$ in $F(S)$.
\end{remark}

\paragraph{Literatur:} Weiterführend empfiehlt sich das Buch \emph{Oleg Bogopolski - Introduction to Group Theory}, aus welchem auch die
wesentlichen Punkte dieses Abschnittes herrühren.
\end{document}
