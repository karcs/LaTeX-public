
\section{Die wichtigsten Regeln zum Simplexverfahren zusammengefasst}

\subsection{Das Simplex-Verfahren}

Wir nehmen an, dass ein Simplextableau der folgenden Gestalt vorliegt.
$$
\begin{matrix}
    & x_N^{\trans} & 1 \\
    \midrule
    x_B = & P & p \\
    \midrule
    z = & q^\trans & q_0 
\end{matrix}
$$

Dabei ist $x_N$ der Vektor der Nichtbasisvariablen, $x_B$ der Vektor der Basisvariablen (Koordinaten). Das Tableau heißt \keyword{primal zulässig}, falls alle Einträge in $p$ (aktuelle Basisvariablen) nicht negativ sind.

\subsubsection{Entscheidbares oder nicht entscheidbares Tableau?}

Ein primal zulässiges Simplextableau heißt \keyword{entscheidbar}, falls einer der folgenden Fälle vorliegt:
\begin{statements}
    \item Die Zielfunktionskoeffizienten, also die Einträge des Vektors $q^\trans$ sind alle $\geq 0$. In diesem Falle ist die Lösung optimal. Sei lautet $x_B=p$ und $x_N=0$. Enthält $q^\trans$ Nullen, so ist die Lösung nicht eindeutig und es ergeben sich entsprechende Ungleichungsrestriktionen für die Lösungsmenge.
    \item Es gibt eine Spalte $j$, in welcher der Eintrag $q_j$ negativ ist, die darüberliegenden Einträge $P_{ij}$ aber alle $\geq 0$. Dann ist das Problem nicht lösbar, da die Zielfunktion von unten unbeschränkt ist im zulässigen Bereich.
\end{statements}

Liegt keiner der obigen Fälle vor, dann heißt das Tableau \keyword{nicht entscheidbar} und es ist ein Austauschschritt zu vollziehen.

\subsubsection{Wie wird das Pivotelement ausgewählt?}

Man geht wie folgt vor:

\begin{steps}
        \item Wähle eine Spalte mit negativem Zielfunktionskoeffizienten. Das ist dann die Pivotspalte (am besten ist der Koeffizient möglichst klein).
        \item Bilde für die negativen Einträge der Pivotspalte (und nur für diese) folgende Quotienten:
    $$
    -\frac{\textrm{Eintrag auf der rechten Seite}}{\textrm{Eintrag in der Pivotspalte}}
    $$
    \item Wähle eine Zeile aus, in der der Quotient am kleinsten ist. Das ist die Pivotzeile.
\end{steps}

Der Eintrag, in dem sich die Pivotzeile und Pivotspalte kreuzen heißt \keyword{Pivotelement}.

\subsubsection{Wie wird das neue Tableau erzeugt?}

Zunächst wird an das alte Tableau eine Kellerzeile angefügt. Diese hat in der Spalte, in der das Pivotelement steht,  einen $*$ stehen. Alle anderen Einträge berechnen sich wie folgt:
$$
\textrm{Kellerzeile}=-\frac{\textrm{Pivotzeile}}{\textrm{Pivotelement}}.
$$
Danach geht man wie folgt vor:
\begin{steps}
        \item An der Stelle, wo das Pivotelement stand, steht jetzt
    $$
    \textrm{neuer Eintrag} = \frac 1 {\textrm{altes Pivotelement}}.
    $$
        \item Die Spalte, die Pivotspalte war, berechnet sich gemäß
    $$
    \textrm{neuer Eintrag} = \frac{\textrm{alte Pivotspalte}}{\textrm{altes Pivotelement}}.
    $$
        \item Die Zeile, die Pivotzeile war, ist gleich der alten Kellerzeile.
        \item Alle anderen Einträge ergeben sich nach der Vorschrift:
    $$
    \eqalign{
        \textrm{neuer Eintrag} &= \textrm{alter Eintrag}\cr
        &\qquad + \textrm{alter Pivotspalteneintrag}\cdot\textrm{alter Kellerzeileneintrag}
    }
    $$
\end{steps}

\subsection{Das duale Simplex-Verfahren}

Wir nehmen an, dass ein Simplextableau der folgenden Gestalt vorliegt.
$$
\begin{matrix}
    & x_N^{\trans} & 1 \\
    \midrule
    x_B = & P & p \\
    \midrule
    z = & q^\trans & q_0 
\end{matrix}
$$
Das Tableau heißt \keyword{dual zulässig}, falls alle Einträge in $p$ (aktuelle Basisvariablen) nicht negativ sind.

\subsubsection{Entscheidbares oder nicht entscheidbares Tableau?}

Ein dual zulässiges Simplextableau heißt \keyword{entscheidbar}, falls einer der folgenden Fälle vorliegt:
\begin{statements}
        \item Die rechte Seite, also die Einträge des Vektors $p$, sind alle $\geq 0$. In diesem Falle ist eine optimale Lösung gefunden.
    \item Es gibt eine Zeile, in welcher der Eintrag von $p$ negativ ist und die daneben liegenden Einträge der Matrix $P$ alle $\leq 0$ sind. In diesem Fall ist das Problem nicht lösbar, da der zulässige Bereich leer ist.
\end{statements}

Liegt keiner der obigen Fälle vor, dann heißt das Tableau \keyword{nicht entscheidbar} und es ist ein Austauschschritt zu vollziehen.

\subsubsection{Wie wird das Pivotelement ausgewählt?}

Man geht wie folgt vor:

\begin{steps}
        \item Wähle eine Zeile $i$ mit negativem $p_i$. Das ist dann die Pivotzeile (am besten ist der Koeffizient möglichst klein).
        \item Bilde für die negativen Einträge der Pivotspalte (und nur für diese) folgende Quotienten:
    $$
    \frac{\textrm{Eintrag des Zielfunktionsvektors}}{\textrm{Eintrag in der Pivotzeile}}
    $$
    \item Wähle eine Spalte aus, in der der Quotient am kleinsten ist. Das ist die Pivotspalte.
\end{steps}

Der Eintrag, in dem sich die Pivotzeile und Pivotspalte kreuzen heißt \keyword{Pivotelement}.
Der Austausch funktioniert analog zum normalen Simplexverfahren.
