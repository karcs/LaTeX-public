\documentclass{article}
\usepackage{amssymb,amsthm,amsmath,amsfonts}

%\usepackage{mathabx} % more and nicer symbols symbols

%\usepackage{mathabx} % more and nicer symbols symbols
% MEINE PACKAGES!!!!!!
\usepackage{color}
\usepackage[usenames,dvipsnames]{pstricks}
\usepackage{epsfig}
\usepackage{pst-grad} % For gradients
\usepackage{pst-plot} % For axes

%tikz
%\usepackage{tikz}

% importing various types of grapics

\usepackage{epsfig}  		% For postscript
\usepackage{epic,eepic}       % For epic and eepic output from xfig


% The following is very useful in keeping track of labels while writing
%\usepackage[notcite]{showkeys}

%textwidth ...



\newtheorem{theorem}{Theorem}[section]
\newtheorem{proposition}[theorem]{Proposition}
\newtheorem{lemma}[theorem]{Lemma}
\newtheorem{corollary}[theorem]{Corollary}

\newtheorem{conjecture}[theorem]{Conjecture} 

\theoremstyle{definition}
\newtheorem{definition}[theorem]{Definition}
\newtheorem{example}[theorem]{Example}

\newtheorem{note}[theorem]{Note}

\theoremstyle{remark}
\newtheorem{remark}[theorem]{Remark}

% common algebraic objects
\newcommand{\reels}{\mathbb{R}}  % The real numbers
\newcommand{\nats}{\mathbb{N}} % The natural numbers
\newcommand{\ints}{\mathbb{Z}} % The integers
\newcommand{\field}{\mathbb{F}}


% new operators and relations

% setproduct
\newcommand{\setprod}{\bigtimes}
\newcommand{\dirprod}{\bigotimes} % direct product for vector spaces
\newcommand{\dirtimes}{\otimes}
\newcommand{\dirsum}{\bigoplus} % direct sum for vector spaces
\newcommand{\dirplus}{\oplus}
\newcommand{\rest}[1]{\left. #1\right\vert}
\DeclareMathOperator{\vecspace}{vec} % forgetful functor to vector spaces
\DeclareMathOperator{\ord}{ord} % forgetful functor to partial orders
\DeclareMathOperator{\lex}{lex}
\DeclareMathOperator{\trans}{T}
\renewcommand{\mod}{\text{mod }}
\newcommand{\lcm}{\text{lcm}}

\newcommand{\norm}[1]{\left\Vert #1 \right\Vert}
\renewcommand{\impliedby}{\Rightarrow}
\renewcommand{\implies}{\Rightarrow}
\newcommand{\equival}{\Leftrightarrow}
\newcommand{\intersect}{\cap}
\newcommand{\unify}{\cup}
\newcommand{\union}{\bigcup}
%\newcommand{\setminus}{\char`\\}
\newcommand{\level}{\boldsymbol{l}}



\begin{document}

\title{On the Road Coloring Problem}
\author{Jakob Schneider \& Sebastian Manecke}

\maketitle

\begin{abstract}
We review Trahtmans proof of the Road Coloring Conjecture published in 2009.
\end{abstract}

\section{Introduction}

% Generated with LaTeXDraw 2.0.8
% Mon Jun 10 15:33:16 CEST 2013

Let $G$ be a finite, strongly connected, directed graph where all vertices have the same out-degree $k\in\nats$. Let $\Sigma$ be the alphabet containing the letters $1, \ldots, k$. A synchronizing coloring (also known as a collapsible coloring) in $G$ is a labeling of the edges in $G$ with letters from $\Sigma$ such that 
\begin{enumerate}
\item each vertex has exactly one outgoing edge with a given label and
\item for every vertex $v$ in the graph, there exists a word $s$ over $\Sigma$ such that all paths in $G$ corresponding to $s$ terminate at $v$.
\end{enumerate}

For such a coloring to exist at all, it is necessary that $G$ be aperiodic. The road coloring theorem states that aperiodicity is also sufficient for such a coloring to exist. Therefore, the road coloring problem can be stated briefly as:

\begin{conjecture}
Every finite strongly connected aperiodic directed graph of uniform out-degree has a synchronizing coloring.
\end{conjecture}

\begin{figure}[hbt]
\centering
%LaTeX with PSTricks extensions
%%Creator: inkscape 0.48.4
%%Please note this file requires PSTricks extensions
\psset{xunit=.5pt,yunit=.5pt,runit=.5pt}
\begin{pspicture}(372.04724121,524.40942383)
{
\newrgbcolor{curcolor}{0 0 1}
\pscustom[linewidth=0.7072311,linecolor=curcolor]
{
\newpath
\moveto(212.43062,310.09723383)
\curveto(248.42658,275.91063383)(233.23962,205.15749383)(223.03908,172.18717383)
}
}
{
\newrgbcolor{curcolor}{0 0 0}
\pscustom[linestyle=none,fillstyle=solid,fillcolor=curcolor]
{
\newpath
\moveto(217.55872167,305.22689947)
\lineto(221.55809608,305.3300064)
\lineto(212.43062,310.09723383)
\lineto(217.6618286,301.22752506)
\lineto(217.55872167,305.22689947)
\closepath
}
}
{
\newrgbcolor{curcolor}{0 0 0}
\pscustom[linewidth=0.7072311,linecolor=curcolor]
{
\newpath
\moveto(217.55872167,305.22689947)
\lineto(221.55809608,305.3300064)
\lineto(212.43062,310.09723383)
\lineto(217.6618286,301.22752506)
\lineto(217.55872167,305.22689947)
\closepath
}
}
{
\newrgbcolor{curcolor}{1 0 0}
\pscustom[linewidth=0.7072311,linecolor=curcolor]
{
\newpath
\moveto(63.912086,278.27184383)
\curveto(10.350709,246.61991383)(36.012525,177.84864383)(42.695152,119.14484383)
}
}
{
\newrgbcolor{curcolor}{0 0 0}
\pscustom[linestyle=none,fillstyle=solid,fillcolor=curcolor]
{
\newpath
\moveto(57.82344665,274.67378124)
\lineto(56.82721594,270.79910047)
\lineto(63.912086,278.27184383)
\lineto(53.94876588,275.67001195)
\lineto(57.82344665,274.67378124)
\closepath
}
}
{
\newrgbcolor{curcolor}{0 0 0}
\pscustom[linewidth=0.7072311,linecolor=curcolor]
{
\newpath
\moveto(57.82344665,274.67378124)
\lineto(56.82721594,270.79910047)
\lineto(63.912086,278.27184383)
\lineto(53.94876588,275.67001195)
\lineto(57.82344665,274.67378124)
\closepath
}
}
{
\newrgbcolor{curcolor}{0 0 1}
\pscustom[linewidth=0.7072311,linecolor=curcolor]
{
\newpath
\moveto(212.43062,310.09723383)
\curveto(180.60522,267.66337383)(62.529018,321.36515383)(63.912086,278.27184383)
}
}
{
\newrgbcolor{curcolor}{0 0 0}
\pscustom[linestyle=none,fillstyle=solid,fillcolor=curcolor]
{
\newpath
\moveto(208.18723295,304.43938531)
\lineto(208.75301753,300.47889109)
\lineto(212.43062,310.09723383)
\lineto(204.22673872,303.87360073)
\lineto(208.18723295,304.43938531)
\closepath
}
}
{
\newrgbcolor{curcolor}{0 0 0}
\pscustom[linewidth=0.7072311,linecolor=curcolor]
{
\newpath
\moveto(208.18723295,304.43938531)
\lineto(208.75301753,300.47889109)
\lineto(212.43062,310.09723383)
\lineto(204.22673872,303.87360073)
\lineto(208.18723295,304.43938531)
\closepath
}
}
{
\newrgbcolor{curcolor}{0 0 1}
\pscustom[linewidth=0.7072311,linecolor=curcolor]
{
\newpath
\moveto(286.68988,299.48877383)
\curveto(265.47295,320.70570383)(212.14997,348.65412383)(212.43062,310.09723383)
}
}
{
\newrgbcolor{curcolor}{0 0 0}
\pscustom[linestyle=none,fillstyle=solid,fillcolor=curcolor]
{
\newpath
\moveto(281.6890009,304.48965293)
\lineto(277.68829762,304.48965293)
\lineto(286.68988,299.48877383)
\lineto(281.6890009,308.4903562)
\lineto(281.6890009,304.48965293)
\closepath
}
}
{
\newrgbcolor{curcolor}{0 0 0}
\pscustom[linewidth=0.7072311,linecolor=curcolor]
{
\newpath
\moveto(281.6890009,304.48965293)
\lineto(277.68829762,304.48965293)
\lineto(286.68988,299.48877383)
\lineto(281.6890009,308.4903562)
\lineto(281.6890009,304.48965293)
\closepath
}
}
{
\newrgbcolor{curcolor}{1 0 0}
\pscustom[linewidth=0.7072311,linecolor=curcolor]
{
\newpath
\moveto(254.86448,161.57871383)
\curveto(265.47295,235.83797383)(244.25601,341.92263383)(212.43062,310.09723383)
}
}
{
\newrgbcolor{curcolor}{0 0 0}
\pscustom[linestyle=none,fillstyle=solid,fillcolor=curcolor]
{
\newpath
\moveto(255.86465622,168.57994451)
\lineto(253.46423443,171.78050727)
\lineto(254.86448,161.57871383)
\lineto(259.06521897,170.98036629)
\lineto(255.86465622,168.57994451)
\closepath
}
}
{
\newrgbcolor{curcolor}{0 0 0}
\pscustom[linewidth=0.7072311,linecolor=curcolor]
{
\newpath
\moveto(255.86465622,168.57994451)
\lineto(253.46423443,171.78050727)
\lineto(254.86448,161.57871383)
\lineto(259.06521897,170.98036629)
\lineto(255.86465622,168.57994451)
\closepath
}
}
{
\newrgbcolor{curcolor}{0 0 1}
\pscustom[linewidth=0.7072311,linecolor=curcolor]
{
\newpath
\moveto(286.68988,299.48877383)
\curveto(309.38455,229.19965383)(270.33714,177.41271383)(254.86448,161.57871383)
}
}
{
\newrgbcolor{curcolor}{0 0 0}
\pscustom[linestyle=none,fillstyle=solid,fillcolor=curcolor]
{
\newpath
\moveto(288.86289953,292.75857549)
\lineto(292.42418667,290.93570397)
\lineto(286.68988,299.48877383)
\lineto(287.040028,289.19728835)
\lineto(288.86289953,292.75857549)
\closepath
}
}
{
\newrgbcolor{curcolor}{0 0 0}
\pscustom[linewidth=0.7072311,linecolor=curcolor]
{
\newpath
\moveto(288.86289953,292.75857549)
\lineto(292.42418667,290.93570397)
\lineto(286.68988,299.48877383)
\lineto(287.040028,289.19728835)
\lineto(288.86289953,292.75857549)
\closepath
}
}
{
\newrgbcolor{curcolor}{1 0 0}
\pscustom[linewidth=0.7072311,linecolor=curcolor]
{
\newpath
\moveto(254.86448,161.57871383)
\curveto(243.90615,163.80557383)(233.64755,168.65102383)(223.03908,172.18717383)
}
}
{
\newrgbcolor{curcolor}{0 0 0}
\pscustom[linestyle=none,fillstyle=solid,fillcolor=curcolor]
{
\newpath
\moveto(247.93382199,162.98710407)
\lineto(244.59820269,160.77819697)
\lineto(254.86448,161.57871383)
\lineto(245.72491489,166.32272338)
\lineto(247.93382199,162.98710407)
\closepath
}
}
{
\newrgbcolor{curcolor}{0 0 0}
\pscustom[linewidth=0.7072311,linecolor=curcolor]
{
\newpath
\moveto(247.93382199,162.98710407)
\lineto(244.59820269,160.77819697)
\lineto(254.86448,161.57871383)
\lineto(245.72491489,166.32272338)
\lineto(247.93382199,162.98710407)
\closepath
}
}
{
\newrgbcolor{curcolor}{1 0 0}
\pscustom[linewidth=0.7072311,linecolor=curcolor]
{
\newpath
\moveto(42.695152,119.14484383)
\curveto(117.06193,76.47947383)(195.1852,125.01407383)(254.86448,161.57871383)
}
}
{
\newrgbcolor{curcolor}{0 0 0}
\pscustom[linestyle=none,fillstyle=solid,fillcolor=curcolor]
{
\newpath
\moveto(48.8295853,115.62542462)
\lineto(52.69112631,116.67143026)
\lineto(42.695152,119.14484383)
\lineto(49.87559094,111.76388362)
\lineto(48.8295853,115.62542462)
\closepath
}
}
{
\newrgbcolor{curcolor}{0 0 0}
\pscustom[linewidth=0.7072311,linecolor=curcolor]
{
\newpath
\moveto(48.8295853,115.62542462)
\lineto(52.69112631,116.67143026)
\lineto(42.695152,119.14484383)
\lineto(49.87559094,111.76388362)
\lineto(48.8295853,115.62542462)
\closepath
}
}
{
\newrgbcolor{curcolor}{0 0 1}
\pscustom[linewidth=0.7072311,linecolor=curcolor]
{
\newpath
\moveto(42.695152,119.14484383)
\curveto(42.695152,-93.02444617)(531.11359,237.27800383)(286.68988,299.48877383)
}
}
{
\newrgbcolor{curcolor}{0 0 0}
\pscustom[linestyle=none,fillstyle=solid,fillcolor=curcolor]
{
\newpath
\moveto(42.695152,112.07253278)
\lineto(45.52407642,109.24360837)
\lineto(42.695152,119.14484383)
\lineto(39.86622758,109.24360837)
\lineto(42.695152,112.07253278)
\closepath
}
}
{
\newrgbcolor{curcolor}{0 0 0}
\pscustom[linewidth=0.7072311,linecolor=curcolor]
{
\newpath
\moveto(42.695152,112.07253278)
\lineto(45.52407642,109.24360837)
\lineto(42.695152,119.14484383)
\lineto(39.86622758,109.24360837)
\lineto(42.695152,112.07253278)
\closepath
}
}
{
\newrgbcolor{curcolor}{1 0 0}
\pscustom[linewidth=0.7072311,linecolor=curcolor]
{
\newpath
\moveto(286.68988,299.48877383)
\curveto(326.02604,438.16696383)(70.887718,389.64905383)(63.912086,278.27184383)
}
}
{
\newrgbcolor{curcolor}{0 0 0}
\pscustom[linestyle=none,fillstyle=solid,fillcolor=curcolor]
{
\newpath
\moveto(288.61980871,306.29266673)
\lineto(286.67022303,309.78619538)
\lineto(286.68988,299.48877383)
\lineto(292.11333735,308.24225241)
\lineto(288.61980871,306.29266673)
\closepath
}
}
{
\newrgbcolor{curcolor}{0 0 0}
\pscustom[linewidth=0.7072311,linecolor=curcolor]
{
\newpath
\moveto(288.61980871,306.29266673)
\lineto(286.67022303,309.78619538)
\lineto(286.68988,299.48877383)
\lineto(292.11333735,308.24225241)
\lineto(288.61980871,306.29266673)
\closepath
}
}
{
\newrgbcolor{curcolor}{1 0 0}
\pscustom[linewidth=0.7072311,linecolor=curcolor]
{
\newpath
\moveto(116.95442,426.79036383)
\curveto(231.66504,491.61218383)(404.61962,400.94076383)(286.68988,299.48877383)
}
}
{
\newrgbcolor{curcolor}{0 0 0}
\pscustom[linestyle=none,fillstyle=solid,fillcolor=curcolor]
{
\newpath
\moveto(123.11164725,430.26975115)
\lineto(124.18278322,434.12439697)
\lineto(116.95442,426.79036383)
\lineto(126.96629308,429.19861517)
\lineto(123.11164725,430.26975115)
\closepath
}
}
{
\newrgbcolor{curcolor}{0 0 0}
\pscustom[linewidth=0.7072311,linecolor=curcolor]
{
\newpath
\moveto(123.11164725,430.26975115)
\lineto(124.18278322,434.12439697)
\lineto(116.95442,426.79036383)
\lineto(126.96629308,429.19861517)
\lineto(123.11164725,430.26975115)
\closepath
}
}
{
\newrgbcolor{curcolor}{1 0 0}
\pscustom[linewidth=0.7072311,linecolor=curcolor]
{
\newpath
\moveto(116.95442,426.79036383)
\curveto(129.92193,490.95595383)(16.398957,485.24355383)(116.95442,426.79036383)
}
}
{
\newrgbcolor{curcolor}{0 0 0}
\pscustom[linestyle=none,fillstyle=solid,fillcolor=curcolor]
{
\newpath
\moveto(118.35537219,433.72252918)
\lineto(116.14288692,437.0557762)
\lineto(116.95442,426.79036383)
\lineto(121.6886192,435.93501445)
\lineto(118.35537219,433.72252918)
\closepath
}
}
{
\newrgbcolor{curcolor}{0 0 0}
\pscustom[linewidth=0.7072311,linecolor=curcolor]
{
\newpath
\moveto(118.35537219,433.72252918)
\lineto(116.14288692,437.0557762)
\lineto(116.95442,426.79036383)
\lineto(121.6886192,435.93501445)
\lineto(118.35537219,433.72252918)
\closepath
}
}
{
\newrgbcolor{curcolor}{0 0 1}
\pscustom[linewidth=0.7072311,linecolor=curcolor]
{
\newpath
\moveto(223.03908,172.18717383)
\curveto(159.38828,193.40411383)(159.38828,129.75331383)(42.695152,119.14484383)
}
}
{
\newrgbcolor{curcolor}{0 0 0}
\pscustom[linestyle=none,fillstyle=solid,fillcolor=curcolor]
{
\newpath
\moveto(216.32969685,174.42363558)
\lineto(212.75135888,172.63446702)
\lineto(223.03908,172.18717383)
\lineto(214.54052828,178.00197355)
\lineto(216.32969685,174.42363558)
\closepath
}
}
{
\newrgbcolor{curcolor}{0 0 0}
\pscustom[linewidth=0.7072311,linecolor=curcolor]
{
\newpath
\moveto(216.32969685,174.42363558)
\lineto(212.75135888,172.63446702)
\lineto(223.03908,172.18717383)
\lineto(214.54052828,178.00197355)
\lineto(216.32969685,174.42363558)
\closepath
}
}
{
\newrgbcolor{curcolor}{0 0 1}
\pscustom[linewidth=0.7072311,linecolor=curcolor]
{
\newpath
\moveto(63.912086,278.27184383)
\curveto(-20.636938,377.65024383)(59.885047,411.21843383)(116.95442,426.79036383)
}
}
{
\newrgbcolor{curcolor}{0 0 0}
\pscustom[linestyle=none,fillstyle=solid,fillcolor=curcolor]
{
\newpath
\moveto(59.32928031,283.6584453)
\lineto(55.34151744,283.97996362)
\lineto(63.912086,278.27184383)
\lineto(59.65079862,287.64620817)
\lineto(59.32928031,283.6584453)
\closepath
}
}
{
\newrgbcolor{curcolor}{0 0 0}
\pscustom[linewidth=0.7072311,linecolor=curcolor]
{
\newpath
\moveto(59.32928031,283.6584453)
\lineto(55.34151744,283.97996362)
\lineto(63.912086,278.27184383)
\lineto(59.65079862,287.64620817)
\lineto(59.32928031,283.6584453)
\closepath
}
}
\end{pspicture}

\caption{AGW graph with synchronizing word $r^5$.}
\end{figure}

This paper adapts the (ingenious) proof presented by Trahtman published in 2009 (see \cite{Tr07}). It adds upon his work in explaining some details left in his proof, some of which were referenced to \cite{Fr90} while others were expected as base knowledge of the reader.



\section{Preliminaries}

At first we want to establish some notation. 

\begin{definition}[strongly connected, aperiodic]
A directed graph $G=(V,E)$ is called \emph{strongly connected} if any two vertices $u,v\in V$ are joined by a directed path. We call $G$ \emph{aperiodic} if the $\gcd$ of the lengths of its circles is one.
\end{definition}

At this point we want to mention that the property of strongly connectedness is equivalent to the property of the adjacency matrix $A$ of $G$ that for any $i,j\in\{1,\ldots,|V|\}$ there exists $n\in\nats$ such that $(A^n)_{ij}>0$. This can be seen easily by recalling that $A^n$ just counts the number of paths of length $n$ from one vertex to another.

% The property of $G$ being \emph{aperiodic} is equivalent to the following one. There is no partition of $V$ in non-empty sets $V_i$ ($i=1,\ldots,l$ where $l>1$) such that every edge starting from $V_i$ ends in $V_{i+1}$ ($\mod l$). 

Another equivalent formulation of aperiodicty is that if there exists $n\in\nats$ such that $(A^n)_{ij}>0$ then there exists $n_0\in\nats$ such that $(A^n)_{ij}>0$ for all $n\geq n_0$.

\begin{definition}
An AGW graph is a \emph{strongly connected}, \emph{aperiodic} directed finite graph which is $k$-regular for some $k\in\nats$.
\end{definition}

Combining both properties (or equivalent formulations for the adjacency matrix, respectively) implies that the adjacency matrix $A$ of an AGW graph admits a strictly positive natural power $A^n$ (for some $n\in\nats$). This property is often called \emph{primitive} with respect to matrices as well as to graphs.  

Hereafter $G=(V,E)$ is an AGW graph, $A$ the adjacency matrix of $G$ and $k$ its out-degree.

\paragraph{The case $k=1$.} This case is trivial but we shall do the short argument here (any coloring is the trivial coloring). Let $G$ be an AGW graph with out-degree $k=1$. Let $C$ be a circle in $G$ of maximal length. If $G=C$ then $C$ must be the trivial circle with one vertex as $G$ is acyclic. In the opposite case $G$ is not strongly connected (it consists of at least two disjoint cycles). Thus $G$ is the trivial graph with one vertex. 

\section{Weights}
We want to introduce the concept of weights of vertices.

\begin{definition}
Let $u$ be a left eigenvector of eigenvalue $k$ with natural entries of the adjacency matrix $A$ of an AGW graph $G$, then we define the weight $w(v_i)$ as the $i$-th component of $u$ (w.l.o.g. we may assume that these entries have $\gcd$ one). For a subset $L \subset V$ we define $w(L):=\sum_{v\in S}{w(v)}$.
\end{definition}

The existence of such eigenvector can easily be seen from the fact that $A$ is primitive. A more precise explanation is given in the following theorem.

\begin{theorem}
Let $k\geq 2$\footnote{This shall also be assumed for the rest of the paper as the case $k=1$ is trivial as mentioned in the preliminaries.}. All of the eigenvalues $\lambda$ of the adjacency matrix $A$ lie in the interval $[-k,k]$ and especially $k$ is an eigenvalue of $A$ of algebraic multiplicity one with an integer-valued positive left and right eigenvector, where the right eigenvector trivially is a multiple of $(1,\ldots,1)^{\trans}$.\footnote{Actually, such weights can be found for any strongly connected graph, as Perron-Frobenius' theorem works for positive irreducible operators (which are the corresponding adjacency matrices).}
\end{theorem}

\begin{proof}
One directly notices that $\norm{A}_\infty=k$ implying that for all eigenvalues $\lambda$ we have $|\lambda|\leq k$. The rest is just applying Perron-Frobenius' theorem.
\end{proof}

Again we need to introduce some notation conventions. Let $ L \subset V$ be a set of vertices of the colored AGW graph $G$ and $s$ be a word of $\Sigma^*$. Then we denote by $Ls$ the set of vertices which are reached from the vertices of $L$ by a $s$-path.

The papers we are referring to are mainly dealing with the following two concepts.

\begin{definition}[$F$-maximal set]
An \emph{$F$-maximal set} set $L$ is a subset of $V$ where $w(L)$ is maximal under the condition that $|Ls|=1$ for some word $s\in\Sigma^*$. This last condition is also called \emph{contractible}.
\end{definition}

\begin{definition}[$F$-clique]
An $F$-clique is a set of the form $Vs$ where $s\in \Sigma^*$ such that every pair of $Vs$ is not \emph{synchronizable}, i.e. for all $u,v\in Vs$ and $t\in \Sigma^*$ we have $ut\neq vt$. Such pairs we also call \emph{deadlocks}.
\end{definition}

The first important step for the solution of the problem is now given by the following 

\begin{theorem}[Partition into $F$-maximal sets, Friedman]\footnote{This theorem does also hold for strongly connected graphs.}
The graph $G$ has a partition into $F$-maximal sets.
\end{theorem}

Friedman actually proved this theorem for $k=2$. But his proof works as well for the general case.

\begin{proof}
Let $L$ be an $F$-maximal set. Then by the definition of the weight, the following identity holds for the backward images of $L$.

\begin{equation}
\sum_{s\in\Sigma}{w(Ls^{-1})}=kw(L)
\end{equation} 

From this identity one observes that all the sets $Ls^{-1}$ ($s\in\Sigma$) must be $F$-maximal sets, because they are contractible (as $L$ is) and must all have the same weight as $L$ does. Now, if $L=V$ we are done. Otherwise, let $r$ be a synchronizing word of $L$ and $v_0\in V$ the vertex such that $Lr=\{v_0\}$. Furthermore, pick $v\in V\setminus L$. By strongly connectedness there is some word $w$ such that $v_0w=v$. Then consider the $F$-maximal sets $Lw^{-1}r^{-1}=:\tilde{L}$ and $L$ (the first one has this property by successive application of inverses of single letters). Then 
it is obvious that $|Lrwr|=|\tilde{L}rwr|=1$. Furthermore, these both sets cannot be synchronized to the same vertex, since $Lr=\tilde{L}rwr =\{v_0\}$ and thus otherwise we would have $v_0wr=vr=v_0$ which would imply $v\in v_0r^{-1}=L$, a contradiction\footnote{This argument uses strict positivity of weights. Otherwise we could only say $L\subset v_0r^{-1}$.}. From this, one deduces that $L$ and $\tilde{L}$ are disjoint. Continuing the argument inductively, one obtains the desired partition.
\end{proof}

A second important theorem is also of significant importance for the proof of the conjecture. It is stated by Trahtman.

At first we need the following

\begin{definition}
A pair $u,v\in V$ is called \emph{stable} if for any word $w\in \Sigma^*$ the pair $uw, vw\in V$ is synchronizable.
\end{definition} 

It is clear, that \emph{stability} defines a congruence relation on the set $V$. Let us denote this relation by $\rho$.  

\begin{theorem}[Quotient graph, Trahtman]
The graph $G/\rho$ is an AGW graph. 
\end{theorem}\label{theo1}

The proof of the theorem is obvious and is omitted\footnote{Circles and paths remain, although they might get 'glued' somewhere}. While this theorem is quiet natural and clear by the definition of $\rho$, it is a very strong tool in the proof, because it ensures that the existence of one single stable pair of vertices in an AGW graph implies (inductively) the existence of a synchronizing coloring.

We now continue with a lemma on $F$-cliques.

\begin{lemma}[Equal cardinality of $F$-cliques, Trahtman]\label{lem1}\footnote{This lemma also holds for strongly connected graphs.}
Let $w$ be the weight of an $F$-maximal set of $G$. Then any $F$-clique $Vs$ is of cardinality $w(V)/w$. Moreover, if $F_1, \ldots F_l$ is a partition of $V$ into $F$-maximal sets then $|F_i\intersect Vs|=1$ (forall $i=1,\ldots,l$).
\end{lemma}

The proof is mainly based on Friedmans theorem.

\begin{proof}
Let $Vs$ be an $F$-clique ($s\in\Sigma^*$). The first thing we note is that for any $F$-maximal set $F$ we have $|F\intersect Vs|\leq 1$, because otherwise $Vs$ would contain a synchronizable pair. Thus we have $|Vs|\leq w(V)/w$ (number of $F$-maximal sets. Besides, for any $v\in Vs$ we have
\begin{equation}
w(vs^{-1})\leq w \label{eq1}
\end{equation}
by the definition of $F$-maximal sets. Furthermore, we have the following identity
\begin{equation}
w(V)=\sum_{v\in Vs}{w(vs^{-1})}\label{eq2}
\end{equation}
Plugging (\ref{eq1}) into equation (\ref{eq2}) gives
\begin{equation}
w(V)\leq |Vs|w
\end{equation}
from which (together with the other inequality) we obtain the desired result $|Vs|=w(V)/w$. The second fact mentioned follows easily.
\end{proof}

The next lemma is an easy one.

\begin{lemma}\label{lem2}
For any $F$-clique $Vs$ and any word $w\in \Sigma^*$ the set $Vsw$ is an $F$-clique. Moreover, any vertex $v$ belongs to some $F$-clique.
\end{lemma}

\begin{proof}
Just use the stability of the binary relation to be deadlock. The second fact is an easy consequence of strongly connectedness.
\end{proof}

The next lemma is essential for the final proof and is based on the idea to prove the existence of a stable couple.

\begin{lemma}\label{lem3}
Assume $G$ via some coloring has no stable couples. Let $A$ and $B$ be distinct $F$-cliques ($|A|=|B|>1$). Then $|A\setminus B|=|B\setminus A|>1$.
\end{lemma}

\begin{proof}
Assume $|A\setminus B|=|B\setminus A|=1$ (the cardinalities are equal by Lemma (\ref{lem1}). Then pick $u\in A\setminus B$ and $v\in B\setminus A$. By condition these two form no stable couple. Thus we find $s\in\Sigma^*$ such that the pair $(us,vs)$ is deadlock. Thus we see immediately that $(A\unify B)s$ must also be an $F$-clique which contradicts Lemma (\ref{lem1}).\footnote{Attention. If $|A\setminus B|\geq 2$ one cannot use the same argument repetitively (twice), because it is not assured that a pair $(u's,v's)$ is unstable for $u'=A\setminus B\setminus\{u\}$ and $v'=B\setminus A\setminus\{v\}$ are unstable ($s\in\Sigma^*$ chosen for $u$ and $v$ as above).}
\end{proof}



Finally, we end up the preparation with an easy one for which we need the following

\begin{definition}
Let $B=\{e_1,\ldots,e_k\}\subset E$ a set of $k$ edges starting from a vertex $u$ and ending in $v\in V$. Then we call $B$ a bunch.
\end{definition}

\begin{lemma}\label{lem4}
Let $G$ have a vertex $v$ with two incoming bunches. Then $G$ has a stable couple.
\end{lemma}

\begin{proof}
The two starting vertices of the two incoming bunches are obvioulsy stable.
\end{proof}

\section{Spanning subgraphs, trees and cycles}

This section gives the main part of the proof. For the understanding of this, the reader is allowed to forget everything about the weights of the graph $G$. They were only a mean to proof the facts about $F$-cliques and $F$-maximal sets.

At first we need some terminology.

\begin{definition}[Spanning subgraph]
Let $R$ be a subgraph of $G$. We call $R$ a \emph{spanning subgraph} of $G$ if the vertex sets of $R$ and $G$ coincide and every vertex of $R$ has out-degree one.\footnote{If $G$ is colored then any monochromatic subgraph is a spanning subgraph. Moreover, for any spanning subgraph there exists such coloring.}
\end{definition}

\begin{definition}[Rooted tree]
A graph $T$ of the spanning subgraph $R$ is called a \emph{rooted tree} of $R$ if $T$ is a maximal tree of $R$ having no edges with cycles of $R$ in common and its root in a cycle of $R$.
\end{definition}

\begin{definition}[Level of a vertex]
Let $T$ be a rooted tree in a spanning subgraph $R$ and $v\in V$. Then the \emph{level} of $v$ is defined as the length of the path from $v$ to the root of $T$ and is denoted by $\level(v)$. Furthermore let $\smash{\level(T) := \displaystyle\max_{v \in T} \level(v)}$  
\end{definition}

From these three definitions it should be clear that any spanning subgraph $R$ of $G$ can be uniquely partitioned into disjoint cycles and rooted trees. The level for each vertex $v\in V$ is then defined by the particular tree in which $v$ lies. If $v$ is member of a cycle, then its level is 0.

In the following, if nothing different is stated, we use the same symbols as in the definitions.

Considering such partitioned spanning subgraph $R$ of $G$ the following lemma should be trivial.

\begin{lemma}[%Connection of $F$-cliques and $n$-level vertices
]\label{lem5}
Let all edges of $R$ have the same color $c$ and let $Vs$ be an $F$-clique then we have $|\level^{-1}[n]\intersect Vs|\leq 1$ ($s\in\Sigma^*$). 
\end{lemma}

\begin{proof}
If $\level^{-1}[n]\intersect Vs$ had more than one member, $Vs$ would contain a synchronizable pair (synchronizing word is $c^n$). 
\end{proof}

For the rest of the section let us assume, that $G$ has more than one vertex.\footnote{This case is sufficiently trivial.}

\begin{lemma}[%Making a new spanning subgraph from a bad one
]\label{lem6}
Let $R$ be some spanning subgraph of $G$ consisting only of disjoint cycles. Then there is another spanning subgraph $R'$ of $G$ with exactly one vertex of maximal level.
\end{lemma}

\begin{proof}
If $R$ is a single cycle then $R$ has (not necessarily distinct) vertices $u,v,w\in V$ such that $u\rightarrow w\in E_G$, $u\rightarrow v\in E_R$\footnote{In this proof we use the subscripts to indicate to which subgraph vertices belong.} and $v\neq w$\footnote{Here we use $k\geq 2$ and the fact that if all edges of $G$ would belong to bunches then we could assign any coloring to $G$ and deal with it like $k=1$.}. We then may take the new spanning subgraph $R'=(V_R,E_R\setminus\{u\rightarrow v\}\unify\{u\rightarrow u\})$.

In the second case, where $R$ is not a single cycle we may apply the same argument.
In the end we have a single tail (that is a tree which is a path) rooted in $w$ and thus a single vertex of maximal length.

\begin{figure}[htb]
\centering
%LaTeX with PSTricks extensions
%%Creator: inkscape 0.48.4
%%Please note this file requires PSTricks extensions
\psset{xunit=.5pt,yunit=.5pt,runit=.5pt}
\begin{pspicture}(524.40942383,372.04724121)
{
\newrgbcolor{curcolor}{1 0 0}
\pscustom[linewidth=0.81980938,linecolor=curcolor]
{
\newpath
\moveto(180,285.00000121)
\curveto(227.71387,110.68843121)(195,90.00000121)(195,90.00000121)
}
}
{
\newrgbcolor{curcolor}{0 0 0}
\pscustom[linestyle=none,fillstyle=solid,fillcolor=curcolor]
{
\newpath
\moveto(182.16442168,277.09278803)
\lineto(186.19307563,274.79567143)
\lineto(180,285.00000121)
\lineto(179.86730508,273.06413409)
\lineto(182.16442168,277.09278803)
\closepath
}
}
{
\newrgbcolor{curcolor}{0 0 0}
\pscustom[linewidth=0.81980938,linecolor=curcolor]
{
\newpath
\moveto(182.16442168,277.09278803)
\lineto(186.19307563,274.79567143)
\lineto(180,285.00000121)
\lineto(179.86730508,273.06413409)
\lineto(182.16442168,277.09278803)
\closepath
}
}
{
\newrgbcolor{curcolor}{1 0 0}
\pscustom[linewidth=0.81980938,linecolor=curcolor]
{
\newpath
\moveto(90,240.00000121)
\curveto(113.13672,288.48008121)(159.94532,284.56095121)(180,285.00000121)
}
}
{
\newrgbcolor{curcolor}{0 0 0}
\pscustom[linestyle=none,fillstyle=solid,fillcolor=curcolor]
{
\newpath
\moveto(93.53097509,247.39871437)
\lineto(91.98387986,251.77058967)
\lineto(90,240.00000121)
\lineto(97.90285039,248.9458096)
\lineto(93.53097509,247.39871437)
\closepath
}
}
{
\newrgbcolor{curcolor}{0 0 0}
\pscustom[linewidth=0.81980938,linecolor=curcolor]
{
\newpath
\moveto(93.53097509,247.39871437)
\lineto(91.98387986,251.77058967)
\lineto(90,240.00000121)
\lineto(97.90285039,248.9458096)
\lineto(93.53097509,247.39871437)
\closepath
}
}
{
\newrgbcolor{curcolor}{1 0 0}
\pscustom[linewidth=0.81980938,linecolor=curcolor]
{
\newpath
\moveto(105,150.00000121)
\curveto(76.37168,181.69301121)(70.914452,192.46047121)(90,240.00000121)
}
}
{
\newrgbcolor{curcolor}{0 0 0}
\pscustom[linestyle=none,fillstyle=solid,fillcolor=curcolor]
{
\newpath
\moveto(99.50467929,156.08360144)
\lineto(94.87311091,156.31891325)
\lineto(105,150.00000121)
\lineto(99.7399911,160.71516982)
\lineto(99.50467929,156.08360144)
\closepath
}
}
{
\newrgbcolor{curcolor}{0 0 0}
\pscustom[linewidth=0.81980938,linecolor=curcolor]
{
\newpath
\moveto(99.50467929,156.08360144)
\lineto(94.87311091,156.31891325)
\lineto(105,150.00000121)
\lineto(99.7399911,160.71516982)
\lineto(99.50467929,156.08360144)
\closepath
}
}
{
\newrgbcolor{curcolor}{1 0 0}
\pscustom[linewidth=0.81980938,linecolor=curcolor]
{
\newpath
\moveto(195,90.00000121)
\curveto(137.74335,74.15351121)(112.7631,145.87661121)(105,150.00000121)
}
}
{
\newrgbcolor{curcolor}{0 0 0}
\pscustom[linestyle=none,fillstyle=solid,fillcolor=curcolor]
{
\newpath
\moveto(187.09892455,87.81328031)
\lineto(184.81318273,83.77816178)
\lineto(195,90.00000121)
\lineto(183.06380602,90.09902213)
\lineto(187.09892455,87.81328031)
\closepath
}
}
{
\newrgbcolor{curcolor}{0 0 0}
\pscustom[linewidth=0.81980938,linecolor=curcolor]
{
\newpath
\moveto(187.09892455,87.81328031)
\lineto(184.81318273,83.77816178)
\lineto(195,90.00000121)
\lineto(183.06380602,90.09902213)
\lineto(187.09892455,87.81328031)
\closepath
}
}
{
\newrgbcolor{curcolor}{0 0 0}
\pscustom[linewidth=0.81980938,linecolor=curcolor]
{
\newpath
\moveto(180,285.00000121)
\curveto(140.85642,244.94497121)(120.10633,213.05158121)(105,150.00000121)
}
}
{
\newrgbcolor{curcolor}{0 0 0}
\pscustom[linestyle=none,fillstyle=solid,fillcolor=curcolor]
{
\newpath
\moveto(174.27016514,279.13674836)
\lineto(174.32353234,274.49951328)
\lineto(180,285.00000121)
\lineto(169.63293006,279.08338116)
\lineto(174.27016514,279.13674836)
\closepath
}
}
{
\newrgbcolor{curcolor}{0 0 0}
\pscustom[linewidth=0.81980938,linecolor=curcolor]
{
\newpath
\moveto(174.27016514,279.13674836)
\lineto(174.32353234,274.49951328)
\lineto(180,285.00000121)
\lineto(169.63293006,279.08338116)
\lineto(174.27016514,279.13674836)
\closepath
}
}
{
\newrgbcolor{curcolor}{0 0 0}
\pscustom[linewidth=0.81980938,linecolor=curcolor]
{
\newpath
\moveto(195,90.00000121)
\curveto(171.03287,214.03546121)(116.45166,227.85073121)(90,240.00000121)
}
}
{
\newrgbcolor{curcolor}{0 0 0}
\pscustom[linestyle=none,fillstyle=solid,fillcolor=curcolor]
{
\newpath
\moveto(193.44466802,98.04920514)
\lineto(189.60285366,100.64675393)
\lineto(195,90.00000121)
\lineto(196.04221681,101.89101951)
\lineto(193.44466802,98.04920514)
\closepath
}
}
{
\newrgbcolor{curcolor}{0 0 0}
\pscustom[linewidth=0.81980938,linecolor=curcolor]
{
\newpath
\moveto(193.44466802,98.04920514)
\lineto(189.60285366,100.64675393)
\lineto(195,90.00000121)
\lineto(196.04221681,101.89101951)
\lineto(193.44466802,98.04920514)
\closepath
}
}
{
\newrgbcolor{curcolor}{0 0 0}
\pscustom[linewidth=0.81980938,linecolor=curcolor]
{
\newpath
\moveto(180,285.00000121)
\curveto(309.02079,215.12106121)(182.23348,430.94813121)(180,285.00000121)
}
}
{
\newrgbcolor{curcolor}{0 0 0}
\pscustom[linestyle=none,fillstyle=solid,fillcolor=curcolor]
{
\newpath
\moveto(187.2086893,281.09570356)
\lineto(191.65388407,282.41746022)
\lineto(180,285.00000121)
\lineto(188.53044596,276.65050879)
\lineto(187.2086893,281.09570356)
\closepath
}
}
{
\newrgbcolor{curcolor}{0 0 0}
\pscustom[linewidth=0.81980938,linecolor=curcolor]
{
\newpath
\moveto(187.2086893,281.09570356)
\lineto(191.65388407,282.41746022)
\lineto(180,285.00000121)
\lineto(188.53044596,276.65050879)
\lineto(187.2086893,281.09570356)
\closepath
}
}
{
\newrgbcolor{curcolor}{0 0 0}
\pscustom[linewidth=0.81980938,linecolor=curcolor]
{
\newpath
\moveto(90,240.00000121)
\curveto(-15.188271,64.82698121)(9.8819755,-5.29024879)(195,90.00000121)
}
}
{
\newrgbcolor{curcolor}{0 0 0}
\pscustom[linestyle=none,fillstyle=solid,fillcolor=curcolor]
{
\newpath
\moveto(85.77962414,232.97168924)
\lineto(86.90279859,228.47221411)
\lineto(90,240.00000121)
\lineto(81.28014902,231.8485148)
\lineto(85.77962414,232.97168924)
\closepath
}
}
{
\newrgbcolor{curcolor}{0 0 0}
\pscustom[linewidth=0.81980938,linecolor=curcolor]
{
\newpath
\moveto(85.77962414,232.97168924)
\lineto(86.90279859,228.47221411)
\lineto(90,240.00000121)
\lineto(81.28014902,231.8485148)
\lineto(85.77962414,232.97168924)
\closepath
}
}
{
\newrgbcolor{curcolor}{0 0 0}
\pscustom[linewidth=1,linecolor=curcolor]
{
\newpath
\moveto(195,90.00000121)
\curveto(230.8603,0.34335121)(287.03474,66.39687121)(195,90.00000121)
}
}
{
\newrgbcolor{curcolor}{0 0 0}
\pscustom[linestyle=none,fillstyle=solid,fillcolor=curcolor]
{
\newpath
\moveto(198.71369607,80.71515003)
\lineto(203.91311497,78.48668798)
\lineto(195,90.00000121)
\lineto(196.48523403,75.51573113)
\lineto(198.71369607,80.71515003)
\closepath
}
}
{
\newrgbcolor{curcolor}{0 0 0}
\pscustom[linewidth=1,linecolor=curcolor]
{
\newpath
\moveto(198.71369607,80.71515003)
\lineto(203.91311497,78.48668798)
\lineto(195,90.00000121)
\lineto(196.48523403,75.51573113)
\lineto(198.71369607,80.71515003)
\closepath
}
}
{
\newrgbcolor{curcolor}{0 0 1}
\pscustom[linewidth=0.81980938,linecolor=curcolor]
{
\newpath
\moveto(426.67144,293.28560121)
\curveto(474.38531,118.97403121)(441.67144,98.28560121)(441.67144,98.28560121)
}
}
{
\newrgbcolor{curcolor}{0 0 0}
\pscustom[linestyle=none,fillstyle=solid,fillcolor=curcolor]
{
\newpath
\moveto(428.83586168,285.37838803)
\lineto(432.86451563,283.08127143)
\lineto(426.67144,293.28560121)
\lineto(426.53874508,281.34973409)
\lineto(428.83586168,285.37838803)
\closepath
}
}
{
\newrgbcolor{curcolor}{0 0 0}
\pscustom[linewidth=0.81980938,linecolor=curcolor]
{
\newpath
\moveto(428.83586168,285.37838803)
\lineto(432.86451563,283.08127143)
\lineto(426.67144,293.28560121)
\lineto(426.53874508,281.34973409)
\lineto(428.83586168,285.37838803)
\closepath
}
}
{
\newrgbcolor{curcolor}{0 0 1}
\pscustom[linewidth=0.81980938,linecolor=curcolor]
{
\newpath
\moveto(336.67144,248.28560121)
\curveto(359.80816,296.76568121)(406.61676,292.84655121)(426.67144,293.28560121)
}
}
{
\newrgbcolor{curcolor}{0 0 0}
\pscustom[linestyle=none,fillstyle=solid,fillcolor=curcolor]
{
\newpath
\moveto(340.20241509,255.68431437)
\lineto(338.65531986,260.05618967)
\lineto(336.67144,248.28560121)
\lineto(344.57429039,257.2314096)
\lineto(340.20241509,255.68431437)
\closepath
}
}
{
\newrgbcolor{curcolor}{0 0 0}
\pscustom[linewidth=0.81980938,linecolor=curcolor]
{
\newpath
\moveto(340.20241509,255.68431437)
\lineto(338.65531986,260.05618967)
\lineto(336.67144,248.28560121)
\lineto(344.57429039,257.2314096)
\lineto(340.20241509,255.68431437)
\closepath
}
}
{
\newrgbcolor{curcolor}{0 0 0}
\pscustom[linewidth=0.81980938,linecolor=curcolor]
{
\newpath
\moveto(351.67144,158.28560121)
\curveto(323.04312,189.97861121)(317.58589,200.74607121)(336.67144,248.28560121)
}
}
{
\newrgbcolor{curcolor}{0 0 0}
\pscustom[linestyle=none,fillstyle=solid,fillcolor=curcolor]
{
\newpath
\moveto(346.17611929,164.36920144)
\lineto(341.54455091,164.60451325)
\lineto(351.67144,158.28560121)
\lineto(346.4114311,169.00076982)
\lineto(346.17611929,164.36920144)
\closepath
}
}
{
\newrgbcolor{curcolor}{0 0 0}
\pscustom[linewidth=0.81980938,linecolor=curcolor]
{
\newpath
\moveto(346.17611929,164.36920144)
\lineto(341.54455091,164.60451325)
\lineto(351.67144,158.28560121)
\lineto(346.4114311,169.00076982)
\lineto(346.17611929,164.36920144)
\closepath
}
}
{
\newrgbcolor{curcolor}{0 0 1}
\pscustom[linewidth=0.81980938,linecolor=curcolor]
{
\newpath
\moveto(441.67144,98.28560121)
\curveto(384.41479,82.43911121)(359.43454,154.16221121)(351.67144,158.28560121)
}
}
{
\newrgbcolor{curcolor}{0 0 0}
\pscustom[linestyle=none,fillstyle=solid,fillcolor=curcolor]
{
\newpath
\moveto(433.77036455,96.09888031)
\lineto(431.48462273,92.06376178)
\lineto(441.67144,98.28560121)
\lineto(429.73524602,98.38462213)
\lineto(433.77036455,96.09888031)
\closepath
}
}
{
\newrgbcolor{curcolor}{0 0 0}
\pscustom[linewidth=0.81980938,linecolor=curcolor]
{
\newpath
\moveto(433.77036455,96.09888031)
\lineto(431.48462273,92.06376178)
\lineto(441.67144,98.28560121)
\lineto(429.73524602,98.38462213)
\lineto(433.77036455,96.09888031)
\closepath
}
}
{
\newrgbcolor{curcolor}{0 0 0}
\pscustom[linewidth=0.81980938,linecolor=curcolor]
{
\newpath
\moveto(426.67144,293.28560121)
\curveto(387.52786,253.23057121)(366.77777,221.33718121)(351.67144,158.28560121)
}
}
{
\newrgbcolor{curcolor}{0 0 0}
\pscustom[linestyle=none,fillstyle=solid,fillcolor=curcolor]
{
\newpath
\moveto(420.94160514,287.42234836)
\lineto(420.99497234,282.78511328)
\lineto(426.67144,293.28560121)
\lineto(416.30437006,287.36898116)
\lineto(420.94160514,287.42234836)
\closepath
}
}
{
\newrgbcolor{curcolor}{0 0 0}
\pscustom[linewidth=0.81980938,linecolor=curcolor]
{
\newpath
\moveto(420.94160514,287.42234836)
\lineto(420.99497234,282.78511328)
\lineto(426.67144,293.28560121)
\lineto(416.30437006,287.36898116)
\lineto(420.94160514,287.42234836)
\closepath
}
}
{
\newrgbcolor{curcolor}{0 0 1}
\pscustom[linewidth=0.81980938,linecolor=curcolor]
{
\newpath
\moveto(441.67144,98.28560121)
\curveto(417.70431,222.32106121)(363.1231,236.13633121)(336.67144,248.28560121)
}
}
{
\newrgbcolor{curcolor}{0 0 0}
\pscustom[linestyle=none,fillstyle=solid,fillcolor=curcolor]
{
\newpath
\moveto(440.11610802,106.33480514)
\lineto(436.27429366,108.93235393)
\lineto(441.67144,98.28560121)
\lineto(442.71365681,110.17661951)
\lineto(440.11610802,106.33480514)
\closepath
}
}
{
\newrgbcolor{curcolor}{0 0 0}
\pscustom[linewidth=0.81980938,linecolor=curcolor]
{
\newpath
\moveto(440.11610802,106.33480514)
\lineto(436.27429366,108.93235393)
\lineto(441.67144,98.28560121)
\lineto(442.71365681,110.17661951)
\lineto(440.11610802,106.33480514)
\closepath
}
}
{
\newrgbcolor{curcolor}{0 0 0}
\pscustom[linewidth=0.81980938,linecolor=curcolor]
{
\newpath
\moveto(426.67144,293.28560121)
\curveto(555.69223,223.40666121)(428.90492,439.23373121)(426.67144,293.28560121)
}
}
{
\newrgbcolor{curcolor}{0 0 0}
\pscustom[linestyle=none,fillstyle=solid,fillcolor=curcolor]
{
\newpath
\moveto(433.8801293,289.38130356)
\lineto(438.32532407,290.70306022)
\lineto(426.67144,293.28560121)
\lineto(435.20188596,284.93610879)
\lineto(433.8801293,289.38130356)
\closepath
}
}
{
\newrgbcolor{curcolor}{0 0 0}
\pscustom[linewidth=0.81980938,linecolor=curcolor]
{
\newpath
\moveto(433.8801293,289.38130356)
\lineto(438.32532407,290.70306022)
\lineto(426.67144,293.28560121)
\lineto(435.20188596,284.93610879)
\lineto(433.8801293,289.38130356)
\closepath
}
}
{
\newrgbcolor{curcolor}{0 0 0}
\pscustom[linewidth=0.81980938,linecolor=curcolor]
{
\newpath
\moveto(336.67144,248.28560121)
\curveto(231.48317,73.11258121)(256.55342,2.99535121)(441.67144,98.28560121)
}
}
{
\newrgbcolor{curcolor}{0 0 0}
\pscustom[linestyle=none,fillstyle=solid,fillcolor=curcolor]
{
\newpath
\moveto(332.45106417,241.25728922)
\lineto(333.57423864,236.7578141)
\lineto(336.67144,248.28560121)
\lineto(327.95158905,240.13411476)
\lineto(332.45106417,241.25728922)
\closepath
}
}
{
\newrgbcolor{curcolor}{0 0 0}
\pscustom[linewidth=0.81980938,linecolor=curcolor]
{
\newpath
\moveto(332.45106417,241.25728922)
\lineto(333.57423864,236.7578141)
\lineto(336.67144,248.28560121)
\lineto(327.95158905,240.13411476)
\lineto(332.45106417,241.25728922)
\closepath
}
}
{
\newrgbcolor{curcolor}{0 0 0}
\pscustom[linewidth=1,linecolor=curcolor]
{
\newpath
\moveto(441.67144,98.28560121)
\curveto(477.53174,8.62895121)(533.70618,74.68247121)(441.67144,98.28560121)
}
}
{
\newrgbcolor{curcolor}{0 0 0}
\pscustom[linestyle=none,fillstyle=solid,fillcolor=curcolor]
{
\newpath
\moveto(445.38513607,89.00075003)
\lineto(450.58455497,86.77228798)
\lineto(441.67144,98.28560121)
\lineto(443.15667403,83.80133113)
\lineto(445.38513607,89.00075003)
\closepath
}
}
{
\newrgbcolor{curcolor}{0 0 0}
\pscustom[linewidth=1,linecolor=curcolor]
{
\newpath
\moveto(445.38513607,89.00075003)
\lineto(450.58455497,86.77228798)
\lineto(441.67144,98.28560121)
\lineto(443.15667403,83.80133113)
\lineto(445.38513607,89.00075003)
\closepath
}
}
\end{pspicture}

\caption{The graphs $R$ (red) and $R'$ (blue) as constructed.}
\end{figure}
\end{proof}

The rest of the proof is ugly for guys who dislike dealing with 'pointers'.

\begin{lemma}\label{lem7}
Every AGW graphs without two incoming bunches has a spanning subgraph such that all of its vertices of maximal level belong to one non-trivial tree.
\end{lemma}
\begin{proof}

For this proof we utilize severe case-by-case analysis. Denote by $C_1$, \ldots, $C_n$ the cycles of a spanning subgraph $R$ with the trees $T_{i,1}$, \ldots, $T_{i,l_i}$ on the cycle $C_i$ and $l_i \leq |C_i|$. To be more precise let $R$ be a spanning subgraph for which the sum of the number of vertices in each cycle $\sum |C_i|$ is maximal.\\
Let us further name some important parts of $R$. Let $T$ be a tree of maximal level $L = \level(T) > 0$, i.e. $\level(T) \geq \level(T_{i, j})$ for all $i$, $j$ and $C$ the corresponding cycle. In addition let $p \in T$ have the level $L$, the root vertex of $T$ be denoted by $r$ with $C$ having a vertex $c$ and an edge $c \rightarrow r$ and $b$ be the vertex on the path $a \rightarrow \ldots \rightarrow b \rightarrow r$ which points to $r$. With $G$ being strongly connected, there exists also an edge $a \rightarrow p$, which is not in $R$. \\


Throughout the proof we want to check, if we can create a new spanning subgraph satisfying the conditions by changing some of the edges of $R$. The edges which are subject to change are $a \rightarrow p$, $b \rightarrow r$ and $c \rightarrow r$. As $L$ is the maximal level, it is sufficient to construct a tree with level $L+1$ (or greater) in one circle.\\

First we want to see what happens if we include $a \rightarrow p$ in our spanning subgraph $R$ and delete the edge $a \rightarrow x \in R$, (which exists as $a$ has out-degree one). If $a \notin T \unify C$ then $T$ is enlarged to a tree with a level greater then $L$ (as $\level(a) = L + 1$), so the new tree with root $r$ includes every vertex with maximal level, else if $a \in T$ a new cycle (of length greater or equal to one) is added to $R$, which contradicts the construction of $R$ as subgraph with maximal number of vertices in cycles. So we can assume in further considerations that $a \in C$. There is a new cycle $C' = a \rightarrow p \rightarrow \ldots \rightarrow b \rightarrow r \rightarrow \ldots \rightarrow a$. Note that if we exchange $a \rightarrow x$ with $a \rightarrow p$ in $R$, we remove the cycle $C$ from $R$ and add $C'$ (without touching any other cycle). By construction of $R$, the number of vertices in $C'$ is lesser or equal to the number of vertices in the cycle $C$. Both cycles have a common 
path from $r$ to $a$, so we can conclude, that the path $a \rightarrow x \rightarrow \ldots \rightarrow c \rightarrow r$ (of length $L' + 1$) in $C$ is longer then or equally long as the path $a \rightarrow p \rightarrow \ldots \rightarrow b \rightarrow r$ of length $L + 1$ in $T$, thus $L' \geq L$. If $L' > L$, then $C'$ has a tree of length $L'$, so by exchanging $C$ with $C'$ we have a new spanning subgraph with the desired properties. So we can assume that $L = L'$.\\

Now we want to examine the case when we replace $b \rightarrow r$ by another edge $b \rightarrow r'$, $r \neq r'$, (we consider the case that were is no such edge, i.e. that there is a bunch from $b$ to $r$ in the last case). Of course, if $b$ has an outgoing edge to a vertex of another tree $T_{i,j}$ (for some $i$, $j$), which is not the root of the tree, we would end up having a single tree which contains every vertex of maximum length. If $b$ has an edge to a vertex in a circle $C_i$, we can combine both methods of changing the outgoing edges of $a$ and $b$ to construct a new spanning subgraph. As previously showed, there is an edge $a \rightarrow x$ in $C$ and an edge $a \rightarrow p$ in $G$. So if $b$ has an outgoing edge to a vertex $r'$ of $C_i$, we can change edges, such that $C_i$ has a tree with the path $x \rightarrow \ldots \rightarrow c \rightarrow r \rightarrow \ldots \rightarrow a \rightarrow p \rightarrow \ldots \rightarrow b \rightarrow r'$, which contains a path of length $L + 1$ ($a \rightarrow p \rightarrow \ldots \rightarrow b \rightarrow r$).\\

The case that $b$ has an outgoing edge to a vertex in $T$ contradicts the definition of $R$ as seen above. So finnally we must consider whether $b$ has an edge to a vertex $r'$ in C. As we had seen previously the path $a \rightarrow x \rightarrow \ldots\ldots \rightarrow c \rightarrow p$ is of length $L + 1$, we can use the same argumentation that either the path $a \rightarrow x \rightarrow \ldots r'$ is of length $L + 1$ or we can find a new spanning subgraph which meet the requirements. But as these paths are both part of the cycle $C$, we see that $r = r'$, contradiction.\\

In the last case we can assume that were is a bunch from the vertex $b$ to the vertex $r$. As were are no two incoming bunches in $r$, there is an edge $c \rightarrow r''$, $r \neq r''$. If $r'' \in T$, we could replace $C$ by a new cycle of greater length, contradicting maximality of $R$. If $r'' \notin C \unify T$, we add a path of length greater then $L$ to another cycle or tree, which would in return have all vertices of maximal level, so $r'' \in C$. There is a path from $r''$ to $r$ in $C$, which leads to the observation, that there is a new circle $C''$ containing edges from $C$ and the edge $r \rightarrow r''$. As $r \notin C''$ the shortest path (in $C$) from $r$ to a vertex of $C''$ is of non-zero length, thus the tree of $C''$ which contains $T$ has a maximal level greater then $L$, proving the claim.
\end{proof}

The last part is now the final theorem.

\begin{theorem}\label{theo3}
The graph $G$ always has a coloring with a stable couple.
\end{theorem}

\begin{proof}
We see immediately that by Lemma \ref{lem4} $G$ has a stable couple, if it contains a vertex with two incoming bunches. In the other case, Lemma \ref{lem7} ensures that $G$ has a spanning subgraph $R$ with all vertices of maximal (positive) level belonging to one tree of $R$. Let $\mathcal{C} = \union{C_i}$ be the set of all vertices from cycles of $R$, ($C_i$ is defined as in Lemma \ref{lem7}). We want to color all edges in $R$ with one color (for example $c \in \sigma$). It is not of importance how every other edge is colored. In regard of Lemma \ref{lem2}, there is a non-empty intersection between the set of vertices of maximal level $N$ and some $F$-clique $F$. By reviewing \ref{lem5}, we observe that $|F \intersect N| = 1$. The word $c^{L - 1} \in \Sigma^{*}$ (with $L$ as the level of the vertices of $N$) translates $F$ to another $F$-clique $F'$ of the same size. With $L$ being the maximal level of all vertices it follows, that $F' \setminus \mathcal{C}$ contains exactly one vertex (i.e. $F' \setminus  
\mathcal{C} = (F \intersect N) c^{L-1})$), as every vertex of $V \setminus N$ is mapped to $\mathcal{C}$. From this we see, that $|N c^{L-1} \intersect F'| = | F' \setminus \mathcal{C}| = 1$ and $|\mathcal{C} \intersect F'| = |F'| - 1$.\\

Let $m = \lcm (|C_1|, \ldots, |C_n|)$ be the least common multiple of the lengths of all cycles, so for $v \in \mathcal{C}$ we have $v c^m = v$. We conclude, that for the $F$-clique $F'' := F' c^m$ holds $F'' \subseteq \mathcal{C}$ and $\mathcal{C} \intersect F' = F' \intersect F''$. It follows that $F'$ and $F''$ have $|F'| - 1$ common vertices, which contradicts Lemma \ref{lem3}, if the examined coloring has no stable couple.
\end{proof}

\begin{corollary}
Any AGW graph is synchronizable.
\end{corollary}
\begin{proof}
Let $G$ be the smallest counter-example. Then the number of vertices is greater then one, as a graph with one vertex is obviously synchronizable. Then we see from Theorem \ref{theo3} that there exists a stable couple, with that we have a non-trivial congruence class of more than one element. Therefore $G$ has at least one further vertex then $G / \rho$, which in return has a synchronizing coloring and implies that $G$ also has a synchronizing coloring.
\end{proof}

\bibliographystyle{unsrt} 
\bibliography{mybib}

\end{document}
