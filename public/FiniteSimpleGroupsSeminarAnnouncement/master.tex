\documentclass{article}
\usepackage{graphicx}
% for use of german language in a document
\usepackage[utf8]{inputenc} % this is needed for umlauts
\usepackage[ngerman]{babel} % this is needed for umlauts
%\usepackage[T1]{fontenc}    % this is needed for correct output of umlauts in pdf


\pagestyle{empty}

\begin{document}

\section*{Arbeitsgruppe: Einfache endliche Gruppen}

Im kommenden Semester soll eine Arbeitsgruppe zum Thema einfache endliche Gruppen angeboten werden. Ziel ist es --- grob gesprochen --- an `die meisten Gruppen', die in CFSG\footnote{Classification Theorem for Finite Simple Groups}  auftauchen, heranzuführen und die damit verbunden algebraischen Strukturen zu studieren.
Ich stehe bei Rückfragen gerne per Email zur Verfügung ({\tt jakob.schneider@tu-dresden.de}).

\paragraph{Inhalt:} Der Inhalt des Seminars soll natürlich auch durch die Hörerschaft mitbestimmt werden.
Am Anfang wird es sicherlich nochmal eine kleine Auffrischung zu sehr grundlegenden Aussagen und Sätzen der Gruppentheorie geben.
Desweiteren wird ein Großteil des Stoffes konkrete Beispiele (Gruppen) sein. Im Wesentlichen soll das Seminar auf dem Buch `The Finite Simple Groups' von Robert Wilson
basieren (aus dem wir wahrscheinlich nur einen Bruchteil schaffen werden).

\begin{figure}[htb]
    \centering
    \includegraphics[width=0.4\textwidth]{FSG.jpg}
\end{figure}

\paragraph{Gestaltung:} Optimal wäre es, wenn die Teilnehmer die Arbeitsgruppe selbst durch eigene Vorträge beleben würden.
Auch Studentinnen und Studenten höherer Semester, die sich mit der Thematik schon ein wenig auseinandergesetzt haben, sind also willkommen.
Grundsätzlich soll die Vermittlung des Stoffes jedoch derart konzipiert sein, dass dieser für Bachelorstudenten ab dem dritten Semester verständlich ist.

\begin{flushright}
    \today,\\
    Jakob Schneider  
\end{flushright}


\end{document}