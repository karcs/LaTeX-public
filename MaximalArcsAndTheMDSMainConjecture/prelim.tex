\section*{Preliminaries}
\makeatletter%%
\def\@currentlabel{Preliminaries}%
\makeatother% TODO better section command
\addcontentsline{toc}{section}{Preliminaries}

The topic of this thesis first came to my mind when I set up my three hard drives to work together as a RAID\footnote{redundant array of independent disks}. Although the linear code behind this so-called RAID5 level is of extreme simplicity, the principle itself (i.e.~the notion of optimal redundancy) turns out to admit highly interesting mathematical aspects.

The notion behind optimal redundancy in a RAID consisting of $n$ hard drives but storing only the information of $k$ hard drives ($k\leq n$) is as follows. Whenever $l\leq n-k$ hard drives stop to work the computer shall still be able to recover the information from the remaining $n-l$ ones. Actually, this means that any $k$ hard drives carry the full information and thus it is somehow `ideally distributed' among the hard drives.

Certainly, there are many ways to treat the above problem mathematically --- but the simplest among these seems to be the concept of MDS codes. Here, we associate to any of the $n$ hard drives a row in the generator matrix of an $(n,k)$-MDS code (i.e.~the minimum weight of the code is $n-k+1$). The statement that the whole information can be recovered from any $k$ hard drives then translates to the property of the generator matrix that any $k\setprod k$-submatrix of it is regular.

Of course, one could consider the above over an arbitrary ring, but for the sake of simplicity of calculation it is self-evident to restrict oneself to a field. Moreover, finite fields are most natural to use when processing `digital' data with computers --- especially fields of charactaristic two.

Now, having agreed on the finite field $\field q$ of $q$ elements, we can ask which ratios $k/n$ we can attain. 
Regrettably, as it often seems to be the case --- small charactaristic does complicate this problem a lot.

The conjecture arising from this last question is the so-called `MDS main conjecture' which was first stated in 1955 by \person{Beniamino Segre} and remains open until today.

However, an astonishing new result contributing to the answer of this problem was recently published by \person{Simeon Ball} and \person{Jean De Beule} which gives an affirmative answer to this question in the case of prime fields. In particular, the methods used in their paper~\cite{ball2012setsI} (and in the subsequent paper~\cite{ball2012setsII}) are of completely elementary nature. We will simplify some `technical aspects' of the proofs given in these two papers and present them appropriately to the reader, \see{sec-msd-nleq2p-2}.

This section can also be seen as the core part of the thesis.

The structure of the thesis is as follows:
% TODO: section label adjust
In \autoref{sec-combcons} we start with some elementary combinatorial considerations of some sets in finite projective geometry which are somehow related to the concept of projective arcs and are natural to consider in this context (these are $(k,m)$-arcs and hyperovals).
In \autoref{sec-mds-connections} we introduce the notion of a \emph{projective arc} and draw connections to some `equivalent concepts'. At the end of this section we formulate the MDS main conjecture for these different terms.
The following section presents some results which hold in a somehow more general context related to the combinatorial aspects of so-called \emph{generic} and \emph{weakly generic} hyperplane arrangements which are closely related to MDS codes as we will have seen in the previous section. The results presented here are some simple generalizations of some facts proven by \person{Zaslavsky} in~\cite{zas}.

In \autoref{sec-reedsolomon} we present the most common class of $(q+1)$-arcs and their coding theoretic version.
Especially, we construct the most relevant representations of these (and generator matrices of related codes).
Moreover, we investigate whether they are complete in some cases.

The subsequent section forms --- as we already mentioned --- the core part of the thesis presenting the proof (and classification of $(q+1)$-arcs) of the MDS main conjecture for $n\leq p$ and $n\leq 2p-2$ (without classification) given by \person{Simeon Ball} and \person{Jean De Beule} which is in my opinion of unobtrusive elegance.

Lastly, we draw some connections to the \person{Kneser} graphs of type $\KG(2n-3,n-2)$ which offer the possibility of an alternative proof of the conjecture for the case ($n\leq p$) by looking carefully at the interpolation system of tangent polynomials. Although this connection is not mentioned anywhere, I found it suitable to mention it at this point. But of course we cannot classify the $(q+1)$-arcs using an argument of that type.

There are many aspects related to the topic which are not discussed here --- especially when posing the question of maximal arcs in a rather topological context (i.e.~in characteristic zero) and also some results I rediscovered which admit a closer relation to the contents of this thesis. However, it is clear that these would certainly have gone beyond the scope of this work.

This thesis is designed to be readable without great effort for an undergraduate student with basic knowledge in projective geometry, group theory and linear algebra.

\begin{flushright}
    Jakob Schneider,\\
    August 8, 2014
\end{flushright}

\subsection*{Notations and Conventions}\label{not-conv}%% TODO fix ref somewhere
\addcontentsline{toc}{subsection}{Notations and Conventions}

We want to introduce several notation conventions used throughout the thesis.

\paragraph{Generalized binomial coefficient.}
Define the generalized binomial coefficient as
$$
\binom[q]{n}{k}:= \prod_{i=1}^{k}{\frac{q^n-q^{n-i}}{q^k-q^{k-i}}}.
$$
This counts the number of $(k-1)$-dimensional projective subspaces of an $(n-1)$-dimensional projective space.
In all sections, $\P$ denotes the projective functor. %% TODO

\paragraph{Definition up to scalar factor.}
When defining a polynomial only up to scalar factor of the underlying field, we write `$\defdoteq$' instead of `$\defeq$'.
%
\paragraph{Groups.} We assume that the reader is familiar with the projective linear group $\PGL(n,K)$, and the projective semilinear group $\PGammaL(n,K)$. Moreover, we write the semidirect product of groups $A$ and $B$ as $A\rtimes B$ (via some map $\phi:B\to\Aut A$ which is mostly clear form the context) and the wreath product of $A$ and $B$ as $A\wr B$ (here $B\leq S_m$ for some $m$ is understood as a group of permutations and $A\wr B\iso A^m\rtimes B$).

\paragraph{Substructures.} By $\Sub A$ we always mean the substructures of a structure as indicated by a substript if necessary (e.g. $\Sub_{\Aff}V$ for a vector space $V$ means the affine subspaces whereas $\Sub V=\Sub_{\Vec} V$ means the linear subspaces of $V$). Similarly, for a set $S$ with no further algebraic structure explained on it $\Sub S=\Sub_{\Set} S$ just means the powerset of $S$\footnote{More generally, we assign multiple meanings to symbols indicating the current interpretation by a subscript of the operator applied to it if necessary}.

\paragraph{Hulls.} The hull generated by some subset $S$ of a complete lattice $L$ will also be written as $\gen{S}_L$ instead of $\Join L$. This notation is especially used for vector spaces (e.g.~$\gen{S}_{\Sub V}$ means the subspace generated by $S$ of $V$). 

\paragraph{Meets and joins.} For most operations which can be interpreted as meet and join of an underlying lattice, we write `$\meet$' and `$\join$'. This applies e.g.~to the lattice of subspaces of a vector space and the greatest common divisor and least common multiple of natural numbers or polynomials (although these are unique only up to a unit).
To express the minimum and maximum of a set of numbers we use $\min$ and $\max$.

\paragraph{Vector spaces.} When we deal with the vectorspace $K^n$ (where possibly $K=\field{q}$) then $e_0,\ldots,e_{n-1}$ always denotes the standard basis, i.e.
$$
e_i:=\seq{\delta_{ij}}_{j=0}^{n-1}.
$$
Moreover, when $x_0,\ldots,x_{n-1}$ is a basis of a (finite dimensional) vector space by $\dual x_0,\ldots,\dual x_{n-1}$ we mean its dual basis, i.e.
$$
\dual x_i(x_j)=\delta_{ij}.
$$

%% TODO: fix section in refs%
\paragraph{Permutations and sequences.} In \autoref{sec-msd-nleq2p-2} we will make exhaustive use of sequences (or permutations) of vectors. For this reason we introduce the following convention. For any given set $\cA$ on which we have assigned a total order $\leq$ by the term $(A_0,\ldots,A_{n-1})$ we mean the sequence given by writing down the elements of $A_0, \ldots, A_{n-1}$ each in the order given by $\leq$. For convenience we write singletons without curly brackets (e.g.~$\seq{\set{1,5,4},2,\set{6,3}}=\seq{1,4,5,2,3,6}$ where the numbers are ordered by size).

\paragraph{Signs and determinants.} Given a $\seq{a_i}_{i=0}^{m-1}$ where all $a_i$ are distinct and $a_i\in\cA$ for some totally ordered set $\cA$ we define the sign of this sequence with respect to the order declared on $\cA$ as the sign of the permutation $\seq{a_i}_{i=0}^{m-1}$ of $\seq{\set{a_i}_{i=0}^{m-1}}$. Similarly, we can apply the determinant $\det$ to an $n$-sequence of vectors of an $n$-dimensional vectorspace --- where the former is of course always taken with respect to some fixed basis $\seq{e_i}_{i=0}^{n-1}$.
