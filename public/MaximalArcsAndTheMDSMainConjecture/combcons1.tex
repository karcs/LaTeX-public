
 --- TO BE SHIFTED OR DELETED ---
\subsection{Combinatorial consideration}
\begin{lemma}[\person{Tallini Scafati}, generalized]
  Let $S\setleq \field{q}^n$ be a $k$-element subset and let $p_i^m$ be the
  number of $m$-dimensional subspaces $V\leq \field{q}^n$ which
  intersect $S$ in exactly $i$ points
  ($i\in\nats$) then it holds for $0\leq j\leq m$ that
  $$
  \sum_{i\in\nats}{\binom{i}{j}p^m_i} = \sum_{\substack{S'\setleq S\\
          \card{S'}=j}}{\binom{n-\dim\gen{S'}}{m-\dim\gen{S'}}_q}\text{.}
  $$
  In the case where any $j$ points of $S$ are in general linear position this
  gives
  $$
  \sum_{i\in\nats}{\binom{i}{j}p^m_i}=\binom{k}{j}\binom[q]{n-j}{m-j}\text{.}
  $$
%%%% test %%%
TEST:%
\makeatletter
\context@push\a%
\def\a{Lemma 2.7}
\newtoks\funtoks%
\funtoks={\a\a}
\newcount\funcount%
\funcount=1%
\context@diff\funtoks\funcount%
\the\t@context%
\makeatother
%%
\end{lemma}

\begin{proof}
  The proof is a classical double counting argument. We count the set
  $\{(V,S')\in\binom[\field{q}]{n}{m}\settimes\binom{S}{j}:V\supseteq
  S'\}$. This is done on the LHS by summing over the subspaces the
  number of $j$-element subsets they contain (partitioning the subspaces
  in those which contain exactly $i$ points of $S$). On the RHS the same
  set is counted by summing over the $j$-element sets $S'\setleq S$
  the number of $m$-dimensional subspaces in which they are contained.
  This proves the first identity.
  The second follows by the additional assumption that any $j$-element
  subset $S'$ of $S$ is in general linear position, i.e. $\dim\gen{S'}=j$.
\end{proof}
