\documentclass{article}


\usepackage{amssymb,amsthm,amsmath,amsfonts}
\usepackage{mathabx} % more and nicer symbols symbols


% importing various types of grapics

\usepackage{epsfig}  		% For postscript
\usepackage{epic,eepic}       % For epic and eepic output from xfig


% The following is very useful in keeping track of labels while writing
%\usepackage[notcite]{showkeys}

\newtheorem{theorem}{Theorem}[section]
\newtheorem{proposition}[theorem]{Proposition}
\newtheorem{lemma}[theorem]{Lemma}
\newtheorem{corollary}[theorem]{Corollary}

\newtheorem{conj}[theorem]{Conjecture} 

\theoremstyle{definition}
\newtheorem{definition}[theorem]{Definition}
\newtheorem{example}[theorem]{Example}

\newtheorem{note}[theorem]{Note}

\theoremstyle{remark}
\newtheorem{remark}[theorem]{Remark}

% common algebraic objects
\newcommand{\reels}{\mathbb{R}}  % The real numbers
\newcommand{\nats}{\mathbb{N}} % The natural numbers
\newcommand{\ints}{\mathbb{Z}} % The integers
\newcommand{\field}{\mathbb{F}}


% new operators and relations

% setproduct
\newcommand{\setprod}{\bigtimes}
\renewcommand{\mod}[1]{\left| #1\right|} % module in Riesz-spaces
\newcommand{\dirprod}{\bigotimes} % direct product for vector spaces
\newcommand{\dirtimes}{\otimes}
\newcommand{\dirsum}{\bigoplus} % direct sum for vector spaces
\newcommand{\dirplus}{\oplus}
\newcommand{\rest}[1]{\left. #1\right\vert}
\DeclareMathOperator{\vecspace}{vec} % forgetful functor to vector spaces
\DeclareMathOperator{\ord}{ord} % forgetful functor to partial orders
\DeclareMathOperator{\lex}{lex}

\renewcommand{\implies}{\Rightarrow}

\newcommand{\equival}{\Leftrightarrow}
\begin{document}

%%
%% The title of the paper goes here.  Edit to your title.
%%

\title{On Order Convergence in Riesz-spaces}

%%
%% Now edit the following to give your name and address:
%% 





%\author{Jakob Schneider}
%\address{Technical University of Dresden,
%Germany}
%\email{jakob.schneider@tu-dresden.de}
%\urladdr{}

%\begin{abstract}
%Great stuff.
%\end{abstract}

% generate title things
\maketitle

% toc
\tableofcontents

\section{Introduction}

In this paper we want to deal with two basic definitions of order convergence in Riesz-spaces and outline their connections.

\begin{definition}[O1-convergence]\label{o1}
Let $R$ be a Riesz-space and $(x_i)_{i\in I}$ a net in $R$. Then we call $(x_i)_{i\in I}$ O1-convergent if and only if there exists some point $x\in V$ and a net $(y_j)_{j\in J}$ in $R$ with $y_j\downarrow 0$ such that
\begin{equation}
\forall j\in J \exists i_0\in I \forall i\geq i_0 : \mod{x-x_i}\leq y_j
\end{equation}
\end{definition}

Indeed, this definition is just another mean than $\limsup$ (of $x_i$), whose existence we cannot assure, to get a good notion of convergence.

The second definition is obviously stronger than the first:

\begin{definition}[O2-convergence]\label{o2}
A net $(x_i)_{i\in I}$ in a Riesz-space $R$ is called O2-convergent if and only if there exists some point $x\in V$ and a net $(y_i)_{i\in I}$ in $R$ with $y_i\downarrow 0$ such that
\begin{equation}
\exists i_0\in I \forall i\geq i_0 : \mod{x-x_i}\leq y_i
\end{equation}
\end{definition}

\section{On finite dimensional Riesz-spaces}

Now having introduced these notions of convergence, several interesting questions arise. As for example which topologies may be induced by them.

In this section we want to discuss this question especially for finite dimensional spaces.
Indeed, the understanding of the character of finite dimensional Riesz-spaces is not that complicated. 
At first, it becomes necessary to establish some notation. 

\begin{definition}[direct lexicographic sum and product]
For some well-order $W$ and Riesz-spaces $R_\omega$ ($\omega\in W$) we define their direct lexicographic sum as
\begin{equation}
\dirsum\limits^{\lex}_{\omega\in W}{R_\omega} := \left(\dirsum\limits_{\omega\in W}{\vecspace{R_\omega}},\leq_{\dirplus}\right)\text{.}
\end{equation}
Here $\leq_{\dirplus}$ is defined by 
\begin{equation}
\leq_{\dirplus} = \rest{\setprod\limits^{\lex}_{\omega\in W}{\ord{R_w}}}_{\dirsum\limits_{\omega\in W}{\vecspace{R_\omega}}}
\end{equation}
and $\bigtimes\limits^{\lex}$ denotes the lexicographic product of the orders. Analogously, one defines the direct lexicographic sum.
\end{definition}

And another definition we will need is  

\begin{definition}[direct pointwise product and sum]
For some well-order $W$ and Riesz-spaces $R_\omega$ ($\omega\in W$) we define their direct pointwise sum and analogously to the lexicographic ones, replacing the order by the pointwise order restricted to the domain.
\end{definition}

We want to start with the following characterization:

\begin{lemma}[Characterization of finite-dimensional Riesz-spaces]\label{chargen}
Let $R$ be a finite-dimensional Riesz-space. Then we have a unique representation
of $R$ as
\begin{equation}
R = \dirsum^{\lex}_{i\in I}{R_i}
\end{equation}
for archimedian subspaces $R_i$ of $R$ and a finite index set (well-order) $I$.
\end{lemma}

For the proof of Lemma (\ref{chargen}) we need a little help by another one.

\begin{lemma}[reformulation of archimedian property] Let $V$ be an Archimedian ordered vectorspace. Then $V^+$ contains no non-trivial affine subspace. If we require additionally that $V$ is a vector lattice, then the last two statements are equivalent. 
\end{lemma}

\begin{proof}
The first statement becomes clear by the following equivalences starting from the negated archimedian property of $V$ ($V^+=\{v\in V:v>0\}$
\begin{equation}
{\exists u,v\in V^+:\field^+ u \leq v}\equival {\exists u,v\in V^+:0\leq \field u + v}\label{eq1}
\end{equation}
This already proofs the first fact stated (this is only an implication, because we just proved the equivalence with $u\in V^+$). WRONG Now if $V$ is a lattice and there is some one dimensional affine subspace $\field u+v$ in $V^+$, we may say w.l.o.g. that $u^+>0$. Then one checks easily that $\field u^+ + v\subset V^+$ and thus we are done by the upper equivalence (\ref{eq1}).  
\end{proof}

\begin{remark}
With $V$ not being a lattice, the equivalence fails by the example $V=\reels^2$ and $V^+=\{0\}\cup \reels^+\times \reels$.
\end{remark}

\begin{lemma}
A Riesz space $R$ is archimedian if and only if $R^+$ does not contain a non-trivial affine subspace.
\end{lemma}

The proof is simple.

We want to come to the conclusion 

\begin{lemma}[Classification of finite-dimensional Riesz-spaces] Every finite dimensional Riesz-space arises from successive pointwise and lexicographic (order) products of subspaces. Especially, every archimedian Riesz-space $R$ with $\dim{R}=n\in\nats$ has exactly $n$ main rays.
\end{lemma}



\end{document}






