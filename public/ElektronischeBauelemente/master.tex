\documentclass{article}

\usepackage{booktabs}
\usepackage[de]{mathenv}
\usepackage{tikz}

%My personal maths package
%\bibliography{Bibliography.bib}

%%%%%%%%%%%%%%%%%%%%%%%%%%%%%%%%%%%%%%%%%%%%%%%%%%%%%%%%%%%%%%%%%%%%%%%%%%%%%%
%%%%% MATH PACKAGES %%%%%%%%%%%%%%%%%%%%%%%%%%%%%%%%%%%%%%%%%%%%%%%%%%%%%%%%%%
%%%%%%%%%%%%%%%%%%%%%%%%%%%%%%%%%%%%%%%%%%%%%%%%%%%%%%%%%%%%%%%%%%%%%%%%%%%%%%

% very good package
%\usepackage{mathtools}

%%% font stuff
\usepackage[T1]{fontenc}        % for capitals in section /paragraph etc.
\usepackage[utf8]{inputenc}     % use utf8 symbols in code

\usepackage{amssymb,amsmath,amsfonts} % amsthm not needed -- use my own envs
%\usepackage{mathtools}
%\mathtoolsset{showonlyrefs}
% further alternative math packages: unicode-math, abx-math
\usepackage{bm}
\usepackage{mathrsfs} % used for: fraktal math letters
\usepackage[bigsqcap]{stmaryrd} % used for: big square cap symbol
\usepackage{xargs}

% standard packages
%\usepackage{color} % for color

%%
%%index

\newcommand*{\keyword}[2][\empty]{\emph{#2}\ifx#1\empty\index{#2}\else\index{#1}\fi}
\newcommand*{\person}[1]{\textsc{#1}}

%%%%%%%%%%%%%%%%%%%%%%%%%%%%%%%%%%%%%%%%%%%%%%%%%%%%%%%%%%%%%%%%%%%%%%%%%%%%%%%%%%%%%%%%%%%%%%%%%%%%%%%%%%%%
%%%%% MATH ALPHABETS & SYMBOLS %%%%%%%%%%%%%%%%%%%%%%%%%%%%%%%%%%%%%%%%%%%%%%%%%%%%%%%%%%%%%%%%%%%%%%%%%%%%%
%%%%%%%%%%%%%%%%%%%%%%%%%%%%%%%%%%%%%%%%%%%%%%%%%%%%%%%%%%%%%%%%%%%%%%%%%%%%%%%%%%%%%%%%%%%%%%%%%%%%%%%%%%%%

% w: http://milde.users.sourceforge.net/LUCR/Math/math-font-selection.xhtml

% ===== Set quick commands for math letters ================================================================
% calagraphic letters (only upper case available; standard)
\newcommand{\cA}{\mathcal{A}}
\newcommand{\cB}{\mathcal{B}}
\newcommand{\cC}{\mathcal{C}}
\newcommand{\cD}{\mathcal{D}}
\newcommand{\cE}{\mathcal{E}}
\newcommand{\cF}{\mathcal{F}}
\newcommand{\cG}{\mathcal{G}}
\newcommand{\cH}{\mathcal{H}}
\newcommand{\cI}{\mathcal{I}}
\newcommand{\cJ}{\mathcal{J}}
\newcommand{\cK}{\mathcal{K}}
\newcommand{\cL}{\mathcal{L}}
\newcommand{\cM}{\mathcal{M}}
\newcommand{\cN}{\mathcal{N}}
\newcommand{\cO}{\mathcal{O}}
\newcommand{\cP}{\mathcal{P}}
\newcommand{\cQ}{\mathcal{Q}}
\newcommand{\cR}{\mathcal{R}}
\newcommand{\cS}{\mathcal{S}}
\newcommand{\cT}{\mathcal{T}}
\newcommand{\cU}{\mathcal{U}}
\newcommand{\cV}{\mathcal{V}}
\newcommand{\cW}{\mathcal{W}}
\newcommand{\cX}{\mathcal{X}}
\newcommand{\cY}{\mathcal{Y}}
\newcommand{\cZ}{\mathcal{Z}}

% bold math letters (standard)
\newcommand{\bfA}{\mathbf{A}}
\newcommand{\bfB}{\mathbf{B}}
\newcommand{\bfC}{\mathbf{C}}
\newcommand{\bfD}{\mathbf{D}}
\newcommand{\bfE}{\mathbf{E}}
\newcommand{\bfF}{\mathbf{F}}
\newcommand{\bfG}{\mathbf{G}}
\newcommand{\bfH}{\mathbf{H}}
\newcommand{\bfI}{\mathbf{I}}
\newcommand{\bfJ}{\mathbf{J}}
\newcommand{\bfK}{\mathbf{K}}
\newcommand{\bfL}{\mathbf{L}}
\newcommand{\bfM}{\mathbf{M}}
\newcommand{\bfN}{\mathbf{N}}
\newcommand{\bfO}{\mathbf{O}}
\newcommand{\bfP}{\mathbf{P}}
\newcommand{\bfQ}{\mathbf{Q}}
\newcommand{\bfR}{\mathbf{R}}
\newcommand{\bfS}{\mathbf{S}}
\newcommand{\bfT}{\mathbf{T}}
\newcommand{\bfU}{\mathbf{U}}
\newcommand{\bfV}{\mathbf{V}}
\newcommand{\bfW}{\mathbf{W}}
\newcommand{\bfX}{\mathbf{X}}
\newcommand{\bfY}{\mathbf{Y}}
\newcommand{\bfZ}{\mathbf{Z}}
\newcommand{\bfa}{\mathbf{a}}
\newcommand{\bfb}{\mathbf{b}}
\newcommand{\bfc}{\mathbf{c}}
\newcommand{\bfd}{\mathbf{d}}
\newcommand{\bfe}{\mathbf{e}}
\newcommand{\bff}{\mathbf{f}}
\newcommand{\bfg}{\mathbf{g}}
\newcommand{\bfh}{\mathbf{h}}
\newcommand{\bfi}{\mathbf{i}}
\newcommand{\bfj}{\mathbf{j}}
\newcommand{\bfk}{\mathbf{k}}
\newcommand{\bfl}{\mathbf{l}}
\newcommand{\bfm}{\mathbf{m}}
\newcommand{\bfn}{\mathbf{n}}
\newcommand{\bfo}{\mathbf{o}}
\newcommand{\bfp}{\mathbf{p}}
\newcommand{\bfq}{\mathbf{q}}
\newcommand{\bfr}{\mathbf{r}}
\newcommand{\bfs}{\mathbf{s}}
\newcommand{\bft}{\mathbf{t}}
\newcommand{\bfu}{\mathbf{u}}
\newcommand{\bfv}{\mathbf{v}}
\newcommand{\bfw}{\mathbf{w}}
\newcommand{\bfx}{\mathbf{x}}
\newcommand{\bfy}{\mathbf{y}}
\newcommand{\bfz}{\mathbf{z}}

% fractal math letters (standard)
\newcommand{\fkA}{\mathfrak{A}}
\newcommand{\fkB}{\mathfrak{B}}
\newcommand{\fkC}{\mathfrak{C}}
\newcommand{\fkD}{\mathfrak{D}}
\newcommand{\fkE}{\mathfrak{E}}
\newcommand{\fkF}{\mathfrak{F}}
\newcommand{\fkG}{\mathfrak{G}}
\newcommand{\fkH}{\mathfrak{H}}
\newcommand{\fkI}{\mathfrak{I}}
\newcommand{\fkJ}{\mathfrak{J}}
\newcommand{\fkK}{\mathfrak{K}}
\newcommand{\fkL}{\mathfrak{L}}
\newcommand{\fkM}{\mathfrak{M}}
\newcommand{\fkN}{\mathfrak{N}}
\newcommand{\fkO}{\mathfrak{O}}
\newcommand{\fkP}{\mathfrak{P}}
\newcommand{\fkQ}{\mathfrak{Q}}
\newcommand{\fkR}{\mathfrak{R}}
\newcommand{\fkS}{\mathfrak{S}}
\newcommand{\fkT}{\mathfrak{T}}
\newcommand{\fkU}{\mathfrak{U}}
\newcommand{\fkV}{\mathfrak{V}}
\newcommand{\fkW}{\mathfrak{W}}
\newcommand{\fkX}{\mathfrak{X}}
\newcommand{\fkY}{\mathfrak{Y}}
\newcommand{\fkZ}{\mathfrak{Z}}
\newcommand{\fka}{\mathfrak{a}}
\newcommand{\fkb}{\mathfrak{b}}
\newcommand{\fkc}{\mathfrak{c}}
\newcommand{\fkd}{\mathfrak{d}}
\newcommand{\fke}{\mathfrak{e}}
\newcommand{\fkf}{\mathfrak{f}}
\newcommand{\fkg}{\mathfrak{g}}
\newcommand{\fkh}{\mathfrak{h}}
\newcommand{\fki}{\mathfrak{i}}
\newcommand{\fkj}{\mathfrak{j}}
\newcommand{\fkk}{\mathfrak{k}}
\newcommand{\fkl}{\mathfrak{l}}
\newcommand{\fkm}{\mathfrak{m}}
\newcommand{\fkn}{\mathfrak{n}}
\newcommand{\fko}{\mathfrak{o}}
\newcommand{\fkp}{\mathfrak{p}}
\newcommand{\fkq}{\mathfrak{q}}
\newcommand{\fkr}{\mathfrak{r}}
\newcommand{\fks}{\mathfrak{s}}
\newcommand{\fkt}{\mathfrak{t}}
\newcommand{\fku}{\mathfrak{u}}
\newcommand{\fkv}{\mathfrak{v}}
\newcommand{\fkw}{\mathfrak{w}}
\newcommand{\fkx}{\mathfrak{x}}
\newcommand{\fky}{\mathfrak{y}}
\newcommand{\fkz}{\mathfrak{z}}

% script math symbols (only uppercase; package: mathrsfs)
\newcommand{\sA}{\mathscr{A}}
\newcommand{\sB}{\mathscr{B}}
\newcommand{\sC}{\mathscr{C}}
\newcommand{\sD}{\mathscr{D}}
\newcommand{\sE}{\mathscr{E}}
\newcommand{\sF}{\mathscr{F}}
\newcommand{\sG}{\mathscr{G}}
\newcommand{\sH}{\mathscr{H}}
\newcommand{\sI}{\mathscr{I}}
\newcommand{\sJ}{\mathscr{J}}
\newcommand{\sK}{\mathscr{K}}
\newcommand{\sL}{\mathscr{L}}
\newcommand{\sM}{\mathscr{M}}
\newcommand{\sN}{\mathscr{N}}
\newcommand{\sO}{\mathscr{O}}
\newcommand{\sP}{\mathscr{P}}
\newcommand{\sQ}{\mathscr{Q}}
\newcommand{\sR}{\mathscr{R}}
\newcommand{\sS}{\mathscr{S}}
\newcommand{\sT}{\mathscr{T}}
\newcommand{\sU}{\mathscr{U}}
\newcommand{\sV}{\mathscr{V}}
\newcommand{\sW}{\mathscr{W}}
\newcommand{\sX}{\mathscr{X}}
\newcommand{\sY}{\mathscr{Y}}
\newcommand{\sZ}{\mathscr{Z}}

%%%%%%%%%%%%%%%%%%%%%%%%%%%%%%%%%%%%%%%%%%%%%%%%%%%%%%%%%%%%%%%%%%%%%%%%%%%%%%%%%%%%%%
%%%% BOLD MATH IN BOLD TEXT ENVIRONMENT %%%%%%%%%%%%%%%%%%%%%%%%%%%%%%%%%%%%%%%%%%%%%%
%%%%%%%%%%%%%%%%%%%%%%%%%%%%%%%%%%%%%%%%%%%%%%%%%%%%%%%%%%%%%%%%%%%%%%%%%%%%%%%%%%%%%%

% for bold math in bold text (e.g. sections)
\makeatletter
\g@addto@macro\bfseries{\boldmath}
\makeatother

\def\brackets#1{\ifx#1\empty\else\left(#1\right)\fi}

%%%%%%%%%%%%%%%%%%%%%%%%%%%%%%%%%%%%%%%%%%%%%%%%%%%%%%%%%%%%%%%%%%%%%%%%%%%%%%%%%%%%%%
%%%%% CATEGORY THEORY %%%%%%%%%%%%%%%%%%%%%%%%%%%%%%%%%%%%%%%%%%%%%%%%%%%%%%%%%%%%%%%%
%%%%%%%%%%%%%%%%%%%%%%%%%%%%%%%%%%%%%%%%%%%%%%%%%%%%%%%%%%%%%%%%%%%%%%%%%%%%%%%%%%%%%%

% ===== Category theory concepts =====================================================

\newcommand{\Ob}{\mathop\mathrm{Ob}}
\newcommand{\Mor}{\mathop\mathrm{Mor}}

\newcommand{\ccoprod}{\bigsqcup}
\newcommand{\cprod}{\bigsqcap}
\newcommand{\cincl}{\mathop\mathrm{incl}}
\newcommand{\cproj}{\mathop\mathrm{pr}}


% ===== Define standard categories ===================================================

% Define sets
\newcommand{\Set}{\mathbf{Set}}
% Define set-builder operator (equivalent to gen for algebras)
\newcommand{\set}[1]{\left\{#1\right\}}
% define interval operator: o - open, c - closed
\newcommand{\intervalcc}[2]{\left[#1,#2\right]}
\newcommand{\intervalco}[2]{\left[#1,#2\right)}
\newcommand{\intervaloc}[2]{\left(#1,#2\right]}
\newcommand{\intervaloo}[2]{\left(#1,#2\right)}

\newcommand{\inter}{\mathop\mathrm{int}}
\newcommand{\face}{\mathop\mathrm{F}}
\newcommand{\Pol}{\mathop\mathrm{Pol}}
\newcommand{\Inv}{\mathop\mathrm{Inv}}
\def\struct#1{\gen{#1}}

\let\originaltimes\times%
\renewcommand{\times}{\mathbin{\sqcap}}
\newcommand\settimes{\originaltimes}
\newcommand{\setleq}{\subseteq}
\newcommand{\setgeq}{\supseteq}

\newcommand{\pderive}[2]{\frac{\partial{#1}}{\partial{#2}}}
\renewcommand{\div}{\mathop\mathrm{div}}

%% Diffgeo
\def\Ric{\mathop\mathrm{Ric}}
\def\ric{\mathop\mathrm{ric}}
\def\tr{\mathop\mathrm{tr}}

\def\cotimes{\mathbin{\sqcup}}

\def\@rightopen#1{\ifx#1]{\right]}\else{\interval@errmessage}\fi}
\def\@leftclosed[#1){\left[#1\right)}
\makeatother

% finite
\newcommand{\fin}{\mathrm{fin}}

% Define groups (optarg: properties such as -> abelian, noetherian (acc), artinian (dcc) etc.)
\newcommand{\Grp}[1][\empty]{\if\empty{#1}{\mathbf{Grp}}\else{\mathbf{Grp}_{#1}}}
\def\PGL{\mathrm{PGL}}
\def\PGammaL{\mathrm{P\Gamma L}}
\def\GL{\mathrm{GL}}
% Define rings
\newcommand{\rg}{\mathrm{rg}} %rank of a matrix
\newcommand{\Rg}[1][\empty]{\if\empty{#1}{\mathbf{Rg}}\else{\mathbf{Rg}_{#1}}}
\edef\units#1{#1^{\settimes}}
\def\dual#1{#1^{\ast}}

%% redefine the command \P to produce the projective functor in math mode
\let\parsymb\P%
\def\P{\ifmmode\mathrm{P}\else\parsymb\fi}
\renewcommand{\iff}{\ifmmode\equival\else{if and only if}\fi}
\newcommand{\quotring}{\mathop{\mathrm{Q}}}
\newcommand{\rad}{\mathrm{rad}}
% Standard rings
% integral domains
\newcommand{\ID}{\mathbf{ID}}
% unique factorization domains
\newcommand{\UFD}{\mathbf{UFD}}
% principal ideal domains
\newcommand{\PID}{\mathbf{PID}}

% Define modules over a group or ring
\newcommand{\Mod}[1]{\mathbf{Mod}_{#1}}
% Define vector space over a field
\renewcommand{\Vec}[1]{\mathbf{Vec}_{#1}}%

% when cases
\def\otherwise{\textrm{otherwise}}

% new concepts
\newcommand{\new}[1]{\emph{#1}}

\usepackage{xifthen,xstring}

% replace the bar command by overline when argument just one character (shorter and better)
%$\let\oldbar\bar
%\renewcommand{\bar}[1]{\StrLen{#1}[\length]\ifthenelse{\length > 1}{\overline{#1}}{\oldbar{#1}}}
\def\bar{\overline}

%% argument in equatoin
\def\arg{\bullet}

% groups and algebras
\newcommand{\Con}{\mathop\mathrm{Con}}
\newcommand{\Sub}{\mathop\mathrm{Sub}}
\newcommand{\Hom}{\mathop\mathrm{Hom}}
\newcommand{\Aut}{\mathop\mathrm{Aut}}
\newcommand{\Out}{\mathop\mathrm{Out}}
\newcommand{\End}{\mathop\mathrm{End}}
\newcommand{\id}{\mathop\mathrm{id}}
\newcommand{\rk}{\mathop\mathrm{rk}} % rank of a group module/ lattice
\newcommandx{\con}[1][1=\empty]{\ifx#1\empty{\mathop{\mathrm{con}}}\else{\mathop{\mathrm{con}}\left(#1\right)}\fi}
\newcommand{\leftsemidirprod}[1][]{\mathbin{\ifx&#1&\ltimes\else{\ltimes_{#1}}\fi}}
\newcommand{\rightsemidirprod}[1][]{\ifx#1\empty\rtimes\else{\rtimes_{#1}}\fi}
\newcommand{\normalisor}[2][]{\ifx#1\empty{\mathrm{N}\left(#2\right)}\else{\mathrm{N}_{#1} \left(#2\right)}\fi}
% support
\newcommand{\spt}{\mathop{\mathrm{spt}}}
% commutator
\newcommand{\gcom}[2]{\left[#1,#2\right]}

% physics stuff
\newcommand{\float}[3][\empty]{\ifx#1\empty{{#2}\cdot{10^{#3}}}\else{{#2}\cdot{{#1}^{#3}}}\fi}
\makeatletter
\def\newunit#1{\@namedef{#1}{\mathrm{#1}}}
\def\mum{\mathrm{\mu m}}
\def\ohm{\Omega}
\newunit{V}
\newunit{mV}
\newunit{kV}
\newunit{s}
\newunit{ms}
\def\mus{\mathrm{\mu s}}
\newunit{m}
\newunit{nm}
\newunit{cm}
\newunit{mm}
\newunit{fF}
\newunit{A}

\newunit{fA}
\newunit{C}

% elements
\def\newelement#1{\@namedef{#1}{\mathrm{#1}}}
\newelement{Si}
\makeatother


% groups
\newcommand{\ord}{\mathop\mathrm{ord}}
\newcommand{\divides}{|}

\newcommand{\conleq}{\trianglelefteq}
\newcommand{\congeq}{\trianglerighteq}

% common algebraic objects
\newcommand{\reals}{\mathbb{R}} 			% real numbers
\newcommand{\nats}{\mathbb{N}} 				% natural numbers
\newcommand{\ints}{\mathbb{Z}} 				% integers
\newcommand{\rats}{\mathbb{Q}}				% rationals
\newcommand{\complex}{\mathbb{C}}			% complex numbers
\newcommand{\field}[1]{\mathbb{F}_{#1}}  		% finite field
\newcommand{\cards}{\boldsymbol{Cn}}                     % The cardinal numbers
\newcommand{\ords}{\boldsymbol{On}}                      % The ordinal numbers

% graphs
\def\KG{\mathop\mathrm{KG}}                     % Knesergraph

\newcommand{\uvect}{\boldsymbol{e}}

% new operators and relations

%%%%%%%%%%%%%%%%%%%
% complex numbers %
%%%%%%%%%%%%%%%%%%%

\renewcommand{\Re}{\mathop\mathrm{Re}}		% real part
\renewcommand{\Im}{\mathop\mathrm{Im}}		% imaginary part
\newcommand{\sgn}{\mathop\mathrm{sgn}}				% the sign operator (0 for 0)

%%%%%%%%%%%%%%%%
% reel numbers %
%%%%%%%%%%%%%%%%

\newcommand{\floor}[1]{\left\lfloor#1\right\rfloor}
\newcommand{\ceil}[1]{\left\lceil#1\right\rceil}

%%%%%%%%%%%%%%%%%%
% set operations %
%%%%%%%%%%%%%%%%%%

\newcommand{\intersect}{\cap}			% intersect to sets
\newcommand{\setjoin}{\cup}				% join two sets
\newcommand{\setmeet}{\cap}                     % intersect to sets
\newcommand{\bigsetjoin}{\bigcup}			% the union of sets ... subscripts to be added
\newcommand{\distunion}{\dot{\bigcup}}	% disjoint union of sets ... subscripts to be added
\newcommand{\bigsetmeet}{\bigcap}		% intersection of sets
\newcommand{\powerset}[1][]{\ifx&#1&\mathcal{P}\else\mathcal{P}_{#1}\fi}		% powerset ... to be customized
\newcommand{\card}[1]{\left|#1\right|}

%%%%%%%%%%%%%%%%%%%%%%%%%%%%%%%%%%%%%%%%
% composition operations of structures %
%%%%%%%%%%%%%%%%%%%%%%%%%%%%%%%%%%%%%%%%

%\newcommand{\setprod}{\bigtimes}			% setproduct - needed
\newcommand{\dirprod}{\bigotimes} 			% direct product for groups and spaces
\newcommand{\dirtimes}{\otimes}				% direct multiply for groups and spaces
\newcommand{\dirsum}{\bigoplus} 			% direct sum for groups and spaces
\newcommand{\dirplus}{\oplus}				% direct add for groups and spaces
\newcommand{\inprod}[2]{\left\langle #1,#2 \right\rangle}

\newcommand{\tuple}{\meet}
\newcommand{\cotuple}{\join}

%%%%%%%%%%%%%%
% categories %
%%%%%%%%%%%%%%

% combinatorics
%%

\renewcommand{\binom}[3][\empty]{\if\empty{#1}{{#2 \choose #3}}\else{{#2 \choose #3}_{#1}}}

%%%%%%%%%%%%%%%%%%%%%%%%%%%%%%%%%%%%%%%%%%%%%%%%%%%%%
% metric spaces and normed spaces and vector spaces %
%%%%%%%%%%%%%%%%%%%%%%%%%%%%%%%%%%%%%%%%%%%%%%%%%%%%%

\newcommand{\dist}{\mathop\mathrm{dist}}				% distance operator ... dist(A,b), where A is a set and b a point
\newcommand{\diam}{\mathop\mathrm{diam}}	% diameter operator for sets
\newcommand{\norm}[1]{\left\Vert #1 \right\Vert}	% norm in a normed space ... subscript to be added
\newcommand{\conv}{\mathop\mathrm{conv}} 			% convex hull - vectorspaces
\newcommand{\lin}{\mathop\mathrm{lin}} 				% linear hull - vectorspaces
\newcommand{\aff}{\mathop\mathrm{aff}}				% affine hull - vectorspaces

%%%%%%%%%%%%%%%%%%%%%%
% operators in rings %
%%%%%%%%%%%%%%%%%%%%%%

\newcommand{\lcm}{\mathop\mathrm{lcm}}				% least common multiple - in euclidean rings
\renewcommand{\gcd}{\mathop\mathrm{gcd}}				% greatest command devisor - in euclidean rings
\newcommand{\res}{\mathop\mathrm{res}}				% residue of p mod q is res(p,q)
\renewcommand{\mod}{\textrm{ mod }}

%%%%%%%%%%%%%%%%%%%
% logical symbols %
%%%%%%%%%%%%%%%%%%%

%\newcommand{\impliedby}{\Leftarrow}				% reverse implicatoin arrow
%\newcommand{\implies}{\Rightarrow}				% implication
\newcommand{\equival}{\Leftrightarrow}				% equivalence

%%%%%%%%%%%%%%%%%%%%%%%%%%%%%%%%%%
% functions - elementary symbols %
%%%%%%%%%%%%%%%%%%%%%%%%%%%%%%%%%%

\newcommand{\rest}[1]{\left. #1\right\vert}		% restriction of a function to a set / also used as restriction in other terms like differential expressions / evaluation of a function
\newcommand{\rto}[3][]{#2\ifx&#1&\rightarrow\else\stackrel{#1}{\rightarrow}\fi#3}
\renewcommand{\to}{\rightarrow}							% arrow between domain and image
\newcommand{\dom}{\mathop\mathrm{dom}}					% domain of a function
\newcommand{\im}{\mathop\mathrm{im}}						% image of a function
\newcommand{\compose}{\circ}							% compose two functions
\newcommand{\cont}{\mathop\mathrm{C}}					% continuous functions from a domain into the reels or complex numbers 

%%%%%%%%%%%%%%%%%%%%
% groups - symbols %
%%%%%%%%%%%%%%%%%%%%

\newcommand{\stab}[1][]{\if&#1{\mathop\mathrm{stab}}&\else{\mathop\mathrm{stab}_{#1}}\fi}					% the stabilizer ... subscripts to be added
\newcommand{\orb}[1][]{\ifx&#1&\mathrm{orb}\else\mathrm{orb}_{#1}\fi} 					% orbit ... subscrit to be added (group)
\newcommand{\gen}[2][\empty]{\ifx#1\empty{\left\langle#2\right\rangle}\else{\left\langle#2\right\rangle_{#1}}\fi}					% generate ... kind of hull operator ---- to be thought of !!!!!!!

\def\Clo{\mathrm{Clo}}
\def\Loc{\mathrm{Loc}}
%% nets
\newcommand{\net}[2][\empty]{\ifx#1\empty{\left(#2\right)}\else{{\left(#2\right)}_{#1}}\fi}

%% open half ray
\newcommand{\ray}[2]{R_{#1}(#2)}

%% new
\let\oldcong\cong%
\newcommand{\iso}{\oldcong}

\def\cong{\equiv}
\newcommand{\base}[2]{\left[#2\right]_{#1}}                                   % base n expansion of some number
%%%%%%%%%%%%%%%%%%%%%%%%%%%%%%%%%
% matrices and linear operators %
%%%%%%%%%%%%%%%%%%%%%%%%%%%%%%%%%

\newcommand{\diag}{\mathop\mathrm{diag}}				% diagonal matrix or operator
\newcommand{\Eig}[1]{\mathop\mathrm{Eig}_{#1}}		% eigenspace for a certain eigenvalie		
\newcommand{\trace}{\mathop\mathrm{tr}}				% trace of a matrix
\newcommand{\trans}{\top} 							% transponse matrix

%%%%%%%%%%%%%
% constants %
%%%%%%%%%%%%%

\renewcommand{\i}{\boldsymbol{i}}			% imaginary unit
\newcommand{\e}{\boldsymbol{e}}				% the Eulerian constant

%%%%%%%%%%%%%%%%%%%%%%%%%%%%%%
% limit operators and arrows %
%%%%%%%%%%%%%%%%%%%%%%%%%%%%%%

\newcommand{\upto}{\uparrow}				% convergence from above
\newcommand{\downto}{\downarrow}			% convergence from below

%%
% other
%%
\newcommand{\cind}{\mathop\mathrm{Ind}}		% Cauchy index
\newcommand{\sgnc}{\sigma}					% sign changes
\newcommand{\wnumb}{\omega}					% winding number
\newcommand{\cfunc}{\mathop\mathrm{Cf}}		% Cauchy function of a compact curve in complex\setminus\{0\}


% evaluation of a function as a difference or single value

\newcommand{\abs}[1]{\left|#1\right|}
\newcommand{\conj}[1]{\overline{#1}}
\newcommand{\diff}{\mathop\mathrm{d}}

%%%%% test
\newcommand{\distjoin}{\mathaccent\cdot\cup}	% to be modified (name)
\newcommand{\cl}{\mathop\mathrm{cl}}				% topological closure
\newcommand{\sphere}{\mathbb{S}} % n-sphere
\newcommand{\ball}{\mathbb{B}} % n-ball
\newcommand{\bound}{\partial}
\newcommand{\bigmeet}{\mathop\mathrm{\bigwedge}}
\newcommand{\bigjoin}{\mathop\mathrm{\bigvee}}
\newcommandx{\rchar}[1][1=\empty]{\mathop\mathrm{char}\brackets{#1}}              % characteristic of a ring
\newcommand{\lgor}{\vee}                               % logical
\newcommand{\lgand}{\wedge}
\newcommand{\codim}{\mathop\mathrm{codim}}
\newcommand{\row}{\mathop\mathrm{row}}
\newcommand{\cone}{\mathop\mathrm{cone}}
\newcommand{\comp}{\mathop\mathrm{comp}}
\newcommand{\proj}{\mathrm{P}}
\def\PG{\mathrm{PG}}           % projective space
\newcommand{\meet}{\wedge}
\newcommand{\join}{\vee}
\newcommand{\col}{\mathop\mathrm{col}}
\newcommand{\vol}{\mathop\mathrm{vol}\nolimits}
%arrangements
\newcommand{\tpert}{\mathop\mathrm{tpert}}
\renewcommand{\epsilon}{\varepsilon}
% groups
\newcommand{\symgr}{\mathop\mathrm{Sym}}
\newcommand{\symalg}{\mathop\mathrm{S}}
\newcommand{\extalg}{\mathop\mathrm{\Lambda}}
\newcommand{\extpow}[1]{\mathop\mathrm{\Lambda}^{#1}}

%% set hulloperators -> define



\newcommandx{\homl}[3][1=1,2=2,3=3]{\ifx#1\empty{\mathrm{Hom}}\else{\mathrm{Hom}_{#1}}\fi(#2,#3)}
\makeatletter
\newenvironment{myproofof}[1]{\par
  \pushQED{\qed}%
  \normalfont \topsep6\p@\@plus6\p@\relax
  \trivlist
  \item[\hskip\labelsep
        \bfseries
    Proof of #1\@addpunct{.}]\ignorespaces
}{%
  \popQED\endtrivlist\@endpefalse
}
\makeatother



%%% environment test with enumerates

    

% Local variables:
% mode: tex
% End:


% for use of german language in a document
\usepackage[utf8]{inputenc} % this is needed for umlauts
\usepackage[ngerman]{babel} % this is needed for umlauts
\usepackage[T1]{fontenc}    % this is needed for correct output of umlauts in pdf


\begin{document}
\section{Formeln und Fakten}

\subsection{Physikalische Grundlagen}

$J$\ldots Stromdichte,
$\rho$\ldots Ladungsdichte,
$q$\ldots Ladung eines Ladungsträgers,
$n$\ldots Teilchendichte,
$\mu$\ldots Beweglichkeit,
$v_D$\ldots Driftgeschwindigkeit,
$E$\ldots elektische Feldstärke,
$\varsigma$\ldots spezifische Leitfähigkeit,
$\varrho$\ldots spezifischer Widerstand,

$$
J = \rho v_D,\ v_D=\mu E,\ \rho=qn,\ \varsigma=\rho\mu=e(\mu_p p+\mu_n n),\ \varrho=\varsigma^{-1} 
$$

\subsection{Der abrupte $pn$-Übergang}
\definecolor{gray}{rgb}{0.9,0.9,0.9}

\begin{figure}[htb]
    \centering
    \begin{tikzpicture}
        % axis
        \draw[->] (-5,0) -- (5,0) node[anchor=north] {$x$};
        \draw[->] (0,-3) -- (0,3) node[anchor=east] {$E,\rho$};

        % dotted lines for dimensioning
        \draw[dotted] (3,-3) -- (3,3);
        \draw[dotted] (-1,-3) -- (-1,3);
        \draw[dotted] (-5,3) -- (-5,-3);
        \draw[dotted] (5,3) -- (5,-3);

        % dimensioning
        \draw[<->] (-5,1) -- (-1,1);
        \draw (-3,1) node[anchor=south] {$w_n$}; % label for w_n
        \draw[<->] (3,-1) -- (5,-1);
        \draw (4,-1) node[anchor=north] {$w_p$}; % label for w_p
        
        % labels
        \draw
        (-5,0) node[anchor=south] {$-W_p$}
        (-1,0) node[anchor=south] {$-x_p$}
        (0,0) node[anchor=north west] {0}
        (3,0) node[anchor=north] {$x_n$}
        (5,0) node[anchor=south] {$W_n$};

        % charge
        \path [fill=gray] (0,0) -- (0,1) -- (3,1) -- (3,0) -- (0,0);
        \draw (2,0.5) node[anchor=east] {$Q^+$};
        \path [fill=gray] (-1,0) -- (0,0) -- (0,-3) -- (-1,-3) -- (-1,0);
        \draw (-0.5,-2.5) node {$Q^-$};
        \draw[thick] (-5,0) -- (-1,0) -- (-1,-3) -- (0,-3) -- (0,1) -- (3,1) -- (3,0) -- (5,0);
        % electric field
        %\draw[thick] (-5,0) -- (-1,0) -- (-1,-3) -- (0,-3) -- (0,1) -- (3,1) -- (3,0) -- (5,0);
        \draw (1.5,1.5) node {$\rho(x)$}; %label
        
        % electric field
        \draw[thick, dashed] (-5,0) -- (-1,0) -- (0,-2) -- (3,0) -- (5,0);
        \draw (1.5,-1.5) node {$E(x)$}; %label

        % potential
        \draw[thick, dotted] (-5,0) -- (-1,0) to [out=0, in=-100] (0,1.5)  to [out=80, in=180] (3,3) -- (5,3);
        \draw (3,3) node[anchor=north] {$\psi(x)$};
    \end{tikzpicture}
\caption{Abrupter $pn$-Übergang}
\end{figure}

Weite der Raumladungszone:
$$
w_0=\sqrt{\frac{2\epsilon}{e}\left(\frac 1 {N_A} + \frac 1 {N_B}\right)U_D},\ w=w_0\sqrt{1-\frac U {U_D}}.
$$
Ausdehnung der Raumladungszone ins $p$- und $n$-Gebiet
$$
x_p=\frac {N_D}{N_A+N_d}w,\ x_n=\frac {N_A}{N_A+N_D}.
$$
Maximale elektische Feldstärke
$$
E_{\max,0}=\sqrt{\frac {2e}{\epsilon}\frac {N_A N_D}{N_A+N_D}U_D}=\frac{e}{\epsilon}N_D x_n=\frac{e}{\epsilon} N_A x_p,\ E_{\max}=E_{\max,0}\sqrt{1-\frac U {U_D}}.
$$
Spannungsabfall über der Raumladungszone
$$
U_D = E_{\max}\frac w 2,\ U_{D,p}=E_{\max}\frac{x_p} 2,\ U_{D,n}=E_{\max}\frac{x_n} 2.
$$
Diffusionslängen der Minoritätsträger im Bahngebiet
$$
L_p=\sqrt{D_p\tau_p},\ L_n=\sqrt{D_n\tau_n}.
$$
Diffusionskonstanten:
$$
D_p=\mu_p U_T,\ D_n = \mu_n U_T.
$$
Einteilung der Bahngebiete
\begin{description}
        \item[Kurzes Bahngebiet:] Es gilt $w_n=W_n-x_n\ll L_p$ bzw. $w_p=W_p-x_p\ll L_n$.
    Sättigungsstromdichte:
    $$
    J_S=e\left(\frac{D_p p_{n0}}{w_n}+\frac{D_n n_{p0}}{w_p}\right)=en_i^2\left(\frac{D_p}{N_D w_n}+\frac{D_n}{N_A w_p}\right).
    $$

        \item[Unendlich langes Bahngebiet:] Es gilt $w_n \gg L_p$ bzw. $w_p \gg L_n$.
    Sättigungsstromdichte:
    $$
    J_S=e\left(\frac {D_p p_{n0}}{L_p}+\frac{D_n n_{p0}}{L_n}\right).
    $$
    \item[Endlich langes Bahngebiet:] ??
\end{description}
Sättigungsstrom:
$$
I_S=J_S A.
$$
Spezifische Leitwert (durch Majoritäten):
$$
\varsigma_p=e N_A \mu_p(N_A)\, \varsigma_n= e N_D \mu_n(N_D).
$$
Spezifischer Widerstand (durch Majoritäten):
$$
\varrho_p=\varsigma_p^{-1},\ \varrho_n=\varsigma_n^{-1}.
$$
Bahnwiderstand (Reihenschaltung):
$$
R_S=R_n+R_p=\frac{1}{A}(\varrho_p w_p+\varrho_n w_n).
$$

\section{Aufgaben}

\def\theexercisecontext{D}

\begin{exercise}
Gegeben ist ein linearer $pn$-Übergang für den die Störstellenverteilung $N_D-N_A=ax$ innerhalb der Raumladungszone ($-x_p\leq x\leq x_p$) angenommen wird. Der Parameter $a$ bezeichnet hier den Störstellengradienten.
Für diesen $pn$-Übergang sollen elektrische Feldstärke $E$ und Potential $\psi$ in Abhängigkeit des Ortes $x$ bestimmt werden.

Folgende Annahmen werden getroffen:
\begin{enumerate}
        \item Die Raumladungszone ist frei von beweglichen Ladungsträgern.
        \item Die Bahngebiete ($-W_p\leq x\leq -x_p$ und $x_n\leq x\leq W_n$) sind feldfrei und somit neutral.
        \item Über dem $pn$-Übergang liege die Spannung $U$ an.
\end{enumerate}

Folgende Zahlenwerte sind gegeben: $a=5\cdot 10^{21}\mathrm{cm}$, $U_T=26\mathrm{mV}$, $U_D=0,7\mathrm{V}$, $\epsilon_{r,\mathrm{Si}}=11,9$.
\end{exercise}

\begin{solution}
Die benötigte Gleichung (\person{Gauß}'sches Gesetz) lautet
$$
\div{D}=\rho,
$$
oder für das elektrische Feld formuliert $\pderive{E}{x}=\frac{\rho}{\epsilon}$.

Wir müssen also zunächst die Ladungsdichte $\rho(x)$ bestimmen, welche gegeben ist durch $\rho(x)=e(N_D-N_A)=eax$ (innerhalb der Raumladungszone). Diese ist also eine lineare Funktion. Damit ergibt sich die Feldstärke zu $E(x)=E(-x_p)+\frac{1}{\epsilon}\int_{-x_p}^x{\rho(x)}\diff{x}=\frac{ea}{2\epsilon}(x-x_p)(x+x_p)$ (da $E(-x_p)=0$ angenommen wurde). Man sieht damit, dass hier implizit $x_p=x_n$ angenommen wird, denn dies sind die beiden sich ergebenden Nullstellen der Feldstärkefunktion.

Man erhält ein entsprechendes Diagramm
\begin{figure}[htb]
    \centering
    \begin{tikzpicture}
        % axis
        \draw[->] (-3,0) -- (3,0) node[anchor=north] {$x$};
        \draw[->] (0,-3) -- (0,2) node[anchor=east] {$E,\rho$};
        
        % labels
        \draw
        (-2,0) node[anchor=south] {$-x_p$}
        (0,0) node[anchor=north west] {0}
        (2,0) node[anchor=south] {$x_n$};

        % boundaries (vertical)
        \draw[dotted] (2,-3) -- (2,2);
        \draw[dotted] (-2,-3) -- (-2,2);
        
        % Us
        \draw[thick] (-3,0) -- (-2,0) -- (-2,-2) -- (2,2) -- (2,0) -- (3,0);
        \draw (-0.5,0.5) node {$\rho(x)$}; %label
        
        % Psis
        \draw[thick] (-3,0) -- (-2,0) parabola bend (0,-2) (2,0) -- (3,0);
        \draw (-0.5,-2.5) node {$E(x)$}; %label
    \end{tikzpicture}
\end{figure}

Es ergibt sich $E_{\min}=E\left(\frac{-x_p+x_n}{2}\right)=-\frac{eax_p^2}{2\epsilon}$ (da $x_n=x_p$), also
$$\abs{E}_{\max}=\frac{eax_p^2}{2\epsilon}.$$

Das Potential ergibt sich dann mit der \person{Poisson}-Gleichung $\Delta{\psi}=-\frac{\rho}{\epsilon}$.
Es ergibt sich also zu $\psi(x)=\psi(-x_p)-\int_{-x_p}^x{E(x)}\diff{x}=\psi(-x_p)-\frac{ea}{6\epsilon}(x^3-3x_p^2x-2x_p^3)$

Um den Potentialwert an den Stellen $\pm \frac{w}{2}$ zu errechnen benötigt man zunächst ein Gleichung
für die Weite der Raumladungszone (wie z.B. beim abrupten Übergang $w=w_0\sqrt{1-\frac{U}{U_D}}$).

Wir leiten diese noch kurz her. Zunächst ist die gesamte über der Raumladungszone abfallende Spannung gleich $U_D-U$ (wobei $U$ die eingeprägte Spannung ist). Dies ergibt dann
$$
\psi(x_n)-\psi(-x_p)=\frac{2eax_p^3}{3\epsilon}=U_D-U,
$$
was zu
$$w_0=2x_p=2\sqrt[3]{\frac{3\epsilon U_D}{2ea}}$$
und
$$w=w_0\sqrt[3]{1-\frac{U}{U_D}}$$
führt.

Damit erhält man die Zahlenwerte $w_0=222,78\mathrm{nm}$, $w(U=-5\mathrm{V})=448,21\mathrm{nm}$, $w(U=0,4\mathrm{V})=167,96\mathrm{nm}$.
Analog ergibt sich für die Feldstärke
$$\abs{E}_{\max}=\abs{E}_{\max,0}{\left(1-\frac{U}{U_D}\right)}^{\frac{2}{3}}.$$
Man erhält $\abs{E}_{\max} (U=0)=\float{-190,7}{3}\frac{\V}{\cm}$, $\abs{E}_{\max}(U=0)=\float{-190,7}{3}\frac{\V}{\cm}$.
\end{solution}

% D2
\begin{exercise}
In dieser Aufgabe soll die Sperrschichtkapazität eines linearen $pn$-Übergangs unter Verwendung von Aufgabe D1 hergeleited und grafisch dargestellt werden. Nehmen Sie dazu die Raumladungszone als einen Plattenkondensator an, mit Plattenabstand $w$ und Ladung $Q$ der Ladung auf der $n$-Seite dieser entsprechend.
Wieder gelte $N_D-N_A=ax$, $x_p=x_n$.

Zahlenwerte sind: $n_i=\float{9,65}{9}\cm^{-3}$, $a=\float{5}{21}\cm^{-4}$, $U_T=26\mV$, $U_D=0,7\V$, $\epsilon_{r,\Si}=11,9$ und $A=100\mum$.
\end{exercise}

\begin{solution}
Es gilt grundsätzlich $C=Q/U$ bei einem Plattenkondensator. 
Die Ladung auf dem Kondensator ergibt sich über $Q=A\int_{0}^{x_n}{\rho(x)}\diff{x}$.
Also
$$
Q=Aea\frac{x_p^2}{2}.
$$
Mit der Lösung von Aufgabe D1 liefert dies
$$
C_j = \frac{Aea}{2}{\left(\frac{3\epsilon}{2ea}\right)}^{\frac{2}{3}}\frac{1}{\sqrt[3]{U_D-U}}.
$$
Man sieht also auch, dass
$$
C_j = C_{j0}\frac{1}{\sqrt[3]{1-\frac{U}{U_D}}},
$$
wobei sich $C_{j0}$ zu
$$
C_{j0}=\frac{A}{2}\sqrt[3]{\frac{9\epsilon^2ea}{4U_D}}
$$
Es ergibt sich dann $C_{j0}=47,3\fF$, $C_j(U=-5\V)=23,5fF$ und $C_j(U=0,4)=62,7\fF$.
\end{solution}

% D5
\begin{exercise}
Gegeben ist ein $pn$-Übergang mit der Fläche $A=1000\mum^2$, den Dotierungen $N_A=10^{16}\cm^{-1}$, $N_D=10^{18}\cm^{-1}$, den Lebensdauern $\tau_n(10^{16}\cm^{-3})= 40\mus$, $\tau_n(10^{18}\cm^{-3})=5\mus$ der Elektronen und $\tau_p(10^{16}\cm^{-3})= 18\mus$, $\tau_p(10^{18}\cm^{-3})=0,9\mus$ der Löcher, den metallurgischen Weiten $W_p=10\mum$ und $W_n=5\mum$ und der Temperaturspannung $U_T=26\mV$.
Weiterhin ist folgendes Diagramm gegeben, dass die Beweglichkeit der Ladungsträger in Abhängigkeit der Dotierung (egal ob $p$- oder $n$-dotiert) angibt.

\begin{figure}[htb]
    \centering
    \begin{tikzpicture}
        % x-Achse (Dotierung)
        \draw[->] (0,0) -- (7,0) node[anchor=north] {$N/\cm^{-3}$};
        \draw (0,0) -- (0,-0.1) node[anchor=north] {$10^{14}$};
        \draw (1,0) -- (1,-0.1) node[anchor=north] {$10^{15}$};
        \draw (2,0) -- (2,-0.1) node[anchor=north] {$10^{16}$};
        \draw (3,0) -- (3,-0.1) node[anchor=north] {$10^{17}$};
        \draw (4,0) -- (4,-0.1) node[anchor=north] {$10^{18}$};
        \draw (5,0) -- (5,-0.1) node[anchor=north] {$10^{19}$};

        % y-Achse (Beweglichkeit)
        \draw[->] (0,0) -- (0,3.5) node[anchor=east] {$\mu_n,\mu_p/\frac{\cm^2}{\V\s}$};
        \draw (0,0) -- (-0.1,0) node[anchor=east] {$0$};
        \draw (0,1) -- (-0.1,1) node[anchor=east] {$500$};
        \draw (0,2) -- (-0.1,2) node[anchor=east] {$1000$};
        \draw (0,3) -- (-0.1,3) node[anchor=east] {$1500$};
        
        % Elektronenbeweglichkeit
        \draw[thick] (0,3) to [out=0, in=140] (2,2.5) to [out=-40, in=160] (4,0.5)  to [out=-20, in=179] (7,0.2);

        % Löcherbeweglichkeit
        \draw[thick] (0,1) to [out=0, in=170] (2,0.8) to [out=-10, in=175] (4,0.2)  to [out=-5, in=179] (7,0.1);
        \draw (2,3) node[anchor=west] {Elektronen}; % label
        \draw (1,1) node[anchor=south] {Löcher}; % label

        % wichtige Punkte
        \draw [fill] (2,2.5) circle [radius=.05] node[anchor=south] {\small $1250$};
        \draw [fill] (2,0.8) circle [radius=.05] node[anchor=south] {\small $400$};
        \draw [fill] (4,0.5) circle [radius=.05] node[anchor=south] {\small $250$};
        \draw [fill] (4,0.2) circle [radius=.05] node[anchor=east] {\small $100$};
    \end{tikzpicture}
    \caption{Ladungsträgerbeweglichkeit in Abhängigkeit der Dotierung}
\end{figure}

\begin{tasks}
        \item Entscheiden Sie mithilfe eine einfachen Abschätzung, ob es sich um ein unendlich langes oder kurzes Bahngebiet handelt. Gehen Sie von einem abrupten $pn$-Übergang aus.
        \item Berechnen Sie den Sättigungsstrom $I_S$ der Diode (Formel, Zahlenwert).
        \item Berechnen Sie den gesamten Bahnwiderstand $R_S$ für den $pn$-Übergang (Formel, Zahlenwert).
\end{tasks}
\end{exercise}

\begin{solution}
    \begin{tasks}
            \item a.
            \item b.
        \item Der Bahndwiderstand ergibt sich nach der Formel aus der Formelsammlung zu:
$$
R_s=R_n+R_p=\frac{1}{A}(\varrho_p w_p +\varrho_n w_n),
$$
mit
$$
\varrho_p=\varsigma_p^{-1}={(e N_A\mu_p(N_A))}^{-1},\ \varrho_n=\varsigma_n^{-1}={(e N_D\mu_n(N_D))}^{-1}.
$$
Zahlenmäßig erhält man $R_p=6,25\Omega$, $R_n=23,5\Omega$, $R_S=29,8\Omega$.

\emph{Wichtig:} Der Bahnwiderstand entsteht nur durch den Driftstrom der Majoritäten, da die Minoritätsträger nahezu keinen Drift haben und nur diffundieren.

\end{tasks}
\end{solution}

% D6
\begin{exercise}
In Aufgabe G1 der ersten Übung wurde die Temperaturabhängigkeit der Eigenleitungsdichte diskutiert und hergeleitet.
Da die Eigenleitungsdichte $n_i$ auch Bestandteil des Sättigungsstroms $I_S$ ist, miss $I_S$ auch temperaturabhängig sein. Es werde der Sättigungsstrom eines kurzen abrupten $pn$-Übergangs betrachtet. Für diesen gilt:
$$
I_S = eAn_i^2\frac{\mu_p U_T}{N_D w_n}.
$$

\emph{Hinweise:} Die Temperaturabhängigkeit der Weite $w_n$ sei vernachlässigbar.
Für die Temperaturabhängigkeit der Ladungsträgerbeweglichkeit gilt: $\mu \propto T^{-a_\mu}$.

\begin{tasks}
        \item Berechnen Sie zahlenmäßig den Sättigungsstrom $I_S$.
        \item Leiten Sie den temperaturabhängigen Sättigungsstrom $I_S(T)$ her. Normieren Sie $I_S$ auf $I_S(T_0)$ ($T_0$ sei die Bezugstemperatur).
\end{tasks}
\end{exercise}

\begin{solution}
    \begin{tasks}
            \item Direktes Einsetzen ergibt: $I_S=0,787\fA$.
            \item Wir haben $I_S(T)=\frac{A}{N_D w_n} n_i^2(T) kT\mu_p(T)$. Daraus ergibt sich mit der Formel
        $$
        n_i(T)=n_i(T_0){\left(\frac T {T_0}\right)}^{\frac 3 2}\exp\left(\frac{E_{g0}}{2kT_0}\left(1-\frac{T_0}{T}\right)\right)
        $$
        aus Aufgabe G1 (bei der wir eine lineare Näherung $E_g=E_{g0}-a_g T$ mit extrapolierter Referenzenergie $E_{g0}$ angenommen haben), dass
        $$
        I_S(T)=I(T_0){\left(\frac{T}{T_0}\right)}^{4-a_\mu}\exp\left(\frac{U_{g0}}{U_{T0}}\left(1-\frac{T_0}{T}\right)\right).
        $$
    \end{tasks}
\end{solution}

% D7
\begin{exercise}
Gegeben ist ein abrupter $pn$-Übergang im thermischen Gleichgewicht ($N_A=10^{17}\cm^{-1}$, $U_T=26\mV$, $U_D=0,934\V$).
\begin{tasks}
        \item Berechnen Sie die Dotierung $N_D$ (Formel, Zahlenwert).
        \item Berechnen sie die maximale Feldstärke $E_{\max}$ und die Weite $w$ der Raumladungszone (Formel, Zahlenwert).
    \item Berechnen Sie die Ausdehnungen $x_n$, $x_p$ der Raumladungszone im $n$- und $p$-Gebiet (Formel, Zahlenwert). 
\end{tasks}
\end{exercise}

\begin{solution}
    \begin{tasks}
            \item Nach der Formel aus Aufgabe D4 gilt
        $$
        U_D=U_T\log\left(\frac{N_A N_D}{n_i^2}\right)
        $$
        und wir erhalten
        $$
        N_D=\frac{n_i^2}{N_A}\exp\left(\frac{U_D}{U_T}\right).
        $$
        Zahlenmäßig erhält man $N_D=\float{0,9}{19}\cm^{-3}$.
            \item Die maximale Feldstärke $E_{\max,0}$ und die Weite $w_0$ der Raumladungszone ergeben sich nach den Formeln aus der Formelsammlung für den abrupten $pn$-Übergang zu
        $$
        w_0=\sqrt{\frac{2\epsilon}{e}\left(\frac 1 {N_A} + \frac 1 {N_B}\right)U_D},\ w=w_0\sqrt{1-\frac U {U_D}}.
        $$
        und
        $$
        \abs{E}_{\max,0}=\frac{2U_D}{w}=\sqrt{\frac {2e}{\epsilon}\left(\frac {N_A N_D}{N_A+N_D}\right)U_D}
        $$
        und zahlenmäßig ergibt sich $\abs{E}_{\max,0}=169\kV\cm^{-1}$, $w_0=110,5\nm$.
            \item Die Ausdehnungen $x_{p0}$ und $x_{n0}$ verhalten sich nach der Formelsammlung reziprok zu den den Dotierungen $N_A$ und $N_D$ und addieren sich zur Weite der Raumladungszone. Es gilt also
        $$
        x_{p0} =\frac{N_D}{N_A+N_D}w_0,\ x_{n0}=\frac{N_A}{N_A+N_D}w_0.
        $$
        Zahlenmäßig ergibt sich $x_{p0}=109,4\nm$ und $x_{n0}=1,1\nm$.
    \end{tasks}
\end{solution}

% D8
\begin{exercise}
Eine $\Si$-Diode mit abruptem $pn$-Übergang wird auf einem mit Bor vordotierten Substrat hergestellt ($N_{\textrm{Bor}}=N_A=10^{16}\cm^{-3}$, $U_T=26\mV$, $\abs{E}_{\max,0}=\float{7,7}{4}\V\cm^{-1}$). An der Diode liegt keine Spannung an.
\begin{tasks}
        \item Wie weit dehnt sich die Raumladungszone in das Substrat aus (Formel, Zahlenwert)? 
        \item Wie hoch muss die Dotierstoffkonzentration in der über dem Substrat liegenden Schicht gewählt werden, damit die Weite der Raumladungszone $w=750\nm$ beträgt (Formel, Zahlenwert)? Mit welchem Dotantentyp ist zu dotieren?
    \item Wie groß ist die Diffusionsspannung $U_D$ (Formel, Zahlenwert)? 
\end{tasks}
\end{exercise}

\begin{solution}
    \begin{tasks}
            \item Wir wenden die Formel
        $$
        \abs{E}_{\max}=\frac e {\epsilon} N_A x_p=\frac e {\epsilon} N_D x_n
        $$
        aus der Formelsammlung an. Dies ergibt
        $$
        x_p=\frac \epsilon e \abs{E}_{\max}
        $$
        und zahlenmäßig $x_p=498\nm$.
            \item Es gilt $w=x_p+x_n$. Damit muss $x_n=w-x_p=252\nm$ gelten.
        Wieder wenden wir die Formel $\abs{E}_{\max}=\frac e \epsilon N_D x_n$ an und erhalten
        $$
        N_D= \frac {\epsilon\abs{E}_{\max}}{ex_n},
        $$
        was zu $N_D=\float{2}{16}\cm^{-1}$ führt. Damit handelt es sich nicht um einen abrupten $pn$-Übergang, da $N_A\not\ll N_D$.
            \item Die Diffusionsspannung ergibt sich dann aus der Formel
        $$
        U_D=U_T\log\left(\frac{N_A N_D}{n_i^2}\right)
        $$
        also $U_D=0,712\V$. Die Formel $2U_D=\abs{E}_{\max}w$ führt hier zum falschen Ergebnis (da kein abrupter Übergang ?).
    \end{tasks}
\end{solution}

% D9
\begin{exercise}
Aus Messungen der Sperrschichtkapazität einer abrupten $\Si$-$pn$-Diode resultiert nachfolgende Tabelle. Es ist bekannt, dass das $n$-Gebiet höher als das $p$-Gebiet dotiert ist, also $N_D\gg N_A$. Zahlenwerte sind: $U_T=26\mV$, $A=10\mum^2$.
\begin{table}[h]
    \centering
    \begin{tabular}{cc}
        $C_j/\fF$ & $U/\V$ \\
        \midrule
        1,6312    & -2,5   \\
        1,8096    & -1,875 \\
        2,0633    & -1,25  \\
        2,4661    & -0,625 \\
        3,2623    & 0
    \end{tabular}
\end{table}
\begin{tasks}
        \item Berechnen Sie näherungsweise unter Zuhilfenahme der Tabelle die Diffusionsspannung $U_D$ (Formel, Zahlenwert).
        \item Bestimmen Sie die Dotierungen des $n$- und $p$-Gebietes (Formel, Zahlenwert).
\end{tasks}
\end{exercise}

\begin{solution}

\end{solution}









\end{document}














