\documentclass{article}

% for use of german language in a document
\usepackage[utf8]{inputenc} % this is needed for umlauts
\usepackage[ngerman]{babel} % this is needed for umlauts
\usepackage[T1]{fontenc}    % this is needed for correct output of umlauts in pdf

\usepackage[de]{mathenv}
%My personal maths package
%\bibliography{Bibliography.bib}

%%%%%%%%%%%%%%%%%%%%%%%%%%%%%%%%%%%%%%%%%%%%%%%%%%%%%%%%%%%%%%%%%%%%%%%%%%%%%%
%%%%% MATH PACKAGES %%%%%%%%%%%%%%%%%%%%%%%%%%%%%%%%%%%%%%%%%%%%%%%%%%%%%%%%%%
%%%%%%%%%%%%%%%%%%%%%%%%%%%%%%%%%%%%%%%%%%%%%%%%%%%%%%%%%%%%%%%%%%%%%%%%%%%%%%

% very good package
%\usepackage{mathtools}

%%% font stuff
\usepackage[T1]{fontenc}        % for capitals in section /paragraph etc.
\usepackage[utf8]{inputenc}     % use utf8 symbols in code

\usepackage{amssymb,amsmath,amsfonts} % amsthm not needed -- use my own envs
%\usepackage{mathtools}
%\mathtoolsset{showonlyrefs}
% further alternative math packages: unicode-math, abx-math
\usepackage{bm}
\usepackage{mathrsfs} % used for: fraktal math letters
\usepackage[bigsqcap]{stmaryrd} % used for: big square cap symbol
\usepackage{xargs}

% standard packages
%\usepackage{color} % for color

%%
%%index

\newcommand*{\keyword}[2][\empty]{\emph{#2}\ifx#1\empty\index{#2}\else\index{#1}\fi}
\newcommand*{\person}[1]{\textsc{#1}}

%%%%%%%%%%%%%%%%%%%%%%%%%%%%%%%%%%%%%%%%%%%%%%%%%%%%%%%%%%%%%%%%%%%%%%%%%%%%%%%%%%%%%%%%%%%%%%%%%%%%%%%%%%%%
%%%%% MATH ALPHABETS & SYMBOLS %%%%%%%%%%%%%%%%%%%%%%%%%%%%%%%%%%%%%%%%%%%%%%%%%%%%%%%%%%%%%%%%%%%%%%%%%%%%%
%%%%%%%%%%%%%%%%%%%%%%%%%%%%%%%%%%%%%%%%%%%%%%%%%%%%%%%%%%%%%%%%%%%%%%%%%%%%%%%%%%%%%%%%%%%%%%%%%%%%%%%%%%%%

% w: http://milde.users.sourceforge.net/LUCR/Math/math-font-selection.xhtml

% ===== Set quick commands for math letters ================================================================
% calagraphic letters (only upper case available; standard)
\newcommand{\cA}{\mathcal{A}}
\newcommand{\cB}{\mathcal{B}}
\newcommand{\cC}{\mathcal{C}}
\newcommand{\cD}{\mathcal{D}}
\newcommand{\cE}{\mathcal{E}}
\newcommand{\cF}{\mathcal{F}}
\newcommand{\cG}{\mathcal{G}}
\newcommand{\cH}{\mathcal{H}}
\newcommand{\cI}{\mathcal{I}}
\newcommand{\cJ}{\mathcal{J}}
\newcommand{\cK}{\mathcal{K}}
\newcommand{\cL}{\mathcal{L}}
\newcommand{\cM}{\mathcal{M}}
\newcommand{\cN}{\mathcal{N}}
\newcommand{\cO}{\mathcal{O}}
\newcommand{\cP}{\mathcal{P}}
\newcommand{\cQ}{\mathcal{Q}}
\newcommand{\cR}{\mathcal{R}}
\newcommand{\cS}{\mathcal{S}}
\newcommand{\cT}{\mathcal{T}}
\newcommand{\cU}{\mathcal{U}}
\newcommand{\cV}{\mathcal{V}}
\newcommand{\cW}{\mathcal{W}}
\newcommand{\cX}{\mathcal{X}}
\newcommand{\cY}{\mathcal{Y}}
\newcommand{\cZ}{\mathcal{Z}}

% bold math letters (standard)
\newcommand{\bfA}{\mathbf{A}}
\newcommand{\bfB}{\mathbf{B}}
\newcommand{\bfC}{\mathbf{C}}
\newcommand{\bfD}{\mathbf{D}}
\newcommand{\bfE}{\mathbf{E}}
\newcommand{\bfF}{\mathbf{F}}
\newcommand{\bfG}{\mathbf{G}}
\newcommand{\bfH}{\mathbf{H}}
\newcommand{\bfI}{\mathbf{I}}
\newcommand{\bfJ}{\mathbf{J}}
\newcommand{\bfK}{\mathbf{K}}
\newcommand{\bfL}{\mathbf{L}}
\newcommand{\bfM}{\mathbf{M}}
\newcommand{\bfN}{\mathbf{N}}
\newcommand{\bfO}{\mathbf{O}}
\newcommand{\bfP}{\mathbf{P}}
\newcommand{\bfQ}{\mathbf{Q}}
\newcommand{\bfR}{\mathbf{R}}
\newcommand{\bfS}{\mathbf{S}}
\newcommand{\bfT}{\mathbf{T}}
\newcommand{\bfU}{\mathbf{U}}
\newcommand{\bfV}{\mathbf{V}}
\newcommand{\bfW}{\mathbf{W}}
\newcommand{\bfX}{\mathbf{X}}
\newcommand{\bfY}{\mathbf{Y}}
\newcommand{\bfZ}{\mathbf{Z}}
\newcommand{\bfa}{\mathbf{a}}
\newcommand{\bfb}{\mathbf{b}}
\newcommand{\bfc}{\mathbf{c}}
\newcommand{\bfd}{\mathbf{d}}
\newcommand{\bfe}{\mathbf{e}}
\newcommand{\bff}{\mathbf{f}}
\newcommand{\bfg}{\mathbf{g}}
\newcommand{\bfh}{\mathbf{h}}
\newcommand{\bfi}{\mathbf{i}}
\newcommand{\bfj}{\mathbf{j}}
\newcommand{\bfk}{\mathbf{k}}
\newcommand{\bfl}{\mathbf{l}}
\newcommand{\bfm}{\mathbf{m}}
\newcommand{\bfn}{\mathbf{n}}
\newcommand{\bfo}{\mathbf{o}}
\newcommand{\bfp}{\mathbf{p}}
\newcommand{\bfq}{\mathbf{q}}
\newcommand{\bfr}{\mathbf{r}}
\newcommand{\bfs}{\mathbf{s}}
\newcommand{\bft}{\mathbf{t}}
\newcommand{\bfu}{\mathbf{u}}
\newcommand{\bfv}{\mathbf{v}}
\newcommand{\bfw}{\mathbf{w}}
\newcommand{\bfx}{\mathbf{x}}
\newcommand{\bfy}{\mathbf{y}}
\newcommand{\bfz}{\mathbf{z}}

% fractal math letters (standard)
\newcommand{\fkA}{\mathfrak{A}}
\newcommand{\fkB}{\mathfrak{B}}
\newcommand{\fkC}{\mathfrak{C}}
\newcommand{\fkD}{\mathfrak{D}}
\newcommand{\fkE}{\mathfrak{E}}
\newcommand{\fkF}{\mathfrak{F}}
\newcommand{\fkG}{\mathfrak{G}}
\newcommand{\fkH}{\mathfrak{H}}
\newcommand{\fkI}{\mathfrak{I}}
\newcommand{\fkJ}{\mathfrak{J}}
\newcommand{\fkK}{\mathfrak{K}}
\newcommand{\fkL}{\mathfrak{L}}
\newcommand{\fkM}{\mathfrak{M}}
\newcommand{\fkN}{\mathfrak{N}}
\newcommand{\fkO}{\mathfrak{O}}
\newcommand{\fkP}{\mathfrak{P}}
\newcommand{\fkQ}{\mathfrak{Q}}
\newcommand{\fkR}{\mathfrak{R}}
\newcommand{\fkS}{\mathfrak{S}}
\newcommand{\fkT}{\mathfrak{T}}
\newcommand{\fkU}{\mathfrak{U}}
\newcommand{\fkV}{\mathfrak{V}}
\newcommand{\fkW}{\mathfrak{W}}
\newcommand{\fkX}{\mathfrak{X}}
\newcommand{\fkY}{\mathfrak{Y}}
\newcommand{\fkZ}{\mathfrak{Z}}
\newcommand{\fka}{\mathfrak{a}}
\newcommand{\fkb}{\mathfrak{b}}
\newcommand{\fkc}{\mathfrak{c}}
\newcommand{\fkd}{\mathfrak{d}}
\newcommand{\fke}{\mathfrak{e}}
\newcommand{\fkf}{\mathfrak{f}}
\newcommand{\fkg}{\mathfrak{g}}
\newcommand{\fkh}{\mathfrak{h}}
\newcommand{\fki}{\mathfrak{i}}
\newcommand{\fkj}{\mathfrak{j}}
\newcommand{\fkk}{\mathfrak{k}}
\newcommand{\fkl}{\mathfrak{l}}
\newcommand{\fkm}{\mathfrak{m}}
\newcommand{\fkn}{\mathfrak{n}}
\newcommand{\fko}{\mathfrak{o}}
\newcommand{\fkp}{\mathfrak{p}}
\newcommand{\fkq}{\mathfrak{q}}
\newcommand{\fkr}{\mathfrak{r}}
\newcommand{\fks}{\mathfrak{s}}
\newcommand{\fkt}{\mathfrak{t}}
\newcommand{\fku}{\mathfrak{u}}
\newcommand{\fkv}{\mathfrak{v}}
\newcommand{\fkw}{\mathfrak{w}}
\newcommand{\fkx}{\mathfrak{x}}
\newcommand{\fky}{\mathfrak{y}}
\newcommand{\fkz}{\mathfrak{z}}

% script math symbols (only uppercase; package: mathrsfs)
\newcommand{\sA}{\mathscr{A}}
\newcommand{\sB}{\mathscr{B}}
\newcommand{\sC}{\mathscr{C}}
\newcommand{\sD}{\mathscr{D}}
\newcommand{\sE}{\mathscr{E}}
\newcommand{\sF}{\mathscr{F}}
\newcommand{\sG}{\mathscr{G}}
\newcommand{\sH}{\mathscr{H}}
\newcommand{\sI}{\mathscr{I}}
\newcommand{\sJ}{\mathscr{J}}
\newcommand{\sK}{\mathscr{K}}
\newcommand{\sL}{\mathscr{L}}
\newcommand{\sM}{\mathscr{M}}
\newcommand{\sN}{\mathscr{N}}
\newcommand{\sO}{\mathscr{O}}
\newcommand{\sP}{\mathscr{P}}
\newcommand{\sQ}{\mathscr{Q}}
\newcommand{\sR}{\mathscr{R}}
\newcommand{\sS}{\mathscr{S}}
\newcommand{\sT}{\mathscr{T}}
\newcommand{\sU}{\mathscr{U}}
\newcommand{\sV}{\mathscr{V}}
\newcommand{\sW}{\mathscr{W}}
\newcommand{\sX}{\mathscr{X}}
\newcommand{\sY}{\mathscr{Y}}
\newcommand{\sZ}{\mathscr{Z}}

%%%%%%%%%%%%%%%%%%%%%%%%%%%%%%%%%%%%%%%%%%%%%%%%%%%%%%%%%%%%%%%%%%%%%%%%%%%%%%%%%%%%%%
%%%% BOLD MATH IN BOLD TEXT ENVIRONMENT %%%%%%%%%%%%%%%%%%%%%%%%%%%%%%%%%%%%%%%%%%%%%%
%%%%%%%%%%%%%%%%%%%%%%%%%%%%%%%%%%%%%%%%%%%%%%%%%%%%%%%%%%%%%%%%%%%%%%%%%%%%%%%%%%%%%%

% for bold math in bold text (e.g. sections)
\makeatletter
\g@addto@macro\bfseries{\boldmath}
\makeatother

\def\brackets#1{\ifx#1\empty\else\left(#1\right)\fi}

%%%%%%%%%%%%%%%%%%%%%%%%%%%%%%%%%%%%%%%%%%%%%%%%%%%%%%%%%%%%%%%%%%%%%%%%%%%%%%%%%%%%%%
%%%%% CATEGORY THEORY %%%%%%%%%%%%%%%%%%%%%%%%%%%%%%%%%%%%%%%%%%%%%%%%%%%%%%%%%%%%%%%%
%%%%%%%%%%%%%%%%%%%%%%%%%%%%%%%%%%%%%%%%%%%%%%%%%%%%%%%%%%%%%%%%%%%%%%%%%%%%%%%%%%%%%%

% ===== Category theory concepts =====================================================

\newcommand{\Ob}{\mathop\mathrm{Ob}}
\newcommand{\Mor}{\mathop\mathrm{Mor}}

\newcommand{\ccoprod}{\bigsqcup}
\newcommand{\cprod}{\bigsqcap}
\newcommand{\cincl}{\mathop\mathrm{incl}}
\newcommand{\cproj}{\mathop\mathrm{pr}}


% ===== Define standard categories ===================================================

% Define sets
\newcommand{\Set}{\mathbf{Set}}
% Define set-builder operator (equivalent to gen for algebras)
\newcommand{\set}[1]{\left\{#1\right\}}
% define interval operator: o - open, c - closed
\newcommand{\intervalcc}[2]{\left[#1,#2\right]}
\newcommand{\intervalco}[2]{\left[#1,#2\right)}
\newcommand{\intervaloc}[2]{\left(#1,#2\right]}
\newcommand{\intervaloo}[2]{\left(#1,#2\right)}

\newcommand{\inter}{\mathop\mathrm{int}}
\newcommand{\face}{\mathop\mathrm{F}}
\newcommand{\Pol}{\mathop\mathrm{Pol}}
\newcommand{\Inv}{\mathop\mathrm{Inv}}
\def\struct#1{\gen{#1}}

\let\originaltimes\times%
\renewcommand{\times}{\mathbin{\sqcap}}
\newcommand\settimes{\originaltimes}
\newcommand{\setleq}{\subseteq}
\newcommand{\setgeq}{\supseteq}

\newcommand{\pderive}[2]{\frac{\partial{#1}}{\partial{#2}}}
\renewcommand{\div}{\mathop\mathrm{div}}

%% Diffgeo
\def\Ric{\mathop\mathrm{Ric}}
\def\ric{\mathop\mathrm{ric}}
\def\tr{\mathop\mathrm{tr}}

\def\cotimes{\mathbin{\sqcup}}

\def\@rightopen#1{\ifx#1]{\right]}\else{\interval@errmessage}\fi}
\def\@leftclosed[#1){\left[#1\right)}
\makeatother

% finite
\newcommand{\fin}{\mathrm{fin}}

% Define groups (optarg: properties such as -> abelian, noetherian (acc), artinian (dcc) etc.)
\newcommand{\Grp}[1][\empty]{\if\empty{#1}{\mathbf{Grp}}\else{\mathbf{Grp}_{#1}}}
\def\PGL{\mathrm{PGL}}
\def\PGammaL{\mathrm{P\Gamma L}}
\def\GL{\mathrm{GL}}
% Define rings
\newcommand{\rg}{\mathrm{rg}} %rank of a matrix
\newcommand{\Rg}[1][\empty]{\if\empty{#1}{\mathbf{Rg}}\else{\mathbf{Rg}_{#1}}}
\edef\units#1{#1^{\settimes}}
\def\dual#1{#1^{\ast}}

%% redefine the command \P to produce the projective functor in math mode
\let\parsymb\P%
\def\P{\ifmmode\mathrm{P}\else\parsymb\fi}
\renewcommand{\iff}{\ifmmode\equival\else{if and only if}\fi}
\newcommand{\quotring}{\mathop{\mathrm{Q}}}
\newcommand{\rad}{\mathrm{rad}}
% Standard rings
% integral domains
\newcommand{\ID}{\mathbf{ID}}
% unique factorization domains
\newcommand{\UFD}{\mathbf{UFD}}
% principal ideal domains
\newcommand{\PID}{\mathbf{PID}}

% Define modules over a group or ring
\newcommand{\Mod}[1]{\mathbf{Mod}_{#1}}
% Define vector space over a field
\renewcommand{\Vec}[1]{\mathbf{Vec}_{#1}}%

% when cases
\def\otherwise{\textrm{otherwise}}

% new concepts
\newcommand{\new}[1]{\emph{#1}}

\usepackage{xifthen,xstring}

% replace the bar command by overline when argument just one character (shorter and better)
%$\let\oldbar\bar
%\renewcommand{\bar}[1]{\StrLen{#1}[\length]\ifthenelse{\length > 1}{\overline{#1}}{\oldbar{#1}}}
\def\bar{\overline}

%% argument in equatoin
\def\arg{\bullet}

% groups and algebras
\newcommand{\Con}{\mathop\mathrm{Con}}
\newcommand{\Sub}{\mathop\mathrm{Sub}}
\newcommand{\Hom}{\mathop\mathrm{Hom}}
\newcommand{\Aut}{\mathop\mathrm{Aut}}
\newcommand{\Out}{\mathop\mathrm{Out}}
\newcommand{\End}{\mathop\mathrm{End}}
\newcommand{\id}{\mathop\mathrm{id}}
\newcommand{\rk}{\mathop\mathrm{rk}} % rank of a group module/ lattice
\newcommandx{\con}[1][1=\empty]{\ifx#1\empty{\mathop{\mathrm{con}}}\else{\mathop{\mathrm{con}}\left(#1\right)}\fi}
\newcommand{\leftsemidirprod}[1][]{\mathbin{\ifx&#1&\ltimes\else{\ltimes_{#1}}\fi}}
\newcommand{\rightsemidirprod}[1][]{\ifx#1\empty\rtimes\else{\rtimes_{#1}}\fi}
\newcommand{\normalisor}[2][]{\ifx#1\empty{\mathrm{N}\left(#2\right)}\else{\mathrm{N}_{#1} \left(#2\right)}\fi}
% support
\newcommand{\spt}{\mathop{\mathrm{spt}}}
% commutator
\newcommand{\gcom}[2]{\left[#1,#2\right]}

% physics stuff
\newcommand{\float}[3][\empty]{\ifx#1\empty{{#2}\cdot{10^{#3}}}\else{{#2}\cdot{{#1}^{#3}}}\fi}
\makeatletter
\def\newunit#1{\@namedef{#1}{\mathrm{#1}}}
\def\mum{\mathrm{\mu m}}
\def\ohm{\Omega}
\newunit{V}
\newunit{mV}
\newunit{kV}
\newunit{s}
\newunit{ms}
\def\mus{\mathrm{\mu s}}
\newunit{m}
\newunit{nm}
\newunit{cm}
\newunit{mm}
\newunit{fF}
\newunit{A}

\newunit{fA}
\newunit{C}

% elements
\def\newelement#1{\@namedef{#1}{\mathrm{#1}}}
\newelement{Si}
\makeatother


% groups
\newcommand{\ord}{\mathop\mathrm{ord}}
\newcommand{\divides}{|}

\newcommand{\conleq}{\trianglelefteq}
\newcommand{\congeq}{\trianglerighteq}

% common algebraic objects
\newcommand{\reals}{\mathbb{R}} 			% real numbers
\newcommand{\nats}{\mathbb{N}} 				% natural numbers
\newcommand{\ints}{\mathbb{Z}} 				% integers
\newcommand{\rats}{\mathbb{Q}}				% rationals
\newcommand{\complex}{\mathbb{C}}			% complex numbers
\newcommand{\field}[1]{\mathbb{F}_{#1}}  		% finite field
\newcommand{\cards}{\boldsymbol{Cn}}                     % The cardinal numbers
\newcommand{\ords}{\boldsymbol{On}}                      % The ordinal numbers

% graphs
\def\KG{\mathop\mathrm{KG}}                     % Knesergraph

\newcommand{\uvect}{\boldsymbol{e}}

% new operators and relations

%%%%%%%%%%%%%%%%%%%
% complex numbers %
%%%%%%%%%%%%%%%%%%%

\renewcommand{\Re}{\mathop\mathrm{Re}}		% real part
\renewcommand{\Im}{\mathop\mathrm{Im}}		% imaginary part
\newcommand{\sgn}{\mathop\mathrm{sgn}}				% the sign operator (0 for 0)

%%%%%%%%%%%%%%%%
% reel numbers %
%%%%%%%%%%%%%%%%

\newcommand{\floor}[1]{\left\lfloor#1\right\rfloor}
\newcommand{\ceil}[1]{\left\lceil#1\right\rceil}

%%%%%%%%%%%%%%%%%%
% set operations %
%%%%%%%%%%%%%%%%%%

\newcommand{\intersect}{\cap}			% intersect to sets
\newcommand{\setjoin}{\cup}				% join two sets
\newcommand{\setmeet}{\cap}                     % intersect to sets
\newcommand{\bigsetjoin}{\bigcup}			% the union of sets ... subscripts to be added
\newcommand{\distunion}{\dot{\bigcup}}	% disjoint union of sets ... subscripts to be added
\newcommand{\bigsetmeet}{\bigcap}		% intersection of sets
\newcommand{\powerset}[1][]{\ifx&#1&\mathcal{P}\else\mathcal{P}_{#1}\fi}		% powerset ... to be customized
\newcommand{\card}[1]{\left|#1\right|}

%%%%%%%%%%%%%%%%%%%%%%%%%%%%%%%%%%%%%%%%
% composition operations of structures %
%%%%%%%%%%%%%%%%%%%%%%%%%%%%%%%%%%%%%%%%

%\newcommand{\setprod}{\bigtimes}			% setproduct - needed
\newcommand{\dirprod}{\bigotimes} 			% direct product for groups and spaces
\newcommand{\dirtimes}{\otimes}				% direct multiply for groups and spaces
\newcommand{\dirsum}{\bigoplus} 			% direct sum for groups and spaces
\newcommand{\dirplus}{\oplus}				% direct add for groups and spaces
\newcommand{\inprod}[2]{\left\langle #1,#2 \right\rangle}

\newcommand{\tuple}{\meet}
\newcommand{\cotuple}{\join}

%%%%%%%%%%%%%%
% categories %
%%%%%%%%%%%%%%

% combinatorics
%%

\renewcommand{\binom}[3][\empty]{\if\empty{#1}{{#2 \choose #3}}\else{{#2 \choose #3}_{#1}}}

%%%%%%%%%%%%%%%%%%%%%%%%%%%%%%%%%%%%%%%%%%%%%%%%%%%%%
% metric spaces and normed spaces and vector spaces %
%%%%%%%%%%%%%%%%%%%%%%%%%%%%%%%%%%%%%%%%%%%%%%%%%%%%%

\newcommand{\dist}{\mathop\mathrm{dist}}				% distance operator ... dist(A,b), where A is a set and b a point
\newcommand{\diam}{\mathop\mathrm{diam}}	% diameter operator for sets
\newcommand{\norm}[1]{\left\Vert #1 \right\Vert}	% norm in a normed space ... subscript to be added
\newcommand{\conv}{\mathop\mathrm{conv}} 			% convex hull - vectorspaces
\newcommand{\lin}{\mathop\mathrm{lin}} 				% linear hull - vectorspaces
\newcommand{\aff}{\mathop\mathrm{aff}}				% affine hull - vectorspaces

%%%%%%%%%%%%%%%%%%%%%%
% operators in rings %
%%%%%%%%%%%%%%%%%%%%%%

\newcommand{\lcm}{\mathop\mathrm{lcm}}				% least common multiple - in euclidean rings
\renewcommand{\gcd}{\mathop\mathrm{gcd}}				% greatest command devisor - in euclidean rings
\newcommand{\res}{\mathop\mathrm{res}}				% residue of p mod q is res(p,q)
\renewcommand{\mod}{\textrm{ mod }}

%%%%%%%%%%%%%%%%%%%
% logical symbols %
%%%%%%%%%%%%%%%%%%%

%\newcommand{\impliedby}{\Leftarrow}				% reverse implicatoin arrow
%\newcommand{\implies}{\Rightarrow}				% implication
\newcommand{\equival}{\Leftrightarrow}				% equivalence

%%%%%%%%%%%%%%%%%%%%%%%%%%%%%%%%%%
% functions - elementary symbols %
%%%%%%%%%%%%%%%%%%%%%%%%%%%%%%%%%%

\newcommand{\rest}[1]{\left. #1\right\vert}		% restriction of a function to a set / also used as restriction in other terms like differential expressions / evaluation of a function
\newcommand{\rto}[3][]{#2\ifx&#1&\rightarrow\else\stackrel{#1}{\rightarrow}\fi#3}
\renewcommand{\to}{\rightarrow}							% arrow between domain and image
\newcommand{\dom}{\mathop\mathrm{dom}}					% domain of a function
\newcommand{\im}{\mathop\mathrm{im}}						% image of a function
\newcommand{\compose}{\circ}							% compose two functions
\newcommand{\cont}{\mathop\mathrm{C}}					% continuous functions from a domain into the reels or complex numbers 

%%%%%%%%%%%%%%%%%%%%
% groups - symbols %
%%%%%%%%%%%%%%%%%%%%

\newcommand{\stab}[1][]{\if&#1{\mathop\mathrm{stab}}&\else{\mathop\mathrm{stab}_{#1}}\fi}					% the stabilizer ... subscripts to be added
\newcommand{\orb}[1][]{\ifx&#1&\mathrm{orb}\else\mathrm{orb}_{#1}\fi} 					% orbit ... subscrit to be added (group)
\newcommand{\gen}[2][\empty]{\ifx#1\empty{\left\langle#2\right\rangle}\else{\left\langle#2\right\rangle_{#1}}\fi}					% generate ... kind of hull operator ---- to be thought of !!!!!!!

\def\Clo{\mathrm{Clo}}
\def\Loc{\mathrm{Loc}}
%% nets
\newcommand{\net}[2][\empty]{\ifx#1\empty{\left(#2\right)}\else{{\left(#2\right)}_{#1}}\fi}

%% open half ray
\newcommand{\ray}[2]{R_{#1}(#2)}

%% new
\let\oldcong\cong%
\newcommand{\iso}{\oldcong}

\def\cong{\equiv}
\newcommand{\base}[2]{\left[#2\right]_{#1}}                                   % base n expansion of some number
%%%%%%%%%%%%%%%%%%%%%%%%%%%%%%%%%
% matrices and linear operators %
%%%%%%%%%%%%%%%%%%%%%%%%%%%%%%%%%

\newcommand{\diag}{\mathop\mathrm{diag}}				% diagonal matrix or operator
\newcommand{\Eig}[1]{\mathop\mathrm{Eig}_{#1}}		% eigenspace for a certain eigenvalie		
\newcommand{\trace}{\mathop\mathrm{tr}}				% trace of a matrix
\newcommand{\trans}{\top} 							% transponse matrix

%%%%%%%%%%%%%
% constants %
%%%%%%%%%%%%%

\renewcommand{\i}{\boldsymbol{i}}			% imaginary unit
\newcommand{\e}{\boldsymbol{e}}				% the Eulerian constant

%%%%%%%%%%%%%%%%%%%%%%%%%%%%%%
% limit operators and arrows %
%%%%%%%%%%%%%%%%%%%%%%%%%%%%%%

\newcommand{\upto}{\uparrow}				% convergence from above
\newcommand{\downto}{\downarrow}			% convergence from below

%%
% other
%%
\newcommand{\cind}{\mathop\mathrm{Ind}}		% Cauchy index
\newcommand{\sgnc}{\sigma}					% sign changes
\newcommand{\wnumb}{\omega}					% winding number
\newcommand{\cfunc}{\mathop\mathrm{Cf}}		% Cauchy function of a compact curve in complex\setminus\{0\}


% evaluation of a function as a difference or single value

\newcommand{\abs}[1]{\left|#1\right|}
\newcommand{\conj}[1]{\overline{#1}}
\newcommand{\diff}{\mathop\mathrm{d}}

%%%%% test
\newcommand{\distjoin}{\mathaccent\cdot\cup}	% to be modified (name)
\newcommand{\cl}{\mathop\mathrm{cl}}				% topological closure
\newcommand{\sphere}{\mathbb{S}} % n-sphere
\newcommand{\ball}{\mathbb{B}} % n-ball
\newcommand{\bound}{\partial}
\newcommand{\bigmeet}{\mathop\mathrm{\bigwedge}}
\newcommand{\bigjoin}{\mathop\mathrm{\bigvee}}
\newcommandx{\rchar}[1][1=\empty]{\mathop\mathrm{char}\brackets{#1}}              % characteristic of a ring
\newcommand{\lgor}{\vee}                               % logical
\newcommand{\lgand}{\wedge}
\newcommand{\codim}{\mathop\mathrm{codim}}
\newcommand{\row}{\mathop\mathrm{row}}
\newcommand{\cone}{\mathop\mathrm{cone}}
\newcommand{\comp}{\mathop\mathrm{comp}}
\newcommand{\proj}{\mathrm{P}}
\def\PG{\mathrm{PG}}           % projective space
\newcommand{\meet}{\wedge}
\newcommand{\join}{\vee}
\newcommand{\col}{\mathop\mathrm{col}}
\newcommand{\vol}{\mathop\mathrm{vol}\nolimits}
%arrangements
\newcommand{\tpert}{\mathop\mathrm{tpert}}
\renewcommand{\epsilon}{\varepsilon}
% groups
\newcommand{\symgr}{\mathop\mathrm{Sym}}
\newcommand{\symalg}{\mathop\mathrm{S}}
\newcommand{\extalg}{\mathop\mathrm{\Lambda}}
\newcommand{\extpow}[1]{\mathop\mathrm{\Lambda}^{#1}}

%% set hulloperators -> define



\newcommandx{\homl}[3][1=1,2=2,3=3]{\ifx#1\empty{\mathrm{Hom}}\else{\mathrm{Hom}_{#1}}\fi(#2,#3)}
\makeatletter
\newenvironment{myproofof}[1]{\par
  \pushQED{\qed}%
  \normalfont \topsep6\p@\@plus6\p@\relax
  \trivlist
  \item[\hskip\labelsep
        \bfseries
    Proof of #1\@addpunct{.}]\ignorespaces
}{%
  \popQED\endtrivlist\@endpefalse
}
\makeatother



%%% environment test with enumerates

    

% Local variables:
% mode: tex
% End:


\usepackage{tikz}
\usepackage{booktabs}

\begin{document}

\person{Gallager}-symmetrisch
schwach-symmetrisch
stark symmetrisch (Zeilen sind Permutationen voneinander) (somit \person{Gallager}- und schwach-symmetrisch)

25
\begin{exercise}
    Die folgenden Kanalmatrizen
    $$
    \begin{pmatrix}
        1/8 & 3/8 & 3/8 & 1/8\\
        3/8 & 3/8 & 3/8 & 1/8
    \end{pmatrix}
    $$
    ist stark symmetrisch.
    $$
    \begin{pmatrix}
        2/3 & 1/3 & 0\\
        0 & 1/3 & 2/3
    \end{pmatrix}
    $$
    schwach und \person{Gallager}-symmetrisch, aber nicht stark.
    $$
    \begin{pmatrix}
        0.7 & 0.2 & 0.1\\
        0.2 & 0.1 & 0.7
    \end{pmatrix}
    $$
    hat keine Symmetrie.
    $$
    \begin{pmatrix}
        0.1 & 0.2 & 0.3 & 0.4\\
        0.2 & 0.1 & 0.4 & 0.3
    \end{pmatrix}
    $$
    \person{Gallager}-symmetrisch.
    $$
    \begin{pmatrix}
        0.3 & 0.2 & 0.5\\
        0.2 & 0.5 & 0.3\\
        0.5 & 0.3 & 0.2
    \end{pmatrix}
    $$
    stark symmetrisch.
    $$
    \begin{pmatrix}
        p & 1-p & 0 & 0\\
        1-p & p & 0 & 0\\
        0 & 0 & q & 1-q\\
        0 & 0 & 1-q & q
    \end{pmatrix}
    $$
    ist stark symmetrisch für $p=1-q$ sonst nichts.
    $$
    \begin{pmatrix}
        0.1 & 0.2 & 0.3 & 0.4\\
        0.2 & 0.4 & 0.3 & 0.1\\
        0.3 & 0.1 & 0.2 & 0.4\\
        0.4 & 0.3 & 0.2 & 0.1
    \end{pmatrix}
    $$
    ist nur schwach symmetrisch.
\end{exercise}

26
\begin{exercise}
    
\end{exercise}

\begin{solution}
    Gegeben ist eine Kanalfolge $K_1$, $K_2$ mit $X_1$ am Eingang und $Y_1$ am Ausgang.
    \begin{tasks}
            \item $$C_1=\max_{p_{X_1}}{\rest{I(X_1,Y_1)}_{p_{X_1}}},$$ dabei sei $p'^\ast_{X_1}$ die optimale Verteilung zu Kanal 1.
        $$C=\rest{I(X_1,Y_2)}_{p^\ast_{X_1}},$$ mit $p^\ast_{X_1}$ die optimale Verteilung für den Gesamtkanal.
        Nun gilt laut Datenverarbeitungsungleichung
        $$
        C = \rest{I(X_1,Y_2)}_{p^\ast_{X_1}} \leq \rest{I(X_1,Y_1)}_{p^\ast_{X_1}}\leq \rest{I(X_1,Y_1)}_{p'^\ast_{X_1}}=C_1 
        $$
            \item Die Kanalmatrix vom Gesamtkanal ermittelt sich zu
        $$
        K = K_1 K_2.
        $$
            \item Gegeben sind zwei symmetrische Binärkanäle (BSC) mit Fehlerwahrscheinlichkeit $\epsilon_1$, Damit
        $$
        K_1 =
        \begin{pmatrix}
            1-\epsilon_1 & \epsilon_1\\
            \epsilon_1 & 1-\epsilon_1
        \end{pmatrix}
        $$
        analog ergibt sich
        $$
        K_2 =
        \begin{pmatrix}
            1-\epsilon_2 & \epsilon_2\\
            \epsilon_2 & 1-\epsilon_2
        \end{pmatrix}.
        $$
        Die Kanalmatrix des Gesamtkanals eine Matrix
        $$
        K = K_1 K_2 =
        \begin{pmatrix}
            1-\epsilon & \epsilon\\
            \epsilon & 1-\epsilon
        \end{pmatrix}
        $$
        mit $\epsilon = \epsilon_1(1-\epsilon_2)+\epsilon_2(1-\epsilon_1)$.
        D.h.~die Gesamtkanalkapazität ist BSC mit Fehlerwahrscheinlichkeit $\epsilon$.
        Also laut Script (3.19)
        $$
        C = 1- H_b(\epsilon).
        $$
            \item Wieder wird ein symmetrischer Binärkanal $K_1$ (BSC) mit einem Binärkanal mit Auslöschung (BEC) hintereinandergeschaltet:
        $$
K_1 =
\begin{pmatrix}
    1-\epsilon_2 & \epsilon_2 & 0\\
    0 & \epsilon_2 & 1-\epsilon_2
\end{pmatrix}
$$
Matrixmultiplikation liefert dann
$$
K =
\begin{pmatrix}
    (1-\epsilon_2)(1-\epsilon_1) & \epsilon_2 & \epsilon_1(1-\epsilon_2)\\
     \epsilon_1(1-\epsilon_2) & \epsilon_2 & (1-\epsilon_2)(1-\epsilon_1)
\end{pmatrix}.
$$

Diese ist \person{Gallager}-symmetrisch, denn die Zeilen sind Permutationen voneinander und es existiert eine Zerlegung in stark symmetrische Matrizen. Die optimale Eingangsverteilung ist also die Gleichverteilung.
Es berechnet sich dann $I(X_1,Y_2)=H(Y_2)-H(Y_2|X_1)$ für $X_1$ gleichverteilt. Es ergibt sich
$$
\begin{matrix}
    Y_1 & 0 & \Delta & 1 \\
    \midrule
    p_{Y_2}(y_2) & \frac 1 2(1-\epsilon_2) & \epsilon_2 & \frac 1 2 (1-\epsilon_2)
\end{matrix}
$$
also $H(Y_2)=(1-\epsilon_2)+H_b(\epsilon_2)$ und $H(Y_2|Y_1)=\frac 1 2 H(Y_2|X_1=0) + \frac 1 2 H(Y_2|X_1=1)=(1-\epsilon_2)H_b(\epsilon_1)+H_b(\epsilon_2)$. Also
$$C=(1-\epsilon_2)(1-H_b(\epsilon_1)).$$
Bei $\epsilon_1=0$ bleibt nur der BEC über, bei $\epsilon_2=0$ nur der BSC.
    \end{tasks}
\end{solution}
27
\begin{solution}
    \begin{tasks}
            \item Die Kanalmatrix für den Gesamtkanal ist zu ermitteln. Diese ergibt sich zu
        $$
        K = \alpha K_1 + (1-\alpha) K_2,
        $$
        da mit Wahrscheinlichkeit $\alpha$ der Kanal $K_1$ und mit $1-\alpha$ Kanal $K_2$ durchlaufen wird.
            \item Für feste Eingangsverteilung ist die Transinformation $I$ konvex bzgl. der Kanalmatrix (1.24), d.h.
        $$
        I(p_X,\alpha K_1+(1-\alpha)K_2)\leq \alpha I(p_X,K_1)+(1-\alpha)I(p_x,K_2).
        $$
        mit $I(X,Y)=:I(p_X,K)$.
        Sei $p_X^{(1)\ast}$ optimale Verteilung zu $K_1$ und
        $p_X^{(2)\ast}$ optimale Verteilung zu $K_2$, sowie
        $p^\ast$ optimale Verteilung zu $K$.
        Daraus folgt
        $$
        \eqalign{
            C & = I(p^\ast_X)\leq \alpha I(p^ast_X,K_1)+(1-\alpha) I(p^\ast_X,K_2)\cr
            & \leq \alpha I(p^{(1)\ast}_X,K_1)+(1-\alpha)I(p^{(2)\ast}_X,K_2)\cr
            & =\alpha C_1+(1-\alpha)C_2,}
        $$
        was die gesuchte Identität ist nach Definition der Informationskapazität.
            \item Kanalmatrix für den Gesamtkanal (wieder zwei BSC's hintereinandergeschaltet) ergibt sich zu
        $$
        K = \alpha
        \begin{pmatrix}
            1-\epsilon_1 & \epsilon_1\\
            \epsilon_1 & 1-\epsilon_1
        \end{pmatrix}
        +(1-\alpha)
        \begin{pmatrix}
            1-\epsilon_2 & \epsilon_2\\
            \epsilon_2 & 1-\epsilon_2
        \end{pmatrix}=
        \begin{pmatrix}
            1-\epsilon & \epsilon\\
            \epsilon & 1-\epsilon
        \end{pmatrix}
        $$
        mit $\epsilon=\alpha\epsilon_1+(1-\alpha)\epsilon_2$. D.h. der Gesamtkanal ist BSC mit Fehlerwahrscheinlichkeit $\epsilon$, damit laut Script $C = 1-H_b(\epsilon)$.
    \end{tasks}
\end{solution}

\section{4. Hausaufgabe zur LV Informationstheorie (28-30)}\hfill{19.06.2014}
\begin{exercise}[Stark- und schwach typische Sequenzen diskreter gedächtnisloser Quellen]
    Gegeben ist eine diskrete Gedächtnislose Quelle $(U_k)_k$ mit einem binären Alphabet $\cU=\set{0,1}$. Die Auftrittswahrscheinlichkeit für das Symbol $0$ sei $q$.
    \begin{align*}
        s_1&=(00100)\\
        s_2&=(10101)\\
        s_3&=(10111)\\
        s_4&=(11111)
    \end{align*}
    Entscheiden Sie für jede dieser Sequenzen, ob die Sequenz $\epsilon$-stark und oder $\epsilon$-schwach-typisch ist, wenn $\epsilon$ und $q$ folgende Werte haben.
    \begin{tasks}
            \item $\epsilon=0,15$, $q=1/3$.
            \item $\epsilon=0,3$, $q=1/7$.
    \end{tasks}
\end{exercise}

\begin{remark}
    Die Sequenz ist \emph{schwach typisch}, wenn sie die Ungleichung $\abs{-\frac 1 n \log_2(p_{\cU}^{(n)}(u^{(n)})-H(U)}\leq \epsilon$ erfüllt, der erste Term heißt dabei empirische Entropie (Bezeichnung $\hat {H(U)}^{(n)}$). Ist die Sequenz stark typisch, so muss die Ungleichung $\abs{\frac 1 n N(a|u^{(n)})-p_U(a)}<\epsilon/\abs{U}$ für alle $a\in\cU$.
\end{remark}

\begin{solution}
    Es gilt $p_U(s_1)=q^3{(1-q)}^2$, $p_U(s_2)=q^2{(1-q)}^3$, $p_U(s_3)=q{(1-q)}^4$, $p_U(s_4)={(1-q)}^5$.
    \begin{tasks}
            \item $\epsilon=0,15$, $q=1/3$, $p_U(s_1)=4/243$, $p_U(s_2)=8/243$, $p_U(s_3)=16/243$, $p_U(s_4)=32/243$.
        Damit erhält man $\hat H(s_1)=1,185$bit, $\hat H(s_2)=0,985$bit, $\hat H(s_3)=0,785$bit und $\hat H(s_4)=$. Es ergibt sich also:
        \begin{align*}
            \abs{\hat H(s_1)-H(U)}&=0,267\\
            \abs{\hat H(s_2)-H(U)}&=0,067\\
            \abs{\hat H(s_3)-H(U)}&=0,133\\
            \abs{\hat H(s_4)-H(U)}&=0,333
        \end{align*}
    
    Damit ist der Abstand für $s_2, s_3$ kleiner als $\epsilon=0,15$ also sind sie schwach typisch.
    Nun testen wir auf start-typisch.
    \begin{align*}
        \begin{matrix}
            s_i & N(0|s_i) & N(1|s_i) & \abs{1/n N(a|s_i)-p_U(a)}\\
            s_1 & 3 & 2 & \abs{3/5-1/3}=0,2667\\ 
            s_2 & 2 & 3 & \abs{2/5-1/3}=0,067\\
            s_3 & 1 & 4 & \abs{1/5-1/3}=0,133\\
            s_4 & 0 & 5 & 1/3
        \end{matrix}
    \end{align*}
    Damit sind $s_2,s_3$ stark- und und schwach-typisch. $s_1,s_4$ sind weder das eine noch das andere.
    \begin{remark}
        Im binären gilt $\abs{1/n N(0|s)-p_U(0)}=\abs{1/n N(1|s)-p_U(1)}$, da $\abs{1/nN(1|s)-p_U(1)}=\abs{1/n(n-N(1|s))-(1-p_U(0))}=\abs{1/n N(0|s)-p_U(0)}$.
    \end{remark}
    \item Analog berechnet man, dass hier $s_3$ stark und schwach typisch ist, $s_4$ nur stark typisch ist und $s_1,s_2$ beides nicht sind.
\begin{remark}
    Es sind also alle Kombinationen aus stark und schwach typisch möglich.
\end{remark}
\end{tasks}
\end{solution}

\begin{exercise}[Informationskapazität]
    Gegeben seien zwei unabhängige Zufallsvariablen $X$ und $Z$ mit Alphabeten $\cX=\set{1,2,3}$. Die Zufallsgröße $Z$ besitzt die Wahrscheinlichkeitsdichte $p_Z$, die wie folgt gegeben ist:
    $$
    \begin{matrix}
        z & 1 & 2 & 3\\
        \midrule
        p_z(\delta) & \epsilon & \delta & 1 -\delta-\epsilon
    \end{matrix}.
    $$
    Es soll $\epsilon,\delta\in[0,1/2]$ gelten. Es sei $Y$ durch $Y:=X+Z \mod 3$ gegeben.
    \begin{tasks}
            \item Gib das Alphabet von $Y$ an.
            \item Bestimme die Wahrscheinlichkeiten $p_{Y|X}(\arg|x)$ für alle $x\in\cX$.
            \item Wir wollen nun $X$ und $Y$ als Eingang bzw.~Ausgang eineds DMC betrachten. Berechnen Sie die Informationskapazität diese DMC in Abhängigkeit von $\epsilon, \delta$ und geben Sie die zugehörige optimale Wahrscheinlichkeitsverteilung an.
        \item Sei $\epsilon=\delta$, für welches $\epsilon$ nimmt nun die in der vorherigen Teilaufgabe berechnete Informationskapazität ihr Minimum und Maximum an.
    \end{tasks}
\end{exercise}

\begin{solution}
    \begin{tasks}
        \item $\cY=\cX$.
            \item Folgende Tabelle ergibt sich:
        $$
        \begin{matrix}
            X & Z & X+Z & Y & P_{Y|X}(y|x)\\
            \midrule
            1 & 1 & 2 & 2 & \delta\\
            1 & 2 & 3 & 2 & \epsilon\\
            1 & 3 & 4 & 1 & 1-\delta-\epsilon\\
            2 & 1 & 3 & 0 & \delta\\
            2 & 2 & 4 & 1 & \epsilon\\
            2 & 3 & 5 & 2 & 1-\delta-\epsilon\\
            3 & 1 & 4 & 1 & \delta\\
            3 & 2 & 5 & 2 & \epsilon\\
            3 & 3 & 6 & 0 & 1-\delta-\epsilon
        \end{matrix}
        $$
        Man hat also die Kanalmatrix:
        $$
        \begin{matrix}
            p_{Y|X} (x,y) & 0 & 1 & 2\\
            1 & \epsilon & 1-\delta-\epsilon & \delta\\
            2 & \delta & \epsilon & 1-\delta-\epsilon\\
            3 & 1-\delta-\epsilon & \delta & \epsilon
        \end{matrix}.
        $$
            \item Der DMC ist symmetrisch. Die Informationskapazität wird durch Gleichverteilung am Eingang erreicht (Skript §3.21). Für die Kapazität gilt (§3.22)
        $$
        C = \log_2\card\cY - H(r)
        $$
        hier also:
        $$
        C = \log_2 3 - (-\epsilon\log_2\epsilon-\delta\log\delta-(1-\delta-\epsilon)\log_2(1-\delta-\epsilon).
        $$
        \item Hier $\epsilon=\delta$. Dann wird die Entropie maximal bei Gleichverteilung, die Kapazität also minimal für $\epsilon=\delta=1/3$.
    $C_{\min}=0$. Die Entropie wird minimal für $\epsilon=0$, dann ist $C_{\max}=\log_2 3$.
\end{tasks}
\end{solution}

\begin{exercise}[Hintereinanderschaltung von Multiplikationskanälen]
    Gegeben sei ein diskreter gedächtnisloser Multiplikationskanal mit binärem Ein- und Ausgang ($X_1, Y_1$).
    Das multiplikative Rauschen wird durch die vom Kanaleingang unabhängige zufallsgröße $Z_1$ mit binärem Alphabet $\cZ_1=\set{0,1}$ beschrieben, wobei $P(Z_1=1)=\epsilon_1$ sei.
    \begin{tasks}
        \item Bestimme die Kanalmatrix und die Informationskapazität.
    \end{tasks}
\end{exercise}

\begin{solution}
    \begin{tasks}
            \item Siehe Aufgabe 16:
        $$
        K_1=
        \begin{pmatrix}
            1 & 0\\
            1-\epsilon_1 & \epsilon_1
        \end{pmatrix}
        $$
        Es gilt $C_1 = \max_{p_X}{I(x;y_1)} = \log_2(1+\epsilon_1{(1-\epsilon_1)}^{\frac{1-\epsilon_1}{\epsilon_1}})$bit mit
            $$
            p_X = \frac 1 {{(1-\epsilon_1)}^{\frac{\epsilon_1-1}{\epsilon_1}}+\epsilon_1}.
            $$
                \item Durch Hintereinanderschaltung zweier Kanäle entsteht die Gesamtkanalmatrix durch Multiplikation der beiden Teilkanalmatrizen ($K=K_1K_2$).
            $$
            K =
            \begin{pmatrix}
                1 & 0 \\
                1-\epsilon_1 & \epsilon_1
            \end{pmatrix}
            \begin{pmatrix}
                1 & 0 \\
                1-\epsilon_2 & \epsilon_2
            \end{pmatrix}
            =
            \begin{pmatrix}
                1 & 0 \\
                1-\epsilon & \epsilon
            \end{pmatrix}
            $$
        \end{tasks}
        mit $\epsilon=\epsilon_1\epsilon_2$. Gesamtkapazität berechnet sich nach derselben Formel
        $$
        C_{\text{ges}} = \log_2(1+\epsilon{(1-\epsilon)}^{\frac{1-\epsilon}{\epsilon}})
        $$
        und
        $$
        p_X = \frac 1 {{(1-\epsilon)}^{\epsilon-1}{\epsilon}+\epsilon}.
        $$
        Der Gesamtkanal ist wieder ein $Z$-Kanal mit Parameter $\epsilon=\epsilon_1\epsilon_2$.
\end{solution}

\hfill{26.06.14}
% 31
\begin{exercise}
    
\end{exercise}

\begin{solution}
    \begin{tasks}
            \item Der Erwartungswert der Zufallsgröße ist $0$, da die Verteilung symmetrisch um $0$ ist, d.h. $f(x)=\frac 1 a\left(1-\abs{\frac x a}\right)$. Die Varianz bestimmen wir, indem wir $E({(X-E(X))}^2)$ berechnen, dann erhalten wir also
        $$
        \int_{-a}^a{\frac {x^2} a\left(1-\abs{\frac x a}\right)}\diff x = 2a^2\int_0^1{y^2(1-y)}\diff y =\frac {a^2} 6. 
        $$
            \item Gegeben ist eine Zufallsgröße $X$ mit und $Y=g(X)$ mit $f_X$ Dichte von $X$ und $f_Y$ Dichte von $Y$. Es sei $g(x)=\exp(x/a)$. Der Transformationssatz für Dichten besagt:
        $$
        f_Y(y)=\frac{f_X(g^{-1}(y))}{\abs{g'(g^{-1}(y))}}
        $$
        Ableitung ist dann $g'(x)=\frac 1 a \exp(x/a)$, Umkehrfunktion $g^{-1}(y)=a\log(y)$ ($y\in (0,\infty)$). Träger von $Y$ ist $\bar{g(-a,a)}=[e^{-1},e]$. Also für $y\in[e^{-1},e]$ gilt:
        $$
        f_Y(y)=\frac{\frac 1 a\left(1-\abs{\frac{a\log(y)}{a}}\right)}{\frac 1 a\exp\left(\frac{a\log(y)}{a}\right)}
        $$
        insgesamt
        $$
        f_Y(y)=
        \begin{cases}
            \frac 1 y\left(1-\abs{\log(y)}\right) &: y\in [e^{-1},e]\\
            0 &: \text{sonst}
        \end{cases}
        $$
            \item Es ist
        \begin{align*}
            h(X) &
            =-\int_{-\infty}^\infty{\log_2(f_X(x))f_X(x)}\\
            &
            = -\int_{-a}^a{\frac 1 a\left(1-\abs{\frac x a}\right)\log_2\left(\frac 1 a\left(1-\abs{\frac x a}\right)\right)}\\
            &
            = 2a^2\int_{y=1/a}^0{y\log_2(y)}\diff y\\
            &
            = \rest{\frac{-2a^2}{\log(2)}\left(y^2\left(\frac{\log(y)}{4}-\frac 1 4\right)\right)}_{1/a}^0\\
            & = \log_2(\sqrt e a)
        \end{align*}
    \end{tasks}
\end{solution}

\begin{solution}
    Die Dichten von $X$ und $Y$ sind wie folgt gegeben durch
    $$
    f_X(x)=\frac 1 {\sqrt{2\pi\sigma_X^2}}\exp(-{(\frac x {\sigma_X})}^2)
        $$ und $f_Y$ analog.
        Dann ist
        \begin{align*}
            D(X||Y) &= \int_{-\infty}^\infty{f_X(x)\log_2\left(\frac{f_X(x)}{f_Y(x)}\right)}\diff x\\
            & = \int_{-\infty}^\infty{\frac 1 {\sqrt{2\pi\sigma_X^2}}\exp\left(-\frac 1 2{\left(\frac x {\sigma_X}\right)}^2\right)\log_2\left(\sqrt{\frac{\sigma_Y^2}{\sigma_X^2}}\exp\left(-\frac 1 2\left(\frac 1 {\sigma_X^2}-\frac 1 {\sigma_Y^2}\right)x^2\right)\right)}\diff x\\
            & =
            \frac 1 2\log_2\left(\frac{\sigma_Y^2}{\sigma_X^2}\right)E(X)+\frac 1{\log(2)}\frac 1 2\left(\frac 1 {\sigma_Y^2}-\frac 1 {\sigma_X^2}\right)V(X)\\
            & = \frac 1 {2\log(2)}\left(2\log\left(\frac{\sigma_Y}{\sigma_X}\right)+\left(\frac{\sigma_X^2}{\sigma_Y^2}-1\right)\right)
        \end{align*}
\end{solution}

% 33
\begin{solution}
    Wir haben $Z_1:=X_1+Y$, $Z_2:=X_2+Y$, $Z_1$ ist normalverteilt, denn die Summe normalverteilter Zufallsgrößen ist normalverteilt. $E(Z_1)=E(X_1)+E(Y)=0+0=0$ und die Varianz ist $V(Z_1)=V(X_1)+V(Y)=\sigma_1^2+\sigma_Y^2$, da $X_1, Y$ unkorreliert (da unabhängig).
    Entsprechend ist $E(Z_2)=0$ und $V(Z_2)=\sigma_2^2+\sigma_Y^2$.
    \begin{tasks}
            \item Mit Aufgabe 32 erhalten wir:
        $$
        D(X_1+Y||X_2+Y)=\frac 1 {2\log(2)}\left(\log\left(\frac{\sigma_2^2+\sigma_Y^2}{\sigma_1^2+\sigma_Y^2}\right)+\frac{\sigma_1^2+\sigma_Y^2}{\sigma_2^2+\sigma_Y^2}-1\right)
        $$
            \item Es gilt $D(X_1||X_2)\geq D(X_1+Y||X_2+Y)$, da
        $$
        \log\left(\frac{b+x}{a+x}\right)+\frac{a+x}{b+x}-1
        $$
        monoton fallend in $x$ ist.
        Die Ableitung ist
        $$
        \frac{a+x}{b+x}\frac{a-b}{{(a+x)}^2}+\frac{b-a}{{(b+x)}^2} = \frac{a-b}{b+x}\left(\frac1 {a+x}-\frac 1 {b+x}\right)=\frac{-{(b-a)}^2}{ {(b+x)}^2(a+x)}<0
        $$
    \end{tasks}
    \begin{remark}
        Durch additive Störung werden die Verteilungen der Zufallsgrößen $X_1$ und $X_2$ ähnlicher, daher $D(\arg || \arg)$ kleiner.
    \end{remark}
\end{solution}

% 34
\begin{solution}
    Äquivalentes Modell: $\tilde Z = Z_1+Z_2$, $\tilde Z$ normalverteilt, und $Y=X+\tilde Z$.
    \begin{tasks}
            \item $E(\tilde Z)=E(Z_1)+E(Z_2)=0$ und $V(\tilde Z)=V(Z_1)+V(Z_2)=\sigma_1^2+\sigma_2^2$ (da $Z_1$ und $Z_2$ unabhängig sind.
        $X$ und $(Z_1,Z_2)$ sind unabhängig, also Modell \person{Gauß}-Kanal (Hintereinanderschaltung von zwei \person{Gauß}-Kanälen).
        $$
        \max_{E(X^2)\leq P}{I(X;Y)} = \max_{E(X^2)\leq P}{I(X;X+Z)} = \frac 1 2\log_2\left(1+\frac{P}{\sigma_1^2+\sigma_2^2}\right)
            $$ nach der Formel für die Informationskapazität des \person{Gauß}-Kanals (§3.51).
                \item Es ergibt sich
            \begin{align*}
                V(\tilde Z) &= V(Z_1)+2C(Z_1,Z_2)+V(Z_2)\\
                &=\sigma_1^2+2\alpha\sigma_1\sigma_2+\sigma_2^2
            \end{align*}
            Wieder muss in die Formel für die Transinformation eingesetzt werden.
        \end{tasks}
\end{solution}
\end{document}






